\chapter{Questionário GQM}\label{apend:gqm}
Para identificar a possibilidade de integrar o monitoramento da saúde do jogador através de jogos eletrônicos à sua rotina diária, foi utilizada a abordagem \textit{Goal, Question, Metric} (GQM). GQM ~\cite{basili94} é uma abordagem hierárquica que inicia com objetivo principal e o divide em atividades que podem ser mensuradas durante a pesquisa. É uma abordagem para integrar objetivos a produtos e perspectivas de qualidade de interesse, baseado nas necessidades do produto~\cite{van1999goal}. 
Foi preparado o questionário GQM mostrado na Tabela~\ref{tab:gqm} para avaliar a possibilidade de monitorar dados motores de forma não invasiva e integrada a rotina diária das pessoas.

\begin{longtable}{|p{\textwidth}|}
\caption{O Questionário GQM}\\
\hline
\endfirsthead
\multicolumn{1}{c}%
{\tablename\ \thetable\ -- \textit{Continuação da página anterior}} \\
\hline
\endhead
\hline \multicolumn{1}{r}{\textit{Continua na próxima página}} \\
\endfoot
\hline
\endlastfoot
\textbf{\textit{Objetivo principal}}: Avaliar a possibilidade de monitorar dados motores de forma não invasiva e integrada a rotina diária das pessoas. \\ \hline
\textbf{\textit{Questão 1}}: O usuário poderia integrar a abordagem \textit{JOGUE-ME} à sua rotina diária ?\\ \hline
\textit{Métrica 1.1}: Numa escala de 1 a 5 qual o grau de diversão do jogo? \\ \hline
\textit{Métrica 1.2}: O jogo traz motivação ao usuário (Sim/Não) \\ \hline
\textit{Métrica 1.3}: Se o usuário tivesse adquirido esse jogo, com que frequência o utilizaria durante a semana? (1 vez/3 vezes/Todos os dias/Nunca usaria) \\ \hline
\textit{Métrica 1.4}: O usuário considera o jogo simples, sem muitas regras e de fácil entendimento ? Ele pode ser aplicado em diferentes idades? (Sim/ Não) \\ \hline
\textit{Métrica 1.5}: O usuário tem o costume de jogar esses jogos casuais em casa? (Sim/ Não) \\ \hline
\textit{Métrica 1.6}: Você agregaria um jogo desse estilo em sua rotina diária? (Sim/ Não) \\ \hline
\textbf{\textit{Questão 2}}: A segurança com a integridade física está de acordo com a faixa etária do usuário ? \\ \hline
\textit{Métrica 2.1}: Uma criança estaria segura jogando esse jogo, ao efetuar os movimentos dos braços? \\ \hline
\textit{Métrica 2.2}: Um adulto estaria seguro ao jogar esse jogo, ao efetuar os movimentos dos braços? \\ \hline
\textit{Métrica 2.3	}: Um idoso estaria seguro ao jogar esse jogo, ao efetuar os movimentos dos braços? \\ \hline
\textit{Métrica 2.4}: Qual opinião do usuário sobre a faixa etária do jogo? (Livre/Crianças/Adultos/Idosos)
\label{tab:gqm}
\end{longtable}


A preocupação principal dessa pesquisa  é avaliar se os uso grau de entretenimento dos jogadores, a possibilidade de integrar jogos para monitoramento na rotina dos jogadores, motivação para jogar, segurança e opinião do jogador em relação ao monitoramento da saúde.
%
%\begin{enumerate}
  %\item Numa escala de 1 a 5 qual o grau de diversão do jogo?
  %\item Você agregaria um jogo desse estilo em sua rotina diária? (Sim/Não)
  %\item Se você tivesse adquirido esse jogo, com que frequência você o utilizaria durante a semana? (1 vez/3 vezes/Todo dia/Nunca)
  %\item Você considera o jogo simples, sem muitas regras, de fácil entendimento e para diferentes idades? (Sim/Não)
  %\item Você sentiria motivado a jogar esse jogo? (Sim/Não)
  %\item Você tem o costume de jogar esses jogos casuais? (Sim/Não)
  %\item Uma criança estaria segura jogando esse jogo, ao efetuar os movimentos dos braços e das pernas (Sim/Não)
  %\item Um adulto estaria seguro ao jogar esse jogo, ao efetuar os movimentos dos braços e das pernas (Sim/Não)
  %\item Um idoso estaria seguro ao jogar esse jogo, ao efetuar os movimentos dos braços e das pernas (Sim/Não)
  %\item A qual faixa etária você acha o jogo é direcionado? (Livre/Crianças/Adultos/Idosos)
%\end{enumerate}

%\begin{description}
	%\item[Objetivo 1] Avaliar se o usuário integraria o jogo à sua rotina diária.
	%\item[Objetivo 2] Avaliar a segurança com a integridade física; identificar a faixa etária do usuário.
%\end{description}