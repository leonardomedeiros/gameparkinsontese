
%% bare_conf.tex
%% V1.3
%% 2007/01/11
%% by Michael Shell
%% See:
%% http://www.michaelshell.org/
%% for current contact information.
%%
%% This is a skeleton file demonstrating the use of IEEEtran.cls
%% (requires IEEEtran.cls version 1.7 or later) with an IEEE conference paper.
%%
%% Support sites:
%% http://www.michaelshell.org/tex/ieeetran/
%% http://www.ctan.org/tex-archive/macros/latex/contrib/IEEEtran/
%% and
%% http://www.ieee.org/

%%*************************************************************************
%% Legal Notice:
%% This code is offered as-is without any warranty either expressed or
%% implied; without even the implied warranty of MERCHANTABILITY or
%% FITNESS FOR A PARTICULAR PURPOSE! 
%% User assumes all risk.
%% In no event shall IEEE or any contributor to this code be liable for
%% any damages or losses, including, but not limited to, incidental,
%% consequential, or any other damages, resulting from the use or misuse
%% of any information contained here.
%%
%% All comments are the opinions of their respective authors and are not
%% necessarily endorsed by the IEEE.
%%
%% This work is distributed under the LaTeX Project Public License (LPPL)
%% ( http://www.latex-project.org/ ) version 1.3, and may be freely used,
%% distributed and modified. A copy of the LPPL, version 1.3, is included
%% in the base LaTeX documentation of all distributions of LaTeX released
%% 2003/12/01 or later.
%% Retain all contribution notices and credits.
%% ** Modified files should be clearly indicated as such, including  **
%% ** renaming them and changing author support contact information. **
%%
%% File list of work: IEEEtran.cls, IEEEtran_HOWTO.pdf, bare_adv.tex,
%%                    bare_conf.tex, bare_jrnl.tex, bare_jrnl_compsoc.tex
%%*************************************************************************

% *** Authors should verify (and, if needed, correct) their LaTeX system  ***
% *** with the testflow diagnostic prior to trusting their LaTeX platform ***
% *** with production work. IEEE's font choices can trigger bugs that do  ***
% *** not appear when using other class files.                            ***
% The testflow support page is at:
% http://www.michaelshell.org/tex/testflow/



% Note that the a4paper option is mainly intended so that authors in
% countries using A4 can easily print to A4 and see how their papers will
% look in print - the typesetting of the document will not typically be
% affected with changes in paper size (but the bottom and side margins will).
% Use the testflow package mentioned above to verify correct handling of
% both paper sizes by the user's LaTeX system.
%
% Also note that the "draftcls" or "draftclsnofoot", not "draft", option
% should be used if it is desired that the figures are to be displayed in
% draft mode.
%
\documentclass[10pt, conference, compsocconf]{IEEEtran}
% Add the compsocconf option for Computer Society conferences.
%
% If IEEEtran.cls has not been installed into the LaTeX system files,
% manually specify the path to it like:
% \documentclass[conference]{../sty/IEEEtran}





% Some very useful LaTeX packages include:
% (uncomment the ones you want to load)


% *** MISC UTILITY PACKAGES ***
%
%\usepackage{ifpdf}
% Heiko Oberdiek's ifpdf.sty is very useful if you need conditional
% compilation based on whether the output is pdf or dvi.
% usage:
% \ifpdf
%   % pdf code
% \else
%   % dvi code
% \fi
% The latest version of ifpdf.sty can be obtained from:
% http://www.ctan.org/tex-archive/macros/latex/contrib/oberdiek/
% Also, note that IEEEtran.cls V1.7 and later provides a builtin
% \ifCLASSINFOpdf conditional that works the same way.
% When switching from latex to pdflatex and vice-versa, the compiler may
% have to be run twice to clear warning/error messages.






% *** CITATION PACKAGES ***
%
%\usepackage{cite}
% cite.sty was written by Donald Arseneau
% V1.6 and later of IEEEtran pre-defines the format of the cite.sty package
% \cite{} output to follow that of IEEE. Loading the cite package will
% result in citation numbers being automatically sorted and properly
% "compressed/ranged". e.g., [1], [9], [2], [7], [5], [6] without using
% cite.sty will become [1], [2], [5]--[7], [9] using cite.sty. cite.sty's
% \cite will automatically add leading space, if needed. Use cite.sty's
% noadjust option (cite.sty V3.8 and later) if you want to turn this off.
% cite.sty is already installed on most LaTeX systems. Be sure and use
% version 4.0 (2003-05-27) and later if using hyperref.sty. cite.sty does
% not currently provide for hyperlinked citations.
% The latest version can be obtained at:
% http://www.ctan.org/tex-archive/macros/latex/contrib/cite/
% The documentation is contained in the cite.sty file itself.

%Compactitem
\usepackage{paralist}

% Graphics
\usepackage[pdftex]{graphicx}
%\usepackage[tight,footnotesize]{subfigure}
\usepackage{subfigure}

%Table background
\usepackage[table,xcdraw]{xcolor}
% *** GRAPHICS RELATED PACKAGES ***

%degree com
\usepackage{gensymb}




% *** GRAPHICS RELATED PACKAGES ***
%
\ifCLASSINFOpdf
  % \usepackage[pdftex]{graphicx}
  % declare the path(s) where your graphic files are
  % \graphicspath{{../pdf/}{../jpeg/}}
  % and their extensions so you won't have to specify these with
  % every instance of \includegraphics
  % \DeclareGraphicsExtensions{.pdf,.jpeg,.png}
\else
  % or other class option (dvipsone, dvipdf, if not using dvips). graphicx
  % will default to the driver specified in the system graphics.cfg if no
  % driver is specified.
  % \usepackage[dvips]{graphicx}
  % declare the path(s) where your graphic files are
  % \graphicspath{{../eps/}}
  % and their extensions so you won't have to specify these with
  % every instance of \includegraphics
  % \DeclareGraphicsExtensions{.eps}
\fi
% graphicx was written by David Carlisle and Sebastian Rahtz. It is
% required if you want graphics, photos, etc. graphicx.sty is already
% installed on most LaTeX systems. The latest version and documentation can
% be obtained at: 
% http://www.ctan.org/tex-archive/macros/latex/required/graphics/
% Another good source of documentation is "Using Imported Graphics in
% LaTeX2e" by Keith Reckdahl which can be found as epslatex.ps or
% epslatex.pdf at: http://www.ctan.org/tex-archive/info/
%
% latex, and pdflatex in dvi mode, support graphics in encapsulated
% postscript (.eps) format. pdflatex in pdf mode supports graphics
% in .pdf, .jpeg, .png and .mps (metapost) formats. Users should ensure
% that all non-photo figures use a vector format (.eps, .pdf, .mps) and
% not a bitmapped formats (.jpeg, .png). IEEE frowns on bitmapped formats
% which can result in "jaggedy"/blurry rendering of lines and letters as
% well as large increases in file sizes.
%
% You can find documentation about the pdfTeX application at:
% http://www.tug.org/applications/pdftex





% *** MATH PACKAGES ***
%
%\usepackage[cmex10]{amsmath}
% A popular package from the American Mathematical Society that provides
% many useful and powerful commands for dealing with mathematics. If using
% it, be sure to load this package with the cmex10 option to ensure that
% only type 1 fonts will utilized at all point sizes. Without this option,
% it is possible that some math symbols, particularly those within
% footnotes, will be rendered in bitmap form which will result in a
% document that can not be IEEE Xplore compliant!
%
% Also, note that the amsmath package sets \interdisplaylinepenalty to 10000
% thus preventing page breaks from occurring within multiline equations. Use:
%\interdisplaylinepenalty=2500
% after loading amsmath to restore such page breaks as IEEEtran.cls normally
% does. amsmath.sty is already installed on most LaTeX systems. The latest
% version and documentation can be obtained at:
% http://www.ctan.org/tex-archive/macros/latex/required/amslatex/math/





% *** SPECIALIZED LIST PACKAGES ***
%
%\usepackage{algorithmic}
% algorithmic.sty was written by Peter Williams and Rogerio Brito.
% This package provides an algorithmic environment fo describing algorithms.
% You can use the algorithmic environment in-text or within a figure
% environment to provide for a floating algorithm. Do NOT use the algorithm
% floating environment provided by algorithm.sty (by the same authors) or
% algorithm2e.sty (by Christophe Fiorio) as IEEE does not use dedicated
% algorithm float types and packages that provide these will not provide
% correct IEEE style captions. The latest version and documentation of
% algorithmic.sty can be obtained at:
% http://www.ctan.org/tex-archive/macros/latex/contrib/algorithms/
% There is also a support site at:
% http://algorithms.berlios.de/index.html
% Also of interest may be the (relatively newer and more customizable)
% algorithmicx.sty package by Szasz Janos:
% http://www.ctan.org/tex-archive/macros/latex/contrib/algorithmicx/




% *** ALIGNMENT PACKAGES ***
%
%\usepackage{array}
% Frank Mittelbach's and David Carlisle's array.sty patches and improves
% the standard LaTeX2e array and tabular environments to provide better
% appearance and additional user controls. As the default LaTeX2e table
% generation code is lacking to the point of almost being broken with
% respect to the quality of the end results, all users are strongly
% advised to use an enhanced (at the very least that provided by array.sty)
% set of table tools. array.sty is already installed on most systems. The
% latest version and documentation can be obtained at:
% http://www.ctan.org/tex-archive/macros/latex/required/tools/


%\usepackage{mdwmath}
%\usepackage{mdwtab}
% Also highly recommended is Mark Wooding's extremely powerful MDW tools,
% especially mdwmath.sty and mdwtab.sty which are used to format equations
% and tables, respectively. The MDWtools set is already installed on most
% LaTeX systems. The lastest version and documentation is available at:
% http://www.ctan.org/tex-archive/macros/latex/contrib/mdwtools/


% IEEEtran contains the IEEEeqnarray family of commands that can be used to
% generate multiline equations as well as matrices, tables, etc., of high
% quality.


%\usepackage{eqparbox}
% Also of notable interest is Scott Pakin's eqparbox package for creating
% (automatically sized) equal width boxes - aka "natural width parboxes".
% Available at:
% http://www.ctan.org/tex-archive/macros/latex/contrib/eqparbox/





% *** SUBFIGURE PACKAGES ***
%\usepackage[tight,footnotesize]{subfigure}
% subfigure.sty was written by Steven Douglas Cochran. This package makes it
% easy to put subfigures in your figures. e.g., "Figure 1a and 1b". For IEEE
% work, it is a good idea to load it with the tight package option to reduce
% the amount of white space around the subfigures. subfigure.sty is already
% installed on most LaTeX systems. The latest version and documentation can
% be obtained at:
% http://www.ctan.org/tex-archive/obsolete/macros/latex/contrib/subfigure/
% subfigure.sty has been superceeded by subfig.sty.



%\usepackage[caption=false]{caption}
%\usepackage[font=footnotesize]{subfig}
% subfig.sty, also written by Steven Douglas Cochran, is the modern
% replacement for subfigure.sty. However, subfig.sty requires and
% automatically loads Axel Sommerfeldt's caption.sty which will override
% IEEEtran.cls handling of captions and this will result in nonIEEE style
% figure/table captions. To prevent this problem, be sure and preload
% caption.sty with its "caption=false" package option. This is will preserve
% IEEEtran.cls handing of captions. Version 1.3 (2005/06/28) and later 
% (recommended due to many improvements over 1.2) of subfig.sty supports
% the caption=false option directly:
%\usepackage[caption=false,font=footnotesize]{subfig}
%
% The latest version and documentation can be obtained at:
% http://www.ctan.org/tex-archive/macros/latex/contrib/subfig/
% The latest version and documentation of caption.sty can be obtained at:
% http://www.ctan.org/tex-archive/macros/latex/contrib/caption/




% *** FLOAT PACKAGES ***
%
%\usepackage{fixltx2e}
% fixltx2e, the successor to the earlier fix2col.sty, was written by
% Frank Mittelbach and David Carlisle. This package corrects a few problems
% in the LaTeX2e kernel, the most notable of which is that in current
% LaTeX2e releases, the ordering of single and double column floats is not
% guaranteed to be preserved. Thus, an unpatched LaTeX2e can allow a
% single column figure to be placed prior to an earlier double column
% figure. The latest version and documentation can be found at:
% http://www.ctan.org/tex-archive/macros/latex/base/



%\usepackage{stfloats}
% stfloats.sty was written by Sigitas Tolusis. This package gives LaTeX2e
% the ability to do double column floats at the bottom of the page as well
% as the top. (e.g., "\begin{figure*}[!b]" is not normally possible in
% LaTeX2e). It also provides a command:
%\fnbelowfloat
% to enable the placement of footnotes below bottom floats (the standard
% LaTeX2e kernel puts them above bottom floats). This is an invasive package
% which rewrites many portions of the LaTeX2e float routines. It may not work
% with other packages that modify the LaTeX2e float routines. The latest
% version and documentation can be obtained at:
% http://www.ctan.org/tex-archive/macros/latex/contrib/sttools/
% Documentation is contained in the stfloats.sty comments as well as in the
% presfull.pdf file. Do not use the stfloats baselinefloat ability as IEEE
% does not allow \baselineskip to stretch. Authors submitting work to the
% IEEE should note that IEEE rarely uses double column equations and
% that authors should try to avoid such use. Do not be tempted to use the
% cuted.sty or midfloat.sty packages (also by Sigitas Tolusis) as IEEE does
% not format its papers in such ways.





% *** PDF, URL AND HYPERLINK PACKAGES ***
%
%\usepackage{url}
% url.sty was written by Donald Arseneau. It provides better support for
% handling and breaking URLs. url.sty is already installed on most LaTeX
% systems. The latest version can be obtained at:
% http://www.ctan.org/tex-archive/macros/latex/contrib/misc/
% Read the url.sty source comments for usage information. Basically,
% \url{my_url_here}.





% *** Do not adjust lengths that control margins, column widths, etc. ***
% *** Do not use packages that alter fonts (such as pslatex).         ***
% There should be no need to do such things with IEEEtran.cls V1.6 and later.
% (Unless specifically asked to do so by the journal or conference you plan
% to submit to, of course. )


% correct bad hyphenation here
\hyphenation{op-tical net-works semi-conduc-tor}


\begin{document}
%
% paper title
% can use linebreaks \\ within to get better formatting as desired
\title{A Game-Based Approach to Monitor Parkinson's Disease Motor Symptoms}


% author names and affiliations
% use a multiple column layout for up to two different
% affiliations

% \author{\IEEEauthorblockN{Authors Name/s per 1st Affiliation (Author)}
% \IEEEauthorblockA{line 1 (of Affiliation): dept. name of organization\\
% line 2: name of organization, acronyms acceptable\\
% line 3: City, Country\\
% line 4: Email: name@xyz.com}
% \and
% \IEEEauthorblockN{Authors Name/s per 2nd Affiliation (Author)}
% \IEEEauthorblockA{line 1 (of Affiliation): dept. name of organization\\
% line 2: name of organization, acronyms acceptable\\
% line 3: City, Country\\
% line 4: Email: name@xyz.com}
% }

%\author{\IEEEauthorblockN{Leonardo Medeiros\IEEEauthorrefmark{1,2},
%Hyggo Almeida\IEEEauthorrefmark{1},
%Leandro Dias\IEEEauthorrefmark{3}, 
%Mirko Perkusich\IEEEauthorrefmark{1} and
%Robert Fischer\IEEEauthorrefmark{2}}
%\IEEEauthorblockA{\IEEEauthorrefmark{1}Federal University of Campina Grande, Campina Grande, Brazil}
%\IEEEauthorblockA{\IEEEauthorrefmark{2}Federal Institute of Alagoas, Maceio, Brazil}
%\IEEEauthorblockA{\IEEEauthorrefmark{3} Federal University of Alagoas, Maceio, Brazil}
%}

% conference papers do not typically use \thanks and this command
% is locked out in conference mode. If really needed, such as for
% the acknowledgment of grants, issue a \IEEEoverridecommandlockouts
% after \documentclass

% for over three affiliations, or if they all won't fit within the width
% of the page, use this alternative format:
% 
%\author{\IEEEauthorblockN{Michael Shell\IEEEauthorrefmark{1},
%Homer Simpson\IEEEauthorrefmark{2},
%James Kirk\IEEEauthorrefmark{3}, 
%Montgomery Scott\IEEEauthorrefmark{3} and
%Eldon Tyrell\IEEEauthorrefmark{4}}
%\IEEEauthorblockA{\IEEEauthorrefmark{1}School of Electrical and Computer Engineering\\
%Georgia Institute of Technology,
%Atlanta, Georgia 30332--0250\\ Email: see http://www.michaelshell.org/contact.html}
%\IEEEauthorblockA{\IEEEauthorrefmark{2}Twentieth Century Fox, Springfield, USA\\
%Email: homer@thesimpsons.com}
%\IEEEauthorblockA{\IEEEauthorrefmark{3}Starfleet Academy, San Francisco, California 96678-2391\\
%Telephone: (800) 555--1212, Fax: (888) 555--1212}
%\IEEEauthorblockA{\IEEEauthorrefmark{4}Tyrell Inc., 123 Replicant Street, Los Angeles, California 90210--4321}}




% use for special paper notices
%\IEEEspecialpapernotice{(Invited Paper)}




% make the title area
\maketitle


\begin{abstract}
Parkinson's disease (PD) is a degenerative neurological disorder. It causes motor symptoms such as resting tremor, bradykinesia and gait disorders. The disease's progressive nature requires continuous monitoring of the motor symptoms to assist the neurologist in managing medication. With this purpose, Health Monitoring Systems (HMS) are used as a decentralized healthcare approach. On the other hand, most patients reject the current HMS solutions because they are invasive and stigmatizing. In this work, we present a non-invasive HMS for PD motor symptoms based on games. Because of the nature of games, the approach is able to collect data from patients without reminding them that they are under a disease's treatment. We validated our approach with 30 research subjects divided between PD group and Control group. We used Support Vector Machine (SVM) to identify the occurrence of PD's bradykinesia motor symptoms and reached a classification \textit{precision} of 92.31\%. Furthermore, 90\% of the patients approved 
our HMS considering it as non-invasive and easily integrated into their routine.
\end{abstract}

\begin{IEEEkeywords}
Health Monitoring System; Parkinson's Disease; Game for Health
\end{IEEEkeywords}


% For peer review papers, you can put extra information on the cover
% page as needed:
% \ifCLASSOPTIONpeerreview
% \begin{center} \bfseries EDICS Category: 3-BBND \end{center}
% \fi
%
% For peerreview papers, this IEEEtran command inserts a page break and
% creates the second title. It will be ignored for other modes.
\IEEEpeerreviewmaketitle



\section{Introduction}

The symptoms associated with PD are caused by a degeneration of dopaminergic neurons in the substantia nigra. Common treatment focuses on drugs that activate dopamine receptors. However, the medication's effectiveness decreases over the years requiring higher dosages \cite{national2006parkinson}. Due to the complex combination of symptoms and the drugs' harmful side-effects, the patient's quality of life strongly depends on the accurate level of medication. Usually, neurologists determine this medication level based on patient's clinic visits and information from caregivers. On the other hand, continuous monitoring at the patients' home provides more data regarding the patient's symptoms and improves the medication management mainly in the disease's intermediate stages~\cite{national2006parkinson}.

Health Monitoring Systems (HMS) are designed to support continuous treatment by moving healthcare services from the hospital to the patients' home. The goal is to have healthcare services that are more cost-effective, frequent, and convenient to the patient. They enable proactive and preventive diagnosis, early detection and treatment of different diseases, and patient wellness support~\cite{cbmshms2015}. They should be seamlessly integrated into the patients' daily routine~\cite{alemdar2015}. Many HMS approaches and technologies have been developed to track the motor symptoms of PD and related disorders. Most of them use wearable sensors attached to the patients' body or clothes to quantify motor abilities \cite{cbmsparkglove2015,patel_monitoring_2009}, analyze of handwriting movements~\cite{cbmshandwriting2015} and bradykinesia (slowness of the movement) \cite{ambulatory2010}. These wearable sensors can communicate with devices in the user's home, which may forward processed data to the 
physician, creating a pervasive environment that monitors the user's health condition \cite{patel_monitoring_2009}. A major challenge for such wearable technologies is the patient's acceptance because they may consider these devices disturbing, stigmatizing, or overly interfering with their privacy \cite{alemdar2015}.

A promising approach to overcome those challenges is using games for health improvement. In the last few years, several approaches have been proposed for rehabilitation for elderly users \cite{brox11} and PD patients \cite{atkinson2010,synnott_wiipd_2012}. 

Atkinson and Narasimhan \cite{atkinson2010} developed a video game using a touch sensor to record the players' movements as they try to reach specific targets. Theoretically, this approach facilitates a better medical diagnosis of PD, but no system trial has yet been reported. Synnott \textit{et al.}~\cite{synnott_wiipd_2012} used a commercial game console to capture PD motor symptoms, more specifically, tremor. In this approach, the player has to perform a series of motor tasks presented in the form of mini-games, while holding a Nintendo Wii Remote Control. A metric analyzer provides objective metrics detailing the user performance and records them in an electronic patient diary. This diary also includes the patient's self-reported medication intake and symptom self-rating. 

%Alteração após revisão:
However, one major disadvantage of focusing on tremor is its dependence of the user's action state \cite{synnott_wiipd_2012}. PD tremor is a rest tremor \cite{national2006parkinson}, and we observed in our own research that the tremor can be suppressed~(mainly unintentionally) while the user is concentrating on a game, specially if the player's hands are involved. So, in our approach we use Ms-Kinect Version 1.0\footnote{www.microsoft.com/en-us/kinectforwindows/} motion sensor, which have accuracy and robustness for full body analysis~\cite{gabel2012}.

In this work we present a game-based approach to monitor PD motor symptoms. Our approach has been built with two major requirements: first, contactless measurement of patient motor symptoms inside the game environment; second, usage of popular consumer electronic devices as input to have a non-invasive, cost-effective solution for home use. Our main goal is to continuously provide neurologists with data regarding patient motor symptoms, while collecting the data of patients without reminding them that they are under a disease's treatment.

We validated our approach with 30 research subjects divided between PD group and Control group. We used a Support Vector Machine (SVM) to identify the occurrence of PD's bradykinesia motor symptom and reached a classification accuracy of 86.66\%. To assess our requirements, we used a survey and evaluated the collected data using GQM. Our goal was to confirm that PD patients perceive our approach as non-invasive and that it could, realistically, be integrated into their routine. The results were positive with acceptance of 90\% of the research subjects. 

This paper is organized as follows: Section~\ref{sec:proposedsystem} introduces the system requirements for our game-based HMS and present our game-based HMS; Section~\ref{sec:experimentalresults} describes the experimental validation with the research subjects. Finally, conclusions and future work are summarized in Section~\ref{conclusion} along with this work's limitations.

%Section~\ref{sec:svm} offers a background on Support Vector Machine (SVM);  
%\section{Support Vector Machine for Data Classification}
%\label{sec:svm}
%In this section we describe a brief introduction of the SVM approach. The SVM is a supervised learning algorithm that creates learning functions from a set of labeled training data. We selected this approach because it requires a relatively small number of samples for training \cite{kantardzic2011data}. This algorithm has a good generalization to discriminate between two classes represented as $n$-dimensional vectors with a discriminant function resulting in a binary output.

%The SVM’s discriminant function is based on the concept of decision planes that defines decision boundaries between classes from the training data. Geometrically, the margin corresponds to the shortest distance between the closest data points to a point on the hyperplane. The choice of the maximum margin hyperplane will lead to maximal generalization when predicting the classification of previously unseen test data \cite{kantardzic2011data}.

%To identify a hyperplane in $n-$dimensional, consider that the data set $D$ be given as pairs of $(x_1,y_1), (x_2, y_2),...,(x_{\left | D \right |} , y_{\left | D \right |})$, where: $x_{\left | D \right |}$ is the set of training data with its respective class labels, $y_{\left | D \right |}$  \cite{dataconcept2011}; and each $y_i$ can take one of two values, either $+1$ or $−1$ (i.e., $y_i \in {+1 -1}$), corresponding to the class values as follows:

%\begin{equation} \label{eq1}
%D = \left \{ ( x_1, y_1),...,(x_{\left | D \right | },y_{\left | D \right | } \right \}, x \in \Re ^ n, y \in {-1,1}
%\end{equation}

%with a hyperplane

%\begin{equation} \label{eq1}
%\langle w,x \rangle + b=0.
%\end{equation}

%The set of vectors is optimally separated by the hyperplane if it is separated without error, and the distance between the closest vectors to the hyperplane is maximal \cite{dataconcept2011}. The notation $<w,x>$ is the inner or scalar product of vectors $w$ and $x$, defined by

%\begin{equation} \label{eq1}
%\langle w,x \rangle =\sum_{1}^{n} w_ix_i
%\end{equation}

%So, we identify our hyperplane if this surface separates two classes correctly, satisfying the following conditions:
%\begin{enumerate}
% \item $\left \langle w,x^i \right \rangle + b > 0$, for all $y^i$ = 1 \\
% \item $\left \langle w,x^i \right \rangle + b < 0$, for all $y^i$ = -1
%\end{enumerate}

%The separation hyperplane is applied for linear separation surfaces. But, most classification cases require more complex models to make an optimal separation. The reason is that the given data set requires nonlinear separation of classes \cite{kantardzic2011data}, which is our case. So, we used Gaussian kernel functions which are more proper for this nonlinear separation case \cite{dataconcept2011}. The kernel function $k$ is the dot product between pairs of transformed data in $\mathbb{F}$, such that $z(x_i, x_j)=\Phi(x_i)\cdot\Phi(x_j)$, and the Gaussian kernel allows a point to be separated from the origin in $\mathbb{F}$, hence it has been chosen for our work: $k(x_i,x_j)=exp(-\lVert x_i - x_j \rVert^2/2\sigma^2$), where $\sigma$ is the width parameter associated with the Gaussian kernel \cite{preditive2014}. Because of that, the decision boundary for each class is defined by the subspaces in $\mathbb{F}$ is $z(x) = w_o \cdot\Phi(x)-\rho_0$ with parameters 

%\begin{equation} \label{eq1}
%w_o = \sum_{i=1}^{N_s}\alpha_i \Phi(s_i)
%\end{equation}

%\begin{equation} \label{eq1}
%\rho_0=\frac{1}{N_s}\sum_{j=1}^{N_s}\sum_{i=1}^{N_s}\alpha_ik(S_i,S_j)
%\end{equation}

%where $s_i$  are the support vectors, of which there are $N_s$, and where $k$ is the Gaussian kernel. Here, $w_0 \in \mathbb{F}$, $\rho_0 \in \mathbb{R}$, and that $\alpha_1$ are Lagrangian multipliers used to solve the dual formulation. So, the test data $x$ will be classified according to the sign of $z(x)$ \cite{preditive2014}.

%In this section we briefly described the application of SVM for linear and nonlinear binary classification. As we explained, the main goal of SVM is to find an optimum hyperplane that separates classes of $n$-dimensional vectors. In our case, we used the grid search technique \cite{gridsearchsvm2010} to identify the best SVM kernel function and its respective parameters, as described in Section~\ref{sec:experimentalresults}. 

\section{HMS Description}\label{sec:proposedsystem}

The game-based HMS uses a contactless motion sensor. In other words, it does not require the players to hold or wear any object. For PD patients, who have motor symptoms, this is a key requirement to enable them to easily play the games. We measure the same type of movement that is commonly used by neurologists to assess PD motor symptoms: arm abduction and adduction. The game engages and entertains the PD patients while collecting data for their monitoring.

\subsection{Requirement Analysis}

PD's motor symptoms are usually monitored through motion sensors or video cameras  \cite{cbmshandwriting2015,patel_monitoring_2009}. Those devices are difficult to be seamlessly integrated into the user routine. In this work, we use a video game with a motion sensor game controller to monitor the symptoms. The decision to use the controller was based on a requirement analysis performed with health professionals. 

To elicit the system's requirement, We applied a semi-structured interview, which combines structured and open-ended questions to guide the interview according to the interviewee answers \cite{practical_guide_re2012}. The interviews were audio-recorded and transcribed for later analysis through a qualitative research tool, QDA-Miner Lite version 1.2~\footnote{http://provalisresearch.com/products/qualitative-data-analysis-software/freeware/}.  As a result, we identified the system requirements from healthcare professionals associated to scientific references.

We interviewed two neurologists and two physiotherapists. According to them, monitoring the arms adduction and abduction of PD patients helps assessing the medication management and bradykinesia motor symptom, and could be applied for physiotherapy treatment and drug dosage medication. The respondents suggested focusing on the bradykinesia motor symptom due to its debilitating progress \cite{national2006parkinson,ambulatory2010}. Thus, treatment benefits could be correlated with the increase of amplitude and angular velocities of an arm's adduction and abduction movements (Fig.~\ref{fig:abduction}), which could be applied to quantify the bradykinesia. Arm abduction occurs when the arm is moved upward and laterally away from the body. The arm adduction occurs when the arm is returned to anatomical position. 
% * <mperkusich@gmail.com> 2015-09-01T12:26:45.636Z:
%
%  precisa mesmo dessa informacao? ta parecendo muito solta aqui...
%
% ^ <mperkusich@gmail.com> 2015-09-01T12:35:17.578Z:
%
%  ~Another PD's sign identified in the semi-structured interviews was the movement asymmetry at PD's initial stages~
%

\begin{figure}[!htb]
  \centering
  \includegraphics[width=0.4\textwidth]{./img/movaddcutctionartist.png}
  \caption{Movement of abduction and adduction.}
  \label{fig:abduction}
\end{figure}

%From the system's point-of-view, several requirements must be considered, including: the elderly user's health and acceptance; technical limitations and cost; and motivation related aspects. In other words, the solution should be safe and affordable to motivate elderly users with motor disorders to employ it at their homes \cite{sacbespoke2014}.

In summary, we identified the following requirements through literature review and confirmed by the health professionals: 

\begin{compactenum}
    \item Easy and safe to use equipment.
    \item Simple user interface with large fonts and buttons \cite{Uzor:2014:ILU:2611247.2557160}.
    \item Not prompt the players to perform movements that could cause them to suffer physical harm.
    \item Incite the player to perform specific movements that are required for the measurement (i.e., arm abduction and adduction on both sides). The data should allow the system to detect the maximal height and angle of the adduction as well as the speed of the movement.
    \item Contactless measurement.
    \item Measurement in a reasonably short time, because elderly players tire quickly and this affects their motivation and distorts the measurement.
    \item Game with clear and entertaining goal, level of difficulty adapted to the user's skills, user progress display to motivate and distract the players\cite{Uzor:2014:ILU:2611247.2557160}.
    \item Processed data should be presented in a clear and informative visual way to the physician.
    \item Use common consumer electronics such as a game console to reduce cost.
\end{compactenum}

\subsection{System Overview}

% as depicted in Fig.~\ref{img:visaogeral}:  
The system consists of four steps: 1) data acquisition through a video-based full body motion sensor; 2) health game monitor as an acquisition system; 3) signal processing, which includes biomechanical and signal processing patterns identification; 4) and data visualization for the healthcare professional. To provide a measure of the severity of the motor symptoms, the classifier in the signal processing step is trained to distinguish between the moving patterns of healthy subjects, and subjects diagnosed by neurologists as suffering from PD.

%\begin{figure}[htb!]
%     \centering
%     \includegraphics[scale=0.33]{./img/systemoverview3.png}
%     \caption{System Overview.}
%     \label{img:visaogeral}
%\end{figure}

Fig.~\ref{img:sysarch} illustrates the client-server architecture of the Health Monitoring System, composed of the Health Game Monitor (HGM) Client, responsible for collecting user data and sending it to the server; and the HGM Server, responsible for processing the data and making the results available to the health professional.

In the HGM Client, the player movements are collected via a MS-Kinect. The device projects a light grid in near infrared onto the player and records it with a video camera to identify and track the 3D coordinates of the player joints and anatomical positions \cite{mcginnis2013biomechanics}. 

The HGM game client was developed using the game engine Unity 3D\footnote{www.unity3d.com} and the Zigfu~\footnote{www.zigfu.com} game component which connects the Ms-Kinnect with the Unity 3D applications. This way, the game was used to capture the player's position and movements in 3D coordinates. The Zigfu and Unity 3D build the core of the HGM client.

By the end of each game session, the user data, including the 3D coordinates of the player's body recorded during the game, are uploaded to the server and saved in the database. From there, the data is processed by a Matlab script\footnote
{http://www.mathworks.com/products/matlab/} to transform coordinates into biomechanical signals and determine the severity or degree of motor symptoms. All technologies involved are under active development, and therefore allow the system to be adapted to (and benefit from) future changes in both hardware and third party software. 

\begin{figure}[!htb]
	\centering
	\includegraphics[width=0.475\textwidth]{img/systemarchitecture3.png}
	\caption{System Architecture.}
	\label{img:sysarch}
\end{figure}

Finally, the results are reported to the healthcare professional in a visual and tabular form, featuring direct biomechanical measures that specialists are familiar with from PD diagnosis handbooks \cite{national2006parkinson}, as well as the mechanisms of estimation of the patient's level of motor deficiency \cite{national2006parkinson}.

In the following subsections we detail the two main components of the architecture: the Health Game Monitor client, using the game \textit{Catch the Spheres} as case study; and the HGM server.

\subsection{Health Game Monitor Client: Catch the Spheres}

The Health Monitor Client is the game itself. We have developed a game prototype named \emph{Catch the Spheres} (Fig.~\ref{img:catch}). It is a game in third person in which the players must catch or escape the balls that comes in their direction. There are two types of balls: blue and red. Initially, all the balls are red and some of these suddenly change to blue when approaching the player. The time interval until the ball changes its color may be smaller or larger, depending on the game level selected. The Kinect sensor captures the player movements and replicates them on the character in the center of screen. The player must touch the blue balls with hands and deflect the red balls. The main purpose of the game is to capture the player's movements, when he executes specific actions. The time interval between the moment when the ball changes color and when the ball is captured by the player measures the player's reflexes, while the speed of the movement is calculated from the distance traveled by the 
hands to capture balls.

From the 3D coordinates of the player's body parts acquired by the MS-Kinect motion sensor with position, orientation, velocity and acceleration of the arms and hands, we calculate the angular displacement of adduction and abduction of arm movements \cite{mcginnis2013biomechanics}. As the 3D measurement is noisy, several filtering steps are applied, some in the Kinect system itself. 

During a short game of typically 5 minutes, the player is prompted roughly 10 times to lift their right or left arm to catch a virtual ball approaching from the horizon. The maximal height or angle, the movement's velocity, and the arm to be lifted can thus be indirectly controlled by the trajectory and speed of the ball. A random selection of the trajectories and speeds prevents the player from preparing a movement in advance.

\begin{figure}[!htb]
	\centering
	\includegraphics[width=0.475\textwidth]{img/catch_colour.png}
	\caption{Screenshot of the game \emph{Catch the Spheres}. The
		player has just successfully caught a sphere that flew at him from the
		horizon.}
	\label{img:catch}
\end{figure}

\subsection{Health Game Monitor Server: signal processing and statistical analysis}

%The data collected through the game is received by the HGM Server at the end of each game session. The complete process for symptom identification is illustrated in Fig.~\ref{fig:biomecproc} and can be divided into two phases: biomechanical signal processing and statistical analysis.
The data collected through the game is received by the HGM Server at the end of each game session. The complete process for symptom identification can be divided into two phases: biomechanical signal processing and statistical analysis. 

The biomechanical analysis of human movement is part of the diagnosis and treatment process for PD, where the patients are asked to lift their arms, one after the other, at the highest amplitude and velocity they are able to, in order to check the bradykinesia progress. The movement of abduction and adduction are joint actions which involve wrist, shoulder, and hip joints. To identify the movement of adduction and abduction of the arms, it is necessary to use a reference joint. Here, we focus on the wrist joint because its signal has higher amplitude when compared to the other joints. We used the \textit{peaks and valleys} technique to identify the beginning and the end of the movement cycle (Fig.~\ref{fig:signalamplitudepeakvaley}). After the cycle identification, we extract a window length with the cycle movement and transform the MS-Kinect data into angles. Thus, we calculate the angular motion of the movement and consequently the angle displacement of the adduction and abduction movements.

% \begin{figure}[!htb]
% 	\centering
% 	\includegraphics[width=0.475\textwidth]{img/biomecprocessor2.png}
% 	\caption{Biomechanical signal processing.}
% 	\label{fig:biomecproc}
% \end{figure}

The peak of the amplitude contains the maximum amplitude of each movement. The time spent between the first valley to the peak in each movement cycle is the time taken for the abduction of the arm. The time spent at the peak to the second valley is the time spent for the adduction of the arm. Then, with the maximum amplitude and the time spent in these movements we calculate the angular velocities of abduction and adduction of the arms. However, the acquired signals of the MS-Kinect motion sensors have a lot of noise. In order to remove incomplete movements we designed a data filter which extracts the mean vector and removes outliers. The output of this phase is a set of feature vectors with user motion information, including right and left arm amplitude for abduction and adduction.
%The extracted feature vectors are described in Table~\ref{table:features}.


%\begin{table}[h]
%\centering
%\caption{The detail of the six feature vectors extracted from the data %collection.}
%\label{table:features}
%\begin{tabular}{|l|l|}
%\hline
%{\bf Feature}  & {\bf %Description}                                       \\ \hline
%MaxAmpLeft     & Max. Amplitude of the Left Arm                       %\\ \hline
%MaxAmpRight    & Max. Amplitude of the Right Arm                      %\\ \hline
%AngVelAbdLeft  & Ang. Vel. of Abduction on the Left \\ \hline
%AngVelAbdRight & Ang. Vel. of Abduction on the Right \\ \hline
%AngVelAddLeft  & Ang. Vel. of Adduction on the Left \\ \hline
%AngVelAddRight & Ang. Vel. of Adduction on the Right \\ \hline
%\end{tabular}
%\end{table}


\begin{figure}[!htb]
	\centering
	\includegraphics[width=0.5\textwidth]{img/signalamplitudepeakvaley-2.png}
	\caption{Example of the angle over time with the peak and valley detection technique.}
	\label{fig:signalamplitudepeakvaley}
\end{figure}

The next phase of this research is to classify the motion data to identify the occurrence of Bradykinesia symptom. Using the theory of statistical learning, we have performed a data analysis to acquire knowledge through supervised learning methods, as explained in Section~\ref{classifier_performance}.

\section{Experiments and discussion}
\label{sec:experimentalresults}

%This research project was approved by the Brazilian Ethic Committee of Federal University of Campina Grande with CAAE: 14408213.9.1001.5182. The research subjects were instructed about the procedures and signed a written informed consent.
This research project was approved by the XXX Ethic Committee of XXXXX University of XXXX with CAAE: XXXXXX. The research subjects were instructed about the procedures and signed a written informed consent. %The research subjects played the game (\emph{Catch the Spheres}) with a MS-Kinect (Version 1.0) motion sensor and the HMS acquired and classified the user's motor data.

\subsection{Research Subjects}

A total number of 30 subjects participated in the study. The group previously diagnosed by neurologists with PD consisted of 15 subjects, 10 men and 5 women, between 51 and 65 years (mean: 58). The control group was composed of 15 subjects without a PD diagnosis, 11 men and 4 women, between 50 and 65 years (mean: 57). All subjects were interacting with exactly the same system in both hardware and software, and were prompted to seamlessly perform abduction and adduction of the arms according to the game's context. All sessions were made in a medical institution under the supervision of a neurologist or physiotherapist to continuously check the subject's health condition. No subject needed to interrupt the experiment to ensure safety.

%At times, players may not complete a cycle, such as if they give up on catching a ball deemed too fast or too high or simply because they get distracted. Such incomplete cycles are detected by comparing the cycle profile to the player's median cycle profile, and outliers are discarded.

\subsection{Data Collection}

The data collection was performed over a period of 2 months, on different days, at the same location, with each subject individually. Due to time constraints and technical requirements, such as to capture with the motion sensor the entire length of the upper arm during abduction and adduction movements, each subject was instructed to perform the following steps:

\begin{compactenum}
	\item The subject stands at distance of 2 meters from the motion sensor at a place marked for that purpose on the ground.
	\item The subject faces a projection of the game on a wall, centered over the motion sensor.
	\item The subject plays the game \textit{Catch the Spheres} for 5 minutes.
	\item The subjects end the game by reaching the virtual exit button (seen as a button with an 'X' in Fig.~\ref{img:catch}).
\end{compactenum} 

\subsection{SVM Classifier Performance}
\label{classifier_performance}

The SVM is a supervised learning algorithm that creates learning functions from a set of labeled training data. We selected this approach because it requires a relatively small number of samples for training \cite{kantardzic2011data}. This algorithm has a good generalization to discriminate between two classes represented as $n$-dimensional vectors with a discriminant function resulting in a binary output.

%The SVM's discriminant function is based on the concept of decision planes that defines decision boundaries between classes from the training data. Geometrically, the margin corresponds to the shortest distance between the closest data points to a point on the hyperplane. The choice of the maximum margin hyperplane will lead to maximal generalization when predicting the classification of previously unseen test data \cite{kantardzic2011data}.

%\begin{figure*}[htbp!]
%\subfigure[Grid Search - Accuracy Classification.]{
%\includegraphics[scale=0.4]{img/gridsearch.png}
%\label{fig:gridaccuracy}
%}
%\subfigure[Grid Search - FPRate.]{
%\includegraphics[scale=0.4]{img/gridsearchfprate.png}
%\label{fig:gridfprate}
%}
%\label{fig:gridsearch}
%\caption{GridSearch for SVM Parameter Optimization.}
%\end{figure*}

In this work, we applied the grid search technique \cite{gridsearchsvm2010} to identify the best SVM parameters, using Leave-One-Out Cross-Validation (LOOCV) \cite{kantardzic2011data}. This technique assesses the accuracy of the predicted model, prevents the over fitting problem in the binary classification and is a practical method to identify the SVM parameters. In this study, to reduce the error rate, we applied a mini max approach to maximize the margin over the hyper plane coefficients and the correct classification.

%The parameters values of grid search for $C$ = [$2^5$, ... ,$2^2$] and $\gamma$ = [$2^{15}$, ... ,$2^3$ ], the step length was $2^2$, and with the identification and selection of a “better” region on the grid. So, we did a finer grid search $C$ = [0.25, 0.5, ... ,2.5]; and $\gamma$ = [1, 2, ...,10] as it can be seen in the Fig.~\ref{fig:gridsearch}.


%As illustrated in Fig.~\ref{fig:gridaccuracy}, we achieved a good classification performance where the worst accuracy was 70.00$\%$ and the best was 86.67$\%$. We obtained very low values for False Positives (FP), with 6.67$\%$ of \textit{FPRate} on the best fit (Fig.~\ref{fig:gridfprate}). This way, $C = 2$ and $\gamma = 3$ were the best kernel parameters, achieving accurate classification and minimizing the \textit{FPRate}. 

The best classification performance of this study is presented in the confusion matrix \cite{kantardzic2011data} for two classes that consist of a matrix $2$\ x $2$\, with (TP, FP, TN and FN) described in Table~\ref{table:resultadomatrizconfusaosvm} and his metrics is presented in the metrics results Table~\ref{table:metricas}.



% shows the classification performance, where the \textit{TpRate} is TP divided by the total number of positives, which is TP + FN; the \textit{FpRate} is FP divided by the total number of negatives, which is FP + TN. The \textit{Accuracy} is the number of correct classifications divided by the total number of classifications \cite{kantardzic2011data}.


%True Positive (TP) indicates correctly classified with abnormal movement.  True Negative (TN) indicates correctly classified with normal movement. False Positives (FP) indicate the normal movement classified as abnormal ones and the False Negatives (FN) indicate the real abnormal movement not correctly detected. 

%The best classification performance of this study is presented in the confusion matrix for two classes that consist of a matrix $2$\ x $2$\, with (TP, FP, TN and FN) described in Table~\ref{table:resultadomatrizconfusaosvm}. True Positive (TP) indicates correctly classified with abnormal movement.  True Negative (TN) indicates correctly classified with normal movement. False Positives (FP) indicate the normal movement classified as abnormal ones and the False Negatives (FN) indicate the real abnormal movement not correctly detected. Table~\ref{table:metricas} shows the classification performance, where the \textit{TpRate} is TP divided by the total number of positives, which is TP + FN; the \textit{FpRate} is FP divided by the total number of negatives, which is FP + TN. The \textit{Accuracy} is the number of correct classifications divided by the total number of classifications \cite{kantardzic2011data}.

\begin{table}[!htbp]
\caption{Confusion Matrix Of SVM Classification With One Leave Out Cross Validation.}
\label{table:resultadomatrizconfusaosvm}
\centering
\begin{tabular}{l|c|c|}
\cline{2-3}
\multicolumn{1}{c}{}                         & \multicolumn{2}{|c|}{\textit{\textbf{Predicted Class}}} \\ \cline{2-3} 
                                             & \textbf{Parkinson}      & \textbf{Control Group}         \\ \hline
\multicolumn{1}{|l|}{\textbf{Parkinson}} & 12       & 3          \\ \hline
\multicolumn{1}{|l|}{\textbf{Control Group}}     & 1           & 14     \\ \hline
\end{tabular}
\end{table}

%The confusion matrix indicates the occurrence of 3 incorrectly classified as PD subjects. But, in the analysis of user's data, we checked they have quite similar amplitude of the Control Group subjects. These subjects may be in the initial stages of PD, where motor symptoms may not yet be developed or successfully suppressed by medication. However, an unexpected result was the occurrence of one incorrectly classified as Control Group subject. This will require new future studies to identify what happened in this classification.

%In the analysis of Parkinson's subject's classification we had 3 incorrectly classified as a Control Group's subject, and in the analyze of user's data we checked they have quite similar amplitude and angular velocities of the Control Group subjects, probably these subjects may be PD's initial stages where the motor symptoms may not yet be developed or successfully suppressed by medication. 

We assessed bradykinesia by measuring the amplitude and angular velocities in the adduction and abduction movements of the subjects. Therefore, the amplitude and angular velocities were the feature vectors for data classification~\cite{kantardzic2011data}. In Table~\ref{table:amplitude}, we show the severity of motor disorder caused by bradykinesia, in which PD's patients present lower amplitudes. Notice that subject Control 10 presents amplitude values similar to the PD's patients. In this case, we checked that this subject has an uninformed motor disorder that caused the incorrect classification. Furthermore, some PD's patients, namely Parkinson 3,8 and 12, did not present the bradykinesia symptom. In this case, we assumed that these subjects did not have the symptom or it was successfully suppressed by medication. 

%In Table~\ref{table:amplitude} we show the amplitude values, subject group, id, and classification result of the research subjects. 

%Table 2 shows, the higher the severity of bradykinesia, the higher are the average and standard deviation of movement time. Patients with bradykinesia have a higher standard deviation and/or average of movement time than healthy subjects. There are also patients (for example patients 6&7), whose average of movement time is lower compared to healthy subjects. The reason is that these patients can not reach great amplitudes during finger tapping. Therefore they are able to tap their finger faster than healthy subjects.

%Bradykinesia was assessed by measuring the movement time during finger tapping. Rigidity could not be detected by means of just one thin film force sensor. To improve the evaluation of rigidity the number and placement of force sensors should be adapted. This will be tested further on in clinical setting, and later on e. g. intraoperatively in brain stimulation surgery to analyze qualitatively movement disorders.

\begin{table}[!h]
\centering
\caption{Results of Arm Abduction Amplitude}
\label{table:amplitude}
\begin{tabular}{|l|l|l|l|}
\hline
\rowcolor[HTML]{EFEFEF} 
\textbf{Subjects} & \textbf{\begin{tabular}[c]{@{}l@{}}Mean of\\ Amplitude \\ Left {[\degree]}\end{tabular}} & \textbf{\begin{tabular}[c]{@{}l@{}}Mean of\\ Amplitude\\ Right {[\degree]}\end{tabular}} & \textbf{\begin{tabular}[c]{@{}l@{}}Predicted\\ Class\end{tabular}} \\ \hline
Control 1         & \multicolumn{1}{r|}{153.62}                                                          & \multicolumn{1}{r|}{151.14}                                                          & Control                                                            \\ \hline
Control 2         & \multicolumn{1}{r|}{165.31}                                                          & \multicolumn{1}{r|}{151.84}                                                          & Control                                                            \\ \hline
Control 3         & \multicolumn{1}{r|}{155.44}                                                          & \multicolumn{1}{r|}{163.31}                                                          & Control                                                            \\ \hline
Control 4         & \multicolumn{1}{r|}{169.12}                                                          & \multicolumn{1}{r|}{169.39}                                                          & Control                                                            \\ \hline
Control 5         & \multicolumn{1}{r|}{157.20}                                                          & \multicolumn{1}{r|}{162.72}                                                          & Control                                                            \\ \hline
Control 6         & \multicolumn{1}{r|}{162.99}                                                          & \multicolumn{1}{r|}{167.25}                                                          & Control                                                            \\ \hline
Control 7         & \multicolumn{1}{r|}{166.90}                                                          & \multicolumn{1}{r|}{166.93}                                                          & Control                                                            \\ \hline
Control 8         & \multicolumn{1}{r|}{154.68}                                                          & \multicolumn{1}{r|}{159.13}                                                          & Control                                                            \\ \hline
Control 9         & \multicolumn{1}{r|}{162.31}                                                          & \multicolumn{1}{r|}{158.17}                                                          & Control                                                            \\ \hline
Control 10         & \multicolumn{1}{r|}{135.22}                                                          & \multicolumn{1}{r|}{131.85}                                                          & \textbf{Parkinson}                                                            \\ \hline
Control 11         & \multicolumn{1}{r|}{162.13}                                                          & \multicolumn{1}{r|}{167.61}                                                          & Control                                                            \\ \hline
Control 12         & \multicolumn{1}{r|}{161.69}                                                          & \multicolumn{1}{r|}{166.78}                                                          & Control                                                            \\ \hline
Control 13         & \multicolumn{1}{r|}{160.47}                                                          & \multicolumn{1}{r|}{155.05}                                                          & Control                                                            \\ \hline
Control 14         & \multicolumn{1}{r|}{174.37}                                                          & \multicolumn{1}{r|}{167.66}                                                          & Control                                                            \\ \hline
Control 15         & \multicolumn{1}{r|}{155.08}                                                          & \multicolumn{1}{r|}{167.83}                                                          & Control                                                            \\ \hline
Parkinson 1         & \multicolumn{1}{r|}{125.80}                                                          & \multicolumn{1}{r|}{119.73}                                                          & Parkinson                                                            \\ \hline
Parkinson 2         & \multicolumn{1}{r|}{131.28}                                                          & \multicolumn{1}{r|}{123.49}                                                          & Parkinson                                                            \\ \hline
Parkinson 3         & \multicolumn{1}{r|}{156.66}                                                          & \multicolumn{1}{r|}{149.46}                                                          & \textbf{Control}                                                            \\ \hline
Parkinson 4         & \multicolumn{1}{r|}{139.90}                                                          & \multicolumn{1}{r|}{142.83}                                                          & Parkinson                                                            \\ \hline
Parkinson 5         & \multicolumn{1}{r|}{147.37}                                                          & \multicolumn{1}{r|}{153.13}                                                          & Parkinson                                                            \\ \hline
Parkinson 6         & \multicolumn{1}{r|}{115.32}                                                          & \multicolumn{1}{r|}{123.56}                                                          & Parkinson                                                            \\ \hline
Parkinson 7         & \multicolumn{1}{r|}{129.75}                                                          & \multicolumn{1}{r|}{133.04}                                                          & Parkinson                                                            \\ \hline
Parkinson 8         & \multicolumn{1}{r|}{166.62}                                                          & \multicolumn{1}{r|}{165.63}                                                          & \textbf{Control}                                                            \\ \hline
Parkinson 9         & \multicolumn{1}{r|}{143.95}                                                          & \multicolumn{1}{r|}{140.45}                                                          & Parkinson                                                            \\ \hline
Parkinson 10         & \multicolumn{1}{r|}{136.86}                                                          & \multicolumn{1}{r|}{151.03}                                                          & Parkinson                                                            \\ \hline
Parkinson 11         & \multicolumn{1}{r|}{156.87}                                                          & \multicolumn{1}{r|}{142.93}                                                          & Parkinson                                                            \\ \hline
Parkinson 12         & \multicolumn{1}{r|}{166.59}                                                          & \multicolumn{1}{r|}{157.81}                                                          & \textbf{Control}                                                            \\ \hline
Parkinson 13         & \multicolumn{1}{r|}{147.99}                                                          & \multicolumn{1}{r|}{142.02}                                                          & Parkinson                                                            \\ \hline
Parkinson 14         & \multicolumn{1}{r|}{141.95}                                                          & \multicolumn{1}{r|}{150.60}                                                          & Parkinson                                                            \\ \hline
Parkinson 15         & \multicolumn{1}{r|}{125.69}                                                          & \multicolumn{1}{r|}{140.62}                                                          & Parkinson                                                            
\\ \hline
\end{tabular}
\end{table}

\begin{table}[htbp!]
\caption{Performance of PD and Control Group classification.}
\label{table:metricas}
\centering
\begin{tabular}{|l|r|}
\hline
\multicolumn{2}{|l|}{\textbf{Classifier Metrics}} \\ \hline
\textbf{TpRate}                    & 80.00$\%$\                 \\ \hline
\textbf{FpRate}                    & 6.67$\%$\                \\ \hline
\textbf{Accuracy}                  & 86.67$\%$\                \\ \hline
\textbf{F-score}                   & 85.71$\%$\                \\ \hline
\textbf{Precision}                  & 92.31$\%$\                \\ \hline
\end{tabular}
\end{table}

\subsection{User Acceptance with the GQM Results}

Goal Question Metric (GQM) \cite{gqmhci2009} is a goal-oriented paradigm to define measurements based on explicit and precisely defined goals. GQM has a hierarchical structure divided into three levels: \texttt{Goal}, the conceptual level defined for an object of measurement, such as products, processes and resources; \texttt{Question}, the operational level in which a set of questions is used to decide if a specific goal was achieved; and \texttt{Metric}, the quantitative level in which a set of data is collected to answer the questions from the \texttt{Question} level in a quantitative way. Based on the GQM paradigm, we defined two goals following the GQM template presented in \cite{gqmhci2009}: 

\begin{compactitem}
	\item G1: Analyze our HMS PD approach for the purpose of evaluating with respect to usability from the view point of the patients in the context of the game \emph{Catch the Spheres}
	\item G2: Analyze our HMS PD approach for the purpose of evaluating with respect to fit to daily routine from the view point of the patients in the context of the game \emph{Catch the Spheres}
\end{compactitem}

For G1 we defined two questions: 

\begin{compactitem}
	\item Q1: Is the game a good entertainment and do PD patients feel motivated to play it again?
	\item Q2: Is the game simple, without many rules, and understandable? Could it be played by PD patients of different ages? 
\end{compactitem}

For G2 we defined three questions:

\begin{compactitem}
	\item Q3: Are PD patients used to playing casual games at home?
	\item Q4: Could PD patients integrate this game into their daily routine?
	\item Q5: Would an elderly user be safe in playing this game using the arms movements?
\end{compactitem}

For each question, we defined a measurement to be collected using a Boolean scale (i.e., \emph{yes} and \emph{no}). The measurements were collected through a questionnaire and we obtained the following result indicating that we reached our goals: 90\% of the users felt motivated with the game (Q1); 93\% found the game simple and could be played by people of different ages (Q2); only 53\% of the respondents are used to play casual games at home (Q3); 80\% would add this game-based monitoring approach into their daily routine (Q4); 75\% considered it safe for elderly users (Q5).

%We must consider that the measurements collected in this study were provided by a game with a research purpose. If the game had a richer graphical interface, its acceptance would probably be greater. Therefore, we conclude that, given the GQM results, the approach was accepted by PD patients and could be applied into their daily routine.

\section{Conclusion and future works}
\label{conclusion}

In this work we presented a game-based approach to monitor the motor symptoms of Parkinson's Disease. The game was applied to motivate the user to be monitored, allowing the collection of biomechanical data measurements. We monitored PD symptoms through the abduction and adduction arm movements, calculating the amplitude and its respective angular velocity to assess bradykinesia motor symptom. 

%This method is commonly applied by neurologists at consultation to distinguish normal and abnormal movements. In this way, we enable the monitoring of PD patients at home, increasing the frequency of symptoms measurement.

To evaluate our approach, we performed an experimental study with 30 research subjects divided in PD and Control group. We used SVM to identify the occurrence of PD's bradykinesia motor symptom and had a classification \textit{Precision} of 92.31\%. Moreover, 90\% of the patients considered our approach non-invasive and easy to integrate into their routine. For future works, we plan to further identify PD symptoms such as dyskinesia and tremor and evaluate our approach with PD patients in different scenarios.

%For future works, we plan to further identify PD symptoms such as dyskinesia and tremor and evaluate our approach with PD patients in different scenarios. Thus, we intend to identify the variance of the PD's motor symptoms during a medication cycle to improve the quality of the information available to the health professional.


\bibliographystyle{IEEEtran}
\bibliography{sigproc2,IEEEFormat}

% \bibliographystyle{abbrv}
 %\bibliography{sigproc2}
% \begin{thebibliography}{10}

% \bibitem{aarhus_negotiating_2010}
% R.~Aarhus and S.~A. Ballegaard.
% \newblock Negotiating boundaries: managing disease at home.
% \newblock In {\em Proceedings of the 28th International Conference on Human
%   Factors in Computing Systems}, 2010.

% \bibitem{gqmhci2009}
% R.~Al-Nanih, H.~Al-Nuaim, and O.~Ormandjieva.
% \newblock New health information systems (his) - quality-in-use model based on
%   the gqm approach and hci principles.
% \newblock In {\em Human-Computer Interaction. Interacting in Various
%   Application Domains}, volume 5613. 2009.

% \bibitem{alemdar2015}
% H.~Alemdar, C.~Tunca, and C.~Ersoy.
% \newblock Daily life behaviour monitoring for health assessment using machine
%   learning: Bridging the gap between domains.
% \newblock {\em Personal Ubiquitous Computing}, 19, 2015.

% \bibitem{practical_guide_re2012}
% F.~Anwar and R.~Razali.
% \newblock A practical guide to requirements elicitation techniques selection -
%   an empirical study.
% \newblock {\em Middle-East Journal of Scientific Research}, 11(8), 2012.

% \bibitem{atkinson2010}
% S.~Atkinson and V.~Narasimhan.
% \newblock Design of an introductory medical gaming environment for diagnosis
%   and management of parkinson's disease.
% \newblock In {\em Trendz in Information Sciences Computing}, 2010.

% \bibitem{brox11}
% E.~Brox, L.~Luque, G.~Evertsen, and J.~Hernandez.
% \newblock Exergames for elderly: Social exergames to persuade seniors to
%   increase physical activity.
% \newblock In {\em Pervasive Computing Technologies for Healthcare
%   (PervasiveHealth)}, 2011.

% \bibitem{sacsvmhms2014}
% H.~Chen, G.-T. Liao, Y.-C. Fan, B.-C. Cheng, C.-M. Chen, and T.-C. Kuo.
% \newblock Design and implementation of a personal health monitoring system with
%   an effective svm-based pvc detection algorithm in cardiology.
% \newblock SAC '14, 2014.

% \bibitem{national2006parkinson}
% N.~I. for Health and C.~E.~G. Britain.
% \newblock {\em Parkinson's Disease: Diagnosis and Management in Primary and
%   Secondary Care}.
% \newblock NICE clinical guideline. National Institute for Health and Clinical
%   Excellence, 2006.

% \bibitem{manumeterjbhi2014}
% N.~Friedman, J.~Rowe, D.~Reinkensmeyer, and M.~Bachman.
% \newblock The manumeter: A wearable device for monitoring daily use of the
%   wrist and fingers.
% \newblock {\em IEEE Journal of Biomedical and Health Informatics}, 18(6), 2014.

% \bibitem{gabel2012}
% M.~Gabel, R.~Gilad-Bachrach, E.~Renshaw, and A.~Schuster.
% \newblock Full body gait analysis with kinect.
% \newblock {\em International Conference of the IEEE Engineering in Medicine and
%   Biology Society (EMBC)}, 2012.

% \bibitem{sacbespoke2014}
% S.~Graziadio, R.~Davison, K.~Shalabi, K.~M.~A. Sahota, G.~Ushaw, G.~Morgan, and
%   J.~A. Eyre.
% \newblock Bespoke video games to provide early response markers to identify the
%   optimal strategies for maximizing rehabilitation.
% \newblock In {\em Proceedings of the 29th Annual ACM Symposium on Applied
%   Computing}, SAC '14, 2014.

% \bibitem{kantardzic2011data}
% M.~Kantardzic.
% \newblock {\em Data Mining: Concepts, Models, Methods, and Algorithms}.
% \newblock John Wiley \& Sons, Piscataway, NJ, USA, 2nd edition, 2011.

% \bibitem{gridsearchsvm2010}
% C.-H. Li, C.-T. Lin, B.-C. Kuo, and H.-H. Ho.
% \newblock An automatic method for selecting the parameter of the normalized
%   kernel function to support vector machines.
% \newblock In {\em International Conference on Technologies and Applications of
%   Artificial Intelligence (TAAI)}, 2010.

% \bibitem{robotgait2014}
% C.-K. Liao, C.~D. Lim, C.-Y. Cheng, C.-M. Huang, and L.-C. Fu.
% \newblock Vision based gait analysis on robotic walking stabilization system
%   for patients with parkinson's disease.
% \newblock In {\em IEEE International Conference on Automation Science and
%   Engineering (CASE)}, 2014.

% \bibitem{mazilu2015}
% S.~Mazilu, U.~Blanke, M.~Dorfman, E.~Gazit, A.~Mirelman, J.~M.~Hausdorff, and
%   G.~Tr\"{o}ster.
% \newblock A wearable assistant for gait training for parkinson's disease with
%   freezing of gait in out-of-the-lab environments.
% \newblock {\em ACM Trans. Interact. Intell. Syst.}, 5, 2015.

% \bibitem{mcginnis2013biomechanics}
% P.~McGinnis.
% \newblock {\em Biomechanics of Sport and Exercise}.
% \newblock Human Kinetics, New York, NY, USA, 3rd edition, 2013.

% \bibitem{patel_monitoring_2009}
% S.~Patel, K.~Lorincz, R.~Hughes, N.~Huggins, J.~Growdon, D.~Standaert, M.~Akay,
%   J.~Dy, M.~Welsh, and P.~Bonato.
% \newblock Monitoring motor fluctuations in patients with parkinson's disease
%   using wearable sensors.
% \newblock {\em IEEE Transactions on Information Technology in Biomedicine},
%   13(6), 2009.

% \bibitem{synnott_wiipd_2012}
% J.~Synnott, L.~Chen, C.~Nugent, and G.~Moore.
% \newblock Wiipd objective home assessment of parkinson's disease using the
%   nintendo wii remote.
% \newblock {\em IEEE Transactions on Information Technology in Biomedicine},
%   16(6), 2012.

% \bibitem{Uzor:2014:ILU:2611247.2557160}
% S.~Uzor and L.~Baillie.
% \newblock Investigating the long-term use of exergames in the home with elderly
%   fallers.
% \newblock In {\em Proceedings of the 32Nd Annual ACM Conference on Human
%   Factors in Computing Systems}, 2014.

% \bibitem{ambulatory2010}
% D.~Zwartjes, T.~Heida, J.~van Vugt, J.~Geelen, and P.~Veltink.
% \newblock Ambulatory monitoring of activities and motor symptoms in parkinson's
%   disease.
% \newblock {\em IEEE Transactions on Biomedical Engineering}, 57(11), 2010.

%\end{thebibliography}
\end{document}