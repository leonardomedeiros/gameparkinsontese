Primeiramente, gostaria de agradecer a meus pais Ricardo Ubirajara de Medeiros e Maria das Graças Melo de Medeiros por me conceberem e darem toda a motivação que me permitiram chegar aqui.

Agradeço aos meus orientadores Leandro Dias da Silva e Hyggo Almeida de Oliveira pela confiança, suporte, compreensão e paciência que tiveram comigo nesses longos anos de pesquisa. O apoio dos senhores foi primordial para a conclusão desta etapa da minha vida.

Ao IFAL que me deu o apoio institucional, financeiro desde o meu ingresso na instituição. Em especial, aos Coordenadores de Curso (Elvys Soares, Jailton Cruz, Eunice Palmeira, Ricardo Rubens, Anderson Barbosa e Marcílio Ferreira) que passaram por mim e tiveram todo o cuidado em ajustar meus horários e me deram o suporte para concluir esta etapa da carreira. Ao Reitor, Sérgio Teixeira e Pró-Reitores Carlos Henrique e Luiz Henrique Gouvêa por terem me concedido o afastamento de 2 anos para a conclusão do curso. Gostaria de agradecer também à Patrícia Galvão por seu empenho na Pró-Reitoria de Pesquisa e Inovação para a concessão da Bolsa Prodoutoral junto à Capes que me forneceu suporte financeiro por determinado período do curso. E, aos professores das Coordenações de Informática do IFAL Maceió por terem assumido minha carga-horária no período de dois anos. Serei eternamente grato e contribuirei da mesma forma para que todos consigam suas qualificações.

Gostaria de agradecer o suporte científico na Área Médica e Fisioterápica à: Cícera Pontes, Flávio Rezende, Ana Cláudia Câmara, Pedro Jatobá, Dayane Vasconcelos, Felipe Rebêlo e Jean Charles Santos.

Gostaria de lembrar a todos também que durante o período de doutoramento várias pessoas passam por nossas vidas e que todo doutorando é eternamente grato por pequenas coisas. Dentre estas, principalmente às pessoas que dividem moradias e compartilham angústias e conquistas neste processo árduo e longo de um doutorado. Dentre essas pessoas, gostaria de destacar meus amigos do Caio Lela: Elthon Alex, Gregory Almeida e Edgard Luiz. "\textit{All izz Well}".

Sou eternamente grato a duas pessoas que tiveram a disponibilidade de me ajudar diretamente para a conclusão desta tese. Meu amigo, Robert Fishcer pelo suporte técnico e científico que muito me auxiliou na conclusão da pesquisa. Meu amigo, Mirko Perkusich por suas contribuições na escrita e na "venda do peixe". E também a João Henrique pelas sugestões na escrita do documento final.

Aos Professores Ângelo Perkusich e Kyller Gorgônio que desde o início do trabalho me deram sugestões para prosseguir, alinhar e melhorar a pesquisa.

Aos colega Antônio Santos Jr. e ao aluno do IFAL Rafael Soares por terem me auxiliado na implementação da arquitetura proposta nesta tese.

Gostaria de agradecer ao Álvaro Alvares, que é um amigo que obtive no CESMAC e que hoje é um grande pesquisador e parceiro.

Não poderia deixar de esquecer dos meus amigos do Tênis, tanto da Associação de Tenistas da Orla (ATO) quanto os amigos do Jarágua Tênis Clube (JTC). Em todos estes anos, o tênis foi fundamental para me dar o equilíbrio necessário para enfrentar o próximo dia. 

A todas as pessoas que passaram por minha vida e que estão em minha memória, são tantos e não caberiam aqui :).

Obrigado, \linebreak
Leonardo Medeiros



