\chapter{Introdu\c{c}\~{a}o} \label{chapter:intro}

A idade média da população mundial está aumentando progressivamente em decorrência da redução da taxa de natalidade e melhora na expectativa de vida da população. Estudos da~\ac{oms}~\cite{ageing2011} indicam que muito em breve teremos mais idosos do que crianças, e, ao considerar que a população idosa possui maior prevalência de doenças crônicas~\cite{prevcronica2009}, surge então a necessidade de monitorar o estado da saúde dessa população. Portanto, diante do crescimento da quantidade de pacientes crônicos, da iminente redução do número de leitos hospitalares disponíveis e da insuficiência de profissionais especializados para atender esta demanda~\cite{healthmonitoring2013}, faz-se necessário transpor serviços de monitoramento dos pacientes crônicos dos leitos hospitalares para o acompanhamento domiciliar~\cite{homecarebrazil2011}. 

Na investigação desta demanda, pesquisadores da computação aplicada à saúde buscam prover mecanismos de monitoramento da saúde~\cite{healthmonitoring2013,bardram2010,aarhus_negotiating_2010} como os ~\ac{sms}. Os ~\ac{sms} permitem ao médico acompanhar à distância o estado de saúde de seus pacientes colaborativamente~\cite{healthmonitoring2013}. Atualmente, os~\ac{sms} realizam tratamento preventivo e pró-ativo do estado de saúde~\cite{bardram2010}; suporte à reabilitação do paciente~\cite{sacbespoke2014}; e auxílio para o paciente atingir uma melhor qualidade de vida~\cite{sacsvmhms2014}. Referente ao monitoramento dos sinais motores, os~\ac{sms} quantificam estes sinais e conseguem quantificar as habilidades motoras~\cite{manumeterjbhi2014,patel_monitoring_2009}, efetuar análise da marcha \cite{robotgait2014} e identificar sinais de bradicinesia~\footnote{Sintoma do Parkinson que consiste na lentidão da execução dos movimentos.}~\cite{ambulatoryparkinson2010}. Contudo, o maior desafio dessas abordagens é motivar e induzir o usuário a executar movimentos específicos para o monitoramento da saúde motora.

Na busca por motivar os usuários a fornecer seus dados motores, foi identificado que os jogos eletrônicos encontram-se presentes na rotina diária de 26\% da população americana acima dos 50 anos~\cite{esa2016}. Com base nesse número, têm-se um público de jogadores idosos beneficiáveis por uma plataforma de monitoramento de dados de saúde embutida num jogo eletrônico. Aliado a esse estudo, foi encontrado, jogos voltados para o público idoso aplicados à melhoria do estado de saúde, tais como jogos para a persuasão da prática de atividades físicas~\cite{brox11} e jogos para a melhoria das capacidades físicas e cognitivas~\cite{arntzen2011}. 

Dentro deste contexto de utilização, é que os jogos eletrônicos foram utilizados como um mecanismo para motivar a frequência do monitoramento da saúde, e induzir a execução dos movimentos específicos necessários para o monitoramento da saúde motora. Mais especificamente, busca-se uma integrar os~\ac{sms} na vida diária de indivíduos através dos jogos, com foco em doenças motoras. Como objeto de estudo escolhemos \ac{dp} por ser uma doença neurodegenerativa crônica, progressiva e com causa desconhecida. É uma doença mais comum em idosos; no entanto, existem casos precoces em indivíduos antes dos 40 anos ou até mesmo abaixo dos 21~\cite{menezes2003}. A incidência da doença é estimada entre 100 a 200 casos por 100.000 habitantes e, com o envelhecimento da população, o contingente de pessoas diagnosticadas com~\ac{dp} tende a aumentar nos próximos anos. Após os 10 anos de tratamento, a doença leva o indivíduo a irreversíveis debilidades: motoras e cognitivas. Logo, a abordagem de monitorar os sinais em diferentes momentos do dia permite um melhor gerenciamento da doença e, por consequência, melhora a qualidade de vida destes indivíduos.









\section{Relevância}\label{section:relevancia}
Nos últimos anos, a criação de tecnologias computacionais para o monitoramento da saúde~\cite{bardram2008} tem sido tema relevante e recorrente na computação~\cite{bradmonitor2015,compapproachparkinson2015,mazilu2015}. Uma área de grande interesse na comunidade científica é o monitoramento não invasivo dos sinais vitais como pressão sanguínea, batimentos cardíacos, glicemia entre outros. Recentemente, Helmman \textit{et. al.}~\cite{autonomparkin2015} realizou um estudo com o monitoramento contínuo não-invasivo da pressão arterial, alterações nos batimentos cardíacos e pressão sanguínea para encontrar potenciais evidências de disfunções autonômicas cardíacas~\footnote{Distúrbio funcional, de natureza primária ou secundária, resultante de alterações  puramente funcionais ou orgânicas localizadas em um  ou em ambos os componentes do sistema nervoso autônomo.} em indivíduos com~\ac{dp}, justamente para conseguir prover um melhor diagnóstico e obter uma maior compreensão dos sintomas não motores da doença. Logo, fica evidenciada a importância da intersecção da Ciência da Computação com outras áreas de conhecimento como a medicina~\cite{bardram2008}, por exemplo. Pois, as abordagens computacionais que permitem mensurar, identificar e quantificar os sintomas do~\ac{dp}, além de possibilitar uma melhora na qualidade de vida destes pacientes, auxiliam numa melhor compreensão clínica da evolução da doença e na identificação de mecanismos científicos para identificar o diagnóstico da doença que ainda não foi estabelecido~\cite{compapproachparkinson2015}.

%O principal objetivo destes trabalhos é permitir um monitoramento da saúde de uma maneira não invasiva e integrada à rotina diária de seus usuários~\cite{bardram08}. 

Atualmente, entidades internacionais de fomento industrial e científico da computação como IEEE~\cite{ieee2016} e ACM~\cite{acm2016} promovem simpósios como o CBMS~\cite{cbms2016}, SAC (\textit{track on Healthcare})~\cite{sachealth2016}, conferências como HEALTHCON~\cite{healthcon2016} e \textit{PervasiveHealth}~\cite{pervasivehealth2016} e até mesmo revistas científicas como \textit{Journal of Biomedical and Health Informatics} (JBHI)~\cite{jbhi2016}, \textit{IEEE Transactions on Biomedical Engineering}~\cite{tbe2016} (TBE). 

No Brasil, a Sociedade Brasileira de Computação~\cite{sbc2016} possui um \textit{Workshop} de Informática Médica (WIM)~\cite{wim2016} em seu principal congresso (CSBC)~\cite{csbc2016} que tem como objetivo reunir pesquisadores, estudantes, professores, empresários e profissionais da computação aplicada à Saúde. Em 2015, o JBHI publicou uma \textit{special issue} cujo tema foi tecnologias para o gerenciamento do~\ac{dp}~\cite{specjbhi2015}. Isto, evidencia a importância científica desta tese. %Desta maneira, evidencia-se a importância deste trabalho tanto nas comunidades científicas nacionais quanto internacionais.

Por fim, a elaboração deste trabalho gerou desdobramentos e discussões científicas no grupo de pesquisa na área de computação aplicada à saúde dentro do Laboratório de Sistemas Embarcados e Computação Pervasiva da UFCG. Como resultado desta sinergia, durante o desenvolvimento desta tese, foi possível colaborar com duas defesas de mestrado~\cite{antonio2013,gustavo2014} e desenvolver 3 trabalhos de iniciação científica. 


\subsection{Monitoramento Da Saúde Motora}
%TODO remover esta seção e condensá-la na anterior
%\section{Atividades Motoras} 
Nas últimas décadas, o monitoramento e quantificação dos sinais motores tem sido objeto de pesquisa recorrente na computação, eletrônica, bioinformática e saúde~\cite{reviewassesenspark2015}. As pesquisas na área, tem sido fundamental para compreender o progresso dos sintomas de doenças como o~\ac{dp} e também para auxiliar o médico no gerenciamento da doença.

Nesta tese, propomos um mecanismo de quantificar, avaliar e identificar o sintoma de bradicinesia do~\ac{dp} induzindo o usuário a executar movimentos de avaliação clínica, de uma forma lúdica e integrada a um jogo eletrônico e longe do contexto de tratamento da saúde.

Do ponto de vista clínico, será possível identificar como está a saúde do paciente e em que momento o tratamento medicamentoso está surtindo efeito. Uma das ferramentas disponíveis atualmente para monitorar flutuações motoras do~\ac{dp} é por meio da avaliação clínica, ou por meio de relatórios dos pacientes. Atualmente, um detalhamento das flutuações motoras do~\ac{dp} é realizado por auto-relatórios em avaliações diárias dos pacientes que informam em que período do dia a medicação está surtindo efeito~\cite{reviewassesenspark2015}. No entanto, para uma avaliação dos sintomas motores mais acurado, é necessário poder induzir a execução de movimentos clínicos de modo a mensurar quantitativamente os sintomas do~\ac{dp}~\cite{wiiassesspark2016}.

Por estes argumentos apresentados, propomos neste tese uma maneira quantitativa de avaliar a eficácia do tratamento através da indução dos movimentos de avaliação clínica dentro de um contexto de jogo. Esta abordagem de monitoramento, trás benefícios aos médicos para um tratamento mais efetivo e acurado da dosagem medicamentosa. Os resultados das análises aqui apresentadas permitiram identificar tarefas motoras adequadas para estimar a gravidade do sintoma e das complicações motoras avaliadas no estudo realizado de caso-controle. Esta observação sugere que os sintomas parkinsonianos e complicações motoras levar a características distintas de movimento que podem ser capturadas num jogo eletrônico, e isto é um resultado bastante relevante para esta tese.


%A quantitative way of assessing treatment efficacy can be a valuable tool for clinicians in disease management. By knowing what happens between outpatient visits, treatment interventions can be fine tuned to the needs of individual patients [99]. Another important application would be for use in randomized clinical trials. By gathering accurate and objective measures of symptoms, one could reduce the number of subjects and the duration of treatment required to observe an effect in a trial of a new therapy. ~\cite{patel2012}
%
%Results of the analyses that we performed to identify motor tasks suitable to estimate the severity of each symptom and motor complication are summarized in Fig. 5. Data features for these results were estimated using a 5 s window. The SVM was built using a third-order polynomial kernel. Results were derived on a subject-by-subject basis and shown as aggregate data via box plots for each symptom and motor complication for the motor tasks that we selected for the study, as shown in Table I. Results are not shown for motor tasks not suitable to build a classifier of a given symptom or motor complication. In general, no major differences were observed in the estimation error values obtained by using feature sets pertaining to different motor tasks, although the use of data from certain tasks appeared to achieve lower estimation error values and variability of the results compared to other tasks. The fact that several tasks appeared to be suitable for estimating the severity of each symptom and motor complication is an important result. This observation suggests that Parkinsonian symptoms and motor complications lead to distinct features of movement that can be captured irrespective of the specific motor task a patient is engaged into. This is a very promising result because it indicates that there is high likelihood that the analyses presented in this manuscript could be extended to motor tasks associated with activities of daily living.~\cite{patel_monitoring_2009}.
%
%
%
%
%
%In patients with Parkinson’s disease (PD), careful medication titration, based on detailed information about symptom response to medication intake, can significantly improve the patient’s quality of life. Medication titration in patients with late stage PD is often challenging as fluctuations in a patient’s motor symptom manifest over several hours and hence cannot be observed in a typical outpatient appointment (often lasting no more than 30 min). Patient diaries are unreliable due to perceptual bias and
%inaccurate reporting about motor status. The abovementioned issues limit the ability of physicians to optimally adjust medication dosage and to test new compounds for the treatment of PD. The use of a sensorbased system to monitor PD symptoms is a promising approach to improve the clinical management of patients in the late stages of the disease. Major PD symptoms have
%typical motor characteristics which can be captured using motion sensors such as accelerometers ~\cite{patel2012}
%
%
%The ability to estimate the severity of symptoms via processing sensor data recorded during activities of daily living is important for practical applications. ~\cite{patel2012}
%
%Movement sensors can also be used to automate clinical testing procedures. Salarian et al. [103] and Weiss et al. [104] have proposed instrumented versions of the timed up-and-go test for identifying gait impairments due to PD. They have shown that instrumented tests lead to an improved sensitivity to gait impairments compared to observation methods. Besides, sensor-based methods can also be extended to long term home monitoring. Based on this body of work, an ambulatory gait analysis system,
%based on wearable accelerometers, for patients with PD has been proposed by Salarian et al. [105]. ~\cite{patel2012}
%
%Finally, Patel et al. [109] showed that, using accelerometers placed on the arm, it is possible to derive accurate estimates of upper extremity functional ability. The authors used a small subset of tasks from the Wolf Functional Ability Scale (FAS) to derive estimates of the total FAS score via analysis of the accelerometer data. As the tasks selected from the FAS closely resemble tasks performed during the performance of activities of daily living, such a system could be used for unobtrusively
%monitoring functional ability in the patients’ home environment. ~\cite{patel2012}
%
%
%
%
%
%Items avaliados:
%They performed three of the bradykinesia-related items of the MDS-UPDRS 
 %- finger tapping
 %- diadochokinesis
 %- toe tapping
%Therefore, in this study a method for automatic, objectve and continuous scoring of three of the bradykinesia-related items of the MDS-UPDRS is proposed. Four clinicians scored these items for 25 patients diagnosed with Parkinson’s disease, within a range of 0-4. Orientation sensors were used to record movement during performance of each item.
%Current assessment of bradykinesia suffers from two important inherent inconveniences: the evaluators’ individual bias and the limited number of categories of the scale (0-4 for each item). The individual bias is due to the subjective appreciation of task execution and the subjective interpretation of the guidelines. The limited number of categories induces that if task performance has characteristics of two consecutive categories (e.g. 2 and 3), it might be scored in either of these categories by different evaluators
%In a separate study [4] we focused on the individual bias problem. We proposed an automatic and objective method for assessment of the bradykinesia-related items of the MDS-UPDRS that used a supervised classification algorithm (support vector machine (SVM) based) that reproduced the evaluators’ classification results.
%In the present study, we propose an approach that overcomes the problem associated with the limited number of categories of the MDS-UPDRS. We employed data recorded using orientation sensors during execution of the bradykinesia-related items of the MDS-UPDRS from the same 25 patients with PD and the scorings of the same four evaluators as in our other, separate study.
%Backward linear regression was applied to obtain a model for performance on each bradykinesia-related item using the features obtained from the Euler angle signals as independent variables and the averaged scores from the four evaluators as dependent variables. Backward linear regression was chosen over other regression approaches because it takes into account suppressor effects, which occur when a predictor has a significant effect but only when another variable is held constant [10]. By averaging the scores of the evaluators we decreased the effect of possible outliers (tasks misclassified) on model estimation.
%In this study we employed backward linear regression to create a continuous model for three bradykinesia-related items of the MDS-UPDRS. Using cross-validation we estimated that we can score a new task performance with an average error of 0.38 for finger tapping compared to the averaged scores of four clinicians. The estimated errors for diadochokinesis (0.56) and toe tapping (0.48) are similar. The still limited number of clinical evaluators may induce bias in the obtained model because of their subjective evaluation. A study with a larger number of evaluators could be helpful to derive an unbiased model. Also, the inclusion of a larger number of participants would allow the utilization of a less optimistic cross-validation technique than LOOCV, such as k-fold cross-validation. Reproducing our results in a study with more evaluators and participants is even more important because there are, to our knowledge, no comparable studies in literature.
%set of features defined there [4] [Appendix 1] and the averaged scores of the evaluators were used to define a continuous model for performance on each bradykinesia-related item of the MDS-UPDRS
%METHODs
%Twenty-five patients diagnosed with PD participated in the study. They performed three of the bradykinesia-related items of the MDS-UPDRS (items 3.4 (finger tapping), 3.6 (diadochokinesis) and 3.7 (toe tapping)) and their movements were videoed and subsequently analyzed by four well-trained evaluators.
%====================
%ATENCAO
%=======================
%Sensior fusion algorithm para combinar sinais do acelerometro, giroscópio e sensores magneticos para obter a orientacao do movimento em ângulos eulerianos
%~\cite{bradmonitor2015}
%
%
%
%
%
%
%------------------------
%Clinical management of patients with Parkinson’s disease requires instead that complex
%interactions of medications with multiple symptoms and motor complications be monitored over time, assessed, and integrated into algorithms for the clinical management of these patients.~\cite{bradmonitor2015}
%
%METHODS
%Subjects were instructed to perform a series of standardized motor tasks utilized clinically to evaluate patients with Parkinson’s
%disease. These motor tasks are part of the motor section of the Unified Parkinson’s Disease Rating Scale [21] and included finger-to-nose (reaching and touching a target), finger tapping, repeated hand movements (opening and closing both hands), heel tapping, quiet sitting, and alternating hand movements (repeated pronation/supination movements of the forearms). Fig. 1
%shows pictures taken during the execution of these tasks. After completion of the baseline trial, subjects took their medications
%and were then tested using the same procedure every 30 min ~\cite{patel_monitoring_2009}
%------------------Justificativa da Escolha da SVM
%We chose to use SVMs [23] due to the success of this approach in many classification problems. SVMs demonstrate good generalization performance [23]. We used the PRTools4 toolbox to implement SVM [24]. The specific SVM implementation we adopted relies on the one-versus-rest approach to tackle the multiclass classification problem. Three different kernels (i.e., exponential, radial basis, and polynomial kernels) were utilized and results were compared.
%==========Escolha das Atividades Motoras
%C. Selecting Suitable Motor Tasks
%-----------------------------------
%Results of the analyses that we performed to identify motor tasks suitable to estimate the severity of each symptom and
%motor complication are summarized in Fig. 5. Data features for these results were estimated using a 5 s window. The SVM
%was built using a third-order polynomial kernel. Results were derived on a subject-by-subject basis and shown as aggregate
%data via box plots for each symptom and motor complication for the motor tasks that we selected for the study, as shown in Table I. Results are not shown for motor tasks not suitable to build a classifier of a given symptom or motor complication. In general, no major differences were observed in the estimation error values obtained by using feature sets pertaining to
%different motor tasks, although the use of data from certain tasks appeared to achieve lower estimation error values and variability of the results compared to other tasks. The fact that several tasks appeared to be suitable for estimating the severity of each symptom and motor complication is an important result. This observation suggests that Parkinsonian symptoms and motor complications lead to distinct features of movement that can be captured irrespective of the specific motor task a patient is engaged into. This is a very promising result because it indicates that there is high likelihood that the analyses presented in this manuscript could be extended to motor tasks associated with activities of daily living.~\cite{patel_monitoring_2009}
%
%
%
%
%
%
%
%
%
%==========================================
%==========================================
%
%~\cite{wiiassesspark2016}
%
%O Mais importante: Cokicar êngase b motor task 
%============================================
%Home-based monitoring depending on fixed motor tasks
%============================================
%Monitoring patients in a continuous way without supervised and standardized motor tasks is not the only way to design a home-based monitoring system. Patients could also provide disease-related information by doing video-guided motor tasks at home (25,31). An algorithm was developed to estimate the clinical scores provided by a neurologist when observing patients performing the same motor tasks. The classification errors of this remote UPDRS assessment ystem were 3.4% for tremor, 2.2% for bradykinesia and 3.2% for dyskinesia, demonstrating that clinical UPDRS
%score can be reliably estimated under the help of the remote UPDRS assessment system based on wearable sensors (23).
%
%Certain motor symptoms like tremor could be detected from some specific metrics using Wii remote. It could also perform as an indicator of PD severity, with a clear correlation between patient self-rating scores of tremor severity and metric obtained (37).
%
%An example system is called PERFORM (49), which is an intelligent closed-loop system integrating a wide range of wearable sensors, allowing physicians to remotely monitor the condition changes of the patients, adjust medication schedules, and individualize therapy. In a study regarding gait measurement, researchers were working on combining the data from wrist and sternum sensors with the data from foot sensors (50).
%
%Patients need to take long-term medication treatment and are followed up continuously for frequent treatment adjustments, as PD can only be potentially slowed, not halted.
%
%Therefore, it is necessary to develop an objective long-term measurement of the motor function in PD patients, with the characteristics of sensitivity, accuracy,
%portability, and objectivity. Fortunately, the evolution of mobile and computer technology has made this possible with many devices available which could satisfy the above requirements, including wearable sensors (e.g., gyroscopes, accelerometers), balance boards (e.g., pressure sensors such as the Nintendo Wii Balance Board) and optical motion capture systems, to name a few. This review mainly introduces the objective measurement of motor function in PD based on wearable sensors from the perspective of practical applications
%
%Researchers were able to differentiate between PD patients and healthy controls in a lab environment using gyroscopes attached to different body segments. These subjects were directed to perform standardized motor tasks which reflect the characteristics of certain PD symptoms (2-7). Furthermore,
%as PD is a multi-symptom disease affecting various body segments and because most of the studies were based on features derived from single body segment, Jens Barth et al. managed to combine two different sensor-based systems analyzing hand motor function and gait together (8).
%
%Thirty-two features were selected from over one thousand features and using an AdaBoost classifier, the classification rates between PD group and healthy control group was improved from 89% for hand motor function and 91% for gait analysis, to 97% in combination (8).
%
%===================
%Home-based monitoring
%===================
%Currently, assessing the impact of medication and changes in a patients’ condition depends a lot on the patients’ diaries and memory, which is neither objective nor reliable. A study suggested that the subjective complaints of PD patients do not necessarily match the findings of quantitative objective
%assessment (21) in reflecting motor fluctuation of PD. An ideal home-based monitoring device should have following functions: record and store movement data continuously over the long-term; portable enough to avoid interfering with the daily activities of the patients; able to test subjects in a home-based and unsupervised environment; reveal the real condition of the patients (work as a remote UPDRS assessment); with the ability to capture the occurrence of
%motor fluctuation and patients’ compliance.
%
%
%============================================
%============================================

%Atualização
%\section{Trabalhos Relacionados}\label{section:trabalhos_relacionados}


\section{Trabalhos Relacionados}\label{section:trabalhos_relacionados}
Devido ao estilo de vida mais sedentário e ao aumento da população obesa, as pesquisas para a promoção da atividade física têm se tornado tópico de interesse para a comunidade científica~\cite{maitland2009,bartolome11,Mandryk2014}. Estudos demostram que uma atividade física regular traz benefícios físicos, cognitivos e emocionais~\cite{Mandryk2014}. Com o surgimento dos jogos comerciais, como o \textit{Wii Sports} da Nintendo~\footnote{http://www.nintendo.com}, em 2006, que aumentou a prática de atividade física de jogadores considerados sedentários~\cite{wiigraves2008}, é que pesquisadores da área de jogos para saúde buscaram apoiar essa prática com desenvolvimento de aplicações que motivassem a prática de atividade física~\cite{stacey2011}. Atualmente, os dispositivos de sensores de movimento permitem desenvolver jogos que promovem a saúde e o bem estar de forma promissora. Por esse motivo, houve um aumento significativo de jogos comerciais com o propósito de promover a saúde e o bem-estar da população~\cite{Papastergiou:2009:EPC:1570538.1570707}.


Os jogos pervasivos móveis motivam a atividade física de forma mais direta, tais como \textit{Transe}, \textit{Feeding Yoshi} e  \textit{Nokia Wellness Diary and Sports Tracker}, que promovem a saúde com a prática de atividade física~\cite{Suhonen:2008:SFE:1457199.1457204} e melhoram as condições de saúde por serem divertidos, imersivos e engajados. Para o público idoso, foram desenvolvidos diferentes jogos para a melhoria da saúde, como: sistemas para reabilitação motora dos idosos~\cite{brox11} e também para pacientes com~\ac{dp}~\cite{atkinson2010,synnott_wiipd_2012,sacbespoke2014}. 

Atkinson e Narasimhan~\cite{atkinson2010} desenvolveram um jogo que utiliza um sensor de toque para quantificar a habilidade motora do paciente com~\ac{dp}. Teoricamente, esta abordagem auxilia no diagnóstico do~\ac{dp}. No entanto, não foi realizado nenhum estudo com os pacientes para avaliar sua eficácia. Synnott \textit{et al.}~\cite{synnott_wiipd_2012} desenvolveu um sistema de gerenciamento medicamentoso e um jogo, utilizando um sensor de captura de movimentos, para identificar o sinal de tremor de~\ac{dp}. No entanto, o tremor de~\ac{dp} é de repouso~\cite{national2006parkinson}. Logo, quando o usuário está concentrado, entra no estado de ação e reduz drasticamente o tremor.

Papastergiou \textit{et al.}~\cite{Papastergiou:2009:EPC:1570538.1570707} identificaram efeitos positivos para a reabilitação através do uso do jogo \textit{Wii Sports} e um potencial mecanismo de prevenção e reeducação motora com o uso do \textit{Wii Fit}. Porém, esses jogos possuem suas limitações e não são substitutos dos esportes reais. Ainda assim, o autor salienta que um ambiente mais controlado, que permite a execução de atividades físicas, inibe a ocorrência de situações de risco como um movimento brusco e que venha causar um dano físico maior. Baseado nessas observações, esse trabalho primou por demonstrar as dificuldades e os efeitos positivos em combinar os jogos sérios de esportes e saúde com as tecnologias de sensores, para a personalização e adaptação dos jogos. Paraskevopoulos \textit{et al.}~\cite{sacbespoke2014} propõem um conjunto de diretrizes para o desenvolvimento de jogos com o objetivo de acompanhar o tratamento fisioterápico dos pacientes com~\ac{dp} e dar suporte à reabilitação 
destes.

Sinclair \textit{et al.}~\cite{Sinclair:2009:UVB:1515604.1515617} consideram que os jogos comerciais para prática de exercício físico (\textit{exergames}) não devem ser usados apenas como um motivador para a prática, mas também podem ser usados para monitorar sinais vitais como batimento cardíaco e reconhecer atividades via acelerômetros. Arntzen~\cite{arntzen2011} se preocupou com os aspectos cognitivos e físicos da aprendizagem baseada em jogos para idosos~\cite{arntzen2011}, defendendo que é necessário identificar quais habilidades cognitivas e físicas precisam ser desenvolvidas, além de considerar a limitação do idoso em relação aos movimentos bruscos no intuito de evitar lesões.

LeMoyne~\cite{lemoyne2010} quantificou os sinais de tremores de \ac{dp} usando um \textit{smartphone} (Figura~\ref{fig:iphone-tremor}). Ele considerou que os \textit{smartphones} estão presentes na rotina dos pacientes e que estes iriam mensurar seus tremores em diferentes momentos do dia. No entanto, o principal problema em mensurar o tremor usando \textit{smartphones} é que o tremor do~\ac{dp} é de repouso~\cite{jankovic2008}. Logo, os pacientes reduzem drasticamente o sinal, o que impacta diretamente na coleta dos dados. Deve-se considerar também que LeMoyne~\cite{lemoyne2010} não realizou avaliações com pacientes ou estudo de caso-controle. 

\begin{figure}
\centering
  \includegraphics[scale=0.3]{./img/moyne-iphone.png}
  % matrixargseg.png: 296x162 pixel, 100dpi, 7.52x4.11 cm, bb=0 0 213 117
  %\caption{Estágio desenvolvimento de jogos ~\cite{fullerton2008game}}
\caption[Aplicação para \textit{smartphone} com a finalidade de identificar sinais de tremor]{Aplicação para iPhone com a finalidade de identificar sinais de tremor ~\cite{lemoyne2010}}
 %  \caption{Estágio desenvolvimento de jogos}
 \label{fig:iphone-tremor}
\end{figure}






% \begin{figure}
%  \centering
%  \includegraphics[scale=0.3]{./img/quantif-parkinson.png}
% \caption[\textit{G-Link Wireless Accelerometer} - Instrumento usado no trabalho de LeMoyne para quantificar o tremor da Doença de Parkinson.]{\textit{G-Link Wireless Accelerometer} - Instrumento usado no trabalho de LeMoyne~\cite{LeMoyne2009} para quantificar o tremor da Doença de Parkinson.} 
% %  \caption{Estágio desenvolvimento de jogos}
%  \label{fig:quantif-parkinson}
% \end{figure}

Normalmente, as soluções existentes para \ac{sms} dos sinais motores utilizam sensores vestíveis (\textit{wearables}), que comumente são incorporados à roupa ou ao corpo do usuário. De acordo com a perspectiva do usuário, estes sensores são considerados invasivos e estereotipados~\cite{aarhus_negotiating_2010}. Por outro lado, o gerenciamento medicamentoso do~\ac{dp} necessita de um cuidado acurado e diário~\cite{quantitativeparkinson2011}. O problema então está em como alcançar o equilíbrio entre necessidade de monitoramento e não-invasividade e, ainda mais, buscando aumentar a motivação.

\section{Objetivos}\label{section:objetivos}
Nesta tese, tem-se como objetivo a conceber uma solução computacional que induza o usuário a executar movimentos para o o monitoramento dos sinais motores. Pretende-se usar jogos eletrônicos como forma de: \textbf{induzir}, \textbf{motivar} e abstrair o monitoramento de dados de saúde de uma maneira \textbf{não invasiva} e longe do \textbf{contexto de tratamento de saúde}.

Neste contexto, utilizou-se a Doença de\ac{dp} como estudo de caso para a avaliação da arquitetura proposta. Objetivou-se criar um\ac{sms} integrado à um jogo eletrônico capaz de: induzir a execução de movimentos clínicos que tornem possível o processamento dos sinais biomecânicos e consequentemente identificar a presença de sintomas do~\ac{dp}. Nesta tese, foi proposto uma arquitetura de software para o desenvolvimento de jogos eletrônicos integrados a um SMS, onde, demonstrou-se a viabilidade destaa arquitetura com a implementação de um jogo capaz de monitorar um sintoma do~\ac{dp}.

A avaliação da tese foi realizada em duas etapas: na primeira, avaliou-se a capacidade de monitoramento dos indivíduos com \ac{dp} em um estudo analítico de caso-controle; na segunda, avaliou-se a possibilidade de inserir este monitoramento na rotina diária dos pacientes. O estudo analítico de caso-controle foi realizado com 30 sujeitos de pesquisa (15 do grupo controle e 15 diagnosticados com~\ac{dp}). Como resultado, identificamos e quantificamos o sintoma da bradicinesia. Para distinguir os grupos (caso-controle e diagnosticados com~\ac{dp}), utilizamos uma~\ac{svm} para classificação dos dados~\cite{datamining2005}, com a qual obtivemos uma acurácia de 86,66\%. Avaliamos a adequação da abordagem de monitoramento dos sinais motores na rotina diária usando jogos eletrônicos, aplicando a técnica~\ac{gqm}~\cite{van1999goal}. Como resultado, 90,00\% dos avaliados consideraram a abordagem não-invasiva e incorporável à rotina diária. 

\section{Metodologia}
Esta pesquisa foi submetida à avaliação pelo Comitê de Ética da UFCG (\textbf{CAAE: 14408213.9.1001.5182})~\footnote{Plataforma Brasil, url: http://aplicacao.saude.gov.br/plataformabrasil/} (Apêndice~\ref{sec:comite}), somente depois da aprovação deste é que os dados foram coletados. A metodologia de pesquisa possui aspectos qualitativos e quantitativos. Referente ao aspecto qualitativo, buscou-se identificar a importância desta tese junto à comunidade de especialistas da área de saúde (Seção~\ref{sec:entrevista_semi_estruturada}). Nos aspectos quantitativos, essa pesquisa fez uma análise do sensores de movimento e avaliou a acurácia da aquisição de sinais motores e possibilidade de identificar os sinais do~\ac{dp} baseado na Cinemática Angular do Movimento Humano. Por meio dos dados coletados, pudemos classificar a normalidade e dificuldade na execução de movimentos de abdução e adução dos braços~\cite{mcginnis2013biomechanics}, como será apresentado na Seção~\ref{sec:resultado_svm}. Para avaliar a aceitabilidade da proposta sob a perspectiva do usuário, utilizamos uma análise~\ac{gqm} a qual é uma abordagem hierárquica que inicia com objetivo principal e o divide em questões mensuráveis~\cite{saraiva2006}, como será apresentado na Seção~\ref{gqm_usuarios}.

Em resumo, três questões foram utilizadas como base para a definição da metodologia do trabalho em três diferentes etapas sequenciais:
	\begin{description}
	\item[QUESTÃO 1] Quais os benefícios de acompanhar os sinais motores do paciente diariamente, do ponto de vista do profissional da saúde?
	\item[QUESTÃO 2] Como melhor adquirir e quantificar sinais motores utilizando sensores de movimento para monitorar os sinais de \ac{dp}?
	\item[QUESTÃO 3] Na perspectiva dos usuários, a abordagem de quantificar os sinais motores é considerada não-invasiva e aplicável à rotina diária?
	\end{description}

As seguintes atividades foram realizadas para a execução do trabalho:

\begin{enumerate}

\item{Realizar revisão bibliográfica e coleta de requisitos junto a profissionais de saúde.}

\item{Definir o conceito da abordagem, denominada \ac{jogue-me}, baseada em captura de sinais motores através de sensores de movimento, utilizando jogos eletrônicos e processamento dos sinais para transformá-los em informações de saúde.}


\item{Analisar a perspectiva dos profissionais de saúde em relação ao acompanhamento dos sinais motores dos pacientes com~\ac{dp} (os profissionais foram indagados sobre a melhora na tomada de decisão quanto ao acompanhamento dos sinais) e verificar se os parâmetros motores, como velocidade angular e amplitude do movimento dos braços, são importantes para realizar o acompanhamento dos sinais do~\ac{dp}. Procurou-se encontrar, junto ao profissional de saúde, a importância do monitoramento dos sinais motores e os benefícios trazidos por este, através de uma abordagem de pesquisa qualitativa. Com esta pesquisa, foi possível validar a \textbf{QUESTÃO 1}, que consiste em verificar a importância do acompanhamento de sinais motores integrados à rotina diária do paciente.}

\item{Validar o uso de sensores para classificação dos dados através da classificação dos sinais motores adquiridos por sensores de movimento utilizados em jogos eletrônicos. A classificação consistiu em aplicar os sinais numa~\ac{svm} para distinguir indivíduos do grupo controle ante indivíduos diagnosticados com~\ac{dp}.
O resultado dessa pesquisa demonstrou a viabilidade da abordagem e, consequentemente, validou a \textbf{QUESTÃO 2} do trabalho.}

\item{Definir a arquitetura de software que viabilizou tecnicamente a abordagem~\ac{jogue-me}. Nesta pesquisa, definimos um arcabouço de software para encapsular o desenvolvimento de jogos com essa abordagem.}

\item{Validar a solução~\ac{jogue-me} do ponto de vista computacional. A solução foi validada através da implementação da arquitetura e do desenvolvimento de jogos. Com esta etapa, demonstrou-se ser possível realizar monitoramento de dados motores de forma não invasiva, ou seja, sem os jogadores perceberem que estão fornecendo dados de saúde.}

\item{Verificar junto ao público alvo (portadores de~\ac{dp}) os requisitos de usabilidade, adequação à rotina diária, segurança física e se a proposta é considerada invasiva na perspectiva do paciente. Com esta avaliação, avaliou-se a \textbf{QUESTÃO 3} da pesquisa.}

\end{enumerate}

\subsection{Termo de Consentimento Livre e Esclarecido (TCLE)}
Antes da realização da coleta dos dados, expomos aos sujeitos da pesquisa as informações necessárias para a realização do estudo. Desta maneira, o indivíduo consentiu com sua participação através da assinatura do Termo de Consentimento Livre e Esclarecido~\footnote{Resolução Nº 196/96, do Conselho Nacional de Saúde, do Ministério da Saúde (CNS/MS).} (Apêndice~\ref{sec:comite}). 

\subsection{Relação Risco Benefício da Pesquisa}
Os riscos inerentes podem decorrer da exposição de dados dos participantes da pesquisa, o que pode acarretar danos morais e/ou psicológicos. Por esse motivo, foram tomados todos os cuidados para que a identidade do indivíduo não fosse revelada, garantindo assim, privacidade e confidência das informações. Todos os dados coletados, estão disponibilizados para pesquisa futura, permitindo o uso para pesquisa a todas instituições envolvidas (UFCG, UFAL e IFAL). No entanto, preservamos a identidade dos participantes da pesquisa e omitimos todos os dados que permitissem sua identificação, conforme descrito no Termo de Consentimento Livre e Esclarecido.

Durante a realização da pesquisa com os participantes da pesquisa, houve uma preocupação referente a possíveis constrangimentos por parte do sujeito da pesquisa. Caso, não conseguisse realizar a pesquisa ou responder alguma pergunta devido ao comprometimento da doença. Os pesquisadores prestaram total assistência, orientando-os adequadamente. Mas, salientamos que os riscos apresentados justificam-se pelo benefício de monitorar os sinais do~\ac{dp} para um melhor tratamento da doença.


\subsection{Confidencialidade}
Os dados do estudo em questão são considerados propriedade conjunta das partes envolvidas (UFCG, UFAL e IFAL). Porém, sua utilização por terceiros necessita de prévia autorização de todos. No entanto, na submissão do Projeto ao Comitê de Ética da UFCG (\textbf{CAAE: 14408213.9.1001.5182}), expressamos o comprometimento em tornar público os resultados da pesquisa, sejam estes favoráveis ou não.


\section{Contribuições}
Atualmente, os jogos são aplicados para melhora da saúde em diferentes contextos. No entanto, nenhum dos trabalhos relacionados pretendem identificar sinais para monitorar o estado de saúde. Logo, este trabalho visa desenvolver um ambiente de jogo que motive a execução de movimentos específicos, com o propósito de quantificar os sinais motores dos usuários.

No entanto, alinhar a jogabilidade e a capacidade de monitoramento dos sinais de saúde não é trivial, pois deve ser levado em consideração o uso dos sensores e deve-se definir quais movimentos ou ações permitem a identificação dos sinais motores. Por este motivo, a proposta de um~\ac{sms} dos sinais motores usando jogos necessita de um acompanhamento de um profissional de saúde para supervisionar e auxiliar nas definições dos movimentos e ações dos usuários. 

%Posteriormente, na posse dessas ações, deverá ser testada a execução dessas atividades e sua aquisição para uma possível classificação dos dados conforme proposto nesta tese.
%os trabalhos já existentes~\cite{Ballegaard:2008:HEL:1357054.1357336,patel_monitoring_2009,visionbased2009,bachlin_parkinsons_2009,albanese2012}.
%De posse dos movimentos e da captura dos dados será feito um levantamento de um \textit{game design} que permita executar os movimentos em  um ambiente lúdico e divertido como um jogo para entretenimento ~\cite{sweetser2005-gameflow}.

Como possível cenário de uso para a pesquisa, supondo que um paciente de uma doença crônica como o~\ac{dp} faz uso de medicamento antiparkinsoniano e possui um jogo de monitoramento de sinais do~\ac{dp} em sua residência, caso ele utilize o jogo em diferentes momentos do dia, os sinais podem ser quantificados sem a presença de um profissional de saúde, que poderia visualizar a melhora ou piora do estado de saúde do seu paciente ao longo dos dias. A partir da presente abordagem, o médico, ao possuir a informação, poderia gerenciar melhor a dosagem medicamentosa e, consequentemente, prolongar a qualidade de vida do paciente~\cite{abn2010}.

\section{Organização do Documento}
O restante deste documento está organizado da seguinte forma:
\begin{itemize}
	\item No Capítulo~\ref{chapter:fundamentacao} está descrita a fundamentação teórica relacionada ao trabalho.
	\item No Capítulo~\ref{chapter:abordagem_gahme} está definida a abordagem \ac{jogue-me} de monitoramento de sinais motores não invasiva usando jogos eletrônicos.
	\item No Capítulo~\ref{chapter:arquitetura_captura} é demonstrada uma implementação da abordagem.
	\item No Capítulo~\ref{chap:avaliacao} são apresentados os experimentos realizados para validar a tese.
	\item No Capítulo~\ref{chapter:conclusoes_futuros} são apresentadas as conclusões do trabalho e propostos trabalhos futuros.
\end{itemize}
