\chapter{Introdu\c{c}\~{a}o} \label{chapter:intro}

A população mundial envelhece progressivamente e, segundo estudos da~\ac{oms}~\cite{ageing2011}, muito em breve teremos mais idosos do que crianças. Ao considerar que a população idosa possui maior prevalência de doenças crônicas~\cite{prevcronica2009}, surge a necessidade de monitorar o estado da saúde dessa população. Portanto, diante do crescimento da quantidade de pacientes crônicos, da iminente redução do número de leitos hospitalares disponíveis e da insuficiência de profissionais especializados para atender esta demanda~\cite{healthmonitoring2013}, faz-se necessário transpor serviços de monitoramento dos pacientes crônicos dos leitos hospitalares para acompanhamento domiciliar~\cite{homecarebrazil2011}. 

Como objeto de estudo escolhemos \ac{dp} por ser uma doença neurodegenerativa crônica, progressiva e com causa desconhecida. É uma doença mais comum em idosos; no entanto, existem casos precoces em indivíduos antes dos 40 anos ou até mesmo abaixo dos 21~\cite{menezes2003}. A incidência da doença é estimada entre 100 a 200 casos por 100.000 habitantes e, com o envelhecimento da população, o contingente de pessoas diagnosticadas com~\ac{dp} tende a aumentar nos próximos anos. Após os 10 anos de tratamento, a doença leva o indivíduo a irreversíveis debilidades: motoras e cognitivas. Logo, a abordagem de monitorar os sinais em diferentes momentos do dia permite um melhor gerenciamento da doença e, por consequência, melhora a qualidade de vida destes indivíduos.

Na averiguação desta demanda, a computação aplicada à saúde busca prover o monitoramento da saúde~\cite{healthmonitoring2013,berg03,bardram2010,aarhus_negotiating_2010}. Os ~\ac{sms} permitem ao médico acompanhar à distância o estado de saúde de seus pacientes colaborativamente~\cite{healthmonitoring2013}. Atualmente, os~\ac{sms} realizam tratamento preventivo e pró-ativo do estado de saúde~\cite{bardram2010}; suporte à reabilitação do paciente~\cite{sacbespoke2014}; e auxílio para o paciente atingir uma melhor qualidade de vida~\cite{sacsvmhms2014}. Referente ao monitoramento dos sinais motores, os~\ac{sms} quantificam estes sinais e conseguem quantificar as habilidades motoras~\cite{manumeterjbhi2014,patel_monitoring_2009}, efetuar análise da marcha \cite{robotgait2014} e identificar sinais de bradicinesia~\footnote{Sintoma do Parkinson que consiste na lentidão da execução dos movimentos.}~\cite{ambulatoryparkinson2010}. Contudo, o maior desafio dessas abordagens é a aceitação do usuário e a consequente inserção na rotina diária~\cite{alemdar2015}.

Na busca por motivar os usuários, identificamos que os jogos eletrônicos encontram-se presentes na rotina diária de 27\% da população americana acima dos 50 anos~\cite{esa2015}. Com base nesse número, percebemos um público de jogadores idosos beneficiáveis por uma plataforma de monitoramento de dados de saúde embutida num jogo eletrônico. Aliado a esse estudo, identificamos jogos voltados para o público idoso aplicados à melhoria do estado de saúde, tais como jogos para a persuasão da prática de atividades físicas~\cite{brox11} e jogos para a melhoria das capacidades físicas e cognitivas~\cite{arntzen2011}. 

É neste contexto de utilização de jogos eletrônicos como meio motivador da utilização de Sistemas de Monitoramento da Saúde que se insere o presente trabalho. Mais especificamente, busca-se uma integração dos SMS na vida diária de indivíduos através dos jogos, com foco em doenças motoras, tendo \ac{dp} como estudo de caso.

\section{Problemática}\label{section:problematica}
%Para a identificação do problema foi realizado uma Revisão Bibliográfica sobre os temas IEEE~\cite{ieee}, ACM~\cite{acm}, PubMed~\cite{pubmed}, Scielo~\cite{scielo}.

%A Revisão da Literatura objetivou encontrar problemas em aberto nos~\ac{sms}. Inicialmente Foi realizado, um estudo complementar nas diretrizes médicas para o suporte científico nesta área. Essa etapa teve como objetivo inicial identificar problemas nesses trabalhos que pudessem ser solucionados por esta tese.

Devido ao estilo de vida mais sedentário e ao aumento da população obesa, as pesquisas para a promoção da atividade física têm se tornado tópico de interesse para a comunidade científica~\cite{maitland2009,bartolome11,Mandryk2014}. Estudos demostram que uma atividade física regular traz benefícios físicos, cognitivos e emocionais~\cite{Mandryk2014}. Com o surgimento dos jogos comerciais, como o \textit{Wii Sports} da Nintendo~\footnote{http://www.nintendo.com}, em 2006, que aumentou a prática de atividade física de jogadores considerados sedentários~\cite{wiigraves2008}, é que pesquisadores da área de jogos para saúde buscaram apoiar essa prática com desenvolvimento de aplicações que motivassem a prática de atividade física~\cite{stacey2011}. Atualmente, os dispositivos de sensores de movimento permitem desenvolver jogos que promovem a saúde e o bem estar de forma promissora. Por esse motivo, houve um aumento significativo de jogos comerciais com o propósito de promover a saúde e o bem-estar da população~\cite{Papastergiou:2009:EPC:1570538.1570707}.


Os jogos pervasivos móveis motivam a atividade física de forma mais direta, tais como \textit{Transe}, \textit{Feeding Yoshi} e  \textit{Nokia Wellness Diary and Sports Tracker}, que promovem a saúde com a prática de atividade física~\cite{Suhonen:2008:SFE:1457199.1457204} e melhoram as condições de saúde por serem divertidos, imersivos e engajados. Para o público idoso, foram desenvolvidos diferentes jogos para a melhoria da saúde, como: sistemas para reabilitação motora dos idosos~\cite{brox11} e também para pacientes com~\ac{dp}~\cite{atkinson2010,synnott_wiipd_2012,sacbespoke2014}. 

Atkinson e Narasimhan~\cite{atkinson2010} desenvolveram um jogo que utiliza um sensor de toque para quantificar a habilidade motora do paciente com~\ac{dp}. Teoricamente, esta abordagem auxilia no diagnóstico do~\ac{dp}. No entanto, não foi realizado nenhum estudo com os pacientes para avaliar sua eficácia. Synnott \textit{et al.}~\cite{synnott_wiipd_2012} desenvolveu um sistema de gerenciamento medicamentoso e um jogo, utilizando um sensor de captura de movimentos, para identificar o sinal de tremor de~\ac{dp}. No entanto, o tremor de~\ac{dp} é de repouso~\cite{national2006parkinson}. Logo, quando o usuário está concentrado, entra no estado de ação e reduz drasticamente o tremor.

Papastergiou \textit{et al.}~\cite{Papastergiou:2009:EPC:1570538.1570707} identificaram efeitos positivos para a reabilitação através do uso do jogo \textit{Wii Sports} e um potencial mecanismo de prevenção e reeducação motora com o uso do \textit{Wii Fit}. Porém, esses jogos possuem suas limitações e não são substitutos dos esportes reais. Ainda assim, o autor salienta que um ambiente mais controlado, que permite a execução de atividades físicas, inibe a ocorrência de situações de risco como um movimento brusco e que venha causar um dano físico maior. Baseado nessas observações, esse trabalho primou por demonstrar as dificuldades e os efeitos positivos em combinar os jogos sérios de esportes e saúde com as tecnologias de sensores, para a personalização e adaptação dos jogos. Paraskevopoulos \textit{et al.}~\cite{sacbespoke2014} propõem um conjunto de diretrizes para o desenvolvimento de jogos com o objetivo de acompanhar o tratamento fisioterápico dos pacientes com~\ac{dp} e dar suporte à reabilitação destes.

Sinclair \textit{et al.}~\cite{Sinclair:2009:UVB:1515604.1515617} consideram que os jogos comerciais para prática de exercício físico (\textit{exergames}) não devem ser usados apenas como um motivador para a prática, mas também podem ser usados para monitorar sinais vitais como batimento cardíaco e reconhecer atividades via acelerômetros. Arntzen~\cite{arntzen2011} se preocupou com os aspectos cognitivos e físicos da aprendizagem baseada em jogos para idosos~\cite{arntzen2011}, defendendo que é necessário identificar quais habilidades cognitivas e físicas precisam ser desenvolvidas, além de considerar a limitação do idoso em relação aos movimentos bruscos no intuito de evitar lesões.

LeMoyne~\cite{lemoyne2010} quantificou os sinais de tremores de \ac{dp} usando \textit{smartphones}. Ele considerou que os \textit{smartphones} estão presentes na rotina dos pacientes e que estes iriam mensurar seus tremores em diferentes momentos do dia. No entanto, o principal problema em mensurar o tremor usando \textit{smartphones} é que o tremor do~\ac{dp} é de repouso~\cite{jankovic2008}. Logo, os pacientes reduzem drasticamente o sinal, o que impacta diretamente na coleta dos dados. Deve-se considerar também que LeMoyne~\cite{lemoyne2010} não realizou avaliações com pacientes ou estudo de caso-controle. 

% \begin{figure}
%  \centering
%  \includegraphics[scale=0.3]{./img/patel-shimmer.png}
%  % matrixargseg.png: 296x162 pixel, 100dpi, 7.52x4.11 cm, bb=0 0 213 117
%  %\caption{Estágio desenvolvimento de jogos ~\cite{fullerton2008game}}
% \caption[Disposição dos Sensores de Movimento (SHIMMER) no corpo no trabalho de Patel]{\textit{Disposição dos Sensores de Movimento (SHIMMER) no corpo no trabalho de Patel ~\cite{patel_monitoring_2009}}}
% %  \caption{Estágio desenvolvimento de jogos}
%  \label{fig:patel-shimmer}
% \end{figure}


% \begin{figure}
%  \centering
%  \includegraphics[scale=0.3]{./img/moyne-iphone.png}
%  % matrixargseg.png: 296x162 pixel, 100dpi, 7.52x4.11 cm, bb=0 0 213 117
%  %\caption{Estágio desenvolvimento de jogos ~\cite{fullerton2008game}}
% \caption[Aplicação para \textit{smartphone} com a finalidade de identificar sinais de tremor]{Aplicação para iPhone com a finalidade de identificar sinais de tremor ~\cite{lemoyne2010}}
% %  \caption{Estágio desenvolvimento de jogos}
%  \label{fig:iphone-tremor}
% \end{figure}

% \begin{figure}
%  \centering
%  \includegraphics[scale=0.3]{./img/quantif-parkinson.png}
% \caption[\textit{G-Link Wireless Accelerometer} - Instrumento usado no trabalho de LeMoyne para quantificar o tremor da Doença de Parkinson.]{\textit{G-Link Wireless Accelerometer} - Instrumento usado no trabalho de LeMoyne~\cite{LeMoyne2009} para quantificar o tremor da Doença de Parkinson.} 
% %  \caption{Estágio desenvolvimento de jogos}
%  \label{fig:quantif-parkinson}
% \end{figure}

Em geral, as soluções existentes para \ac{sms} dos sinais motores utilizam sensores vestíveis (\textit{wearables}), que comumente são incorporados à roupa ou ao corpo do usuário. De acordo com a perspectiva do usuário, estes sensores são considerados invasivos e estereotipados~\cite{aarhus_negotiating_2010}. Por outro lado, o gerenciamento medicamentoso do~\ac{dp} necessita de um cuidado acurado e diário~\cite{quantitativeparkinson2011}. O problema então está em como alcançar o equilíbrio entre necessidade de monitoramento e não-invasividade e, ainda mais, buscando aumentar a motivação.

\section{Objetivos}
Neste trabalho, tem-se como objetivo a concepção de uma abordagem computacional não invasiva para o monitoramento de sinais motores. Jogos eletrônicos são utilizados como forma de motivar o monitoramento e abstrair o contexto de tratamento da saúde para os pacientes.

A Doença de \ac{dp} é utilizada como estudo de caso para a abordagem. Objetiva-se que o \ac{sms} integrado ao jogo eletrônico seja capaz de armazenar dados de sensores, processar sinais biomecânicos e identificar a presença do sinal de bradicinesia de \ac{dp}. Propõe-se uma arquitetura de software para SMS integrada a jogos eletrônicos e demonstra-se a viabilidade da mesma através da implementação de jogos com videogames disponíveis no mercado. 

A validação se baseia em duas etapas: na primeira, avalia-se a capacidade de monitoramento dos indivíduos com \ac{dp} em um estudo analítico de caso-controle; na segunda, avalia-se a possibilidade de inserir este monitoramento na rotina diária dos pacientes. O estudo analítico de caso-controle foi realizado com 30 sujeitos de pesquisa (15 do grupo controle e 15 diagnosticados com~\ac{dp}). Como resultado, identificamos e quantificamos o sintoma da bradicinesia. Para distinguir os grupos (caso-controle e diagnosticados com~\ac{dp}), utilizamos uma~\ac{svm} para classificação dos dados~\cite{datamining2005}, com a qual obtivemos uma acurácia de 86,66\%. Avaliamos a adequação da abordagem de monitoramento dos sinais motores na rotina diária usando jogos eletrônicos, aplicando a técnica~\ac{gqm}~\cite{van1999goal}. Como resultado, 90,00\% dos avaliados consideraram a abordagem não-invasiva e incorporável à rotina diária. 

\section{Metodologia}
A metodologia de pesquisa foi pautada pelo protocolo de pesquisa avaliado pelo Comitê de Ética da UFCG (Apêndice~\ref{sec:comite})(\textbf{CAAE: 14408213.9.1001.5182}). Este trabalho possui aspectos qualitativos e quantitativos. Referente ao aspecto qualitativo, buscou-se identificar a importância desta tese junto à comunidade de especialistas da área de saúde. Nos aspectos quantitativos, demonstramos que a abordagem consegue diferenciar indivíduos diagnosticados com \ac{dp} perante indivíduos sem o diagnóstico estabelecido, por meio de dados capturados por sensores de movimento usados em jogos. Ao final do trabalho avaliamos a aceitabilidade da proposta na perspectiva do usuário, utilizando uma análise~\ac{gqm}~\cite{van1999goal}. 



Em resumo, três questões foram utilizadas como base para a definição da metodologia do trabalho em três diferentes etapas sequenciais:
	\begin{description}
	\item[ETAPA 1] Quais os benefícios de acompanhar os sinais motores do paciente diariamente, do ponto de vista do profissional da saúde?
	\item[ETAPA 2] Como melhor adquirir e quantificar sinais motores utilizando sensores de movimento para monitorar os sinais de \ac{dp}?
	\item[ETAPA 3] Na perspectiva dos usuários, a abordagem de quantificar os sinais motores é considerada não-invasiva e aplicável à rotina diária?
	\end{description}

As seguintes atividades foram realizadas para a execução do trabalho:

\begin{enumerate}

\item{Realizar revisão bibliográfica e coleta de requisitos junto a profissionais de saúde.}

\item{Definir o conceito da abordagem, denominada \ac{jogue-me}, baseada em captura de sinais motores através de sensores de movimento, utilizando jogos eletrônicos e processamento dos sinais para transformá-los em informações de saúde.}


\item{Analisar a perspectiva dos profissionais de saúde em relação ao acompanhamento dos sinais motores dos pacientes com~\ac{dp} (os profissionais foram indagados sobre a melhora na tomada de decisão quanto ao acompanhamento dos sinais) e verificar se os parâmetros motores, como velocidade angular e amplitude do movimento dos braços, são importantes para realizar o acompanhamento dos sinais do~\ac{dp}. Procurou-se encontrar, junto ao profissional de saúde, a importância do monitoramento dos sinais motores e os benefícios trazidos por este, através de uma abordagem de pesquisa qualitativa. Com esta pesquisa, foi possível validar a \textbf{ETAPA 1}, que consiste em verificar a importância do acompanhamento de sinais motores integrados à rotina diária do paciente.}

\item{Validar o uso de sensores para classificação dos dados através da classificação dos sinais motores adquiridos por sensores de movimento utilizados em jogos eletrônicos. A classificação consistiu em aplicar os sinais numa~\ac{svm} para distinguir indivíduos do grupo controle ante indivíduos diagnosticados com~\ac{dp}.
O resultado dessa pesquisa demonstrou a viabilidade da abordagem e, consequentemente, validou a \textbf{ETAPA 2} do trabalho.}

\item{Definir a arquitetura de software que viabilizou tecnicamente a abordagem~\ac{jogue-me}. Nesta etapa, definimos um arcabouço de software para encapsular o desenvolvimento de jogos com essa abordagem.}

\item{Validar a solução~\ac{jogue-me} do ponto de vista computacional. A solução foi validada através da implementação da arquitetura e do desenvolvimento de jogos. Com esta etapa, demonstrou-se ser possível realizar monitoramento de dados motores de forma não invasiva, ou seja, sem os jogadores perceberem que estão fornecendo dados de saúde.}

\item{Verificar junto ao público alvo (portadores de~\ac{dp}) os requisitos de usabilidade, adequação à rotina diária, segurança física e se a proposta é considerada invasiva na perspectiva do paciente. Com esta avaliação, validou-se a \textbf{ETAPA 3} da pesquisa.}

\end{enumerate}

\section{Contribuições}
Atualmente, os jogos são aplicados para melhora da saúde em diferentes contextos. No entanto, nenhum dos trabalhos relacionados pretendem identificar sinais para monitorar o estado de saúde. Logo, este trabalho visa desenvolver um ambiente de jogo que motive a execução de movimentos específicos, com o propósito de quantificar os sinais motores dos usuários.

No entanto, alinhar a jogabilidade e a capacidade de monitoramento dos sinais de saúde não é trivial, pois deve ser levado em consideração o uso dos sensores e deve-se definir quais movimentos ou ações permitem a identificação dos sinais motores. Por este motivo, a proposta de um~\ac{sms} dos sinais motores usando jogos necessita de um acompanhamento de um profissional de saúde para supervisionar e auxiliar nas definições dos movimentos e ações dos usuários. 

%Posteriormente, na posse dessas ações, deverá ser testada a execução dessas atividades e sua aquisição para uma possível classificação dos dados conforme proposto nesta tese.
%os trabalhos já existentes~\cite{Ballegaard:2008:HEL:1357054.1357336,patel_monitoring_2009,visionbased2009,bachlin_parkinsons_2009,albanese2012}.
%De posse dos movimentos e da captura dos dados será feito um levantamento de um \textit{game design} que permita executar os movimentos em  um ambiente lúdico e divertido como um jogo para entretenimento ~\cite{sweetser2005-gameflow}.

Como possível cenário de uso para a pesquisa, supondo que um paciente de uma doença crônica como o~\ac{dp} faz uso de medicamento antiparkinsoniano e possui um jogo de monitoramento de sinais do~\ac{dp} em sua residência, caso ele utilize o jogo em diferentes momentos do dia, os sinais podem ser quantificados sem a presença de um profissional de saúde, que poderia visualizar a melhora ou piora do estado de saúde do seu paciente ao longo dos dias. A partir da presente abordagem, o médico, ao possuir a informação, poderia gerenciar melhor a dosagem medicamentosa e, consequentemente, prolongar a qualidade de vida do paciente~\cite{rodrigues2006}.

\section{Organização do Documento}
O restante deste documento está organizado da seguinte forma:
\begin{itemize}
	\item No Capítulo~\ref{chapter:fundamentacao} está descrita a fundamentação teórica relacionada ao trabalho.
	\item No Capítulo~\ref{chapter:abordagem_gahme} está definida a abordagem \ac{jogue-me} de monitoramento de sinais motores não invasiva usando jogos eletrônicos.
	\item No Capítulo~\ref{chapter:arquitetura_captura} é demonstrada uma implementação da abordagem.
	\item No Capítulo~\ref{chap:avaliacao} são apresentados os experimentos realizados para validar a tese.
	\item No Capítulo~\ref{chapter:conclusoes_futuros} são apresentadas as conclusões do trabalho e propostos trabalhos futuros.
\end{itemize}
