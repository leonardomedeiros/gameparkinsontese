\chapter{Conclusões e Trabalhos Futuros}\label{chapter:conclusoes_futuros}
Neste capítulo são apresentadas as conclusões sobre o trabalho apresentado. Na Seção~\ref{section:conclusoes} são apresentadas as conclusões desta tese juntamente com os resultados alcançados. Na Seção~\ref{section:limitacoes} são apresentadas as limitações encontradas durante os experimentos. Já na Seção~\ref{section:trabalhos_futuros} são propostos possíveis trabalhos futuros. Por fim, na Seção~\ref{section:publicacoes}, são listadas as publicações em conferências internacionais provenientes deste trabalho.




\section{Conclusões}\label{section:conclusoes}
O objetivo deste trabalho foi prover um mecanismo de monitoramento dos sinais motores que induzisse o indivíduo a executar movimentos específicos para avaliação motora de uma maneira não-invasiva. Por este motivo, nesta tese, foi definida e implementada uma arquitetura de \textit{software} que proporcionou usar jogos eletrônicos como um mecanismo para induzir e motivar os usuários a executar movimentos monitoráveis.


Como detalhado no Capítulo~\ref{chap:avaliacao}, os experimentos realizados para avaliar a tese consideraram as questões definidas na metodologia (Seção~\ref{section:metodologia}), são elas:
	\begin{description}
	\item[QUESTÃO 1] Quais os benefícios de acompanhar os sinais motores do paciente diariamente, do ponto de vista do profissional da saúde?
	\item[QUESTÃO 2] Como melhor adquirir e quantificar sinais motores utilizando sensores de movimento para monitorar os sinais de \ac{dp}?
	\item[QUESTÃO 3] Na perspectiva dos usuários, a abordagem de quantificar os sinais motores é considerada não-invasiva e aplicável à rotina diária?
	\end{description}

Referente à \textit{Questão 1}, obteve-se como resultado a necessidade de mensurar a amplitude do movimento, e a sua respectiva velocidade angular, com o propósito de monitorar sinais motores da Doença de Parkinson (Seção ~\ref{sec:entrevista_semi_estruturada}).

Na avaliação da \textit{Questão 2}, por meio de um estudo caso-controle, usando aprendizagem de máquina, obteve-se a viabilidade da arquitetura implementada com um jogo para o monitoramento do sintoma de bradicinesia do~\ac{dp}. Como resultado, atingiu-se uma taxa de acurácia de 86,67\% e falsos positivos de 6,67\% . Portanto, conclui-se que a arquitetura de \textit{software} implementada nesta tese, permitiu quantificar as complicações motoras num estudo de caso-controle e identificar a ocorrência do sintoma da bradicinesia. Considera-se que este é um resultado bastante relevante.

Por fim, na \textit{Questão 3}, avaliou-se a integração dos pacientes com~\ac{dp} ao monitoramento da saúde, usando jogos eletrônicos. Utilizou-se, nesta avaliação, a análise \textit{Goal, Question, Metric} (GQM) para avaliar se o monitoramento desenvolvido era não-invasivo e integrado a rotina diária dos usuários. As métricas obtidas nesta análise quantificaram que um percentual de 83\% dos avaliados integrariam este monitoramento em sua rotina; 91,67\% dos avaliados consideraram o jogo simples e de fácil entendimento, o que facilita a inclusão de idosos; e 76,67\% dos avaliados consideraram o jogo seguro para os idosos. 

Deve-se considerar que as métricas obtidas nessa pesquisa foram extraídas de um jogo desenvolvido com baixo custo. Caso este fosse aperfeiçoado, possivelmente a aceitabilidade da abordagem de monitoramento fosse ainda maior. Portanto, diante desses resultados, conseguiu-se atingir o principal objetivo deste trabalho, ao permitir que indivíduos com comprometimento motor pudessem ser monitorados de maneira não-invasiva e no conforto de seus lares.


\section{Limitações do Trabalho}\label{section:limitacoes}
A avaliação motora realizada neste trabalho foi o método utilizado para diferenciar os movimentos executados de indivíduos diagnosticados com o~\ac{dp} ante os indivíduos do grupo controle. Esta tese não pretende estabelecer um diagnóstico para o~\ac{dp}, ou até mesmo provar que os movimentos utilizados pelos participantes da pesquisa servem como diagnóstico. No entanto, esta tese consegue quantificar as diferenças entre os dois grupos no experimento com o jogo eletrônico e identificar a ocorrência do sintoma da bradicinesia presente no~\ac{dp}.

Com os resultados obtidos, pressupõe-se que esta abordagem pode ser aplicada a outras doenças motoras. No entanto, somente foi avaliados os indivíduos com Parkinson e os de grupo controle.

\section{Trabalhos Futuros}\label{section:trabalhos_futuros}
A partir dos resultados apresentados nesta tese e na sua extensão, alguns trabalhos futuros são propostos para contribuição científica.

\begin{itemize}
	\item Coletar uma amostra maior de pacientes com~\ac{dp}, agrupá-los de acordo com o estágio da doença e aplicar técnicas de multi-classificação de dados~\cite{multisvm2011} para identificar o progresso do~\ac{dp} de acordo com as escalas de avaliação~\cite{goul05}.
	\item Comparar o sinal da bradicinesia em diferente momentos do dia para monitorar a eficácia do tratamento medicamentoso e identificar as ocorrências das flutuações motoras\footnote{Respostas motoras flutuantes ao tratamento medicamentoso, com encurtamento da duração de seu efeito (fenômeno do \textit{wearing off}) e interrupção súbita de sua ação~\cite{protpar010}.}.
	\item Quantificar o sinal de tremor de repouso\footnote{O tremor de repouso é característico do~\ac{dp} e ocorre quando o indivíduo está parado ou desatento~\cite{protpar010}.} nos membros superiores, no momento em que o usuário esteja desatento.
	\end{itemize}



\section{Publicações}\label{section:publicacoes}
Foram publicados três artigos, em conferências internacionais, relacionados à tese: 
  \begin{itemize}
   \item \textit{Abstract}: \textit{Monitoring Parkinson related Gait Disorders with Eigengaits}, no, \textit{XX World Congress on Parkinson's Disease and Related Disorders} (2013)~\cite{lmmeigengaits2013};
   \item \textit{Full Paper}: \textit{A Game-Based Approach to Monitor Parkinson’s Disease: The bradykinesia symptom classification}, no, \textit{International Symposium on Computer-Based Medical Systems} (CBMS 2016)~\cite{lmmcbmsgame2016};
   \item \textit{Full Paper}: \textit{A Gait Analysis Approach to Track Parkinson’s Disease Evolution Using Principal Component Analysis}, no, \textit{International Symposium on Computer-Based Medical Systems} (CBMS 2016)~\cite{lmmcbmsgait2016}.
  \end{itemize}




