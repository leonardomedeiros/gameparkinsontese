\chapter{Conclusões e Trabalhos Futuros}\label{chapter:conclusoes_futuros}
Neste capítulo são apresentadas as conclusões sobre o trabalho apresentado e propostos trabalhos.  

%Já na Seção~\ref{sec:trab_futuros} são propostos possíveis trabalhos futuros. Por fim, no final do capítulo apresentamos as considerações finais na Seção~\ref{sec:cons_finais}.

Nos experimentos realizados, conseguimos demonstrar junto à comunidade de saúde (Seção~\ref{sec:entrevista_semi_estruturada}), a importância do acompanhamento dos sinais motores integrados à rotina diária do paciente. Identificou-se também, a importância de acompanhar a amplitude do movimento e a sua respectiva velocidade angular para acompanhamento da saúde motora.

Os estudos de aprendizagem de máquina com os dados motores adquiridos por meio de sensores de movimento usados em jogos eletrônicos, identificou a viabilidade do desenvolvimento de jogos para o monitoramento, pois, obtivemos uma taxa de acurácia de 86,67\% e falsos positivos de 6,67\% . A~\ac{svm} foi a técnica estatística de aprendizagem utilizada para distinguir os movimentos executados por indivíduos diagnosticados com~\ac{dp} ante os indivíduos de grupo controle. Esse estudo não teve a pretensão de estabelecer um diagnóstico da \ac{dp}, ou até mesmo provar que os movimentos utilizados pelos participantes da pesquisa servem para um diagnóstico. Contudo, este trabalho demonstrou que as diferenças nos movimentos, entre essas duas classes, permitem a identificação do sinal da bradicinesia, e que essas diferenças podem ser adquiridas por um sensor de movimento usado em jogos eletrônicos. A presente abordagem pode ser aplicada a outras doenças motoras; no entanto, testamos somente com indivíduos com~\ac{dp} e grupo controle.

Para identificar a possibilidade de integrar o monitoramento da saúde do jogador através de jogos eletrônicos à sua rotina diária, foi utilizada a análise \textit{Goal, Question, Metric} (GQM)~\cite{basili94} para avaliar a possibilidade de monitorar dados motores de forma não invasiva e integrada a rotina diária das pessoas. As métricas da análise quantificaram que um percentual de 83\% de usuários integrariam em sua rotina a solução de monitoramento proposta. Deve-se levar em consideração, também, que as métricas obtidas nessa pesquisa foram extraídas de um protótipo de jogo, e, caso este fosse aperfeiçoado é possível que a aceitabilidade da abordagem seja ainda maior. Desta maneira, conseguimos atingir o principal objetivo deste trabalho, ao permitir que indivíduos com comprometimento motor pudessem ser monitorados de maneira não-invasiva e no conforto de seus lares.




\subsubsection{Publicações}
Foram publicados três artigos, em conferências internacionais, relacionados à tese: 
  \begin{itemize}
   \item \textit{Abstract}: \textit{Monitoring Parkinson related Gait Disorders with Eigengaits}, no, \textit{XX World Congress on Parkinson's Disease and Related Disorders} (2013)~\cite{lmmeigengaits2013};
   \item \textit{Full Paper}: \textit{A Game-Based Approach to Monitor Parkinson’s Disease: The bradykinesia symptom classification}, no, \textit{International Symposium on Computer-Based Medical Systems} (CBMS 2016)~\cite{lmmcbmsgame2016};
   \item \textit{Full Paper}: \textit{A Gait Analysis Approach to Track Parkinson’s Disease Evolution Using Principal Component Analysis}, no, \textit{International Symposium on Computer-Based Medical Systems} (CBMS 2016)~\cite{lmmcbmsgait2016}.
  \end{itemize}

A partir dos resultados apresentados nesta tese e extensão da mesma, alguns trabalhos futuros são propostos para contribuição científica.

Como foi explanado na Seção~\ref{section:escalas_avaliacao}, a abordagem permite monitorar a progressão da doença e a eficácia do tratamento medicamentoso~\cite{updrs87,goul05}. Desta forma, é importante coletar uma amostra maior de pacientes com~\ac{dp}, agrupá-los de acordo com o estágio da doença~\cite{goul05}, e aplicar técnicas de multi-classificação de dados~\cite{multisvm2011} para identificar o progresso do~\ac{dp} de acordo com as escalas de avaliação. Em decorrência das ``Flutuações Motoras''~\footnote{Referente a respostas motoras flutuantes ao tratamento medicamentoso, com encurtamento da duração de seu efeito (fenômeno do \textit{wearing off}) e interrupção súbita de sua ação.}~\cite{protpar010},  é necessário comparar o sinal da bradicinesia em diferentes momentos do dia, para verificar a eficácia do tratamento medicamentoso~\cite{protpar010}.

Nos estudos realizados com os sinais adquiridos pelo MS-Kinnect, foi possível identificar a amplitude como apresentamos no estudo do movimento de abdução e adução do braço (Seção~\ref{sec:resultado_obtido_svm}). Todavia, a captura de um movimento mais sutil como um tremor é um desafio. Por esse motivo, foi desenvolvido e testado um jogo para celular que pudesse adquirir o sinal de tremor (Seção~\ref{sec:tremor}). Contudo, como o tremor do~\ac{dp} é de repouso~\cite{protpar010}, não foi possível quantificar o sinal. No entanto, ao analisarmos os vídeos dos pacientes com~\ac{dp}, identificamos, que ao levantar um dos membros, alguns indivíduos iniciavam o sinal de tremor no membro parado. Desta maneira, pode ser possível quantificar o sinal de tremor na análise do membro em repouso. No entanto, devido ao ruído existente na aquisição do sinal pelo MS-Kinnect~\cite{kinnect2013} pode-se inviabilizar a quantificação deste sinal. Como trabalhos futuros, uma investigação aprofundada destas questões podem gerar maiores contribuições para a área.








