\chapter{Conclusões e Trabalhos Futuros}\label{chapter:conclusoes_futuros}
Neste capítulo são apresentadas as conclusões sobre o trabalho apresentado. Na Seção~\ref{section:conclusoes} são apresentadas as conclusões desta tese juntamente com os resultados alcançados. Na Seção~\ref{section:limitacoes} são apresentadas as limitações encontradas durante os experimentos. Já na Seção~\ref{section:trabalhos_futuros} são propostos possíveis trabalhos futuros. Por fim, na Seção~\ref{section:publicacoes}, são listadas as publicações em conferências internacionais provenientes deste trabalho.

\section{Conclusões}\label{section:conclusoes}

%Nesta tese, tem-se como objetivo a conceber uma solução computacional que induza o usuário a executar movimentos para avaliação motora. Pretende-se usar jogos eletrônicos como forma de: \textbf{induzir}, \textbf{motivar} e abstrair o monitoramento de dados de saúde de uma maneira \textbf{não invasiva} e longe do \textbf{contexto de tratamento de saúde}.

%Nesta tese, foi proposto uma arquitetura de software para o desenvolvimento de jogos eletrônicos integrados a um SMS, onde, demonstrou-se a viabilidade desta arquitetura com a implementação de um jogo capaz de monitorar um sintoma do~\ac{dp}, como estudo de caso.

%Objetivou-se criar um~\ac{sms} integrado à um jogo eletrônico capaz de: induzir a execução de uma avaliação motora que torne possível o processamento dos sinais biomecânicos e consequentemente identificar a presença de sintomas do~\ac{dp}. 

%%----------------------------------
%
%Em resumo, três questões foram utilizadas como base para a definição da metodologia do trabalho:
	%\begin{description}
	%\item[QUESTÃO 1] Quais os benefícios de acompanhar os sinais motores do paciente diariamente, do ponto de vista do profissional da saúde?
	%\item[QUESTÃO 2] Como melhor adquirir e quantificar sinais motores utilizando sensores de movimento para monitorar os sinais de \ac{dp}?
	%\item[QUESTÃO 3] Na perspectiva dos usuários, a abordagem de quantificar os sinais motores é considerada não-invasiva e aplicável à rotina diária?
	%\end{description}
%
%A execução da metodologia de trabalho foi realizada de acordo com as seguintes atividades:
%\begin{enumerate}
%
%\item{Realização da revisão bibliográfica e coleta de requisitos junto a profissionais de saúde.}
%
%\item{Definição da abordagem, denominada \ac{jogue-me}, baseada em captura de sinais motores através de sensores de movimento, utilizando jogos eletrônicos e processamento dos sinais para transformá-los em informações de saúde.}
%
%
%\item{Análise da perspectiva dos profissionais de saúde relacionados ao acompanhamento dos sinais motores dos pacientes com~\ac{dp} (os profissionais foram indagados sobre a melhora na tomada de decisão quanto ao acompanhamento dos sinais). Procurou-se encontrar, junto ao profissional de saúde, a importância do monitoramento dos sinais motores e os benefícios trazidos por este, através de uma abordagem de pesquisa qualitativa. Com esta pesquisa, foi possível avaliar a \textbf{QUESTÃO 1}, que consiste em verificar a importância do acompanhamento de sinais motores integrados à rotina diária do paciente.}
%
%\item{Validação do uso de sensores para classificação dos dados através do processamento dos sinais motores adquiridos por sensores de movimento utilizados em jogos eletrônicos. A classificação consistiu em aplicar os sinais numa~\ac{svm} para distinguir indivíduos do grupo controle ante indivíduos diagnosticados com~\ac{dp}.
%O resultado dessa pesquisa demonstrou a viabilidade da abordagem e, consequentemente, validou a \textbf{QUESTÃO 2} do trabalho.}
%
%\item{Definição da arquitetura de software que viabilizou tecnicamente a abordagem~\ac{jogue-me}. Nesta pesquisa, definimos um arcabouço de software para encapsular o desenvolvimento de jogos com essa abordagem.}
%
%\item{Validação a solução~\ac{jogue-me} do ponto de vista computacional. A solução foi validada através da implementação da arquitetura e do desenvolvimento de jogos. Com esta etapa, demonstrou-se ser possível realizar monitoramento de dados motores de forma não invasiva, ou seja, sem os jogadores perceberem que estão fornecendo dados de saúde.}
%
%\item{Avaliação junto ao público alvo (portadores de~\ac{dp}) sobre: os requisitos de usabilidade, adequação à rotina diária, segurança física e se a proposta é considerada não-invasiva na perspectiva do paciente. Com esta avaliação, avaliou-se a \textbf{QUESTÃO 3} da pesquisa.}






%===================
A avaliação da tese foi realizada em duas etapas: na primeira, avaliou-se a capacidade de monitoramento dos indivíduos com \ac{dp} em um estudo analítico de caso-controle; na segunda, avaliou-se a possibilidade de inserir este monitoramento na rotina diária dos pacientes. O estudo analítico de caso-controle, onde foi realizado uma avaliação com 30 sujeitos de pesquisa (15 do grupo controle e 15 diagnosticados com~\ac{dp}). Como resultado, foi identificado e quantificado o sintoma da bradicinesia. Para distinguir os grupos (caso-controle e diagnosticados com~\ac{dp}), utilizamos uma~\ac{svm} para classificação dos dados~\cite{datamining2005}, com a qual obteve-se uma acurácia de 86,66\%. Avaliou-se a adequação da abordagem de monitoramento dos sinais motores na rotina diária usando jogos eletrônicos, aplicando a técnica~\ac{gqm}~\cite{van1999goal}. Nesta avaliação, 90,00\% dos avaliados consideraram a abordagem não-invasiva e incorporável a rotina diária. 


%---------------------------
% Removido da qualificação
%---------------------------
%Nos experimentos realizados, conseguimos demonstrar a importância do acompanhamento
%de sintomas motores, integrados à rotina diária do paciente do ponto de vista do profissional
%de saúde Hipótese H1. Identificou-se nessa pesquisa a importância de acompanhar a amplitude
%do movimento e a sua respectiva velocidade angular para acompanhamento da saúde
%motora.
%Os estudos de aprendizagem de máquina com os dados motores adquiridos por meio de
%sensores de movimento usados em jogos eletrônicos, identificou a viabilidade do desenvolvimento
%de jogos para o monitoramento, que valida a Hipótese H2. Pois, obtivemos uma
%taxa de identificação de verdadeiros positivos de 80,00% e falsos positivos de 16,67% .
%A Hipótese H3, foi validada por meio de uma análise GQM aplicada a possíveis usuários
%finais da abordagem. Essa pesquisa forneceu indícios de que a abordagem GAHME apresentada
%nessa Proposta permite o monitoramento de dados de forma não invasiva, e factível
%de integrar a solução a rotina diária dos usuários. Entretanto, o tempo utilizado para jogar
%foi insuficiente para aplicar as técnicas de processamento dos dados apresentados nesta
%abordagem, pois os jogadores tiveram bastante liberdade de movimento e poucos efetuaram
%os movimentos de abdução e adução do braço. Caso, os mesmos indivíduos participassem
%de um tempo maior no jogo, consequentemente eles poderiam efetuar o movimento e seria
%possível adquirir esses dados. Para chegarmos a resultados semelhantes aos apresentados
%na Seção 6.3 em um espaço de tempo menor, é necessário desenvolver um novo jogo com
%as ações específicas de realização




Nos experimentos realizados foi ressaltado junto a comunidade de saúde (Seção~\ref{sec:entrevista_semi_estruturada}) da importância do acompanhamento dos sinais motores integrados à rotina diária do paciente. Foi identificado também que o acompanhamento da amplitude do movimento de abdução e adução dos braços e sua respectiva velocidade angular permitem um monitoramento do estado de saúde dos pacientes com~\ac{dp}.

Os estudos de aprendizagem de máquina com os dados motores adquiridos por meio de sensores de movimento usados em jogos eletrônicos, identificou a viabilidade do desenvolvimento de jogos para o monitoramento, pois, obtivemos uma taxa de acurácia de 86,67\% e falsos positivos de 6,67\% . A~\ac{svm} foi a técnica estatística de aprendizagem utilizada para distinguir os movimentos executados por indivíduos diagnosticados com~\ac{dp} ante os indivíduos de grupo controle. Esse estudo não teve a pretensão de estabelecer um diagnóstico da \ac{dp}, ou até mesmo provar que os movimentos utilizados pelos participantes da pesquisa servem para um diagnóstico. Contudo, este trabalho demonstrou que as diferenças nos movimentos, entre essas duas classes, permitem a identificação do sinal da bradicinesia, e que essas diferenças podem ser adquiridas por um sensor de movimento usado em jogos eletrônicos. A presente abordagem pode ser aplicada a outras doenças motoras; no entanto, testamos somente com indivíduos com~\ac{dp} e grupo controle.

Para identificar a possibilidade de integrar o monitoramento da saúde do jogador através de jogos eletrônicos à sua rotina diária, foi utilizada a análise \textit{Goal, Question, Metric} (GQM)~\cite{basili94} para avaliar a possibilidade de monitorar dados motores induzindo o movimento de avaliação motora de forma não invasiva e integrada a rotina diária das pessoas. As métricas da análise quantificaram que um percentual de 90\% de usuários que integrariam em sua rotina diária a solução de monitoramento proposta. Deve-se levar em consideração, também, que as métricas obtidas nessa pesquisa foram extraídas de um protótipo de jogo, e, caso este fosse aperfeiçoado é possível que a aceitabilidade da abordagem seja ainda maior. Desta maneira, conseguimos atingir o principal objetivo deste trabalho, ao permitir que indivíduos com comprometimento motor pudessem ser monitorados de maneira não-invasiva e no conforto de seus lares.


%Colocar isto com outras palavras
%Os riscos inerentes podem decorrer da exposição de dados dos participantes da pesquisa, o que pode acarretar danos morais e/ou psicológicos. Por esse motivo, foram tomados todos os cuidados para que a identidade do indivíduo não fosse revelada, garantindo assim, privacidade e confidência das informações. Todos os dados coletados, estão disponibilizados para pesquisa futura, permitindo o uso para pesquisa a todas instituições envolvidas (UFCG, UFAL e IFAL). No entanto, preservamos a identidade dos participantes da pesquisa e omitimos todos os dados que permitissem sua identificação, conforme descrito no Termo de Consentimento Livre e Esclarecido.

%Durante a realização da pesquisa com os participantes da pesquisa, houve uma preocupação referente a possíveis constrangimentos por parte do sujeito da pesquisa. Caso, não conseguisse realizar a pesquisa ou responder alguma pergunta devido ao comprometimento da doença. O pesquisador prestou total assistência, orientando-os adequadamente. Mas, salienta-se que os riscos apresentados justificam-se pelo benefício de monitorar os sinais do~\ac{dp} para um melhor tratamento da doença.



\section{Limitações do Trabalho}\label{section:limitacoes}
A avaliação motora avaliada neste trabalho foi o método utilizado para diferenciar os movimentos executados de indivíduos diagnosticados com o~\ac{dp} ante os indivíduos do grupo controle. Nesta tese não se pretende estabelecer um diagnóstico para o~\ac{dp}, ou até mesmo provar que os movimentos utilizados pelos participantes da pesquisa servem para um diagnóstico. No entanto, esta tese conseguiu-se demonstrar as diferenças entre os dois grupos no experimento com o jogo eletrônico com o proóvoltado para monitoramento da saúde motora.

por um sensor de movimento i

usando um jogo eletrônico, e que essas diferenças podem ser classificadas utilizando uma
abordagem de aprendizagem de máquina.

sem o diagnóstico estabelecido, foi uma técnica estatística
de aprendizagem denominada de SVM. 


Apesar do questionário ter avaliado a opinião dos jogadores quanto ao jogo apresentado, pode-se generalizar que as opiniões são válidas para outros jogos usando a arquitetura~\ac{jogue-me}. Deve-se levar em consideração, também, que as métricas obtidas nessa pesquisa foram extraídas de um jogo na fase de protótipo. Caso ele fosse aperfeiçoado é possível que sua aceitabilidade seria ainda maior. Por esse motivo, o resultado obtido com a pesquisa GQM foi positivo, e considera-se que é viável desenvolver um jogo com o objetivo de induzir a execução de movimentos para monitorar os sinais motores, de forma não invasiva, e integrada à rotina diária dos usuários.


\section{Trabalhos Futuros}\label{section:trabalhos_futuros}
A partir dos resultados apresentados nesta tese e extensão da mesma, alguns trabalhos futuros são propostos para contribuição científica.

\begin{itemize}
	\item coletar uma amostra maior de pacientes com~\ac{dp}, agrupá-los de acordo com o estágio da doença~\cite{goul05}, e aplicar técnicas de multi-classificação de dados~\cite{multisvm2011} para identificar o progresso do~\ac{dp} de acordo com as escalas de avaliação.
	\item Em decorrência das ``Flutuações Motoras''~\footnote{Referente a respostas motoras flutuantes ao tratamento medicamentoso, com encurtamento da duração de seu efeito (fenômeno do \textit{wearing off}) e interrupção súbita de sua ação.}~\cite{protpar010}, é necessário comparar o sinal da bradicinesia em diferentes momentos do dia, para verificar a eficácia do tratamento medicamentoso~\cite{protpar010}. 
	\item Todavia, a captura de um movimento mais sutil como um tremor é um desafio. Em um estudo preliminar durante esta tese foi desenvolvido e testado um jogo para celular que pudesse adquirir o sinal de tremor (Seção~\ref{sec:tremor}), mas como o tremor do~\ac{dp} é de repouso~\cite{protpar010}, não foi possível quantificar o sinal. Ao analisar os vídeos dos pacientes com~\ac{dp} foi identificado que ao levantar um dos membros os indivíduos com tremor iniciavam o sinal no membro parado. , pode ser possível quantificar o sinal de tremor na análise do membro em repouso. No entanto, devido ao ruído existente na aquisição do sinal pelo MS-Kinnect~\cite{kinnect2013} pode-se inviabilizar a quantificação deste sinal.
\end{itemize}


%Como foi explanado na Seção~\ref{section:escalas_avaliacao}, a abordagem permite monitorar a progressão da doença e a eficácia do tratamento medicamentoso~\cite{updrs87,goul05}. Desta forma, é importante coletar uma amostra maior de pacientes com~\ac{dp}, agrupá-los de acordo com o estágio da doença~\cite{goul05}, e aplicar técnicas de multi-classificação de dados~\cite{multisvm2011} para identificar o progresso do~\ac{dp} de acordo com as escalas de avaliação. Em decorrência das ``Flutuações Motoras''~\footnote{Referente a respostas motoras flutuantes ao tratamento medicamentoso, com encurtamento da duração de seu efeito (fenômeno do \textit{wearing off}) e interrupção súbita de sua ação.}~\cite{protpar010}, é necessário comparar o sinal da bradicinesia em diferentes momentos do dia, para verificar a eficácia do tratamento medicamentoso~\cite{protpar010}. Como trabalhos futuros, uma investigação aprofundada destas questões podem gerar maiores contribuições para a área.
%
%Nos estudos realizados com os sinais adquiridos pelo MS-Kinnect, foi possível identificar a amplitude como apresentamos no estudo do movimento de abdução e adução do braço (Seção~\ref{sec:resultado_obtido_svm}). Todavia, a captura de um movimento mais sutil como um tremor é um desafio. Em um estudo preliminar durante esta tese foi desenvolvido e testado um jogo para celular que pudesse adquirir o sinal de tremor (Seção~\ref{sec:tremor}), mas como o tremor do~\ac{dp} é de repouso~\cite{protpar010}, não foi possível quantificar o sinal. Ao analisar os vídeos dos pacientes com~\ac{dp} foi identificado que ao levantar um dos membros os indivíduos com tremor iniciavam o sinal no membro parado. , pode ser possível quantificar o sinal de tremor na análise do membro em repouso. No entanto, devido ao ruído existente na aquisição do sinal pelo MS-Kinnect~\cite{kinnect2013} pode-se inviabilizar a quantificação deste sinal. Como trabalhos futuros, uma investigação aprofundada destas questões podem gerar maiores contribuições para a área.




\section{Publicações}\label{section:publicacoes}
Foram publicados três artigos, em conferências internacionais, relacionados à tese: 
  \begin{itemize}
   \item \textit{Abstract}: \textit{Monitoring Parkinson related Gait Disorders with Eigengaits}, no, \textit{XX World Congress on Parkinson's Disease and Related Disorders} (2013)~\cite{lmmeigengaits2013};
   \item \textit{Full Paper}: \textit{A Game-Based Approach to Monitor Parkinson’s Disease: The bradykinesia symptom classification}, no, \textit{International Symposium on Computer-Based Medical Systems} (CBMS 2016)~\cite{lmmcbmsgame2016};
   \item \textit{Full Paper}: \textit{A Gait Analysis Approach to Track Parkinson’s Disease Evolution Using Principal Component Analysis}, no, \textit{International Symposium on Computer-Based Medical Systems} (CBMS 2016)~\cite{lmmcbmsgait2016}.
  \end{itemize}




