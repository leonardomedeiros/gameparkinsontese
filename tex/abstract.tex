The use of Health Monitoring System (HMS) can improve the patients' quality life by allowing the physician to have access to information of patients' health status. This allows early identification of symptoms or identify the occurrence of health critical situations. However, the motor monitoring requires a motor evaluation of the the patient's motility through movement wich allows an motor assessment. This is a real challenge to design a noninvasive and engaged HMS into users' daily routine.
The proposed system architecture, allows a HMS to electronic games that uses motion detection sensors to acquire motor signals from the kinetic actions of the user. Thus, within the context of an electronic game, the user is induced to perform movements that assess their health status in a playful environment and away from the health care context. To evaluate this architecture, it was developed an electronic game using this architecture and held an analytical case-control study to detect individuals diagnosed with Parkinson's disease (Parkinson). So, in this experiment we quantified the motion characteristics using the angular kinects of the human movement, where the acquired data were applied to a Support Vector Machine (SVM) responsible to identify the occurrence of a Parkinson's motor symptom. In our results, we obtained a data classification accuracy of 86.67\% and false positive rate of 6.67\% Moreover, in users' acceptance evaluation, 90\% felt motivated with the developed game and 
aswered they would integrate this HMS into his daily routine. So, according this experiments, we system architecture allows the development of games with the monitoring purpose of the users' motor evaluation integrated into his daily routine.




%VERSÂO ANTERIOR
Health Monitoring Systems (HMS) can improve users' quality life by remotely providing information to remotely provide information about the health status, allowing early identification of critical situations. However, to monitor the users' motor health is necessary to perform clinical movements that allow the motor evaluation. For this reason, the design of a noninvasive SMS for motor health is still a multidisciplinary challenge.


However, the developemnt of a non-invasive health monitoring system for motor data is a multidisciplinary challenge. These systems, despite technology advancements, are still invasive and stereotyped, what makes difficult their dissemination. So, these systems have not been applied to the users daily activities, undermining motor symptoms monitoring.

This work proposes the use of video games to motivate and disregard health monitoring, integrating it in users daily routine. The proposed approach allows integrating the HMS architecture to electronic games that use motion detection sensors to capture users' kinetic actions. This way, the user performs specific movements inside the context of an electronic game that quantifies motion signals and monitors health.

To validate the approach, we performed a case-control analytic study to detect individuals diagnosed with Parkinson's Disease (PD) using motion capturing sensors through video games. We evaluated health data acquisition possibilities based on Human Motion Angular Kinetic characteristics. The data was applied in a Support Vector Machine (SVM) to classify the data. As a result, we had an accuracy rate of 86.67\% true positive identification and 6.67\% rate of false positive. This way, we concluded that the proposed approach allows developing video games to monitor motion data non-invasively.