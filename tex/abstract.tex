Health Monitoring Systems (HMS) allow doctors to gain a better picture of their patient’s health status. An early identification of symptoms can support the disease's diagnostic and prevent critical situations.

In order to monitor a patient’s motor abilities, it is necessary to record and evaluate specific movements. This makes it difficult to design a HMS that non-obtrusively integrates into the patient’s daily routine. The approach presented in this thesis makes use of the motivational power of electronic games. While playing the game, the user is incited to make the relevant movements, so that an optical sensor can detect and measure them. The playful situation distracts the user from thinking about health issues and therefore encourages more natural movements with improved validity for a health examination.

To evaluate this approach, a game has been developed and employed to detect Parkinson related motor symptoms.  In a study with patients diagnosed as affected by the Parkinson Disease and a healthy control group, the angular movements of the arms were used to measure motor abilities. The data was then processed and applied to a Support Vector Machine (SVM) to predict, based on the detected movements, whether a subject shows Parkinson related symptoms or should be classified as healthy. The system classified the subjects with an accuracy of 86.67\% and a rate of 6.67\% false positives. Furthermore, the user acceptance of the game-based approach was studied and showed that 90\% of the users felt motivated to play the game as part of their daily routine. These results demonstrate that the game-based approach presented in this thesis has the potential to become a base for HMS that monitor motor symptoms.

Keywords: Health Monitoring Systems; Health Games; Parkinson Disease; Support Vector Machine.