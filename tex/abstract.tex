The use of Health Monitoring System (HMS) can improve the patients' quality life by allowing the physician to have access to information of patients' health status. This allows early identification of symptoms or identify the occurrence of health critical situations. However, the motor monitoring requires a motor evaluation of the the patient's motility through movement which allows an motor assessment. This is a real challenge to design a noninvasive and engaged HMS into users' daily routine.
The proposed system architecture, allows a HMS to electronic games that uses motion detection sensors to acquire motor signals from the kinetic actions of the user. Thus, within the context of an electronic game, the user is induced to perform movements that assess their health status in a playful environment and away from the health care context. To evaluate this architecture, it was developed an electronic game using this architecture and held an analytical case-control study to detect individuals diagnosed with Parkinson's disease (Parkinson). So, in this experiment we quantified the motion characteristics using the angular kinetics of the human movement, where the acquired data were applied to a Support Vector Machine (SVM) responsible to identify the occurrence of a Parkinson's motor symptom. In our results, we obtained a data classification accuracy of 86.67\% and false positive rate of 6.67\% Moreover, in users' acceptance evaluation, 90\% felt motivated with the developed game and 
answered they would integrate this HMS into his daily routine. So, according this experiments, we system architecture allows the development of games with the monitoring purpose of the users' motor evaluation integrated into his daily routine.



