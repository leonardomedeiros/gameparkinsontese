Os Sistemas de Monitoramento da Saúde (SMS) podem melhorar a qualidade de vida dos usuários ao fornecer remotamente informações sobre o estado de saúde, permitindo identificar precocemente situações potencialmente críticas. Por outro lado, a concepção de um SMS de dados motores não invasivo ainda é um grande desafio multidisciplinar. Estes sistemas, apesar do avanço na tecnologia, ainda são invasivos e estereotipados, o que dificulta sua disseminação. Portanto, o uso destes sistemas não tem sido incorporado na rotina dos usuários, inviabilizando o monitoramento dos sintomas motores. 

Neste trabalho propõe-se utilizar jogos eletrônicos para motivar os usuários e abstrair o monitoramento de dados de saúde, tornando-o integrado à rotina diária. A abordagem proposta permite integrar a arquitetura do SMS a jogos eletrônicos que utilizam sensores de detecção de movimento para capturar as ações cinéticas do usuário. Sendo assim, o usuário executa movimentos específicos, em um jogo eletrônico, para quantificar seus sinais motores e monitorar seu estado de saúde.

Para a validação da abordagem, foi realizado um estudo analítico de caso-controle para detectar indivíduos diagnosticados com a Doença de Parkinson (Parkinson), utilizando sensores de captura de movimento em jogos eletrônicos. Buscou-se avaliar as possibilidades de aquisição de dados de saúde, com base nas características de Cinemática Angular do Movimento Humano. Estes dados foram aplicados em uma Máquina de Vetor de Suporte (SVM) para sua classificação. Como resultado, foi obtida uma taxa de identificação com acurácia de 86,67\% e falsos positivos de 6,67\%. Dessa forma, concluiu-se que a abordagem proposta permite desenvolver jogos eletrônicos que servem como uma forma não invasiva para monitorar dados motores.



