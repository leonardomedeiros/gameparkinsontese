Os Sistemas de Monitoramento da Saúde (SMS) possibilitam aos médicos obterem informações sobre o estado de saúde de seus pacientes. Além disso, a identificação dos sintomas das doenças podem auxiliar no diagnóstico precoce e prevenir a ocorrência de situações críticas.

Para acompanhar e avaliar a saúde motora de um paciente, é necessário realizar uma avaliação motora por meio de movimentos específicos. Isto dificulta a concepção de um SMS de dados motores não-invasivo e engajados na rotina diária dos pacientes. A abordagem apresentada nesta tese, utiliza os jogos eletrônicos como fator motivacional para o fornecimento dos dados motores. Durante o jogo, o usuário é induzido a executar movimentos relevantes, de modo que um sensor de movimento possa adquiri-los e quantificá-los. Este ambiente lúdico, de jogo eletrônico, abstrai o usuário do contexto de tratamento da saúde e incentiva a execução dos movimentos de um maneira mais natural do que a imposta por um exame clínico.

Para avaliar esta abordagem, foi desenvolvido um jogo com a arquitetura proposta para identificar sintomas motores relacionadas com a Doença de Parkinson. Num estudo de caso-controle, foram avaliados 
os movimentos angulares dos braços ​​para quantificar as habilidades motoras desses grupos. Os dados coletados foram processados ​​e aplicados a uma Máquina de Vetor de Suporte (SVM) para classificar a ocorrência do sintoma da bradicinesia do Parkinson. Obteve-se uma classificação com uma acurácia de 86,67\% e falsos positivos de 6,67\%. Além disso, em uma experimento para avaliação da aceitação dos usuários, 90\% ficaram motivados com o jogo desenvolvido e afirmaram que integrariam o SMS em sua rotina diária. Estes resultados demonstram que a abordagem de monitoramento baseado em jogos, apresentada nesta tese, tem potencial para ser um SMS para monitoramentos dos sintomas motores.







