Os Sistemas de Monitoramento da Saúde (SMS) podem melhorar a qualidade de vida dos usuários ao fornecer remotamente informações sobre o estado de saúde, permitindo identificar precocemente situações potencialmente críticas. No entanto, para monitorar a saúde motora é executar movimentos que possibilitem avaliar a motricidade do usuário. Por esse motivo, a concepção de um SMS de dados motores não invasivo ainda é um grande desafio multidisciplinar. Estes sistemas, apesar do avanço na tecnologia, ainda são invasivos e estereotipados, o que dificulta sua disseminação. Portanto, o uso destes sistemas não tem sido incorporado na rotina dos usuários, inviabilizando o monitoramento dos sintomas motores. 

Neste trabalho propõe-se utilizar jogos eletrônicos para induzir a execução de movimentos específicos dos usuários e abstrair o monitoramento de dados de saúde, tornando-o integrado à rotina diária. A abordagem proposta permite integrar a arquitetura do SMS a jogos eletrônicos que utilizam sensores de detecção de movimento para capturar as ações cinéticas do usuário. Desta maneira, será induzido a executar uma avaliação motora no contexto de um jogo eletrônico que permita quantificar seus sinais motores e consequentemente monitorar seu estado de saúde.

Para a avaliação da abordagem foi realizado um estudo analítico de caso-controle para detectar indivíduos diagnosticados com a Doença de Parkinson (Parkinson), utilizando sensores de captura de movimento em jogos eletrônicos. Buscou-se avaliar as possibilidades de aquisição de dados de saúde, com base nas características de Cinemática Angular do Movimento Humano. Estes dados foram aplicados em uma Máquina de Vetor de Suporte (SVM) para sua classificação. Como resultado, obteve-se uma classificação dos dados com acurácia de 86,67\% e taxa de falsos positivos de 6,67\%. Dessa forma, concluiu-se que a arquitetura proposta permite o desenvolvimento de jogos eletrônicos que induzem a execução de movimentos os quais permitem avaliar a motricidade dos usuários de uma forma não invasiva.



