\chapter{Estado Atual do Trabalho}\label{chapter:conclusoes_futuros}
Nos experimentos realizados, conseguimos demonstrar a importância do acompanhamento dos sinais motores, integrados à rotina diária do paciente do ponto de vista do profissional de saúde Hipótese \textit{H1}. Identificou-se nessa pesquisa a importância de acompanhar a amplitude do movimento e a sua respectiva velocidade angular para acompanhamento da saúde motora.

Os estudos de aprendizagem de máquina com os dados motores adquiridos por meio de sensores de movimento usados em jogos eletrônicos, identificou a viabilidade do desenvolvimento de jogos para o monitoramento. Pois, obtivemos uma taxa de acurácia de 86,67\% e falsos positivos de 6,67\% .

A análise~\ac{gqm} forneceu indícios de que a abordagem~\ac{jogue-me} apresentada nessa Proposta permite o monitoramento de dados de forma não invasiva, e factível de integrar a solução a rotina diária dos usuários. Entretanto, o tempo utilizado para jogar foi insuficiente para aplicar as técnicas de processamento dos dados apresentados nesta abordagem, pois os jogadores tiveram bastante liberdade de movimento e poucos efetuaram os movimentos de abdução e adução do braço. Caso, os mesmos indivíduos participassem de um tempo maior no jogo, consequentemente eles poderiam efetuar o movimento e seria possível adquirir esses dados. Para chegarmos a resultados semelhantes aos apresentados na Seção~\ref{sec:resultado_svm} em um espaço de tempo menor, é necessário desenvolver um novo jogo com as ações específicas de realização do movimento de adução e abdução do braço, além de realizar uma nova coleta de dados.

Ao final desta tese, concluímos que a abordagem de ~ac\sms} baseada em jogos para acompanhar os sinais do~\ac{dp}. Conseguiu, motivar o usuário a fornecer sinais motores, permitindo o acompanhamento do tratamento a partir dos dados biomecânicos fornecidos. Nesta tese, foi possível monitorar o sintoma da bradicinesia do ~\ac{dp} usando os movimentos de adução e abdução. No qual, foi calculada a amplitude e respectiva velocidade angular deste movimento para avaliar o sintoma de bradicinesia. Este método é normalmente utilizado por neurologistas para distinguir os movimentos normais e anormais. Logo, conseguimos atingir o principal objetivo da tese, ao permitir que os pacientes do~\ac{dp} pudessem ser monitorados de maneira não-invasiva e no conforto de seus lares.

Como trabalho futuro, pretendemos identificar o progresso do~\ac{dp} de acordo com as fases da doença~\cite{protpar010}.