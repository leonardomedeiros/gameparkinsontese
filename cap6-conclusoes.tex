\chapter{Conclusões e Trabalhos Futuros}\label{chapter:conclusoes_futuros}
Nos experimentos realizados, conseguimos demonstrar a importância de monitorar os sinais motores, de forma integrada à rotina diária do paciente. De acordo com as entrevistas com profissionais de saúde, identificamos a importância de acompanhar a amplitude do movimento e a velocidade angular para monitorar os sinais de doenças motoras como o~\ac{dp}.

Atualmente, a análise da biomecânica do movimento humano faz parte da avaliação clínica para diagnóstico e tratamento do~\ac{dp}. Durante as consultas, os médicos solicitam aos pacientes levantar e baixar os braços para identificar a amplitude do movimento, a velocidade angular e a habilidade motora. Dessa maneira, conseguem avaliar clinicamente o progresso do sintoma da bradicinesia. Em entrevistas, os médicos informaram sobre a necessidade de quantificar esses movimentos para um melhor acompanhamento do tratamento. Por este motivo, foram aplicados os conceitos da cinemática do movimento angular para quantificar a bradicinesia, utilizando o movimento de adução e abdução dos braços.

Para quantificar o movimento da adução e abdução do braço, utilizou-se como referência as articulações do quadril, do ombro e do punho. Através de um estudo caso-controle com 30 sujeitos (15 com~\ac{dp} e 15 indivíduos saudáveis), pudemos avaliar os movimentos desses indivíduos e identificamos consideráveis diferenças na execução dos movimentos. Os dados obtidos foram aplicados em um estudo de aprendizagem de máquina, no qual obtivemos uma taxa de acurácia de 86,67\% e falsos positivos de 6,67\% . O propósito desta classificação foi avaliar a acurácia da identificação de um sinal do~\ac{dp}, a partir de um sensor de aquisição de movimento usado em jogos eletrônicos. Dessa maneira, pudemos desenvolver um jogo eletrônico  capaz de adquirir sinais biomecânicos, processá-los e identificar características de sinais do~\ac{dp} a partir do movimento angular. A presente abordagem pode ser aplicada a outras doenças motoras; no entanto, testamos somente com indivíduos com~\ac{dp} e grupo controle.

A avaliação dos usuários por~\ac{gqm} forneceu suporte ao monitoramento de sinais motores de forma não invasiva, o que facilita sua integração à rotina diária dos usuários. Assim, a abordagem de um~\ac{sms} dos sinais motores, como interface de entrada de dados (\ac{jogue-me}), conseguiu motivar o usuário  a fornecer sinais motores e permitir o acompanhamento do tratamento a partir dos dados biomecânicos fornecidos. Logo, conseguimos atingir o principal objetivo da tese, ao permitir que os pacientes do~\ac{dp} pudessem ser monitorados de maneira não-invasiva e no conforto de seus lares.

Como trabalhos futuros pretendemos identificar o progresso do~\ac{dp} de acordo com as fases da doença~\cite{protpar010}, monitorar um indivíduo em diferentes momentos do dia e realizar estudos de multiclassificação perante outras doenças motoras.


%In this work we presented a game-based approach to monitor the motor symptoms of Parkinson's Disease. The game was applied to motivate the user to be monitored, allowing the collection of biomechanical data measurements. We monitored PD symptoms through the abduction and adduction arm movements, calculating the amplitude and its respective angular velocity to assess bradykinesia motor symptom. This method is commonly applied by neurologists at consultation to distinguish normal and abnormal movements. In this way, we enable the monitoring of PD patients at home, increasing the frequency of symptoms measurement.

%To evaluate our approach, we performed an experimental study with 30 research subjects divided in PD and Control group. We used SVM to identify the occurrence of PD's bradykinesia motor symptom and had a classification \textit{Precision} of 92.31\%. Moreover, 90,00\% of the patients considered our approach non-invasive and easy to integrate into their routine. 