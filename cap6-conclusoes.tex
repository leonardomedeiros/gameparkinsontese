\chapter{Conclusões e Trabalhos Futuros}\label{chapter:conclusoes_futuros}
Neste capítulo são apresentadas as conclusões sobre o trabalho apresentado. Na Seção~\ref{sec:contribuicoes} são apresentadas as contribuições deste trabalho juntamente com os resultados alcançados e uma breve discussão sobre os mesmos. Na Seção~\ref{sec:limitacoes} são apresentadas as limitações encontradas durante o processo de avaliação do trabalho. Já na Seção~\ref{sec:trab_futuros} são propostos possíveis trabalhos futuros. Por fim, as considerações finais são descritas na Seção~\ref{sec:cons_finais}.



Nos experimentos realizados, conseguimos demonstrar a importância do acompanhamento dos sinais motores, integrados à rotina diária do paciente do ponto de vista do profissional de saúde. Identificou-se nessa pesquisa a importância de acompanhar a amplitude do movimento e a sua respectiva velocidade angular para acompanhamento da saúde motora.

Os estudos de aprendizagem de máquina com os dados motores adquiridos por meio de sensores de movimento usados em jogos eletrônicos, identificou a viabilidade do desenvolvimento de jogos para o monitoramento, ois, obtivemos uma taxa de identificação de verdadeiros positivos de 86,67\% e falsos positivos de 6,67\% .

A Hipótese \textit{H3}, foi validada por meio de uma análise~\ac{gqm} aplicada a possíveis usuários finais da abordagem. Essa pesquisa forneceu indícios de que a abordagem \textit{GAHME} apresentada nessa Proposta permite o monitoramento de dados de forma não invasiva, e factível de integrar a solução a rotina diária dos usuários. Entretanto, o tempo utilizado para jogar foi insuficiente para aplicar as técnicas de processamento dos dados apresentados nesta abordagem, pois os jogadores tiveram bastante liberdade de movimento e poucos efetuaram os movimentos de abdução e adução do braço. Caso, os mesmos indivíduos participassem de um tempo maior no jogo, consequentemente eles poderiam efetuar o movimento e seria possível adquirir esses dados. Para chegarmos a resultados semelhantes aos apresentados na Seção~\ref{sec:resultado_svm} em um espaço de tempo menor, é necessário desenvolver um novo jogo com as ações específicas de realização do movimento de adução e abdução do braço, além de realizar uma nova coleta 
de dados.

\subsubsection{Publicações}
Foram publicados três artigos, em conferências internacionais, relacionados à tese: 
  \begin{itemize}
   \item \textit{Abstract}: \textit{Monitoring Parkinson related Gait Disorders with Eigengaits}, no, \textit{XX World Congress on Parkinson's Disease and Related Disorders} (2013)~\cite{lmmeigengaits2013};
   \item \textit{Full Paper}: \textit{A Game-Based Approach to Monitor Parkinson’s Disease: The bradykinesia symptom classification}, no, \textit{International Symposium on Computer-Based Medical Systems} (CBMS 2016)~\cite{lmmcbmsgame2016};
   \item \textit{Full Paper}: \textit{A Gait Analysis Approach to Track Parkinson’s Disease Evolution Using Principal Component Analysis}, no, \textit{International Symposium on Computer-Based Medical Systems} (CBMS 2016)~\cite{lmmcbmsgait2016}.
  \end{itemize}


\subsection{Contribuições}\label{sec:contribuicoes}



Este capítulo apresentou a estrutura geral do arcabouço, detalhando suas partes e operações. Foram apresentadas as operações disponibilizadas pelo \emph{webservice} e as classes, e seus respectivos métodos, que fazem parte do arcabouço. Por fim, foram descritas a forma de utilização do arcabouço na construção de um jogo e a forma de extensão do arcabouço, para adicionar novas funcionalidades.


Este capítulo mostrou como exemplos de jogos simples podem explorar superficialmente o potencial do arcabouço. Jogos mais complexos podem contemplar ações para prover um suporte mais completo ao monitoramento da saúde. Um projeto de jogo mais elaborado, aliado a uma melhor apresentação gráfica, entre outros itens de projetos de jogos que mantém a atenção do jogador, podem criar uma experiência de monitoramento contínua e prolongada.

Durante as coletas, não houve a necessidade de nenhuma interrupção da coleta devido a problemas de saúde ou qualquer outra eventualidade.

\subsection{Limitações Do Trabalho}\label{sec:limitacoes}

O método utilizado para diferenciar os movimentos executados de indivíduos diagnosticados com \ac{dp} ante os indivíduos de grupo controle, foi uma técnica estatística de aprendizagem denominada de~\ac{svm}. Esse estudo não teve a pretensão de estabelecer um diagnóstico da \ac{dp}, ou até mesmo provar que os movimentos utilizados pelos participantes da pesquisa servem para um diagnóstico. Contudo, este trabalho demonstrou que as diferenças nos movimentos, entre essas duas classes, permitem a identificação do sinal da bradicinesia, e estes podem ser capturadas por um sensor de movimento usado em jogos eletrônicos. 

Apesar do questionário ter avaliado a opinião dos jogadores quanto ao jogo apresentado, pode-se generalizar que as opiniões são válidas para outros jogos usando a abordagem~\ac{jogue-me}. Deve-se levar em consideração, também, que as métricas obtidas nessa pesquisa foram extraídas de um jogo na fase de protótipo. Caso ele fosse aperfeiçoado é possível que sua aceitabilidade seria ainda maior. Por esse motivo, o resultado obtido com a pesquisa~\ac{gqm} foi positivo, e considera-se que é viável desenvolver um jogo com o objetivo de monitorar dados motores, de forma não invasiva, e integrada à rotina diária dos usuários.


\subsection{Trabalhos Futuros}\label{sec:trab_futuros}

Alguns trabalhos podem ser desenvolvidos a partir do que fora apresentado nesta tese.

O progresso do~\ac{dp} é avaliado por escalas, como foi explanado na Seção~\ref{section:escalas_avaliacao}, pois permitem monitorar a progressão da doença e a eficácia do tratamento medicamentoso~\cite{updrs87,goul05}. Desta forma, é importante coletar uma amostra maior de pacientes com~\ac{dp}, agrupa-los de acordo com o estágio da doença~\cite{goul05}, e aplicar técnicas de multi-classificação de dados~\cite{multisvm2011} para identificar o seu estágio.

Em decorrência das ``Flutuações Motoras''~\footnote{Referente a respostas motoras flutuantes ao tratamento medicamentoso, com encurtamento da duração de seu efeito (fenômeno do \textit{wearing off}) e interrupção súbita de sua ação.}~\cite{protpar010},  é necessário comparar o sinal da bradicinesia em diferentes momentos do dia, para verificar a eficácia do tratamento medicamentoso~\cite{protpar010}.

Nos estudos realizados com os sinais adquiridos pelo MS-Kinnect, foi possível identificar a amplitude como apresentamos no estudo do movimento de abdução e adução do braço (Seção~\ref{sec:resultado_obtido_svm}). Todavia, a captura de um movimento mais sutil como um tremor é um desafio. Por esse motivo, foi desenvolvido e testado um jogo para celular que pudesse adquiri o sinal de tremor(Seção~\ref{sec:tremor}). Contudo, como o tremor do~\ac{dp} é de repouso~\cite{protpar010}, não foi possível quantificar o sinal. No entanto, ao analisarmos os vídeos dos pacientes com~\ac{dp}, identificamos, que ao levantar um dos membros, alguns indivíduos iniciava o sinal de tremor no membro parado. Desta maneira, pode ser possível quantificar o sinal de tremor na análise do membro em repouso. No entanto, devido ao ruído existente na aquisição do sinal pelo MS-Kinnect~\cite{kinnect2013} pode inviabilizar a quantificação deste sinal.

\subsection{Considerações Finais}\label{sec:cons_finais}

Nos experimentos realizados, conseguimos demonstrar a importância de monitorar os sinais motores, de forma integrada à rotina diária do paciente. De acordo com as entrevistas com profissionais de saúde, identificamos a importância de acompanhar a amplitude do movimento e a velocidade angular para monitorar os sinais de doenças motoras como o~\ac{dp}.

Atualmente, a análise da biomecânica do movimento humano faz parte da avaliação clínica para diagnóstico e tratamento do~\ac{dp}. Durante as consultas, os médicos solicitam aos pacientes levantar e baixar os braços para identificar a amplitude do movimento, a velocidade angular e a habilidade motora. Dessa maneira, conseguem avaliar clinicamente o progresso do sintoma da bradicinesia. Em entrevistas, os médicos informaram sobre a necessidade de quantificar esses movimentos para um melhor acompanhamento do tratamento. Por este motivo, foram aplicados os conceitos da cinemática do movimento angular para quantificar a bradicinesia, utilizando o movimento de adução e abdução dos braços.

Para quantificar o movimento da adução e abdução do braço, utilizou-se como referência as articulações do quadril, do ombro e do punho. Através de um estudo caso-controle com 30 sujeitos (15 com~\ac{dp} e 15 indivíduos saudáveis), pudemos avaliar os movimentos desses indivíduos e identificamos consideráveis diferenças na execução dos movimentos. Os dados obtidos foram aplicados em um estudo de aprendizagem de máquina, no qual obtivemos uma taxa de acurácia de 86,67\% e falsos positivos de 6,67\% . O propósito desta classificação foi avaliar a acurácia da identificação de um sinal do~\ac{dp}, a partir de um sensor de aquisição de movimento usado em jogos eletrônicos. Dessa maneira, pudemos desenvolver um jogo eletrônico  capaz de adquirir sinais biomecânicos, processá-los e identificar características de sinais do~\ac{dp} a partir do movimento angular. A presente abordagem pode ser aplicada a outras doenças motoras; no entanto, testamos somente com indivíduos com~\ac{dp} e grupo controle.

A avaliação dos usuários por~\ac{gqm} forneceu suporte ao monitoramento de sinais motores de forma não invasiva, o que facilita sua integração à rotina diária dos usuários. Assim, a abordagem de um~\ac{sms} dos sinais motores, como interface de entrada de dados (\ac{jogue-me}), conseguiu motivar o usuário  a fornecer sinais motores e permitir o acompanhamento do tratamento a partir dos dados biomecânicos fornecidos. Logo, conseguimos atingir o principal objetivo da tese, ao permitir que os pacientes do~\ac{dp} pudessem ser monitorados de maneira não-invasiva e no conforto de seus lares.


