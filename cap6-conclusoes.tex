\chapter{Conclusões e Trabalhos Futuros}\label{chapter:conclusoes_futuros}
Nos experimentos realizados, conseguimos demonstrar a importância do acompanhamento dos sinais motores, integrados à rotina diária do paciente do ponto de vista do profissional de saúde. Identificou-se nessa pesquisa: a importância de acompanhar a amplitude do movimento e a velocidade angular para acompanhamento da saúde motora.

A aquisição de dados motores usando sensores de movimento usados em jogos eletrônicos,,Os estudos de aprendizagem de máquina com os dados motores adquiridos por meio de sensores de movimento usados em jogos eletrônicos, identificou a viabilidade do desenvolvimento de jogos para o monitoramento. Pois, obtivemos uma taxa de acurácia de 86,67\% e falsos positivos de 6,67\% .

The biomechanical analysis of human movement is part of the diagnosis and treatment process for PD, where the patients are asked to lift their arms, one after the other, at the highest amplitude and velocity they are able to, in order to check the bradykinesia progress. The movement of abduction and adduction are joint actions which involve wrist, shoulder, and hip joints. To identify the movement of adduction and abduction of the arms, it is necessary to use a reference joint. Here, we focus on the wrist joint because its signal has higher amplitude when compared to the other joints. We used the \textit{peaks and valleys} technique to identify the beginning and the end of the movement cycle (Fig.~\ref{fig:signalamplitudepeakvaley}). After the cycle identification, we extract a window length with the cycle movement and transform the MS-Kinect data into angles. Thus, we calculate the angular motion of the movement and consequently the angle displacement of the adduction and abduction movements.

A análise~\ac{gqm} forneceu indícios de que a abordagem~\ac{jogue-me} apresentada nessa Proposta permite o monitoramento de dados de forma não invasiva, e factível de integrar a solução a rotina diária dos usuários. Entretanto, o tempo utilizado para jogar foi insuficiente para aplicar as técnicas de processamento dos dados apresentados nesta abordagem, pois os jogadores tiveram bastante liberdade de movimento e poucos efetuaram os movimentos de abdução e adução do braço. Caso, os mesmos indivíduos participassem de um tempo maior no jogo, consequentemente eles poderiam efetuar o movimento e seria possível adquirir esses dados. Para chegarmos a resultados semelhantes aos apresentados na Seção~\ref{sec:resultado_svm} em um espaço de tempo menor, é necessário desenvolver um novo jogo com as ações específicas de realização do movimento de adução e abdução do braço, além de realizar uma nova coleta de dados.

Ao final desta tese, concluímos que a abordagem de ~ac\sms} baseada em jogos para acompanhar os sinais do~\ac{dp}. Conseguiu, motivar o usuário a fornecer sinais motores, permitindo o acompanhamento do tratamento a partir dos dados biomecânicos fornecidos. Nesta tese, foi possível monitorar o sintoma da bradicinesia do ~\ac{dp} usando os movimentos de adução e abdução. No qual, foi calculada a amplitude e respectiva velocidade angular deste movimento para avaliar o sintoma de bradicinesia. Este método é normalmente utilizado por neurologistas para distinguir os movimentos normais e anormais. Logo, conseguimos atingir o principal objetivo da tese, ao permitir que os pacientes do~\ac{dp} pudessem ser monitorados de maneira não-invasiva e no conforto de seus lares.

Como trabalhos futuros pretendemos: identificar o progresso do~\ac{dp} de acordo com as fases da doença~\cite{protpar010}, monitorar um indivíduo em diferentes momentos do dia e realizar estudos de multiclassificação com outras doenças motoras.


%In this work we presented a game-based approach to monitor the motor symptoms of Parkinson's Disease. The game was applied to motivate the user to be monitored, allowing the collection of biomechanical data measurements. We monitored PD symptoms through the abduction and adduction arm movements, calculating the amplitude and its respective angular velocity to assess bradykinesia motor symptom. This method is commonly applied by neurologists at consultation to distinguish normal and abnormal movements. In this way, we enable the monitoring of PD patients at home, increasing the frequency of symptoms measurement.

%To evaluate our approach, we performed an experimental study with 30 research subjects divided in PD and Control group. We used SVM to identify the occurrence of PD's bradykinesia motor symptom and had a classification \textit{Precision} of 92.31\%. Moreover, 90,00\% of the patients considered our approach non-invasive and easy to integrate into their routine. 