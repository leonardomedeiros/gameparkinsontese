\chapter{Saúde Conectada}\label{chapter:saude_conectada}

Atualmente os dispositivos médicos têm sido domesticados e integrados na rotina diária de seus usuários, principalmente para a parcela populacional preocupada em preservar a saúde e o bem estar, com a prática: de atividade física, educação alimentar junto a profissionais de saúde que realizam o acompanhamento nutricional e físico. Um outro grupo de usuários são pacientes de doênças crônicas, o quais possuem a necessidade de acompanhar o estado de saúde com o objetivo de prolongar a qualidade de vida e possíveis agudizações do estado de saúde~\cite{ibmconnec2011}.  Logo, a partir do ano de 2009, houve um avanço da indústria dispositivos médicos voltados para este público com um faturamento de 900 bilhões de dólares somente nos Estados Unidos~\cite{ibmconnec2011}. 

Por meio As soluções computacionais voltadas para a saúde e bem-estar, 

%In academia there has also been significant recent interest in adopting and advancing IT for effective healthcare. The 2009 US National Research Council report on “Computational Technology for Effective Health Care” suggests an overarching research grand challenge of developing “patient-centered cognitive support.”~\cite{smartwellbeing}

Os dispositivos de saúde conectada possuem dois segmentos de consumidores~\cite{ibmconnec2011}:

\begin{description}
 \item [Usuários Que Preservam O Bem-Estar]: Este usuário, busca monitorar seus dados de saúde na busca de atingir objetivos específicos como: redução de peso, melhorar condicionamento físico e acompanhar estado de saúde.
 \item [Usuários Crônicos Que Necessitam de Monitoramento]: Este usuário, possui um estado de saúde debilitado e frequentemente necessita de um cuidador. Os dispositivos médicos são prescritos pelos médicos e acompanham o estado de saúde do paciente, permitindo um melhor gerenciamento das doenças.
\end{description}


~\cite{smarthealth2013}