\chapter{Introdu\c{c}\~{a}o} \label{chapter:intro}

A população mundial envelhece progressivamente e, segundo estudos da~\ac{oms}~\cite{ageing2011}, muito em breve teremos mais idosos do que crianças. Ao considerar que a população idosa possui maior prevalência de doenças crônicas~\cite{prevcronica2009}, surge a necessidade de monitorar o estado da saúde dessa população. Portanto, diante do crescimento da quantidade de pacientes crônicos, da iminente redução do número de leitos hospitalares disponíveis e, insuficiência de profissionais especializados para atender esta demanda~\cite{healthmonitoring2013}: faz-se necessário transpor serviços de monitoramento dos pacientes crônicos, dos leitos hospitalares para acompanhamento domiciliar~\cite{homecarebrazil2011}. 

Como objeto de estudo escolhemos o~\ac{dp} por ser uma doença neurodegenerativa crônica, progressiva e com causa desconhecida. É uma doença mais comum em idosos, no entanto, existem casos precoces em indivíduos antes dos 40 anos ou até mesmo abaixo dos 21~\cite{menezes2003}. A incidência da doença é estimada entre 100 a 200 casos por 100.000 habitantes e, com o envelhecimento da população, o contingente de pessoas diagnosticadas com~\ac{dp} tende a aumentar nos próximos anos. Após os 10 anos de tratamento, a doença leva o indivíduo a irreversíveis debilidades: motoras e cognitivas. Logo, a abordagem de monitorar os sinais em diferentes momentos do dia, permite um melhor gerenciamento da doença e, por consequência, melhora a qualidade de vida destes indivíduos.

Na averiguação desta demanda, a computação aplicada à saúde busca prover ~\ac{sms}~\cite{healthmonitoring2013,berg03,bardram2010,Ballegaard:2008:HEL:1357054.1357336,aarhus_negotiating_2010}. Estes~\ac{sms} permitem ao médico acompanhar a distância o estado de saúde de seus pacientes colaborativamente~\cite{healthmonitoring2013}. Atualmente, os~\ac{sms} realizam: um tratamento preventivo e pró-ativo do estado de saúde~\cite{bardram2010}; suporte à reabilitação do paciente~\cite{sacbespoke2014}; auxilia o paciente a atingir uma melhor qualidade de vida~\cite{sacsvmhms2014}. Referente ao monitoramento dos sinais motores, os~\ac{sms} quantificam estes sinais e conseguem: quantificar as habilidades motoras~\cite{manumeterjbhi2014,patel_monitoring_2009}, efetuar análise da marcha \cite{robotgait2014} e identificar sinais de bradicinesia~\cite{ambulatoryparkinson2010}. Contudo, o maior desafio dessas abordagens é a aceitação do usuário e consequente inserção na rotina diária~\cite{alemdar2015}.
Na averiguação desta demanda, a computação aplicada à saúde busca prover ~\ac{sms}~\cite{healthmonitoring2013,berg03,bardram2010,Ballegaard:2008:HEL:1357054.1357336,aarhus_negotiating_2010}. Estes~\ac{sms} permitem ao médico acompanhar a distância o estado de saúde de seus pacientes colaborativamente~\cite{healthmonitoring2013}. Atualmente, os~\ac{sms} realizam: um tratamento preventivo e pró-ativo do estado de saúde~\cite{bardram2010}; suporte à reabilitação do paciente~\cite{sacbespoke2014}; auxilia o paciente a atingir uma melhor qualidade de vida~\cite{sacsvmhms2014}. Referente ao monitoramento dos sinais motores, os~\ac{sms} quantificam estes sinais e conseguem: quantificar as habilidades motoras~\cite{manumeterjbhi2014,patel_monitoring_2009}, efetuar análise da marcha \cite{robotgait2014} e identificar sinais de bradicinesia~\cite{ambulatoryparkinson2010}. Contudo, o maior desafio dessas abordagens é a aceitação do usuário e consequente inserção na rotina diária~\cite{alemdar2015}.

Na busca por motivar os usuários, identificamos que os jogos eletrônicos se encontram presentes na rotina diária de 27\% da população americana acima dos 50 anos~\cite{esa2015}. Com base nesse número, percebemos um público de jogadores idosos beneficiáveis por: uma plataforma de monitoramento de dados de saúde embutida num jogo eletrônico. Aliado a esse estudo, identificamos jogos voltados para o público idoso aplicados à melhora do estado de saúde como: jogo para a persuasão da prática de atividades físicas~\cite{brox11}, jogo para a melhora das capacidades físicas e cognitivas~\cite{arntzen2011}. 

Nesta tese, é apresentada uma abordagem de monitoramento dos sinais motores do~\ac{dp} baseada em jogos eletrônicos. Este trabalho foi desenvolvido com o objetivo de: adquirir e quantificar sinais motores por meio de sensores de movimento; utilizar dispositivos eletrônicos comerciais para reduzir custos e facilitar a replicação dos experimentos; prover informações da saúde motora dos pacientes com~\ac{dp} por meio de um sistema de monitoramento não invasivo e lúdico. Desta maneira, possibilitamos que o paciente de~\ac{dp} se desvencilhe do contexto do tratamento da doença e forneça dados sobre seu estado de saúde colaborativamente. 

A validação ocorreu em duas etapas: na primeira, avaliamos a capacidade de monitoramento dos indivíduos com~\ac{dp} em um estudo analítico de caso-controle; na segunda, avaliamos a possibilidade de adequar este monitoramento na rotina diária dos pacientes. O estudo analítico de caso-controle foi realizado com 30 sujeitos de pesquisa (15 grupo controle e 15 diagnosticados com~\ac{dp}). Como resultado, identificamos e quantificamos o sintoma da bradicinesia~\cite{protpar010}, que consiste na lentidão da execução dos movimentos. Para distinguir os grupos (caso-controle e diagnosticados com~\ac{dp}) utilizamos uma~\ac{svm} para classificação dos dados~\cite{datamining2005}, no qual obtemos uma acurácia de 86,66\%. Avaliamos a adequação da abordagem de monitoramento dos sinais motores usando jogos eletrônicos, aplicando a técnica~\ac{gqm}~\cite{van1999goal}. Como resultado, 90,00\% dos avaliados consideraram a abordagem não-invasiva e incorporável à rotina diária. 



\section{Problemática}\label{section:problematica}
%Para a identificação do problema foi realizado uma Revisão Bibliográfica sobre os temas IEEE~\cite{ieee}, ACM~\cite{acm}, PubMed~\cite{pubmed}, Scielo~\cite{scielo}.

%A Revisão da Literatura objetivou encontrar problemas em aberto nos~\ac{sms}. Inicialmente Foi realizado, um estudo complementar nas diretrizes médicas para o suporte científico nesta área. Essa etapa teve como objetivo inicial identificar problemas nesses trabalhos que pudessem ser solucionados por esta tese.

Atualmente, os~\ac{sms} dos sinais motores utilizam sensores vestíveis (\textit{wearable}), que comumente, são incorporados à roupa ou ao corpo do usuário (Figuras: \ref{fig:quantif-parkinson},\ref{fig:patel-shimmer}). De acordo com a perspectiva do usuário~\cite{aarhus_negotiating_2010}, são considerados invasivos e estereotipados~\cite{aarhus_negotiating_2010}. No entanto, o gerenciamento medicamentoso do~\ac{dp}, necessita de um cuidado acurado e diário~\cite{quantitativeparkinson2011}. Por esta razão, buscou-se nesta tese: \textbf{desenvolver um~\ac{sms} dos sinais motores do~\ac{dp}, integrada à rotina diária dos usuários, em um ambiente lúdico como um jogo eletrônico}. Nossa perspectiva é que este ambiente venha a fornecer informações sobre os sinais motores de uma maneira: colaborativa, não invasiva e no conforto do lar. 

\begin{figure}
 \centering
 \includegraphics[scale=0.3]{./img/patel-shimmer.png}
 % matrixargseg.png: 296x162 pixel, 100dpi, 7.52x4.11 cm, bb=0 0 213 117
 %\caption{Estágio desenvolvimento de jogos ~\cite{fullerton2008game}}
\caption[Disposição dos Sensores de Movimento (SHIMMER) no corpo no trabalho de Patel]{\textit{Disposição dos Sensores de Movimento (SHIMMER) no corpo no trabalho de Patel ~\cite{patel_monitoring_2009}}}
%  \caption{Estágio desenvolvimento de jogos}
 \label{fig:patel-shimmer}
\end{figure}

\begin{figure}
 \centering
 \includegraphics[scale=0.3]{./img/quantif-parkinson.png}
 % matrixargseg.png: 296x162 pixel, 100dpi, 7.52x4.11 cm, bb=0 0 213 117
 %\caption{Estágio desenvolvimento de jogos ~\cite{fullerton2008game}}
\caption[\textit{G-Link Wireless Accelerometer} - Instrumento usado no trabalho de LeMoyne para quantificar o tremor da Doença de Parkinson]{\textit{G-Link Wireless Accelerometer} - Instrumento usado no trabalho de LeMoyne~\cite{LeMoyne2009} para quantificar o tremor da Doença de Parkinson} 
%  \caption{Estágio desenvolvimento de jogos}
 \label{fig:quantif-parkinson}
\end{figure}


Em trabalhos relacionados, LeMoyne~\cite{lemoyne2010} quantificou os sinais de tremores do~\ac{dp} usando um \textit{smartphones}(Figura \ref{fig:iphone-tremor}). LeMoyne~\cite{lemoyne2010} considerou que: os \textit{smartphones} estão presentes na rotina dos pacientes e que estes iriam mensurar seus tremores em diferentes momentos do dia. No entanto, o principal problema em mensurar o tremor usando \textit{smartphones} é que o tremor do~\ac{dp} é de repouso~\cite{jankovic2008}. Logo, os pacientes reduzem drasticamente o sinal o que impacta diretamente na coleta dos dados. Deve-se considerar também que LeMoyne~\cite{lemoyne2010} não realizou avaliações com pacientes ou estudo de caso-controle.



\begin{figure}
 \centering
 \includegraphics[scale=0.3]{./img/moyne-iphone.png}
 % matrixargseg.png: 296x162 pixel, 100dpi, 7.52x4.11 cm, bb=0 0 213 117
 %\caption{Estágio desenvolvimento de jogos ~\cite{fullerton2008game}}
\caption[Aplicação para \textit{smartphone} com a finalidade de identificar sinais de tremor]{Aplicação para iPhone com a finalidade de identificar sinais de tremor ~\cite{lemoyne2010}}
%  \caption{Estágio desenvolvimento de jogos}
 \label{fig:iphone-tremor}
\end{figure}










%No entanto, em nossos estudos como o uso dos acelerômetros de celulares junto aos pacientes de~\ac{dp} identificamos que o tremor de repouso presente no~\ac{dp}(Seção~\ref{sec:tremor}) era reduzido drasticamente por que estes pacientes entravam em modo de ação~\cite{jankovic2008}.





%Então, nesta tese, propomos monitorar os pacientes de~\ac{dp} de uma forma: não invasiva, por intermédio de sensores de movimento, motivando  utilizando um jogos eletrônico desenvolvido neste trabalho para \textbf{motivar} a aquisição dos dados e fazer com que os usuários abstraiam que estejam em sessões de monitoramento dos dados de saúde. Nosso objetivo foi prover um mecanismo de monitoramento durante um período lúdico e descontraído como um jogo eletrônico.









\section{Objetivos}
\subsection{Objetivo Principal}
Neste tese, tem-se como objetivo: a concepção de uma abordagem computacional não invasiva para o monitoramento de sinais motores. Pretende-se usar jogos eletrônicos como forma de: \textbf{motivar} o monitoramento dos sinais motores e abstrair do contexto de \textbf{tratamento da saúde}.


%Partindo da necessidade de monitorar dados motores de uma forma não invasiva por intermédio de sensores de movimento, pretendemos usar jogos eletrônicos como forma de \textbf{motivar} e abstrair o monitoramento de dados de saúde de forma \textbf{não invasiva} e longe do \textbf{contexto de tratamento de saúde}.
Para atingir esse objetivo, pretende-se desenvolver um~\ac{sms} capaz de: armazenar, processar sinais biomecânicos e identificar a presença do sinal de bradicinesia do~\ac{dp}. 



%O objetivo principal desta pesquisa é prover um mecanismo de monitoramento de sinais biomecânicos dos sinais motores por meio dos jogos eletrônicos habilitados a adquirir estes sinais.
%Em um primeiro momento, serão verificados junto a profissionais de saúde a relevância desse trabalho e se este poderia auxiliar no monitoramento dos sinais parkinsonianos.

%O monitoramento dos sinais motores proporciona os seguintes benefícios: auxiliar o médico na tomada de decisão em relaOs benefícios proporcionados pelo monitoramento dos sinais motores, está relacionado ao auxílio no suporte ao médico quanto comprometimento motor dos pacientes com~\ac{dp}. Então, neste trabalho, desenvolvemos um mecanismo de de captura de sinais motores embutidos num jogo eletrônico, o qual permite: monitorar e quantificar os sinais motores de uma forma lúdica e não-invasiva.

Para atingir o objetivo principal da pesquisa subdividimos em etapas:
	\begin{description}
	\item[ETAPA 1] Qual o benefício de acompanhar diariamente os sinais motores do paciente do ponto de vista do profissional da saúde?
	\item[ETAPA 2] Como melhor adquirir e quantificar sinais motores utilizando sensores de movimento para monitorar os sinais do~\ac{dp}?
	\item[ETAPA 3] Avaliar a perspectiva dos usuários quanto: a abordagem de capturar os sinais motores de uma maneira não-invasiva e aplicável à rotina diária.
	\end{description}


%Para atingir estes objetivos, foram elencadas e testadas as seguintes hipóteses:
	%\begin{description}
	%\item[H1] O acompanhamento de sinais motores, integrados à rotina diária do paciente traz benefícios ao tratamento e qualidade de vida do mesmo do ponto de vista do profissional da saúde.
	%\item[H2] É possível capturar dados motores por meio de sensores de movimento utilizados em jogos eletrônicos. Esses dados auxiliam no acompanhamento de doenças com comprometimento motor.
	%\item[H3] É possível desenvolver um jogo que tenha mecanismos de captura de dados motores embutidos, e que permita monitorar e quantificar esses dados de maneira não-invasiva.
	%k\end{description}
	
% \subsection{Objetivos Específicos}
% \begin{enumerate}
% 		\item Identificar a importância de realizar monitoramento de dados de saúde em diferentes momentos do dia junto a uma comunidade de profissionais de saúde.
% 		%\item Usar algoritmos de classificação em base de dados de saúde já consolidadas, para desenvolver e testar novas abordagens de monitoramento, além de aumentar o número de casos pesquisados.
% 		\item Identificar viabilidade técnica para mensurar dados de saúde por meio de sensores de movimento utilizados em jogos eletrônicos.
% 		\item Definir e implementar a abordagem apresentada nesta tese capaz de: adquirir, processar, classificar e transformar em informações de saúde motora os sinais biomecânicos de usuários obtidos a partir do jogo eletrônico desenvolvido.
% 		%\item Definir processo de desenvolvimento de jogos que consiga realizar o monitoramento de dados de saúde não-invasiva ao usuário;
% 		\item Realizar experimentos no sentido de validar o trabalho.
% 		\item Avaliar junto aos pacientes de~\ac{dp} a aceitabilidade da proposta.
% \end{enumerate}

\section{Metodologia}
A metodologia de pesquisa deste trabalho possui aspectos qualitativos e quantitativos. Referente ao aspecto qualitativo, buscou-se identificar a importância desta tese junto à comunidade de especialistas da área de saúde. Nos aspectos quantitativos, demonstramos que a abordagem definida consegue diferenciar indivíduos diagnosticados com~\ac{dp} perante indivíduos sem o diagnóstico estabelecido, por meio de dados capturados por sensores de movimento usados em jogos. Ao final do trabalho avaliamos a aceitabilidade da proposta na perspectiva do usuário utilizando uma análise~\ac{gqm}~\cite{van1999goal}. %Logo, o desfecho da pesquisa se fez com resultados qualitativos e quantitativos.

\subsection{Etapas da Pesquisa}
\begin{enumerate}

\item{Realizar revisão bibliográfica e coleta de requisitos junto a profissionais de saúde.}

\item{Definir a abordagem de um~\ac{jogue-me}, baseada em captura de dados motores através de sensores de movimento utilizando jogos eletrônicos e processamento dos dados para transformar os dados adquiridos por meio dos sensores em informações de saúde.}
%Pode ser também GAME-HMS-E (Game with a Health Monitoring System Embedded)

\item{Analisar a perspectiva dos profissionais de saúde em relação ao acompanhamento dos sinais motores dos pacientes com~\ac{dp}. Os profissionais foram indagados sobre a melhora na tomada de decisão quanto ao acompanhamento dos sinais, e verificar se os parâmetros motores como velocidade angular, amplitude do movimento dos braços são importantes para realizar o acompanhamento dos sinais parkinsonianos. Procurou-se encontrar junto ao profissional de saúde, a importância do monitoramento de dados de saúde e os benefícios trazidos por este através de uma abordagem de pesquisa qualitativa. Com esta pesquisa foi possível validar a \textbf{ETAPA 1} da pesquisa que consiste de verificar a importância do acompanhamento de sinais motores integrados à rotina diária do paciente.}

\item{Validar o uso de sensores para classificação dos dados através de: aquisição de sinais motores utilizando sensores de movimento de usados em jogos eletrônicos. Esses dados foram aplicados em um classificador ~\ac{svm} para distinguir indivíduos do grupo controle com indivíduos diagnosticados com~\ac{dp};
O resultado dessa pesquisa demonstrou que foi possível adquirir dados motores utilizando esta abordagem e, consequentemente validou a \textbf{ETAPA 2} do trabalho.

\item{Definir a arquitetura de software que viabilizou tecnicamente a abordagem~\ac{jogue-me}. Nesta etapa, definimos um arcabouço de software para encapsular o desenvolvimento de jogos com essa abordagem.}

\item{Validar a solução~\ac{jogue-me} do ponto de vista computacional. A solução foi validada através da implementação da arquitetura e desenvolvimento de jogos com base na arquitetura. Com jogos, demonstrou-se ser possível realizar monitoramento de dados motores de forma não invasiva, ou seja, sem os jogadores perceberem que estão fornecendo dados de saúde.}

\item{Verificar junto ao público alvo (Pacientes portadores de~\ac{dp}) os requisitos de usabilidade, adequação à rotina diária, segurança física e se a proposta é considerada invasiva na perspectiva do paciente. Com esta avaliação, buscamos validar a \textbf{ETAPA 3} da pesquisa.}

    
%\item{Definir um conjunto de atividades que permitam o desenvolvimento de jogos voltados para o monitoramento de dados de saúde (\ac{jogue-me}). Desta maneira, pretende-se com esse trabalho disseminar o conhecimento adquirido a partir de um conjunto de passos que tornem exequível o desenvolvimento de um \ac{jogue-me}.}

\end{enumerate}

%\subsection{Desenho da Pesquisa} \label{sec:desenho_pesquisa}
%Para uma melhor compreensão da pesquisa temos o Desenho da Pesquisa a ser realizada na Figura \ref{figure:desenho_pesquisa} e a descrição de cada um dos passos.
%
%A Problemática desse trabalho, já foi descrita na Seção \ref{section:problematica}. A revisão da literatura consistiu de uma descrição sobre a doença estudada como estudo de caso (Doença de Parkinson) que está descrito no Capítulo \ref{section:doenca_parkinson}, bem como da possibilidade de integrar o monitoramento da doença de Parkinson por intermédio de jogos eletrônicos.
%
%\begin{figure}[!htp]
    %\centering
    %\includegraphics[width=.7\textwidth]{./img/metodologia-pesquisa.png}
    %\caption{Desenho da Pesquisa}
    %\label{figure:desenho_pesquisa}
%\end{figure}

%\subsection{Pesquisa Qualitativa}
%Essa pesquisa tem como objetivo identificar a importância de realizar o monitoramento de dados de saúde por intermédio jogos eletrônicos. Em um primeiro momento, serão verificados junto a comunidade médica a relevância desse trabalho e se este poderia auxiliar no monitoramento dos sinais parkinsonianos.
%
%Os pesquisadores que adotam a abordagem da pesquisa qualitativa a fazem por sua flexibilidade, sem regras rígidas, aplicáveis a uma ampla gama de casos e formalizações pré-definidas, possibilitando a construção de modelos abrangentes \cite{Gom00}.
%Nesse estudo a pesquisa qualitativa está orientada a um estudo de caso, partindo da análise das expressões e comportamento das pessoas. Na busca por entender os fenômenos sociais complexos, o estudo de caso permite uma investigação significativa nas mudanças desses processos\cite{Yin05}.
%
%\subsubsection{Entrevista Semi-Estruturada Com Especialistas do Domínio (3)}\label{section:entrevista_semi-estruturada}
%
%O objetivo deste procedimento foi entender como é feito o acompanhamento dos sinais da doença de parkinson juntamente aos profissionais de saúde (neurologistas que prescrevem a dosagem medicamentosa e fisioterapeutas que fazem o acompanhamento motor do paciente ao longo de seu tratamento). Os mesmos foram indagados se poderia haver melhora na tomada de decisão caso eles pudessem acompanhar o surgimento dos sinais em diferentes momentos do dia por intermédio de um monitoramento frequente dos sinais. Procurou-se encontrar dentro do contexto de estudo, a importância do monitoramento de dados de saúde e os benefícios trazidos por este.
%
%As entrevistas foram realizadas de maneira presencial, onde primeiramente fez-se perguntas não estruturadas e havendo uma maior estruturação no decorrer da entrevista com a preocupação de que seja evitada a referência do entrevistador seja sobre os pontos de vista do entrevistado conforme prega o método científico \cite{FLI04}. Durante a pesquisa foi utilizado recursos de gravação, de modo a facilitar a transcrição dos dados coletados.
%
%\subsection{Descoberta de Conhecimento usando Base de Dados de Saúde}
%Para esse estudo, foram utilizados dados do \textit{PhysioBank} ~\cite{physionet}. O \textit{PhysioBank} armazena sinais fisiológicos e dados relacionados a comunidade de pesquisa biomédica, incluindo sinais vitais, motores de indivíduos saudáveis e de pacientes com uma variedade de sinais como distúrbios neurológicos ou envelhecimento.
%
%A análise dos dados da \textit{Parkinson Disease Database} ~\cite{physionet}, incluiu 50 pacientes de Parkinson e 50 indivíduos que não possuíam desenvolvido a doença (segundo as informações fornecidas pela própria base de dados). Foram calculados, normalizados e escalonados em 80 \textit{frames} um total de 120 ciclos de marcha (Seção \ref{section:analise_marcha}) que foram usados para efetuar a técnica de aplicar a técnica \textit{eigengaits} ~\cite{medeiros2013} que é uma adaptação da Análise dos Componentes Principais que será explicada na Seção \ref{section:analise_marcha_pca}.
%
%\subsection{Estudo Analítico Caso Controle com \textit{Support Vector Machine} (SVM)}
%Esta etapa da pesquisa foi pautada pelo protocolo de pesquisa submetido à avaliação do Comitê de Ética da UFCG, somente após a aprovação deste (\textbf{CAAE: 14408213.9.1001.5182}) é que os dados foram coletados.
%Os resultados que se desejou alcançar com a pesquisa foi o descobrimento de mecanismos para a identificação e classificação de duas classes de indivíduos os  diagnosticados com a doença de parkinson ante indivíduos sem o diagnóstico estabelecido. Durante a pesquisa também analisou-se o sensor de movimento MS-Kinnect ~\cite{kinnect2013} para avaliar a possibilidades de aquisição de dados de saúde baseada na Cinemática Linear do Movimento Humano.
%
%O propósito dessa classificação é explorar a possibilidade de obter dados de saúde de forma contínua e não invasiva a partir de um sensor de captura de movimento usado em jogos eletrônicos (Ms-Kinnect).
%Os dados adquiridos foram processados extraindo características do movimento angular e postos em uma máquina de de Aprendizagem do tipo \textit{Suppport Vector Machine} para realizar a classificação das duas classes de dados.

\section{Trabalhos Relacionados}\label{section:jogos_saude}

Devido ao estilo de vida mais sedentário e do aumento da população obesa as pesquisas para a promoção da atividade física tem se tornado tópico de interesse para a comunidade científica~\cite{maitland2009,bartolome11,Mandryk2014}. Estudos demostram que uma atividade física regular traz benefícios: físicos, cognitivos e emocionais~\cite{Mandryk2014}. Com o surgimento dos jogos comerciais como o \textit{Wii Sports} da Nintendo em 2006 que aumentou a prática de atividade física de jogadores considerados sedentários~\cite{wiigraves2008} é que pesquisadores da área de jogos para saúde buscaram apoiar essa prática com desenvolvimento de aplicações que motivassem a prática de atividade física~\cite{stacey2011}. Atualmente, os dispositivos de sensores de movimento permitem desenvolver jogos que promovem a saúde e bem estar de forma promissora. Por esse motivo, houve um aumente significativo de jogos comerciais e, com o propósito de promover a saúde e o bem-estar~\cite{Papastergiou:2009:EPC:1570538.1570707}.


Os jogos pervasivos móveis motivam a atividade física de forma mais direta, jogos como: \textit{Transe}, \textit{Feeding Yoshi}, e  \textit{Nokia Wellness Diary and Sports Tracker} promovem a saúde com a prática de atividade física~\cite{Suhonen:2008:SFE:1457199.1457204} e, melhoram as condições de saúde por serem: divertidos, imersivos e engajados. Para o público idoso, foram desenvolvidos diferentes jogos para a melhora da saúde como: sistemas para reabilitação motora dos idosos~\cite{brox11} e, também para pacientes com~\ac{dp}~\cite{atkinson2010,synnott_wiipd_2012,sacbespoke2014}. 

Atkinson e Narasimhan~\cite{atkinson2010} desenvolveram um jogo que utiliza um sensor de toque para quantificar a habilidade motora do paciente com~\ac{dp}. Teoricamente, esta abordagem auxilia no diagnóstico do~\ac{dp}. No entanto, não foi realizado nenhum estudo com os paciente para avaliar sua eficácia. Synnott \textit{et al.}~\cite{synnott_wiipd_2012} desenvolveu um sistema de gerenciamento medicamentoso e um jogo utilizando um sensor de captura de movimentos para identificar o sinal de tremor do~\ac{dp}. No entanto, o tremor do~\ac{dp} é de repouso~\cite{national2006parkinson}}. Logo, quando o usuário está concentrado, entra no estado de ação e reduz drasticamente o tremor.


Papastergiou \textit{et al.}~\cite{Papastergiou:2009:EPC:1570538.1570707} identificou efeitos positivos para a reabilitação através do uso do jogo \textit{Wii Sports} e um potencial mecanismo de prevenção e reeducação motora com o uso do \textit{Wii Fit}. Porém, esses jogos possuem suas limitações e não são substitutos dos esportes reais. Ainda assim, o autor salienta que um ambiente mais controlado e que permite a execução de atividades físicas inibe a ocorrência de situações de risco como um movimento brusco e que venha causar um dano físico maior. Baseado nessas observações, esse trabalho primou por demonstrar as dificuldades e efeitos positivos em combinar os jogos sérios de esportes e saúde com as tecnologias de sensores, para a personalização e adaptação dos jogos. Paraskevopoulos \textit{et al.}~\cite{sacbespoke2014} propõe um conjunto de diretrizes para o desenvolvimento de jogos com o objetivo de: acompanhar o tratamento fisioterápico dos pacientes com~\ac{dp} e, suporte à reabilitação destes.

Sinclair \textit{et al.}~\cite{Sinclair:2009:UVB:1515604.1515617} considera que os jogos comerciais para prática de exercício físico ou \textit{exergames} não devem ser usados apenas como um motivador para a prática, mas também podem ser usados para monitorar sinais vitais como batimento cardíaco e reconhecer atividades via acelerômetros. Arntzen~\cite{arntzen2011} se preocupou com os aspectos cognitivos e físicos da aprendizagem baseada em jogos para idosos~\cite{arntzen2011}, defendendo que é necessário identificar quais habilidades cognitivas e físicas que precisam ser desenvolvidas. Além de considerar a limitação do idoso em relação aos movimentos bruscos no intuito de evitar lesões.

Nesta tese, propomos monitorar os pacientes de~\ac{dp} de uma forma: não invasiva, por intermédio de sensores de movimento, usamos jogos eletrônicos para motivar a aquisição dos dados e, fazer com que os usuários abstraiam do tratamento. Logo, nosso principal objetivo foi: prover um mecanismo de monitoramento dos sinais motores, em um ambiente lúdico e descontraído como um jogo eletrônico.


\section{Contribuições}
Atualmente, os jogos são aplicados para melhora da saúde em diferentes contextos. No entanto, nenhum dos trabalhos relacionados pretendem identificar sinais para monitorar o estado de saúde. Logo, este trabalho visa desenvolver um ambiente de jogo que motive  a execução de movimentos específicos, com o propósito de quantificar os sinais motores dos usuários.

No entanto, alinhar a jogabilidade e a capacidade de monitoramento dos sinais de saúde não é trivial. Pois deve ser levado em consideração o uso dos dispositivos e definir quais movimentos ou ações permitem a identificação dos sinais motores. Por este motivo, a proposta de um~\ac{sms} dos sinais motores usando jogos necessita de um acompanhamento de um profissional de saúde, para supervisionar e auxiliar nas definições dos movimentos e ações do usuário. 

%Posteriormente, na posse dessas ações, deverá ser testada a execução dessas atividades e sua aquisição para uma possível classificação dos dados conforme proposto nesta tese.
%os trabalhos já existentes~\cite{Ballegaard:2008:HEL:1357054.1357336,patel_monitoring_2009,visionbased2009,bachlin_parkinsons_2009,albanese2012}.
%De posse dos movimentos e da captura dos dados será feito um levantamento de um \textit{game design} que permita executar os movimentos em  um ambiente lúdico e divertido como um jogo para entretenimento ~\cite{sweetser2005-gameflow}.

Como possível cenário de uso para a pesquisa, supondo que um paciente de uma doença crônica como o~\ac{dp} faz uso de medicamento antiparkinsoniano e, possui um jogo de monitoramento de sinais do~\ac{dp} em sua residência. Caso, o mesmo utilize o jogo em diferentes momentos do dia, os sinais podem ser identificados e quantificados sem a presença de um profissional de saúde. Onde este, poderia visualizar a melhora ou piora do estado de saúde do seu paciente ao longo dos dias. A partir da presente abordagem, o médico de posse da informação poderia gerenciar melhor a dosagem medicamentosa e, consequentemente, prolongar a qualidade de vida do paciente~\cite{rodrigues2006}.

\section{Organização do Documento}
O restante deste documento está organizado da seguinte forma:
\begin{itemize}
	\item No Capítulo~\ref{chapter:fundamentacao} está descrita a fundamentação teórica relacionada ao trabalho.
	\item No Capítulo~\ref{chapter:abordagem_gahme} está definida a abordagem \ac{jogue-me} de monitoramento de dados motores não invasivo usando jogos eletrônicos.
	\item No Capítulo~\ref{chapter:arquitetura_captura} é demonstrada uma implementação da abordagem.
	%\item No Capítulo~\ref{chap:processo_desenvolvimento}, é apresentado um processo de desenvolvimento de jogos eletrônicos para monitoramento de dados de saúde.
	\item No Capítulo~\ref{chap:avaliacao} são apresentados os experimentos realizados para validar a tese.
	\item No Capítulo~\ref{chapter:conclusoes_futuros} são apresentadas as conclusões do trabalho e, propostos temas para trabalhos futuros.
\end{itemize}
