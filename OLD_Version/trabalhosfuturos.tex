\chapter{Estado Atual do Trabalho}\label{chapter:trabalhos_futuros}
Nos experimentos realizados, conseguimos demonstrar a importância do acompanhamento dos sinais motores, integrados à rotina diária do paciente do ponto de vista do profissional de saúde Hipótese \textit{H1}. Identificou-se nessa pesquisa a importância de acompanhar a amplitude do movimento e a sua respectiva velocidade angular para acompanhamento da saúde motora.

Os estudos de aprendizagem de máquina com os dados motores adquiridos por meio de sensores de movimento usados em jogos eletrônicos, identificou a viabilidade do desenvolvimento de jogos para o monitoramento, que valida a Hipótese \textit{H2}. Pois, obtivemos uma taxa de identificação de verdadeiros positivos de 80,00\% e falsos positivos de 16,67\% .

A Hipótese \textit{H3}, foi validada por meio de uma análise~\ac{gqm} aplicada a possíveis usuários finais da abordagem. Essa pesquisa forneceu indícios de que a abordagem \textit{GAHME} apresentada nessa Proposta permite o monitoramento de dados de forma não invasiva, e factível de integrar a solução a rotina diária dos usuários. Entretanto, o tempo utilizado para jogar foi insuficiente para aplicar as técnicas de processamento dos dados apresentados nesta abordagem, pois os jogadores tiveram bastante liberdade de movimento e poucos efetuaram os movimentos de abdução e adução do braço. Caso, os mesmos indivíduos participassem de um tempo maior no jogo, consequentemente eles poderiam efetuar o movimento e seria possível adquirir esses dados. Para chegarmos a resultados semelhantes aos apresentados na Seção~\ref{sec:resultado_svm} em um espaço de tempo menor, é necessário desenvolver um novo jogo com as ações específicas de realização do movimento de adução e abdução do braço, além de realizar uma nova coleta 
de dados.

Para a entrega da Tese pretendemos uma melhora nos resultados, para isso pretendemos (Tabela~\ref{table:cronograma}) realizar novos estudos com a base de dados do estudo analítico de caso-controle. Então, iremos apresentar na Tese:
\begin{enumerate}
	\item O resultado da aprendizagem obtido pela~\ac{svm} com o \textit{kernel} linear demonstrou que os dados são linearmente separáveis. Logo, para a Tese pretendemos realizar um estudo de regressão linear nessa base de dados.
	\item Realizar estudos de curva de aprendizagem na base de dados usando a~\ac{svm}.
	\item Refinar o processo de desenvolvimento de jogos para monitoramento de saúde adicionando as fases de Construção e Pós-Validação.
	%\item Nos estudos realizados com os dados do MS-Kinnect, conseguimos identificar que ele consegue capturar dados com amplitude como apresentamos o movimento de abdução e adução do braço. Todavia, a captura de um movimento mais sutil como um tremor é um desafio. Por esse motivo, foi desenvolvido e testado um jogo para celular que pudesse capturar o sintoma de tremor. Contudo, como o tremor da doença de parkinson é de repouso não obtivemos sucesso com a captura do sintoma. Ao analisarmos os vídeos dos pacientes de parkinson, notamos que ao levantar um dos membros alguns indivíduos iniciava o sintoma de tremor no membro parado. Pretendemos então, na Tese analisar esse membro parado e comparar os resultados com os indivíduos sem o diagnóstico da ~\ac{dp}. Contudo, devido ao ruído existente no sinal do MS-Kinnect a captura do tremor pode ser inviável.
	\item Analisar o motivo da ocorrência de 2 indivíduos de controle terem sido classificados como portadores da~\ac{dp}. 
\end{enumerate}

% Please remember to add \use{multirow} to your document preamble in order to suppor multirow cells
\begin{table}[h]
\center
\caption{Cronograma de Conclusão}
\label{table:cronograma}
\begin{tabular}{|r|r|r|r|r|r|}
\hline
\multirow{\textbf{Trabalhos Futuros}} & \multicolumn{4}{|c|}{\textbf{Trimestre}}                          \\ \cline{2-5} 
                                            & \textbf{1º}             & \textbf{2º} & \textbf{3º} & \textbf{4º} \\ \hline
Item 1                                      & \multicolumn{1}{|c|}{x} &             &             &             \\ \hline
Item 2                                      &                         & x           &             &             \\ \hline
Item 3                                      &                         &             & x           &             \\ \hline
Item 4                                      &                         &             & x           & x           \\ \hline
\end{tabular}
\end{table}