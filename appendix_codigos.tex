\chapter{Principais Algoritmos do Arcabou�o} \label{app:codigos}

%Um maior detalhamento da estrutura do arcabou�o pode ser visto no diagrama de classes resumido na Figura~\ref{img:classd}. As informa��es enviadas ao \textit{web service} s�o processadas pela classe \texttt{DataResource}, que s�o enviadas para o m�dulo \emph{Gerenciador de Dados} (classe \texttt{DataManager}) e s�o gravadas pelo \emph{Gerenciador do Banco de Dados} (classe \texttt{DatabaseManager}) no \emph{Banco de Dados} para an�lise posterior, ou enviadas diretamente para o \emph{Analisador de Dados} (classe \texttt{DataAnalyzer}) para an�lise. As classes \texttt{FilterModule}, \texttt{ParserModule} e \texttt{WriterModule} implementam, respectivamente, os m�dulos de filtragem, interpreta��o e escrita de dados. \texttt{MovementAnalyzer}, \texttt{TremorAnalyzer} e \texttt{RuleManager} s�o o analisador de movimentos, analisador de tremor e gerenciador de regras, respectivamente. \texttt{IWriter} � a interface que encapsula um arquivo de dados. \texttt{WriterImpl} re�ne o comportamento comum das classes que encapsulam arquivos de dados \texttt{ARFFWriter}, \texttt{CSVWriter} e \texttt{DATWriter}. Todos os elementos s�o abordados nas se��es a seguir.