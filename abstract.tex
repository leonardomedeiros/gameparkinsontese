%VERSÂO SAC 2016
%Parkinson’s disease (PD) is a degenerative neurological disorder. It causes motor symptoms such as resting tremor, bradykinesia and gait disorders. The disease’s progressive nature requires continuous monitoring of the motor symptoms to assist the neurologist in managing medication. With this purpose, Health Monitoring Systems (HMS) are used as a decentralized healthcare approach. On the other hand, most patients reject the current HMS solutions because they are invasive and stigmatizing. In this work, we present a non-invasive HMS for PD motor symptoms based on games. Because of the nature of games, the approach is able to collect data from patients without reminding them that they are under a disease’s treatment. We validated our approach with 30 research subjects divided between PD group and Control group. We used Support Vector Machine (SVM) to identify the occurrence of PD’s bradykinesia motor symptoms and reached a classification accuracy of 86.66\%\. Furthermore, 90\%\ of the patients approved our HMS considering it as non-invasive and easily integrated into their routine.

%VERSÂO ANTERIOR
Applying pervasive computing to health's context can improve users' life quality by remotely providing information regarding their health. On the other hand, developing a non-invasive health monitoring pervasive systems is a multidisciplinary challenge. These systems, despite technology advancements, are still visible and stereotyped which complicates its dissemination. So, these systems have not been applied to the users activity daily living, undermining motor symptoms monitoring.

Given the difficulties of developing such a system, this work proposes using video games to motivate and disregard health monitoring. The American industry stated that, in 2011, video game players averaged 37 years-old and 29\% are over 50 years-old. Since 2005, video games use motion detection sensors to capture users' kinetic actions. This way, this work proposes an approach based in video games to quantify motion signals and monitor health.

This work's relevance was assessed through qualitative research where a semi-structured interview with health professionals. To validate the approach, we executed a case-control analytic study to detect individuals diagnosed with Parkinson's Disease using motion capturing sensors through video games. We evaluated health data acquisition possibilities based on Human Motion Angular Kinetic characteristics. The data was applied in a Support Vector Machine (SVM) to classify the data. As a result, we had an accuracy rate of 86.67\% true positive identification and 6.67\% rate of false positive. This way, we concluded that the proposed approach allows to develop video games that serves to monitor data motion non-invasively.