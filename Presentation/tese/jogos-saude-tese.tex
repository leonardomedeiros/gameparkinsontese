%%\documentclass[t,handout]{beamer}
\documentclass{beamer}


\usepackage[utf8]{inputenc}
\usepackage[portuguese]{babel}
\usepackage[tight]{subfigure}
\usepackage{graphicx}
\usepackage{color}
\usepackage{url}
% \usepackage{listings}
%\usepackage[alf]{abntcite}

%Pacote de listagem de c�digo
\usepackage{listings}
\lstset{numbers=left, stepnumber=1, firstnumber=1,
numberstyle=\tiny, extendedchars=true, breaklines=true, frame=tb,
basicstyle=\footnotesize, stringstyle=\ttfamily,
showstringspaces=false }


\usetheme{Frankfurt} 

\usecolortheme[RGB={134,153,188}]{structure}
\setbeamertemplate{footline}[frame number]
\setbeamertemplate{navigation symbols}{}

\author[L. Medeiros]{Aluno: Leonardo Melo de Medeiros}
\date{\today}
\institute[]{Orientador: Leandro Dias da Silva\\
						 Orientador: Hyggo Oliveira de Almeida \\ 
						 Universidade Federal de Campina Grande - UFCG}
    \title{Uma Abordagem de Monitoramento dos Sinais Motores da Doença de Parkinson Baseada em Jogos Eletrônicos}
\logo{\includegraphics[width=0.2\linewidth]{img/logo.png}}
\subtitle{Defesa de Tese}

\begin{document}

\begin{frame}
  \titlepage
\end{frame}

[]
{\frame{
\frametitle{Roteiro}
\tableofcontents
}
}

\section{Introdução}
\subsection{}
\begin{frame}{Sistemas de Monitoramento de Saúde}
  \begin{block}{}
      \center \includegraphics[height=1.8 in]{img/sismonsaude.png}
  \end{block}
  \begin{block}{}  
A computação aplicada ao contexto de saúde permite monitorar remotamente o estado de saúde dos usuários. Entretanto, a concepção de um sistema não invasivo de monitoramento é um grande desafio~\cite{alemdar}.
  \end{block}
\end{frame}


\begin{frame}{Estratégias de Monitoramento da Saúde}
  \begin{block}{}
      \center \includegraphics[height=1.8 in]{img/estrategmonitorament.png}
  \end{block}
  \begin{block}{}
As tecnologias de monitoramento para serem aceitas precisam preservar a privacidade do usuário e integrar-se à sua rotina diária~\cite{aarh10}.
  \end{block}
\end{frame}


\begin{frame}{Aplicações dos SMS}  
  \begin{block}{}
  Atualmente, os~\ac{sms} permitem ao médico acompanhar à distância o estado de saúde de seus pacientes colaborativamente~\cite{healthmonitoring2013}, realizando:  
  \begin{itemize}
   \item Tratamento preventivo e pró-ativo do estado de saúde~\cite{bardram2010}; 
   \item suporte à reabilitação do paciente~\cite{sacbespoke2014};
   \item auxílio para o paciente atingir uma melhor qualidade de vida~\cite{sacsvmhms2014}. 
  \end{itemize} 
  \end{block} 
\end{frame}


\begin{frame}{SMS da Saúde Motora}  
  \begin{block}{}
  
  Referente ao monitoramento dos sinais motores, os~\ac{sms}:  
  \begin{itemize}
   \item quantificam as habilidades motoras~\cite{manumeterjbhi2014,patel_monitoring_2009};
   \item efetuam análise da marcha \cite{robotgait2014}
   \item identificam sinais de bradicinesia~\cite{ambulatoryparkinson2010}. 
  \end{itemize}   
  \end{block} 
  
  \begin{alertblock}{}
   Contudo, o maior desafio dessas abordagens é a aceitação do usuário e sua consequente inserção na rotina diária~\cite{alemdar2015}.
  \end{alertblock}
\end{frame}



\begin{frame}{Motivação para uso de jogos para monitoramento dos dados motores}
	\begin{block}{}
	\begin{itemize}[<+->]
	    \item Percentual expressivo de adultos e idosos que são usuários de jogos, e os utiliza em sua rotina diária (27\% acima dos 50 anos~\cite{esa2015});
	    \item O jogo é uma experiência autotélica, logo o usuário joga por puro prazer, sem esperar qualquer benefício por seu uso ~\cite{sweetser2005-gameflow};
	    \item As tecnologias de sensores de movimento estão presentes no contexto dos jogos eletrônicos;
	    \item Possibilita a reprodução de movimentos específicos em um ambiente controlado para a aquisição de sinais motores.
	\end{itemize}
	\end{block}
	
	\begin{block}
	A abordagem de monitorar os sinais em diferentes momentos do dia permite um melhor gerenciamento da doença e, por consequência, melhora a qualidade de vida destes indivíduos.
	\end{block}
\end{frame}



\begin{frame}{Objetivo Principal}
  \begin{block}{}<1->
Neste trabalho, tem-se como objetivo a concepção de uma abordagem computacional para o monitoramento de dados motores. Pretende-se usar jogos eletrônicos como forma de motivar e abstrair o monitoramento de dados de saúde de uma maneira não invasiva e longe do contexto de tratamento de saúde.
  \end{block}
\end{frame}

%REMOVER HIPOTESES E ADICIONAR ETAPAS
% \begin{frame}{Hipóteses do Trabalho}
% 	\begin{block}{}
% 			\begin{itemize}[<+->]
%        \item \textbf{H1} - O acompanhamento de sintomas motores, integrados à rotina diária do paciente traz benefícios ao tratamento e qualidade de vida do mesmo do ponto de vista do profissional da saúde.
%        \item \textbf{H2} - É possível capturar dados motores por meio de sensores de movimento utilizados em jogos eletrônicos. Esses dados auxiliam no acompanhamento de doenças com comprometimento motor.
% 			\item \textbf{H3} - É possível desenvolver um jogo que tenha mecanismos de captura de dados motores embutidos, e que permita monitorar e quantificar esses dados de maneira não-invasiva.
% 			\end{itemize}
% 	\end{block}
\end{frame}

%TODO:  Remover menção a base de dados de marcha
% \begin{frame}{Objetivos Específicos}
% 	\begin{block}{}
% 	\begin{itemize}[<+->]
%       \item Identificar a importância de realizar monitoramento de dados junto a profissionais de saúde;
% 			\item Usar bases de dados de saúde já consolidadas para testar abordagens de monitoramento;
% 			\item Identificar viabilidade técnica da aquisição de sintomas por sensores de movimento utilizados em jogos eletrônicos;
% 			\item Definir e implementar a arquitetura de software da abordagem;
% 			\item Realizar experimentos para validar as hipóteses.
% 	\end{itemize}
% 	\end{block}
% \end{frame}

%TODO: Foi utilzado estudo de caso ou objeto de estudo ?
\section{Estudo de Caso}
\subsection{}
\begin{frame}{Doença de Parkinson}
  \begin{block}{}
    A doença de Parkinson (DP) é uma afecção do sistema nervoso central, a qual é expressa de forma crônica e progressiva. 
      \begin{itemize}[<+->]
       \item Causada pela morte dos neurônios produtores de dopamina da substância negra ~\cite{protpar010}. 
       \item Caracterizada pelos sinais cardinais de rigidez, bradicinesia, tremor e instabilidade postural ~\cite{menezes2003}.
      \end{itemize}
  \end{block}

  
  %TODO: Reforçar da Bradicinesia
%   \begin{block}{Termos: Tremor de Repouso e Bradicinesia}<3->
%       \begin{itemize}
%        \item \textbf{Tremor de Repouso:} sintoma mais frequente e perceptível;
%        \item \textbf{Bradicinesia:} lentidão na execução do movimento;
%       \end{itemize}
%   \end{block}
\end{frame}

\begin{frame}{Estágios da Doença}
  \begin{block}{Escala Unificada de Avaliação da Doença de Parkinson (UPDRS)}
    A escala UPDRS ~\cite{updrs87} avalia tanto o nível de estrutura e função corporal quanto o nível das atividades.
      A escala contém itens referentes a:
	\begin{itemize}[<+->]
	 \item Mental, comportamento e humor;
	 \item atividades da vida diária;
	 \item exame motor;
	 \item complicações no tratamento.
	\end{itemize}
 \end{block}
\end{frame}

\begin{frame}{Escala (UPDRS)} 
    \begin{block}{Fenômeno (\textit{On/Off})}
      \center \includegraphics[height=2.4 in]{img/updr1-sel.png}
    \end{block}		
\end{frame}

\begin{frame}{Escala (UPDRS)} 
    \begin{block}{Impacto nas Atividades Diárias}
      \center \includegraphics[height=2.0 in]{img/updr2-sel.png}
    \end{block}
\end{frame}

\begin{frame}{Entrevista Semi-Estruturada com Profissionais de Saúde} 
    \begin{block}{Objetivo da Pesquisa}
    %TODO:Remover hipótese e colocar etapa
      %Validar a Hipótese \textbf{H1}: O acompanhamento de sintomas motores, integrados à rotina diária do paciente, traz benefícios ao tratamento e qualidade de vida do mesmo, do ponto de vista do profissional da saúde.
    \end{block}
		\begin{block}{Participantes}
			\begin{table}[h]
			%\caption{Perfil dos Participantes}
			%\label{table:perfil_analise_participantes}
			\begin{tabular}{|l|l|c|c|}
			\hline
			\textbf{LEGENDA} & \textbf{PROFISSÃO}             & \multicolumn{1}{|l|}{\textbf{EXPERIÊNCIA (ANOS)}} \\ \hline
			FIS\_01          & Fisioterapeuta & 10                                                \\ \hline
			FIS\_02          & Fisioterapeuta    & 10                                                \\ \hline
			NEU\_01          & Neurologista            & 15                                                \\ \hline
			NEU\_02          & Neurologista            & 30                                                \\ \hline
			\end{tabular}
			\end{table}
    \end{block}
\end{frame} 

\begin{frame}{Resultado da Entrevista} 
    \begin{block}{}
			\begin{itemize}[<+->]
				\item Com base na rastreabilidade dos fragmentos da entrevista, pode-se concluir que existiram muitas ocorrências sobre: 
					\begin{enumerate}
						\item tremor;
						\item bradicinesia;
						%TODO remover análise de marcha
						%\item análise da marcha.
					\end{enumerate}
				\item Para o acompanhamento e monitoramento da doença, os profissionais de saúde citaram a importância de calcular:
					\begin{enumerate}
						\item amplitude dos movimentos de abdução e adução dos braços;
						\item a velocidade angular desse movimento.
					\end{enumerate}
			\end{itemize}
    \end{block}
\end{frame} 


\section{Desenv. de Jogos}
\subsection{}

%TODO Remover o processo de desenvolvimento
% \begin{frame}{Processo de Desenvolvimento de um Jogo para Monitoramento de Dados de Saúde}
% \begin{block}{}
% Este trabalho pretende usar um ambiente de jogo para a execução de movimentos específicos com o propósito de quantificar os sinais motores dos usuários e consequentemente realizar o monitoramento. 
% \end{block}
% \begin{block}{}
% O ambiente será denominado de \textbf{GAHME}.
% \end{block}
% \end{frame}
% 
% \begin{frame}{Fases de Um Processo de Desenvolvimento de Jogos}
%   \begin{block}{}
%       \center \includegraphics[height=2.8 in]{img/stages-game-development.png}
% 			%\caption{Modificações no Jogo ao Longo das Fases de Desenvolvimento~\cite{fullerton2008game}}
%   \end{block}
% \end{frame}
% 
% \begin{frame}{Fase de Conceito de um GAHME}
%   %\begin{block}{}
%       \center \includegraphics[height=2.8 in]{img/gahme-fase-conceito.png}
% 			%\caption{Modificações no Jogo ao Longo das Fases de Desenvolvimento~\cite{fullerton2008game}}
%   %\end{block}
% \end{frame}
% 
% \begin{frame}{Fase de Pré-Produção de um GAHME}
%   %\begin{block}{}
%       \center \includegraphics[height=2.1 in]{img/gahme-fase-pre-producao.png}
% 			%\caption{Modificações no Jogo ao Longo das Fases de Desenvolvimento~\cite{fullerton2008game}}
%   %\end{block}
% \end{frame}

%TODO Definir o JOGUE-ME
%TODO Alterar para Jogueme
% \section{GAHME}
% \subsection{}
% \begin{frame}{GAHME – \textit{Health Monitor Environment}}
%       \top \includegraphics[height=2.6 in]{img/systemoverview.png}
% \end{frame}
% 
%TODO verificar se os requisitos batem  
% \begin{frame}{GAHME – \textit{Health Monitor Environment}}
% 	\begin{block}{}
% 		\begin{itemize}[<+->]
% 			\item	\textbf{REQ-GAHME-01} - Pontuação e Taxa de Acerto;
% 			\item	\textbf{REQ-GAHME-02} - Progresso e Evolução do Jogador e dos Desafios;
% 			\item	\textbf{REQ-GAHME-03} - Estado de Fluxo;
% 			\item	\textbf{REQ-GAHME-04} - Preocupação com Integridade Física do Jogador;
% 			\item	\textbf{REQ-GAHME-05} - Captura e Armazenamento de Sinais Motores;
% 			\item	\textbf{REQ-GAHME-06} - Mecanismo de Identificação de Sintomas Motores;
% 			\item	\textbf{REQ-GAHME-07} - Mecanismo de Visualização dos Parâmetros Motores do Usuário.
% 		\end{itemize}
% 	\end{block}
% \end{frame}
% 
% \begin{frame}{Arquitetura GAHME}
%   \begin{block}{}
%       \center \includegraphics[height=2.0 in]{img/arquitetura.png}
%   \end{block}
% \end{frame}
% 
% \begin{frame}{Cinemetria}
%   \begin{block}{}
%       \begin{itemize}
% 				\item A Cinemetria consiste de um conjunto de métodos para medir os valores dos parâmetros cinemáticos;
% 				\item Movimento Cinético é o estudo das forças e momentos que resultam no movimento do corpo e seus segmentos, incluindo a mensuração da Força Vertical de Reação ao Solo (FVRS) e análise cinética.
% 			\end{itemize}
%   \end{block}
\end{frame}

\begin{frame}{Sensor de Captura de Movimentos}
  \begin{block}{\textit{Ms-Kinnect 1.0} e os Pontos Selecionados}
      \center \includegraphics[height=2.6 in]{img/articulacoes-sel.png}
			%\caption{Modificações no Jogo ao Longo das Fases de Desenvolvimento~\cite{fullerton2008game}}
  \end{block}
\end{frame}

\begin{frame}{Movimento Angular}
  \begin{block}{Movimento de Abdução e Adução do Braço ~\cite{mcginnis2013biomechanics}}
      \center \includegraphics[width=4cm]{img/abducao-angulo.png}
  \end{block}
\end{frame}

\begin{frame}{Mecanismo de Identificação de Sintomas Motores}
      \center \includegraphics[height=3 in]{img/exsinalposicaoypunhodireito.png}
\end{frame}

\begin{frame}{Técnicas de Picos e Vales do Sinal}
      \center \includegraphics[height=3 in]{img/deteccaopicosvales.png}
			%\caption{Modificações no Jogo ao Longo das Fases de Desenvolvimento~\cite{fullerton2008game}}
\end{frame}

\begin{frame}{Extração de Início e Fim dos Ciclos de Movimento}
      \center \includegraphics[height=3 in]{img/remocaoruidociclo.png}
			%\caption{Modificações no Jogo ao Longo das Fases de Desenvolvimento~\cite{fullerton2008game}}
\end{frame}




\begin{frame}{Cálculo da Velocidade Angular do Movimento de Abdução e Adução}
      \center \includegraphics[height=2.8 in]{img/amplitude-braco.png}
			%\caption{Modificações no Jogo ao Longo das Fases de Desenvolvimento~\cite{fullerton2008game}}
\end{frame}

\begin{frame}{Filtragem de Dados: Remoção de Ciclos Incompletos}
   \begin{block}{}
   
   \begin{columns}[c]
     \begin{column}{0.5\linewidth}
				\includegraphics[width=5.5cm]{img/ciclonormalizadoescalonado.png}
     \end{column}

     \begin{column}{0.55\linewidth}
				\includegraphics[width=5.5cm]{img/ciclomovimentoremovido.png}
    \end{column}
\end{columns}
\end{block}
\end{frame}

\begin{frame}{Classificador de Dados}
\begin{block}{}
			O classificador de dados, é utilizado na abordagem para identificar de possíveis usuários com problemas motores. Desta forma, o classificador irá auxiliar o profissional de saúde no acompanhamento de seus pacientes.
\end{block}
\end{frame}

\begin{frame}
   \frametitle{Máquina de Vetor de Suporte (SVM)}
   \begin{block}{}
   
   \begin{columns}[c]
     \begin{column}{0.5\linewidth}
			 \begin{itemize}
				\item Uma SVM utiliza vetores de separação através de uma técnica de hiperplano de separação ótima.

				\item Formalmente, classificadores que separam os dados por meio de um hiperplano utilizam um discriminante linear~\ref{eq:hiperplano}.
			\end{itemize}

			\begin{equation}
			f(x)=w^Tx+b
			\label{eq:hiperplano}
			\end{equation}.
     \end{column}

     \begin{column}{0.5\linewidth}
				\includegraphics[width=4cm]{img/svmhyperplane.png}
    \end{column}
\end{columns}
\end{block}
\end{frame}

\begin{frame}{Visualização do Vetor Médio do Movimento de Abdução e Adução do Braço}
      \center \includegraphics[height=2.6 in]{img/vetormedioaducao.png}
\end{frame}

\begin{frame}{Ciclos de Movimento de Abdução e Adução do Braço}
      \center \includegraphics[height=3 in]{img/ciclosmovimentokinnect-2.png}
\end{frame}


\begin{frame}{Visualização das Características do Movimento}
  \begin{block}{}
      \center \includegraphics[height=2.8 in]{img/caracteristicas-tabela.png}
  \end{block}
\end{frame}

\section{Experimentos}
\subsection{}
%\subsection{Estudo Analítico de Caso-Controle}
\begin{frame}{Estudo Analítico de Caso-Controle: Identificação da Bradicinesia} 
    \begin{block}{Objetivo da Pesquisa}<1->
      Validar a Hipótese \textbf{H2}: É possível capturar dados motores por meio de sensores de movimento utilizados em jogos eletrônicos. Esses dados auxiliam no companhamento de doenças  com comprometimento motor.

    \end{block}
		\begin{block}{Coleta de Dados}<2->
			\begin{itemize}
				\item Protocolo de pesquisa submetido aprovado junto ao CEP da UFCG (\textbf{CAAE: 14408213.9.1001.5182})
				\item Coleta realizada nas instituições:
					\begin{enumerate}
						\item Hospital Universitário da UFAL;
						\item Fundação Pestalozzi;
						\item Clínica Fisioterapia do CESMAC;
						\item Instituto Federal de Alagoas;
						\item Universidade Federal de Campina Grande.
					\end{enumerate}				
			\end{itemize}
    \end{block}
\end{frame}

\begin{frame}{Amostra} 
    \begin{block}{}
			\begin{itemize}
				\item A técnica de amostragem utilizada para seleção, foi por conveniência, composta por:
				\begin{enumerate}
					\item 15 indivíduos portadores de DP;
					\item 12 sem o diagnostico, como grupo controle.
				\end{enumerate}
					\item No grupo de portadores de DP, foram inclusos indivíduos até o Estágio 3 (Doença bilateral leve a moderada com alguma instabilidade postural e capacidade para viver independente), segundo a UPDRS.
				\end{itemize}
    \end{block}
\end{frame}

\begin{frame}{Coleta dos Dados Utilizando o Jogo: \textit{Catch the Spheres}}
      \center \includegraphics[height=2.2 in]{img/catch-the-spheres.png}
\end{frame}

%TODO: REVER A COLETA
% \begin{frame}{Processo de Coleta de Dados}
%    \begin{block}{}   
%    \begin{columns}[c]
%      \begin{column}{0.5\linewidth}
% 				\begin{itemize}[<+->]
% 					\item Voluntário se posiciona a 2m. do sensor de movimento;
% 					\item Voluntário inicia o jogo;
% 					\item Voluntário abduz e aduz o braço esquerdo, e depois o direito 10 vezes o mais rápido possível;
% 					\item Voluntário fecha o jogo.
% 				\end{itemize}
%      \end{column}
% 
%      \begin{column}{0.55\linewidth}
% 				\includegraphics[width=5.5cm]{img/capturaducaokinnect.png}
%     \end{column}
% \end{columns}
% \end{block}
% \end{frame}



\begin{frame}{Processo de Coleta de Dados}
  %\begin{block}{}
      \center \includegraphics[height=2.6 in]{img/capturaducaokinnect.png}
\end{frame}

\begin{frame}{Características Extraídas do Movimento}
	\begin{block}{}
		\begin{itemize}[<+->]
			\item	Ciclo de movimento, normalizado e escalonado em 20 amostras;
			\item	amplitude do movimento de abdução do braço esquerdo e direito;
			\item	velocidade angular de abdução dos braços esquerdo e direito;
			\item velocidade angular de adução do braço esquerdo e direito.
		\end{itemize}
	\end{block}
\end{frame}

\begin{frame}{Classificação dos Dados}
	\begin{block}{}
		\begin{itemize}[<+->]
			\item	Com os dados coletados, realizou-se uma classificação usando SVM com núcleo linear e \textit{bias} de 0,10.
			\item	O resultado com o núcleo linear foi o mais expressivo ante o Polinomial, Radial e MLP.
		\end{itemize}
	\end{block}
\end{frame}

\begin{frame}{Matriz de Confusão}
	\begin{block}{Resultado da Matriz de Confusão do Estudo Analítico Caso-Controle Usando SVM Linear}
\begin{table}[!htbp]
		\label{table:resultadomatrizconfusaosvm}
		\centering
		\begin{tabular}{l|c|c|}
		\cline{2-3}
		\multicolumn{1}{c}{}                         & \multicolumn{2}{|c|}{\textit{\textbf{Classe Preditiva}}} \\ \cline{2-3} 
																								 & \textbf{Parkinson}      & \textbf{Não-Parkinson}         \\ \hline
		\multicolumn{1}{|l|}{\textbf{Parkinson}} & 12       & 3           \\ \hline
		\multicolumn{1}{|l|}{\textbf{Não Parkinson}}     & 2           & 10     \\ \hline
		\end{tabular}
\end{table}
	\end{block}
\end{frame}

\begin{frame}
   \frametitle{Métricas da Classificação}
   \begin{block}{}
   		\begin{table}[!htbp]
				\label{table:metricasmatrizconfusao}
				\centering
				\begin{tabular}{|l|r|}
				\hline
				\multicolumn{2}{|l|}{\textbf{Métricas}} \\ \hline
				\textbf{TpRate}                    & 80,00$\%$\                 \\ \hline
				\textbf{FpRate}                    & 16,67$\%$\                \\ \hline
				\textbf{Precision}                 & 85,71$\%$\                \\ \hline
				\textbf{Accuracy}                  & 81,48$\%$\                \\ \hline
				\textbf{F-Measure}                 & 82,76$\%$\                \\ \hline
				\end{tabular}
				\end{table}
	\end{block}
     \begin{block}{}
				\begin{description}
				\item [\textit{TpRate}]: taxa de acerto obtido;
				\item [\textit{FpRate}]: taxa de falso alarme obtido;
				\item [\textit{Precision}]: taxa de acerto de uma instância em determinada classe;
				\item [\textit{Accuracy}]: taxa de acerto de todo o classificador;
				\item [\textit{F-Measure}]: análise de classificador binário que mede a acurácia.
				\end{description}
    \end{block}
\end{frame}

\begin{frame}{Limitações do Método}
	\begin{block}{}
	A aprendizagem estatística deste trabalho é apenas um indicador, o qual necessita da interpretação do profissional de saúde.
	\end{block}
  \begin{block}{}
      \center \includegraphics[height=1 in]{img/visualizacaomedico.png}
  \end{block}
\end{frame}

\begin{frame}{Outros Experimentos}
	\begin{block}{Uso de Jogo em \textit{Smartphone} Para Detecção de Tremor}
	\center \includegraphics[height=1 in]{img/pinball_world.png}
	\end{block}
	\begin{block}{Insucesso na Quantificação}
			\begin{itemize}[<+->]
			\item Tremor da DP é de repouso.
			\item Indivíduos quando utilizavam o jogo reduziam drasticamente o sintoma.
			\item Como os dados não seriam satisfatórios, logo a coleta tornou-se inviável.
		\end{itemize}
	\end{block}
\end{frame}






\section{GQM}
\subsection{}
\begin{frame}{Análise GQM com Usuários} 
    \begin{block}{Objetivo da Pesquisa}
      Validar a Hipótese \textbf{H3}: É possível desenvolver um jogo que tenha mecanismos de captura de dados motores embutidos, e que permita monitorar e quantificar esses dados de maneira não-invasiva.
    \end{block}
		\begin{block}{Participantes}
		Foram entrevistados um total de 24 indivíduos das seguintes instituições:
			\begin{itemize}
				\item Universidade Federal de Campina Grande;
				\item Instituto Federal de Alagoas;
				\item Clínica de Fisioterapia do CESMAC;
				\item Fundação Pestalozzi.
			\end{itemize}
    \end{block}
\end{frame} 

\begin{frame}{Questões da Pesquisa} 
    \begin{block}{}
			\begin{enumerate}
				\item Se o usuário integraria a abordagem GAHME à sua rotina diária.
				\item Se a segurança com a integridade física está de acordo com a faixa etária do usuário.
			\end{enumerate}
    \end{block}
\end{frame}

\begin{frame}{Integrar a Abordagem à Rotina Diária} 
    \begin{block}{Métrica 1.1: Escala de Diversão do Jogo}
			\center \includegraphics[height=2.6 in]{img/chart_1-.png}
    \end{block}
\end{frame}

\begin{frame}{Integrar a Abordagem à Rotina Diária} 
    \begin{block}{Métrica 1.3: Integrar o Jogo À Rotina Diária}
			\center \includegraphics[height=2.6 in]{img/chart_3-.png}
    \end{block}
\end{frame}

\begin{frame}{Integrar a Abordagem à Rotina Diária} 
    \begin{block}{}
			\center \includegraphics[height=1.4 in]{img/metricasq1.png}
    \end{block}
\end{frame}

\begin{frame}{Segurança à Integridade Física} 
    \begin{block}{Métrica 2.4: Faixa Etária do Jogo}
			\center \includegraphics[height=2.6 in]{img/chart_10-.png}
    \end{block}
\end{frame}

\begin{frame}{Segurança à Integridade Física} 
    \begin{block}{}
			\center \includegraphics[height=1.4 in]{img/metricasq2.png}
    \end{block}
\end{frame}


\section{Finalização}
\subsection{}
\begin{frame}{Publicações}
\begin{block}{}
Foram publicados três artigos, em conferências internacionais, relacionados à tese: 
  \begin{itemize}
   \item \textit{Abstract}: \textit{Monitoring Parkinson related Gait Disorders with Eigengaits}, no, \textit{XX World Congress on Parkinson's Disease and Related Disorders} (2013)~\cite{lmmeigengaits2013};
   \item \textit{Full Paper}: \textit{A Game-Based Approach to Monitor Parkinson’s Disease: The bradykinesia symptom classification}, no, \textit{International Symposium on Computer-Based Medical Systems} (CBMS 2016)~\cite{lmmcbmsgame2016};
   \item \textit{Full Paper}: \textit{A Gait Analysis Approach to Track Parkinson’s Disease Evolution Using Principal Component Analysis}, no, \textit{International Symposium on Computer-Based Medical Systems} (CBMS 2016)~\cite{lmmcbmsgait2016}.
  \end{itemize}
\end{block}
\end{frame}

\begin{frame}{Trabalhos Futuros}
\begin{enumerate}[<+->]
	\item Realizar estudo de Regressão Linear nos Dados do estudo Caso-Controle (Ms-Kinnect);
	\item Realizar estudos de curva de aprendizagem nos Dados do estudo Caso-Controle;
	\item Refinar o processo de desenvolvimento para as fases de Construção e Pós-Validação;
	\item Analisar o motivo da ocorrência de 2 indivíduos de controle que foram classificados como Parkinsonianos.
\end{enumerate}
\end{frame}
\begin{block}{}
A partir dos resultados apresentados nesta tese e extensão da mesma, alguns trabalhos futuros são propostos para contribuição científica:
  \begin{itemize}
   \item Coletar uma amostra maior de pacientes com~\ac{dp}, e agrupá-los de acordo com o estágio da doença~\cite{goul05};
   \item Usar técnicas de multi-classificação de dados~\cite{multisvm2011} para identificar o progresso do~\ac{dp} de acordo com as escalas de avaliação (ex.: UPDRS~\cite{UPDRS};
   \item Avaliar o sinal da bradicinesia em diferentes momentos do dia, para verificar a eficácia do tratamento medicamentoso~\cite{protpar010}.   
  \end{itemize}
\end{block}
\end{frame}


%\subsection{Dúvidas}
\begin{frame}
  \begin{center}
  DÚVIDAS ?
  \end{center}
\end{frame}

\bibliographystyle{authordate2}
\bibliography{biblmm} % arquivos com as entradas bib.

\end{document}
	