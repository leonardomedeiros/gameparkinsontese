Nas últimas décadas, os jogos eletrônicos tornaram-se bastante presentes no cotidiano das pessoas. Estatísticas da indústria americana de jogos constataram que em 2011 os jogadores de videogame possuíam em média 37 anos, dos quais 29$\%$  estão acima dos 50 anos. Desde 2005, os jogos eletrônicos fazem uso de sensores detecção de movimento através de vídeo para capturar as ações do usuário dentro do jogo. Por outro lado, a computação pervasiva aplicada ao contexto de saúde pode melhorar a qualidade de vida dos usuários ao fornecer informações sobre o estado de saúde por intermédio do monitoramento remoto dos dados de saúde. Porém, a concepção de um sistema pervasivo de monitoramento de dados motores, que seja não invasivo, ainda é um grande desafio multidisciplinar pois, apesar dos avanços da tecnologia de sensores, estes ainda são visíveis e estereotipados, dificultando assim sua disseminação. Portanto, o uso de sistemas pervasivos de monitoramento de dados de saúde não tem sido incorporados na rotina dos usuários, perdendo a viabilidade do monitoramento dos sintomas motores para averiguação do seu estado da saúde.

Diante da dificuldade de realizar a monitorização não invasiva aliada à rotina diária dos usuários, busca-se nesse trabalho desenvolver um mecanismo não perceptível de monitoramento de dados motores integrado à rotina diária dos usuários. Desse modo, tem-se como objetivo principal realizar a monitoração por intermédio de jogos eletrônicos, como forma de motivar e abstrair o monitoramento de dados e longe do contexto de tratamento de saúde. Na abordagem proposta neste trabalho, os usuários executam movimentos específicos, em ambientes controlados, como um jogo eletrônico, com o propósito de quantificar os sinais motores e, consequentemente, monitorar o seu estado de saúde.

A relevância desse trabalho foi averiguada por intermédio de pesquisa qualitativa onde foi realizada uma entrevista semi-estruturada junto a profissionais de saúde. Por fim, para validação da pesquisa foi realizado um estudo analítico de caso controle de indivíduos, utilizando sensores de captura de movimento em jogos eletrônicos, separados em duas classes: diagnosticados com a doença de parkinson e indivíduos sem o diagnóstico como classe de controle. Nessa etapa da pesquisa, buscou-se avaliar as possibilidades de aquisição de dados de saúde baseada nas características de Cinemática Linear do Movimento Humano. Esses dados são aplicados em uma máquina de aprendizagem Máquina de Vetor de Suporte(SVM) para realizar a classificação entre as classes definidas.


