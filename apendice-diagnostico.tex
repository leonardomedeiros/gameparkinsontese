\chapter{Critérios Estabelecidos de Diagnóstico da Doença de Parkinson} \label{apendice:diagnostico_parkinson}
Critérios estabelecidos para diagnóstico da Doença de Parkinson pela \text{National Hospital for Neurology and Neurosurgery} de Londres~\cite{national2006parkinson}. O paciente será diagnosticado com \ac{dp} se apresentar lentidão no movimento (bradicinesia) e pelo menos 3 critérios de suporte positivo.
\begin{itemize}
	\item Critérios necessários para diagnóstico do~\ac{dp}	
		\begin{itemize}
			\item bradicinesia (e pelo menos um dos seguintes sintomas abaixo);
			\item rigidez muscular;
			\item tremor de repouso (4-6 Hz) avaliado clinicamente
			\item instabilidade postural não causada por distúrbios visuais, vestibulares, cerebelares ou proprioceptivos.
		\end{itemize}
	\item Critérios negativos (excludentes) para o~\ac{dp}
		\begin{itemize}
			\item história de AVC de repetição;
			\item história de trauma craniano grave;
			\item história definida de encefalite;
			\item crises oculogíricashistória de AVC de repetição;
			\item história de trauma craniano grave;
			\item história definida de encefalite;
			\item crises oculogíricas;
			\item tratamento prévio com neurolépticos;
			\item remissão espontânea dos sintomas;
			\item quadro clínico estritamente unilateral após 3 anos;
			\item paralisia supranuclear do olhar;
			\item sinais cerebelares;
			\item sinais autonômicos precoces;
			\item demência precoce;
			\item liberação piramidal com sinal de Babinski;
			\item presença de tumor cerebral ou hidrocefalia comunicante;
			\item resposta negativa a altas doses de levodopa;
			\item exposição a metilfeniltetraperidínio.
		\end{itemize}
	\item Critérios de suporte positivo para o diagnóstico do~\ac{dp} (3 ou mais são necessários para o diagnóstico)
		\begin{itemize}
			\item início unilateral;
			\item presença de tremor de repouso;
			\item doença progressiva;
			\item persistência da assimetria dos sintomas;
			\item boa resposta a levodopa;
			\item presença de discinesias induzidas por levodopa;
			\item resposta a levodopa por 5 anos ou mais;
			\item evolução clínica de 10 anos ou mais.
		\end{itemize}
\end{itemize}