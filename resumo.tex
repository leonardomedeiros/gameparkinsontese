%A Doença de Parkinson (Parkinson) é uma doença neurodegenerativa que causa sintomas motores como: tremor de repouso, bradicinesia e anormalidade na marcha. A natureza progressiva da doença requer um monitoramento contínuo dos sintomas motores para auxiliar o neurologista no gerenciamento medicamentoso. Com este propósito, os Sistemas de Monitoramento da Saúde (SMS) são utilizados para prover esse cuidado com a saúde de forma descentralizada dos hospitais e ambientes clínicos. Todavia, a maioria dos pacientes rejeitam as soluções de SMS atuais, porque as consideram invasivas e estigmatizadas. Neste trabalho, é apresentado uma abordagem não-invasiva de SMS baseada em jogos eletrônicas voltada para o monioramento dos sintomas motores do Parkinson. Devido a natureza lúdica dos jogos eletrônicos, esta abordagem permite coletar dados dos pacientes sem relembrá-los que estão sob o tratamento da doença. Nós validamos esta abordagem junto a 30 sujeitos de pesquisa dividos em Grupo de Parkinson e Grupo Controle. A aplicação dos dados numa Máquina de Vetor Suporte (SVM) identificou a ocorrência do sintoma motor da bradicinesia no Grupo de Parkinson obtendo uma acurácia de 86,66\%. Além disto, 90\% dos pacientes aprovaram a solução de SMS considerando-o não-invasivo e de fácil integração à rotina dos usuários.



%Abstract antigo
%Com o envelhecimento da população mundial e um consequente aumento da incidência de doenças crônicas faz-se necessário o desenvolvimento de mecanismos para melhorar a eficácia no tratamento para melhorar a qualidade de vida desta população. Por esse motivo, nos últimos anos houve um aumento na pesquisa da computação aplicada no contexto do tratamento da saúde mais especificamente numa área denominada de saúde conectada, a qual estuda mecanismos de monitoramento de dados de saúde de forma remota, melhorando o suporte a decisão do profissional de saúde. Pois, por meio de mecanismos de monitoramento de dados de saúde, é possível realizar um tratamento de saúde colaborativo entre pacientes com condições crônicas e profissionais de saúde permitindo predizer diagnósticos, acompanhar a evolução da doença em relação ao tempo e consequentemente melhorar o gerenciamento no tratamento da saúde. 

%Contudo, apesar dos avanços tecnológicos, os sistemas de monitoramento de dados motores, fazem uso de dispositivos vestíveis, estereotipados e de baixa aceitabilidade por serem considerados invasivos e inadequados para integrar à rotina diária dos usuários. Estudos indicam que para que um sistema de monitoramento de saúde seja utilizado é necessário desenvolver a tecnologia de acordo com a rotina diária de seus usuários, com base nesta premissa e nas estatísticas da indústria de entretenimento que demonstram que no ano de 2011 os jogadores americanos possuíam em média 37 anos e 29$\%$ tinham mais de 50 anos.

%Este trabalho busca atender a esse público de jogadores, que já fazem uso de sensores de movimento como dispositivos de entrada dos jogos e possibilitando uma abordagem não invasiva de monitoramento de dados motores e integrada a rotina diária de seus usários. A presente proposta torna viável desenvolver jogos eletrônicos com o propósito de monitorar sinais motores dos usuários de forma lúdica e sem evocar o tratamento da saúde.  Para validar o trabalho foi desenvolvido um jogo eletrônico utilizando a abordagem proposta aplicando a Doença de Parkinson como estudo de caso, no qual por meio de um estudo analítico de caso-controle com 27 sujeitos de pesquisa (12 grupo controle e 15 diagnosticados com parkinson). Demonstramos a viabilidade da abordagem na identificação e quantificação de um sintoma bastante debilitante da doença denominado de bradicinesia que consiste na lentidão da execução dos movimentos.

O uso da computação pervasiva aplicada ao contexto de saúde pode melhorar a qualidade de vida dos usuários ao fornecer remotamente informações sobre o estado de saúde. Por outro lado, a concepção de um sistema pervasivo não invasivo de monitoramento de dados motores ainda é um grande desafio multidisciplinar. Estes sistemas, apesar do avanço na tecnologia, ainda são visíveis e estereotipados, dificultando assim sua disseminação. Portanto, o uso destes sistemas não tem sido incorporado na rotina dos usuários, inviabilizando o monitoramento dos sintomas motores.
 
Diante da dificuldade de desenvolver um sistema com as características descritas, neste trabalho, propõe-se utilizar jogos eletrônicos para motivar e abstrair o monitoramento de dados de saúde. Estatísticas da indústria americana de jogos constataram que, em 2011, os jogadores de videogame possuíam em média 37 anos e 29$\%$  estão acima dos 50 anos. Desde 2005, os jogos eletrônicos utilizam sensores de detecção de movimento para capturar as ações cinéticas do usuário. Desta forma, na abordagem proposta neste trabalho, o usuário executa movimentos específicos em um jogo eletrônico para quantificar seus sinais motores e monitorar seu estado de saúde.

A relevância deste trabalho foi validada por meio de pesquisa qualitativa na qual foi realizada uma entrevista semi-estruturada junto a profissionais de saúde. Para a validação da abordagem, foi realizado um estudo analítico de caso-controle para detectar indivíduos diagnosticados com a Doença de Parkinson, utilizando sensores de captura de movimento em jogos eletrônicos. Buscou-se avaliar as possibilidades de aquisição de dados de saúde baseada nas características de Cinemática Angular do Movimento Humano. Estes dados foram aplicados em uma Máquina de Vetor de Suporte (SVM) para classificação dos dados. Como resultado, foi obtida uma taxa de identificação de \texiti{accuracy} de 86,67\% e falsos positivos de 6,67\%. Desta forma, concluiu-se que a abordagem proposta permite desenvolver jogos eletrônicos que servem como uma forma não invasiva para monitorar dados motores.