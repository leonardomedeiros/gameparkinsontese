\chapter{Identificação de Padrões da Análise de Movimento}\label{sec:analise_movimento_identificacao_padroes}
%O reconhecimento dos padrões de movimento de Parkinsonianos é bastante importante para o diagnóstico da \ac{dp}. Entretanto, existem poucos métodos  que identificam esses padrões específicos da doença. 



%O que é ..
%Para que serve
%Como funciona9
%Importância para Identificação de Padrões


\section{}
%Para a avaliação dos resultados do trabalho é necessário  realizar os testes juntamente com Pacientes de Parkinson e Voluntários como grupo de controle. A presente pesquisa teve o projeto aprovado junto ao conselho de ética médica aprovado pelo Comitê de Ética em Pesquisa (CAAE: 14408213.9.1001.5182) e os pacientes foram informados sobre os procedimentos e assinaram um termo de consentimento livre e esclarecido. 
%Contudo, o número de voluntários é insuficiente para testar a abordagem com máquinas de aprendizagem (11 voluntários, 5 pacientes com AVC e 6 diagnosticados com Doença de Parkinson). Por esse motivo recorremos a base de dados da Physionet para avaliarmos o presente trabalho e consequentemente aumentarmos nosso espaço amostral.




%\subsection{Avaliação dos Resultados}
%Para a avaliação dos resultados do trabalho é necessário  realizar os testes juntamente com Pacientes de Parkinson e Voluntários como grupo de controle, no momento esta sendo aguardado o parecer de ética médica.
%Contudo, foi a abordagem de processamento de sinal, reconhecimento de padrões e classificação usando SVM a ser usada no trabalho está sendo testada com a base de dados de paciente de Parkinson da Physionet. Essa base de dados contêm a\ac{fvrs}, capturada através de 8 sensores em cada pé.
%Uma etapa importante para a aprendizagem de máquina é a seleção dos dados. Baseados nas diretrizes médicas dos sintomas parkinsonianos,  como entrada para a SVM foram processados os ciclos de movimento e para cada ciclo foram calculados: 
%A média da VGRF: Informará se existe muita mudança na força do movimento como consequência alteração no equilíbrio
%SwingPhase/Stance Phase (Figura 2.) e a diferença da média da força do pé esquerdo sobre o pé direito, o parkinsoniano tem um passo mais curto logo a Fase de Balanço será mais curta.
%Diferença da força do pé da esquerda sobre o da direita: A partir da fase 2 da \ac{dp} existe uma assimetria do movimento logo o parkinsoniano demonstraria essa diferença.



