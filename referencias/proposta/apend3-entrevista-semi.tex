\chapter{Questionário Entrevista Semi-Estruturada} \label{apendice:entrevista-semi-estruturada}
% Palavras-chave do resumo em Português
%\begin{keywords}
%projetos de pesquisa, processo de desenvolvimento, processo de
%requisitos.
%\end{keywords}

\section{Entrevista com Profissionais de Neurologia}

Esse documento contém um conjunto de perguntas a serem respondidas em entrevistas semi-estruturadas, a profissionais que trabalham diretamente com doenças neurológicas envolvidos no acompanhamento de pacientes com a doença de parkinson. A pretensão dessa pesquisa é a identificação de mecanismos que auxiliem no monitoramento dos sintomas da doença para auxiliar os neurologistas no gerenciamento da dosagem do medicamento antiparkinsoniano.

\subsection{Sintomas da Doença de Parkinson}
\begin{itemize}
    \item Como é realizado o diagnóstico da doença de parkinson ?
		\item Quais são os sintomas mais frequentes ?
		\item Quais sintomas são amenizados pela dosagem medicamentosa ?
\end{itemize}

\subsection{Monitoramento de dados Motores}
\begin{itemize}
    \item O movimento  de adução e abdução do braço, é um movimento relevante para a identificação da doença de parkinson.
		\item É importante que o profissional de saúde acompanhe a amplitude máxima desses movimentos?
		\item Quão importante é monitorar a velocidade angular do movimento de adução em º/s para a avaliação do sintoma de bradicinesia da doença de parkinson ? Esse sintoma pode ser avaliado em outras doenças ? Cite exemplos.
		\item A doença de parkinson apresenta assimetria do movimento como um dos seus sintomas. Ou seja um lado do braço tem uma amplitude maior do que o outro lado.
		\item Demonstrar a amplitude máxima do movimento de abdução poderia ser aplicado para outras doenças que impactam na mobilidade? Cite exemplos ?		
		\item Um mecanismo que pudesse monitorar os sintomas da doença de parkinson como: tremor, bradicinesia e discinesia. Poderia auxiliar na eficácia da medicação?
		\item Qual a relação do tremor com o uso dos medicamentos antiparkinsonianos ?
		\item A bradicinesia e discinesia são influenciadas pelos medicamentos antiparkinsonianos ?
		\item As diretrizes médicas citam tabelas de evolução da doença de Parkinson como a UPDRS, você as utiliza na sua prática clínica?		
\end{itemize}

\subsection{Benefícios}
\begin{itemize}
    \item A literatura informa que a doença de parkinson é progressiva e devido ao uso de medicamentação estas passam a não surtir efeito necessitando aumentar a dosagem medicamentosa além de efeitos colaterais incapacitantes causados pelo uso da medicação. Diante desses problemas como a dosagem medicamentosa é definida para o paciente ?
		\item Como profissional,seria importante acompanhar: amplitude do movimento, velocidade angular de abdução, velocidade angular de adução?
		\item Esses valores permitiriam visualizar a melhora ou o comprometimento do paciente?
		\item Você acha interessante ser auxiliado por uma máquina de aprendizagem que análise esses dados para facilitar o seu trabalho e melhorar na avaliação dos pacientes ?
		\item Sabendo que muitos indivíduos usam jogos eletrônicos em sua rotina, supondo que dentro desses jogos que são usados em momentos de descontração ou para entretenimento. Se dentro desses jogos houvesse mecanismos de monitoramento de sintomas de parkinson como tremor, bradicinesia e discinesia. Será que o monitoramento desses sintomas identificados durante a rotina diária viria auxiliar na melhora da qualidade de vida do paciente, já que o profissional teria acesso ao surgimento dos sintomas ao longo do dia?
		\item Qual a importância do uso de uma dosagem mínima dos medicamento antiparkinsonianos na qualidade de vida do paciente ?				
		\item Dado que a doença de parkinson é incapacitante, qual a importância de um diagnóstico precoce da doença de parkinson? 
\end{itemize}

