\chapter{Avalia\c{c}\~{a}o Experimental} \label{chap:avaliacao}
Neste capítulo são descritos os experimentos realizados para verificar as hipóteses deste trabalho.

A realização do monitoramento de dados motores de uma maneira não-invasiva é um desafio, esta proposta fornece um mecanismo de monitoramento contínuo de dados de saúde por meio de jogos eletrônicos integrados à rotina diária dos usuários e de maneira lúdica. Pois, a partir de um \textit{GAHME} é possível capturar, processar e classificar os movimentos cinéticos exercidos pelos usuários.

Para atingir os objetivos da pesquisa as seguintes hipóteses foram elencadas e testadas:
	\begin{description}
	\item[H1] O acompanhamento de sintomas motores, integrados à rotina diária do paciente, traz benefícios ao tratamento e qualidade de vida do mesmo, do ponto de vista do profissional da saúde.
	\item[H2] É possível capturar dados motores por meio de sensores de movimento utilizados em jogos eletrônicos. Esses dados auxiliam no acompanhamento de doenças com comprometimento motor.
	\item[H3] É possível desenvolver um jogo que tenha mecanismos de captura de dados motores embutidos, e que permita monitorar e quantificar esses dados de maneira não-invasiva.
	\end{description}


\section{H1 - Entrevista Semi-Estruturada com Profissionais de Saúde}\label{chapter:entrevista_semi_estruturada}

A interpretação de dados é o cerne da pesquisa qualitativa. Esse método tem como função desenvolver a teoria, servindo ao mesmo tempo de base para a decisão sobre quais dados adicionais devem ser coletados \cite{FLI04}, através de técnicas de codificação seletiva. Esta técnica, permite elaborar uma categorização nos dados, demonstrando ao pesquisador quais são os fenômenos salientes da pesquisa para que este possa ponderá-los. O procedimento da interpretação dos dados, assim como a integração de material adicional são encerrados quando se atinge a "saturação teórica", ou seja quando o avanço na codificação não resulta em novos conhecimentos \cite{FLI04}.

Para análise dos textos provenientes da pesquisa (transcrição da entrevista com os neurologistas e fisioterapeutas especialistas em neurologia) foi utilizada a codificação seletiva através de criação de categorias a \emph{posteriori}. As categorias foram criadas e organizadas de acordo com o conteúdo de cada texto. As respostas de cada participante foram analisadas, e a partir da identificação das categorias, incluídas na árvore de categorias do QDA Miner \cite{qda13}, que pôde armazenar a transcrição de cada entrevista. Admitindo-se que uma classificação, para ser adequada, não pode ser feita arbitrariamente, a categorização da árvore foi criada e reformulada várias vezes durante o processo de análise de acordo com o método de pesquisa qualitativa \cite{FLI04}.

\subsection{Objetivo da Pesquisa}
O objetivo da entrevista semi-estruturada foi entender como é feito o acompanhamento do paciente com sintomatologia da doença de parkinson, juntamente aos profissionais de saúde: neurologistas que prescrevem a dosagem medicamentosa e fisioterapeutas que fazem o acompanhamento motor do paciente, ao longo de seu tratamento. Os mesmos foram indagados se poderia haver melhora na tomada de decisão caso eles pudessem acompanhar o surgimento dos sintomas em diferentes momentos do dia por intermédio de um monitoramento contínuo dos sintomas. Procurou-se encontrar dentro do contexto de estudo, a importância do monitoramento de dados de saúde e os benefícios trazidos por este.

As entrevistas foram realizadas presencialmente, com perguntas não estruturadas e, no decorrer da entrevista, uma maior estruturação foi estabelecida sempre com a preocupação de evitar a referência do entrevistador sobre os pontos de vista do entrevistado, conforme prega o método científico \cite{FLI04}. 

\subsubsection{Instrumento de Análise dos Dados da Pesquisa Qualitativa} \label{section:analise_dados} 
A pesquisa qualitativa assistida por computador (\textit{software}) permite uma melhor categorização das informações obtidas em modo texto. O \textit{software} QDA Miner \cite{qda13} auxilia o pesquisador na organização dos registros da pesquisa e das interpretações dos mesmos, justificando-se o uso da ferramenta devido a dificuldade de classificar e analisar os dados obtidos. Nessa análise, foram consideradas as atividades referentes ao acompanhamento dos sintomas motores em pacientes com ~\ac{dp}, e como um possível cenário de monitoramento dos sintomas por intermédio de jogos eletrônicos, poderia auxiliar os profissionais de saúde no tratamento dos pacientes.

Nesta seção, faz-se um detalhamento do resultado da entrevista semi-estruturada, descrevendo a opinião dos entrevistados e coletando requisitos baseado nas necessidades expostas pelos mesmos. Por meio desta etapa da pesquisa, foi testada a Hipótese \textbf{H1} pelo método qualitativo:

	\begin{description}
	\item[H1] O acompanhamento de sintomas motores integrados à rotina diária do paciente, traz benefícios ao tratamento e na qualidade de vida do mesmo, do ponto de vista do profissional da saúde.
	\end{description}
	
%\subsubsection{Análise da Entrevista Semi-Estruturada}
%A análise qualitativa ~\cite{FLI04} permite identificar as práticas dos profissionais de saúde referentes ao acompanhamento dos sintomas motores em pacientes de parkinson e como essas práticas podem ser aperfeiçoadas num cenário em que haja o monitoramento dos sintomas. 


\subsection{Perfil dos Participantes}
O perfil dos participantes é composto por quatro profissionais da saúde, dos quais: dois são fisioterapeutas com especialização em neurologia, e dois são médicos neurologistas. A escolha desse perfil se fez mister, pois tais profissionais desempenham funções distintas, porém complementares. Os neurologistas realizam o diagnóstico e acompanham os sintomas motores juntamente com as informações obtidas do paciente ou de seu cuidador, e baseado nas informações realiza o gerenciamento da dosagem medicamentosa da doença. Por outro lado, os fisioterapeutas fazem o acompanhamento dos sintomas motores em sessões de fisioterapia promovendo a aprendizagem motora desses pacientes. Logo, esses profissionais possuem visões e preocupações distintas inerentes a sua profissão. 


Para manter a confidencialidade de informação, os entrevistados receberam uma \textbf{LEGENDA} que identifica o perfil profissional seguido por um número sequencial que identifica o entrevistado, mas preserva sua identidade, como descrito na Tabela~\ref{table:perfil_analise_participantes}.

\begin{table}[h]
\caption{Perfil dos Participantes}
\label{table:perfil_analise_participantes}
\begin{tabular}{|l|l|c|c|}
\hline
\textbf{LEGENDA} & \textbf{PROFISSÃO}             & \multicolumn{1}{|l}{\textbf{IDADE (ANOS)}} & \multicolumn{1}{|l|}{\textbf{EXPERIÊNCIA (ANOS)}} \\ \hline
FIS\_01          & Fisioterapia em Neurologia & 40                                         & 10                                                \\ \hline
FIS\_02          & Fisioterapia em Neurologia     & 39                                         & 10                                                \\ \hline
NEU\_01          & Médico Neurologista            & 42                                         & 15                                                \\ \hline
NEU\_02          & Médico Neurologista            & 67                                         & 30                                                \\ \hline
\end{tabular}

\end{table}

\subsubsection{Questionário de Pesquisa}
Para a formulação do questionário foram realizadas análises nas diretrizes médicas ~\cite{protpar010,national2006parkinson} e na tabela UPDRS ~\cite{updrs87} sobre o progresso da ~\ac{dp} e dos sintomas monitoráveis por sensores de movimento. Para a entrevista foram elaboradas 15 perguntas agrupadas em 3 seções (Apêndice \ref{apendice:entrevista-semi-estruturada}) com os seguintes temas: sintomas da doença de parkinson, monitoramento da saúde motora e os benefícios advindos do monitoramento. Devido as diferenças existentes na abordagem utilizada por cada profissional o entrevistador pôde selecionar as questões de acordo com as habilidades e responsabilidades do entrevistado.

\subsection{Análise}
Durante a análise dos textos foram extraídos fragmentos, e a nomenclatura utilizada contém o prefixo \textbf{FRAGMENTO} mais um número sequencial identificando o mesmo. Esse procedimento permite identificar \textbf{requisitos} que orientem a proposta de monitoramento de dados motores por intermédio de jogos eletrônicos. Logo, os requisitos extraídos nesta abordagem foram obtidos a partir da perspectiva do profissional de saúde em relação ao tratamento e acompanhamento da ~\ac{dp}. 

\subsubsection{Diagnóstico}\label{section:analise_diagnostico}

Na entrevista junto aos neurologistas, foi indagado como o diagnostico da \ac{dp} é realizado.  A entrevista corroborou com a literatura médica~\cite{tolosa06,vedolin2003}, que informa que \ac{dp} possui um diagnóstico de exclusão~\cite{national2006parkinson,protpar010}.  

Todos os profissionais informaram que o sintoma mais comum é o tremor de repouso, e que este inicialmente é unilateral e seguido de uma bradicinesia como podemos perceber nos ([FRAGMENTO-01][FRAGMENTO-02]). Ainda no ([FRAGMENTO-01]), existe uma ocorrência do \textbf{[NEU\_01]} em que o mesmo evoca sobre a importância da técnica de \textit{Finger Taps} ~\cite{updrs87} para avaliação da bradicinesia.


\begin{quote}
\textbf{[FRAGMENTO-01][NEU\_01]} - 
\emph{
O diagnóstico da doença de Parkinson é quando o paciente chega se queixando de tremor. Esse sintoma começa com um tremor unilateral geralmente pelas mãos, lentamente progressivo e de repouso. Além do tremor esse paciente exibe também uma lentidão que a gente consegue detectar pelo \textit{Finger Taps}. Essa técnica consiste em tocar o polegar ao primeiro e segundo dedo simultaneamente para ver se há ou não lentidão. Faz-se uma comparação sempre com o outro lado para visualizar possíveis diferenças. 
Existe também uma rigidez no braço, quando faz-se uma flexão e extensão do membro e percebe-se que o tônus desse paciente comparado com o outro lado exibe uma diferença.
}
\end{quote}

\begin{quote}
\textbf{[FRAGMENTO-02][NEU\_02]} -
\emph{
O diagnóstico da doença de Parkinson é feito com uma das queixa iniciais do paciente é um tremor de repouso, geralmente associado a uma dificuldade na marcha. Então normalmente os pacientes reclamam de uma perna presa e um tremor de repouso. 
}
\end{quote}

Uma ocorrência no ([FRAGMENTO-03]) que deve ser ressaltado, é o que o entrevistado referiu como ``\textit{boa resposta ao prolopa}''. Essa ocorrência é denominada de diagnóstico diferencial da \ac{dp} ~\cite{protpar010}, consiste na redução dos sintomas parkinsonianos em decorrência da resposta ao tratamento medicamentoso. 

\begin{quote}
\textbf{[FRAGMENTO-03][NEU\_01]} - 
\emph{
Então os sintomas é o tremor em repouso, a lentidão e a rigidez. Inicialmente apenas de um lado, por exemplo começa no braço direito e depois vai para a perna direita, depois para o braço esquerdo e depois a perna esquerda. Isso lentamente progressivo, a gente faz a exclusão com outras doenças através de outros exames como tomografia, ressonância ou a uma boa resposta ao prolopa.
}
\end{quote}

\subsubsection{Sintomas}
Nesta seção estão expostos sintomas para o acompanhamento da sintomatologia da ~\ac{dp}.

%\subsubsection{Tremor}
O sintoma de tremor, além de ter sido referenciado durante o diagnóstico da doença na Seção ~\ref{section:analise_diagnostico} por todos os entrevistados, possui particularidades como a dificuldade de controlar o sintoma por intermédio do tratamento medicamentoso ([FRAGMENTO-04]), e não é tão incapacitante quanto a bradicinesia ~\ac{dp}. No ([FRAGMENTO-05]) o [NEU\_01] reforçou sobre a importância de controlar os sintomas de lentidão do movimento ante os de tremor ~\cite{do2007parkinson}.

\begin{quote}
\textbf{[FRAGMENTO-04][NEU\_01]} - 
\emph{
Mas a gente tem que ver, porque as vezes o tremor é muito mais difícil de você controlar. Porque está relacionada ao emocional do paciente. Quanto mais emocionalmente desequilibrado o paciente tiver, mais tremor ele tem.
}
\end{quote}

\begin{quote}
\textbf{[FRAGMENTO-05][NEU\_01]} - 
\emph{
O controle do tremor é um pouco complicado porque é um sintoma mais difícil de ser controlado com as medicações que temos hoje.  Então você poderia ver nesse seu projeto a lentidão. Porque o paciente quer tremer mas ele não quer ficar lento.
}
\end{quote}


%\subsubsection{Bradicinesia}\label{section:analise_bradicinesia}

O pesquisador indagou se o sintoma da bradicinesia era considerado o mais debilitante da ~\ac{dp} e como resposta ele obteve a afirmação de que a bradicinesia impacta diretamente na qualidade de vida do paciente, privando-o de realizar atividades diárias ([FRAGMENTO-06]).

\begin{quote}
\textbf{[FRAGMENTO-06][NEU\_01]} - 
\emph{
É ele atrapalha né, principalmente no levantar no andar, para você se levantar, pentear o cabelo o tremor é prejudicial. Porém mais prejudicial ainda é a lentidão do movimento.
}
\end{quote}

Ao indagar se o movimento de adução e abdução do braço seria relevante para a identificação da~\ac{dp}, o [NEU\_01] informou que a bradicinesia é um sintoma que traz lentidão em todo o corpo e possivelmente seria afetada por este movimento. Pois, devido à redução dos movimentos automáticos ([FRAGMENTO-07]), traz outros impactos físicos ao paciente ([FRAGMENTO-08]).

\begin{quote}
\textbf{[FRAGMENTO-07][NEU\_01]} - 
\emph{
Na verdade o movimento em si, vai ver o quão lento está. Porque você não tem um déficit motor. O comprometimento na doença de Parkinson está no comprometimento piramidal, o comprometimento extra-piramidal não vai estar alterando a força motora. O que vai estar vai ser exatamente a lentidão. Por exemplo, é um paciente que está andando você que os movimentos dele automáticos estão reduzidos, principalmente no balançar dos braços. Você vai andando, vai andando, você vê aquele paciente que está com a força, ele está com toda a estrutura piramidal tudo normal. Mas ela anda lento em consequência da lentidão do movimento porque os movimentos automáticos estão reduzidos.
}
\end{quote}


\begin{quote}
\textbf{[FRAGMENTO-08][FIS\_01]} - 
\emph{
Os sintomas mais frequentes a gente tem a bradicinesia que é a lentificação do movimento, a gente tem um padrão postural que começa a ficar bem nítido que o paciente apresentar o Parkinson. Você percebe uma perda da movimentação automática da cintura escapular e aí ele começa a apresentar uma diminuição no volume da voz que é uma dipofonia, e apresenta uma maior rigidez muscular. Eles reclamam bastante e a bradicinesia que tornam os movimentos cada vez mais lentos.
}
\end{quote}

%\subsubsection{Marcha}
Foi identificada uma ocorrência na dificuldade do andar do paciente de~\ac{dp}, quando o [NEU\_01] cita no ([FRAGMENTO-07]) (``\textit{Você vai andando, vai andando, você vê aquele paciente que está com a força, ele está com toda a estrutura piramidal tudo normal. Mas ela anda lento em consequência da lentidão do movimento ...}''. O [NEU\_02] corrobora com a mesma opinião ao citar a dificuldade de iniciar a marcha no ([FRAGMENTO-09]). O fisioterapeuta no papel de realizar o acompanhamento da marcha nas sessões fisioterápicas, fornece um aprendizado motor para a melhora da qualidade de vida do paciente de acordo com suas limitações ([FRAGMENTO-10]).

\begin{quote}
\textbf{[FRAGMENTO-09][NEU\_02]} - 
\emph{
Problema na marcha. Dificuldade de iniciar a marcha, certa dificuldade de um lado comprometido. Mesmo quando o sintoma está unilateral eles sentem dificuldade para iniciar a marcha.
}
\end{quote}

\begin{quote}
\textbf{[FRAGMENTO-10][FIS\_01]} - 
\emph{
Numa marcha, o doente de Parkinson tem a tendência de estar olhando para o chão. Mas a gente sabe que isso não é compatível com uma boa marcha a tendência é cair, para piorar eles têm os passos miúdos e também um passo arrastado. Então esse passo favorece a queda, poise ele perdeu a marcha automática que é aquela que a gente adquire na infância. O que a gente faz nas sessões de fisioterapia é tentar aplicar auto-correções para adaptar o paciente à nova realidade para que ele tenha uma aprendizado motor e no futuro um automatismo do movimento.
}
\end{quote}

Devido a quantidade de ocorrências sobre a análise da marcha para o acompanhamento da ~\ac{dp}, e a possibilidade da ocorrência de quedas dos indivíduos. Esses, dois fatos corroboraram com o uso de base de dados contendo dados sobre a marcha, pois isto vai além do custo financeiro para a aquisição dos sensores que capturam a ~\ac{fvrs}. Logo, ao usar bases contendo esses dados para a pesquisa, preserva-se a integridade física dos pacientes.



\subsubsection{Monitoramento Motor}
Nesta seção está exposta a importância do monitoramento dos sinais capturados no estudo analítico de caso controle definido no método de pesquisa na Seção~\ref{section:estudo_caso_controle}. Nesse estudo também pretende-se identificar as características dos movimentos que possam ser extraídos desses sinais, e que venham fornecer subsídios para diferenciar indivíduos diagnosticados com a ~\ac{dp} ante indivíduos sem o diagnóstico.

%\subsubsection{Amplitude do Movimento dos Braços}

Ao indagar ao [FIS\_02] se o movimento de adução e abdução do braço seria relevante para a identificação da~\ac{dp}, este informou que mesmo não sendo um teste específico para a identificação da doença, existem diferenças significativas encontradas em indivíduos diagnosticados com parkinson [FRAGMENTO-11].
\begin{quote}
\textbf{[FRAGMENTO-11][FIS\_02]}-
\emph{
Sim. Existe alterações sim, mas eu nunca vi especificamente esse teste como sendo usado para diagnóstico da doença. Mas que realmente existem mudanças no movimento de adução e abdução de uma pessoa normal ante a um parkinsoniano.
}
\end{quote}

O [FIS\_01], explicou os motivos que levam a perda da mobilidade no movimento de adução e abdução ([FRAGMENTO-12]) e consequentemente, reforça que esse movimento poderia ser monitorado para verificar o comprometimento da doença. Em um outro fragmento ([FRAGMENTO-13]) o mesmo fisioterapeuta menciona a importância de monitorar a amplitude do movimento, pois permite visualizar a resposta do paciente ao tratamento oferecido.

\begin{quote}
\textbf{[FRAGMENTO-12][FIS\_01]}-
\emph{
Têm, porque uma das grandes perdas que eles apresentam é na cintura escapular e consequentemente é pegando a parte de ombro. Pois caso ela seja mais fixa, porque geralmente o paciente de Parkinson abduz o ombro. O ombro fica abduzido junto ao tronco e ai ele perde a mobilidade do cotovelo e punho e também o movimento fica comprometido por conta disso.
}
\end{quote}

\begin{quote}
\textbf{[FRAGMENTO-13][FIS\_01]}-
\emph{
Mesmo sabendo que a tendência é uma lentificação (bradicinesia).  As outras doenças também, porque um dos objetivos nossos é o aumento da amplitude. Então é um meio interessante para a gente conseguir visualizar se o tratamento está dando certo ou não.
}
\end{quote}



\subsubsection{Velocidade do Movimento De Adução e Abdução dos Braços}
Um ponto de convergência, entre os profissionais entrevistados, é a importância de monitorar a velocidade angular dos pacientes. Os profissionais tentam associar o tratamento fisioterápico e medicamentoso para a melhora da bradicinesia. Logo, para estes profissionais a melhora está condicionada a um aumento na velocidade do movimento ([FRAGMENTO-14],[FRAGMENTO-15])

\begin{quote}
\textbf{[FRAGMENTO-14][NEU\_01]} - 
\emph{
É como eu falei para mim seria melhor se capturássemos se ele está mais lento. Se através dessa amplitude você conseguir por intermédio do computador identificar que ele está mais lento de um lado do que do outro, e conseguir visualizar a velocidade de um lado e do outro. Então isso é interessante.
}
\end{quote}


\begin{quote}
\textbf{[FRAGMENTO-15][FIS\_01]} - 
\emph{
É e consequentemente a velocidade, porque nesse caso o tratamento é diretamente relacionado a isso quanto mais veloz o parkinsoniano é melhor para a gente melhor prognóstico a gente pode ter lá na frente. Mesmo sabendo que a tendência é uma lentificação.
}
\end{quote}

%\subsubsection{Assimetria do Movimento}

A assimetria do movimento acomete os pacientes que estão nos estágios iniciais da doença. Por esse motivo, geralmente ela é identificada durante o diagnóstico [FRAGMENTO-03]. Porém, alguns pacientes parkinsonianos apresentam a assimetria do movimento quando um dos lados é mais comprometido que o outro. Por essa razão é que o [NEU\_01] afirmou \textit{``Se através dessa amplitude você conseguir por intermédio do computador identificar que ele está mais lento de um lado do que do outro''}. Pois, a tendência natural da evolução da ~\ac{dp} é a redução na assimetria do movimento conforme a opinião do [NEU\_01] no [FRAGMENTO-16] e na tabela UPDRS ~\cite{updrs87} em sua escala de avaliação do progresso da doença (Seção ~\ref{section:escalas_avaliacao}).

\begin{quote}
\textbf{[FRAGMENTO-16][NEU\_01]}-
\emph{
No início. Geralmente o paciente se queixa de uma diminuição de força de um lado do corpo. Mas na progressão, ele vai sentir dificuldade global. Mas aqueles parkinsonianos iniciais geralmente eles se queixam na diminuição do movimento de um dos lados.
}
\end{quote}

\subsubsection{Benefícios Advindos do Monitoramento}


%\subsubsection{Quantificação dos Sintomas}
As relativas aos benefícios advindo do monitoramento, estavam relacionadas à quantificação dos sintomas como: amplitude de movimento de adução e abdução do braço,  velocidade angular deste movimento e se estes valores permitem um monitoramento dos sintomas. Essa análise trouxe dois grupos de respostas: o primeiro reconhecia da importância da quantificação dos dados para identificar a melhora ou piora do paciente [FRAGMENTO-17], e outro relatava que essa informação tinha mais validade científica do que prática [FRAGMENTO-18]. Todavia, caso esses profissionais tivessem acesso a um sistema que permitisse o monitoramento motor, possivelmente eles iriam perceber os benefícios da abordagem e modificar sua prática atual ao adotar uma nova proposta.

\begin{quote}
\textbf{[FRAGMENTO-17][FIS\_02]}-
\emph{
É preciso ter parâmetros sim. Pois atualmente usamos muito o olho clínico e ai vai de cada profissional. Se tivermos números facilitam bastante porque se tornam fatos e basearmos nossas conclusões em números é bem melhor.
}
\end{quote}

\begin{quote}
\textbf{[FRAGMENTO-18][FIS\_01]}-
\emph{
É interessante em termos de pesquisa. Em termos de clínica a geralmente a gente vai no geral. Por exemplo: Eu faço uma flexão de ombro com bastão e anotei no meu exame que ele ia até mais ou menos 70º e após 15 dias eu vejo que ele está levantando acima de 90º. Então está marcado a minha evolução. Então eu faço a avaliação nesse sentido. Então esse sistema seria bom para pesquisa mesmo.
}
\end{quote}

%\subsubsection{Gerenciamento da Dosagem Medicamentosa}
Foi indagado junto aos profissionais se monitoramento dos dados motores auxiliaria no gerenciamento da dosagem medicamentosa. Os profissionais informaram que sentem a necessidade de visualizar a eficácia do tratamento diante do paciente. O [FIS\_01] no [FRAGMENTO-19] cita a importância de avaliar tanto o tratamento medicamentoso quanto se a sua atividade fisioterápica traz benefícios ao paciente. Os neurologistas citam ([NEU\_01] e [NEU\_01]) a importância de reajustar a dosagem medicamentosa e que a quantificação do sintoma identifica o resultado do efeito medicamentoso. Outra opinião bastante pertinente é que o agravamento da ~\ac{dp} é bastante sutil do ponto de vista do [NEU\_01] no [FRAGMENTO-21]. Logo se for possível, mostrar a evolução da doença em períodos mais longos, o tratamento seria mais efetivo e, consequentemente, traria uma melhor qualidade de vida aos pacientes.

\begin{quote}
\textbf{[FRAGMENTO-19][FIS\_01]} - 
\emph{
 É interessante porque teremos uma ideia de até que ponto a medicação está sendo efetiva, até quando a patologia está progredindo e também avaliar se o nosso tratamento fisioterápico está dando resultados ao tentar frear a evolução da doença.
}
\end{quote}


\begin{quote}
\textbf{[FRAGMENTO-20][NEU\_02]} - 
\emph{
Sim. Dentro do que você propõe. Com certeza sim. Essa avaliação desses movimentos. Porque a gente consegue visualizar se a medicação está surtindo efeito, se precisa ser reajustada.
}
\end{quote}

\begin{quote}
\textbf{[FRAGMENTO-21][NEU\_01]} - 
\emph{
Se esse mecanismo acontecesse. Você poderia avaliar a dosagem de um paciente por exemplo. Veja avalie durante uma semana, não melhorou. Então a gente poderia fazer um teste com tremor, lentidão e a rigidez, se houvesse esse aspecto.  A gente poderia aumentar a dosagem e visualizaria a eficácia da dosagem com o decorrer do tempo, com o decorrer da evolução. E verificaria se realmente o paciente está melhorando. Porque o paciente da doença de Parkinson ele piora lentamente, as vezes é tão sutil que o próprio paciente não consegue. Então é como eu disse, cada paciente a evolução é diferente num existe. Mas poderia assim, se você conseguisse detectar as amplitudes do tremor por exemplo.
}
\end{quote}


\subsection{Requisitos Identificados}
A \ac{er} é o processo de descobrir o propósito do software, identificando os principais envolvidos do sistema com suas respectivas necessidades e documentando a análise para uma implementação posterior \cite{bas00}. Contudo, é um processo que deve ser continuamente repetido para que as necessidades dos envolvidos sejam satisfeitas. As técnicas para identificação de requisitos são derivadas principalmente das ciências sociais, que se baseiam em pesquisa qualitativa onde são analisados a teoria do objeto de estudo com a experiência prática dos envolvidos na pesquisa ~\cite{elicquest05,zowghi2005}.

Uma das técnicas de identificação de requisitos, que é baseada em pesquisa qualitativa, é a entrevista semi-estruturada em que o entrevistador possui um conjunto de perguntas pré-definidas e guia a entrevista de acordo com a opinião do entrevistado ~\cite{FLI04}. A identificação dos requisitos de um sistema representa o início da elicitação das necessidades da solução proposta. Então, os requisitos definem quais serão os serviços que o sistema deve prover além de um conjunto de restrições existentes na operação do mesmo \cite{sommerville2011}. Logo, foi utilizada o resultado da análise da pesquisa qualitativa, para identificar os requisitos deste trabalho. Ficou definido, que cada requisito deve ser importante para os entrevistados e a nomenclatura estabelecida é de \textbf{REQ-ENTREVISTAS} seguida por um número sequencial correspondente à sua apresentação. Para demonstrar a relevância dos requisitos, a teoria foi confrontada com o que é aplicado na prática pelos profissionais de saúde, por esse motivo, foram citadas referências científicas que corroboram com a análise. 



%O uso do termo "Engenharia" implica que técnicas sistemáticas e repetidas devem ser aplicadas para certificar que os requisitos de um sistema estejam completos, consistentes e relevantes \cite{sommerville2011}. O emprego correto da Engenharia de Requisitos é um passo fundamental para o desenvolvimento de um bom  produto, onde a satisfação do usuário deve ser o principal objetivo a ser atingido \cite{zowghi2005}.

\begin{description}
	\item[REQ-ENTREVISTAS-01 :] Identificar e quantificar o tremor parkinsoniano ~\cite{tolosa06,keijsers2006,lemoyne2010}.
	\item[REQ-ENTREVISTAS-02 :] Identificar a bradicinesia ~\cite{patel_monitoring_2009}. %Para identificar a bradicinesia pode ser calculada a velocidade angular do movimento de abdução e adução do braço.
	\item[REQ-ENTREVISTAS-03 :] Avaliar bradicinesia usando ~\textit{finger-tapping} ~\cite{finger2012}.
	\item[REQ-ENTREVISTAS-04 :] Considerar e identificar a assimetria do movimento nos estágios iniciais ~\cite{national2006parkinson}.	%Além da velocidade angular podemos calcular a amplitude do movimento, assim identificarems a assimetria do movimento com mais clareza.
	\item[REQ-ENTREVISTAS-05 :] Fornecer mecanismos para possibilitar o Diagnóstico Diferencial \cite{protpar010} da Doença de Parkinson. %caso os sintomas de bradicinesia e marcha sejam identificados. Caso os indivíduos venham a surtir efeito ao medicamento e consequentemente reduzir a gravidade do sintoma, então foi possível realizar um diagnóstico diferencial.
	\item[REQ-ENTREVISTAS-06 :] Analisar a Marcha ~\cite{gaitusingsensorsreview2012}. Medir a marcha e comparar o padrão do movimento com indivíduos com e sem o diagnóstico da ~\ac{dp} para classificar a marcha como saudável ou parkinsoniana.
	\item[REQ-ENTREVISTAS-07 :] Calcular e armazenar a amplitude do movimento de adução e abdução dos braços, para realizar o monitoramento da saúde motora e poder acompanhar o tratamento.
	\item[REQ-ENTREVISTAS-08 :] Calcular e armazenar a velocidade angular do movimento de adução e abdução dos braços. Para poder avaliar o sintoma da bradicinesia.
	\item[REQ-ENTREVISTAS-09 :] Avaliar estado emocional e avaliar o comprometimento do tremor. 
\end{description}



\subsubsection{Inviabilidade Técnica}
Alguns requisitos identificados não podem ser implementados com a tecnologia de sensor de movimento usada nesse trabalho. A importância destes requisitos é reconhecida e pode ser implementada em trabalhos futuros, desde que as barreiras tecnológicas sejam resolvidas como descrito:
\begin{itemize}
  \item O \textbf{REQ-ENTREVISTAS-01}, o tremor de repouso é um dos principais sintomas da doença de parkinson. Sabíamos da sua importância, inclusive foi desenvolvido um jogo para \textit{Smartphone} que pudesse quantificar o tremor. Porém, no teste junto aos usuários, foi percebido que no momento do uso os pacientes de parkinson cessavam o tremor, inviabilizando assim sua quantificação. 
	\item O \textbf{[REQ-ENTREVISTAS-03]}, a técnica de \textit{finger-tapping} não pode ser avaliadas utilizando o MS-Kinnect 1.0, pois nessa versão não existe a captura do movimento dos dedos, conforme ilustrado na Figura~\ref{fig:articulacoeskinnect}.
	\item O \textbf{REQ-ENTREVISTAS-09}, por envolver estado emocional e parâmetros que não estamos levando em consideração nesse trabalho, esse requisito está fora do escopo. Entretanto, com mecanismos de detecção de batimentos cardíacos presente em versões mais atuais do MS-Kinnect, pode ser averiguada a relação dos batimentos cardíacos com o tremor.
\end{itemize}

\subsubsection{Matriz de Rastreabilidade - Fragmento x Requisitos}
A Matriz de Rastreabilidade (Fragmento x Requisitos) mapeia os \textbf{REQUISITOS} aos \textbf{FRAGMENTOS} que de forma direta ou indireta estejam correlacionados (Tabela \ref{table:matrix_rastreabilidade}). Ao final, é obtido um campo de quantidade de ocorrências quantificando a sua ocorrência nos fragmentos.

%\begin{center}
\begin{table}[!htbp]
\caption{Matriz Rastreabilidade: Fragmento x Requisitos}
\label{table:matrix_rastreabilidade}
\begin{tabular}{||p{6.06cm}||ccccccccc|}
\hline
 \multicolumn{1}{|p{6.06cm}|}{\centering \textbf{FRAGMENTOS / REQUISITOS}} &  01 &  02 &  03 &  04 &  05 &  06 &  07 &  08 & 09 \\ 
\hline 
 \multicolumn{1}{|p{6.06cm}|}{\centering 01} &  x &  x &  x &  x &   &   &   &   &  \\ 
 \multicolumn{1}{|p{6.06cm}|}{\centering 02} &  x &   &   &   &   &  x &   &   &  \\ 
 \multicolumn{1}{|p{6.06cm}|}{\centering 03} &  x &  x &   &  x &  x &   &   &   &  \\ 
 \multicolumn{1}{|p{6.06cm}|}{\centering 04} &  x &   &   &   &   &   &   &   & x \\ 
 \multicolumn{1}{|p{6.06cm}|}{\centering 05} &  x &  x &   &   &   &   &   &   &  \\ 
 \multicolumn{1}{|p{6.06cm}|}{\centering 06} &   &  x &   &   &   &  x &   &   &  \\ 
 \multicolumn{1}{|p{6.06cm}|}{\centering 07} &   &  x &   &   &   &  x &  x &  x &  \\ 
 \multicolumn{1}{|p{6.06cm}|}{\centering 08} &   &  x &   &   &   &  x &  x &  x &  \\ 
 \multicolumn{1}{|p{6.06cm}|}{\centering 09} &   &   &   &  x &   &  x &   &   &  \\ 
 \multicolumn{1}{|p{6.06cm}|}{\centering 10} &   &   &   &   &   &  x &   &   &  \\ 
 \multicolumn{1}{|p{6.06cm}|}{\centering 11} &   &   &   &   &   &   &  x &   &  \\ 
 \multicolumn{1}{|p{6.06cm}|}{\centering 12} &   &  x &   &   &   &   &  x &  x &  \\ 
 \multicolumn{1}{|p{6.06cm}|}{\centering 13} &   &   &   &   &   &   &  x &   &  \\ 
 \multicolumn{1}{|p{6.06cm}|}{\centering 14} &   &  x &   &   &   &   &   &   &  \\ 
 \multicolumn{1}{|p{6.06cm}|}{\centering 15} &   &  x &   &  x &   &   &   &  x &  \\ 
 \multicolumn{1}{|p{6.06cm}|}{\centering 16} &   &   &   &  x &   &   &   &   &  \\ 
 \multicolumn{1}{|p{6.06cm}|}{\centering 17} &   &   &   &   &   &   &   &   &  \\ 
 \multicolumn{1}{|p{6.06cm}|}{\centering 18} &  x &   &  x &  x &   &  x &  x &  x & x \\ 
 \multicolumn{1}{|p{6.06cm}|}{\centering 19} &  x &   &   &   &  x &  x &  x &  x &  \\ 
 \multicolumn{1}{|p{6.06cm}|}{\centering 20} &  x &   &   &   &  x &  x &  x &  x &  \\ 
 \multicolumn{1}{|p{6.06cm}|}{\centering 21} &  x &   &   &   &  x &  x &  x &  x &  \\ 
\hline 
 \multicolumn{1}{|p{6.06cm}|}{\centering \textbf{QTD. OCORRÊNCIAS}} &  9 &  9 &  2 &  6 &  4 &  10 &  9 &  8 & 2 \\ 
\hline 
\end{tabular}
\end{table}
%\end{center}

\subsubsection{Matriz de Rastreabilidade - Requisitos x Implementação}
A Matriz de Rastreabilidade (Tabela~\ref{table:RequisitoImplementado}) mapeia os \textbf{REQUISITOS} implementados neste trabalho e, os que devido a restrições técnicas, ainda estão em aberto. Isso demonstra também o estado atual do trabalho e pode direcionar os trabalhos futuros.

% Please remember to add \use{multirow} to your document preamble in order to suppor multirow cells
% Please remember to add \use{multirow} to your document preamble in order to suppor multirow cells
\begin{table}[h]
\centering
\caption{Requisitos Implementados}
\label{table:RequisitoImplementado}
\begin{tabular}{|c|c|c|}
\hline
\textbf{REQUISITO} & \textbf{IMPLEMENTADO} & \textbf{INVIABILIDADE TÉCNICA}\\ \hline
REQ-ENTREVISTA-01                   &                                                             & X                                                                                                                \\ \hline
REQ-ENTREVISTA-02                   & X                                                           &                                                                                                                  \\ \hline
REQ-ENTREVISTA-03                   &                                                             & X                                                                                                                \\ \hline
REQ-ENTREVISTA-04                   & X                                                           &                                                                                                                  \\ \hline
REQ-ENTREVISTA-05                   & X                                                           &                                                                                                                  \\ \hline
REQ-ENTREVISTA-06                   & X                                                           &                                                                                                                  \\ \hline
REQ-ENTREVISTA-07                   & X                                                           &                                                                                                                  \\ \hline
REQ-ENTREVISTA-08                   & X                                                           &                                                                                                                  \\ \hline
REQ-ENTREVISTA-09                   &                                                             & X                                                                                                                \\ \hline
\end{tabular}
\end{table}



\subsection{Conclusão}
Como dito no início do capítulo, o intuito dessa entrevista semi-estruturada é avaliar a Hipótese \textbf{H1}. Com isso foi verificado, junto aos profissionais de saúde, que um sistema de monitoramento de dados pode promover benefícios na qualidade de vida e na eficácia terapêutica dos usuários.

Com base na rastreabilidade dos fragmentos da entrevista, pode-se concluir que existiram muitas ocorrências nos requisitos de Identificação de sintomas como: tremores ([\textbf{REQ-ENTREVISTAS-01}]), bradicinesia [\textbf{REQ-ENTREVISTAS-02}] e análise da marcha [\textbf{REQ-ENTREVISTAS-06}]. Para o acompanhamento e monitoramento da doença, os profissionais de saúde citaram a importância de calcular, tanto a amplitude dos movimentos de abdução e adução dos braços ([\textbf{REQ-ENTREVISTAS-07}]), quanto a velocidade angular ([\textbf{REQ-ENTREVISTAS-08}]). Baseado nessas considerações, podemos validar qualitativamente a Hipótese \textbf{H1}, como propõe o método de pesquisa utilizado neste trabalho.


\section{H2 - Estudo de Caso 1: Análise dos Componentes Principais na \textit{Parkinson Disease Database}}\label{section:analise_marcha_pca}
Essa etapa da pesquisa foi auxiliar a este trabalho. No estudo, foi realizada uma análise de marcha da base de dados \textit{Parkinson Disease Database} disponível na \textit{Physionet}~\cite{physionet}. Esta base contém registros de sensores de força localizados nos pés que armazenam a ~\ac{fvrs} de indivíduos do grupo de controle e portadores de~\ac{dp}. Que serviram de dados de entrada que foram processados, extraídos os ciclos de movimento de cada indivíduo e aplicados técnicas de aprendizagem de máquina.

A abordagem utilizada foi de processamento de sinal, reconhecimento de padrões e classificação usando~\ac{pca} por meio da distância euclidiana dos indivíduos selecionados para teste junto aos dados de treinamento. Os dados foram projetados no autoespaço (Seção \ref{section:autoespaço}), e por meio das características identificadas no algoritmo foi possível identificar dois grupos distintos: sujeitos diagnosticados com a doença de parkinson e sujeitos sem o diagnóstico da doença de acordo com os dados armazenados na base. A base de dados utilizada nesse estudo contém a \ac{fvrs}, capturada por 8 sensores em cada pé dos sujeitos da pesquisa.


%
%A média da \textit{Fase de Apoio/Fase de Balanço} e a diferença da média da força do pé esquerdo sobre o pé direito, o parkinsoniano tem um passo mais curto logo a Fase de Balanço será mais curta ~\cite{protpar010,Jankovic_2008,ambulatoryparkinson2010}. Outra característica significativa poderia ser a diferença da força do pé da esquerda sobre o da direita, já que durante a fase 2 da \ac{dp} existe uma assimetria do movimento logo o parkinsoniano poderia demonstraria tal característica durante a análise da marcha. 
%A média da \textit{Fase de Apoio/Fase de Balanço} e a diferença da média da força do pé esquerdo sobre o pé direito, o parkinsoniano tem um passo mais curto logo a Fase de Balanço será mais curta ~\cite{protpar010,Jankovic_2008,ambulatoryparkinson2010}. Outra característica significativa poderia ser a diferença da força do pé da esquerda sobre o da direita, já que durante a fase 2 da \ac{dp} existe uma assimetria do movimento logo o parkinsoniano poderia demonstraria tal característica durante a análise da marcha. 


\subsection{Materiais}
Para a presente pesquisa, foi utilizada a base de dados "\textit{Parkinson´s Disease Database}" ~\cite{physionet}. Esta base, contém a marcha de 93 pacientes com ~\ac{dp} (idade média: 66,3 anos, 63$\%$ homens) e 73 como controle ​​(com idade média de 66,3 anos, 55$\%$ homens). Inclui também registros da \ac{fvrs} de indivíduos enquanto eles caminhavam normalmente em seu próprio ritmo por aproximadamente 2 minutos em um terreno plano. Debaixo de cada pé foram postos oito sensores como ilustrado na Figura \ref{fig:posicaosensores} onde são capturados os sinais da ~\ac{fvrs} como pode ser visualizado na Figura \ref{fig:dynography} que mensuravam a força desempenhada pelo corpo sobre o solo, medida em Newtons. 

\begin{figure}[!htbp]
 \centering
 \includegraphics[scale=0.7]{./img/ultraflexposition.png}
 % matrixargseg.png: 296x162 pixel, 100dpi, 7.52x4.11 cm, bb=0 0 213 117
 %\caption{Estágio desenvolvimento de jogos ~\cite{fullerton2008game}}
\caption{Posição dos sensores}
%  \caption{Estágio desenvolvimento de jogos}
 \label{fig:posicaosensores}
\end{figure}

As saídas de cada um destes 16 sensores foram armazenadas e gravadas em 100 amostras por segundo. Nesses registros estão inclusos dois sinais que refletem a soma da medição dos oito sensores para cada um dos pés (esquerdo e direito), denominada de \ac{fvrs}~\cite{gaitusingsensorsreview2012}. Neste trabalho, o resultado desta soma foi utilizado para realizar a análise de movimento.

\begin{figure}[!htbp]
 \centering
 \includegraphics[scale=0.4]{./img/ultraflexdinografia.png}
 % matrixargseg.png: 296x162 pixel, 100dpi, 7.52x4.11 cm, bb=0 0 213 117
 %\caption{Estágio desenvolvimento de jogos ~\cite{fullerton2008game}}
\caption{\textit{Ultraflex Computer Dyno Graphy}}
%  \caption{Estágio desenvolvimento de jogos}
 \label{fig:dynography}
\end{figure}



A análise dos dados~\cite{physionet}, incluiu 50 pacientes de Parkinson, e 50 indivíduos em grupo controle (segundo as informações fornecidas pela própria base de dados). Foram calculados, normalizados e escalonados, em 80 \textit{frames}, um total de 120 ciclos de marcha (Seção \ref{section:analise_marcha}). Foi aplicada nesse conjunto de dados a técnica definida de \textit{eigengaits} ~\cite{medeiros2013}. Esta técnica é uma adaptação da Análise dos Componentes Principais que foi detalhada na Seção \ref{section:analise_pca}.



\subsection{Aplicação do Método}
Cada indivíduo recebeu uma identificação na base de dados para podermos projetá-los e identificá-los no \textit{autoespaço} como visto na Figura \ref{fig:projecao_autoespaco}. Utilizando a técnica de PCA ~\cite{Shlens05atutorial} foi possível identificar 2 grupos bem distintos de indivíduos portadores da \ac{dp} e indivíduos não diagnosticados com a doença. Um indivíduo identificado na base como portador da \ac{dp} é representado como um ponto neste \textit{autoespaço}. As mudanças no padrão da marcha de um parkinsoniano podem resultar em uma mudança de localidade no \textit{autoespaço}, seja em direção aos indivíduos que não possuem a doença o que podemos presumir que houve uma melhora no estado do indivíduo ou em direção aos indivíduos diagnosticados com a \ac{dp} o que nesse modelo indica um agravamento do sintoma.

%\subsection{Projeção no Autoespaço}
A distância calculada é a medida de semelhança e pode ser expressa pela distância $ D $\ os dois pontos. Contudo, a distância euclidiana, quando estimada a partir das variáveis originais, apresenta problemas por ser influenciada: pela escala de medida, número de variáveis e pela correlação existente entre as mesmas. Por esse motivo, a normalização nos dados (descrito na Seção~\ref{section:filtro_dados}), se faz necessária para manter a mesma variância nos dados e permitir identificar as similaridades entre eles~\cite{vicini2005}. Nesta pesquisa, os resultados foram avaliados por meio da distância euclidiana entre o vetor de um indivíduo de teste projetado no \textit{autoespaço}, comparado com todos os indivíduos do grupo de treinamento. Logo, se o indivíduo de teste estiver mais próximo a uma indivíduo diagnosticado com \ac{dp} ele é classificado como parkinsoniano e caso o indivíduo de teste fique mais próximo a um indivíduo sem o diagnóstico da \ac{dp}, ele é classificado como controle.


\begin{figure}[!htbp]
 \centering
 \includegraphics[scale=0.70]{./img/projecaopca.png}
 \caption[Projeção das componentes principais no autoespaço dos dados da \textit{Parkinson Disease Database}]{Projeção das componentes principais no autoespaço dos dados da \textit{Parkinson Disease Database} ~\cite{physionet}}
 \label{fig:projecao_autoespaco}
\end{figure}

\subsection{Resultados}
A projeção dos dados no \textit{autoespaço} tridimensional exibido na Figura \ref{fig:projecaopcaparkinson}, foi realizada por meio das três componentes principais de maior significância. Como ilustrado na projeção dos dados no \textit{autoespaço} na Figura \ref{fig:projecao_autoespaco}, as componentes principais criam um novo eixo de coordenadas contendo uma maior proximidade por meio do reconhecimento das principais características entre este conjunto de dados.

\begin{figure}
 \centering
 \includegraphics[scale=0.45]{./img/projecao-pca-parkinson-healthy.png}
 % matrixargseg.png: 296x162 pixel, 100dpi, 7.52x4.11 cm, bb=0 0 213 117
 %\caption{Estágio desenvolvimento de jogos ~\cite{fullerton2008game}}
\caption{Projeção no autoespaço e visualização da mudança de projeção de 1 Indivíduo por grupo}
%  \caption{Estágio desenvolvimento de jogos}
 \label{fig:projecaopcaparkinson}
\end{figure}

Como pode ser visto na projeção do \textit{autoespaço}, na Figura \ref{fig:projecaopcaparkinson}, existem dois grupos bastante distintos e um indivíduo de cada grupo possui três medições em diferentes momentos. Esses indivíduos foram projetados em cada uma de suas medições, e realizou-se o trajeto da mudança na projeção dos diferentes momentos. Com base nessa figura, pode ser percebido que o indivíduo não diagnosticado com \ac{dp} não sofre grandes alterações na projeção dos estados, mas o individuo classificado como portador da~\ac{dp} teve significativa alteração em dois momentos, projetando-se mais próximo aos indivíduos sem o diagnóstico e em outro momento foi projetado bem distante.

%
%\begin{figure}
 %\centering
 %\includegraphics[scale=0.7]{./img/TestPersonsEigenSpace.png}
 %% matrixargseg.png: 296x162 pixel, 100dpi, 7.52x4.11 cm, bb=0 0 213 117
 %%\caption{Estágio desenvolvimento de jogos ~\cite{fullerton2008game}}
%\caption{Projeção no Autoespaço Grupo de Teste}
%%  \caption{Estágio desenvolvimento de jogos}
 %\label{fig:projeigentest}
%\end{figure}


\subsubsection{Validação Cruzada}\label{sec:validacao_cruzada_database}
Para verificar a precisão da classificação de indivíduos saudáveis perante os parkinsonianos foi aplicada a técnica de Validação Cruzada (cross-validation) ~\cite{datamining2005}. A validação cruzada é uma técnica empregada para avaliar a capacidade de generalização de um modelo de predição em um conjunto de dados. Em um primeiro momento, realiza-se a partição do conjunto de dados em subconjuntos mutualmente exclusivos e ,posteriormente, usa-se este subconjunto para verificar a acurácia do modelo ~\cite{datamining2005}. 
%O conceito central das técnicas de validação cruzada é o particionamento do conjunto de dados em subconjuntos mutualmente exclusivos, e posteriormente, usar alguns destes subconjuntos como dados de treinamento para encontrar valores que permitam a predição. O restante dos subconjuntos serão usados como dados de teste utilizados para avaliar o modelo preditivo.

A escolha do grupo de treinamento deve contemplar todas as classes no conjunto de dados, este procedimento é chamado de estratificação ou \textbf{Validação Cruzada Estratificada}~\cite{datamining2005}. Essa técnica é muito utilizada para mitigar a ocorrência de viés na pesquisa, pois o processo de treinamento é repetido por várias vezes com amostras diferentes incluindo diferentes casos em cada classe. Dada uma amostra única, um método de predição de taxa de erro na aprendizagem de máquina é usar a Validação-Cruzada dez vezes (\textit{10 K-Fold Cross-Validation})~\cite{datamining2005}, onde os dados são aleatórios e divididos em 10 grupos com a mesma proporção de classe. O processo de aprendizagem é executado por 10 vezes onde 1 grupo é selecionado como grupo de teste e os demais são selecionados como grupo de treinamento. A taxa de erro é calculada em cada processo de aprendizagem executado e no final é calculada a taxa de erro global. 

A escolha do número 10 para a quantidade de grupos não foi de forma aleatória, a literatura recomenda que sejam utilizados 10 grupos para obter a melhor estimativa de erro ~\cite{datamining2005}. Contudo, é reconhecido que esse número de 10 não é único ou insubstituível e em determinados casos, grupos de 5, 20 ou quaisquer outros números podem trazer melhores resultados.

Outro dado relevante para a escolha da técnica da estimativa de erro, é que uma única aplicação de Validação Cruzada \textit{10 K-Fold} pode não ser o suficiente para uma taxa de erro confiável. Pois, diferentes experimentos de Validação Cruzada podem produzir resultados distintos devido a natureza aleatória da escolha dos grupos. A estratificação reduz essa variação, contudo não a elimina completamente. Na busca por uma estimativa de erro mais exata, um procedimento que tem se transformado padrão nas técnicas de aprendizagem de máquina é repetir o processo de Validação Cruzada \textit{10 K-Fold} por 10 vezes. Ou seja, consiste em invocar o algoritmo de aprendizagem 100 vezes em um conjunto de dados, dessa forma aumenta-se a amostra de forma considerável e por consequência reduz-se a variação da taxa de erro na execução dos experimentos.

Desta maneira, o método escolhido de validação cruzada foi repetir e calcular por 10 vezes a \textit{10 K-Fold}. Os subjconjuntos são estratificados antes de cada um dos 10 processos de aprendizagem e a taxa de erro estimada em cada etapa processo de aprendizagem está exposta na Figura \ref{fig:erroestimadopca}. A taxa de classificação global é a média da taxa de acerto calculada em todo o processo.

\begin{figure}
 \centering
 \includegraphics[scale=0.6]{./img/boxplot-eigengaits-parkinsondatabase-error-kfold.png}
 % matrixargseg.png: 296x162 pixel, 100dpi, 7.52x4.11 cm, bb=0 0 213 117
 %\caption{Estágio desenvolvimento de jogos ~\cite{fullerton2008game}}
\caption{Erro estimado em cada etapa da aprendizagem usando o método de validação cruzada \textit{10-K-Fold}}
%  \caption{Estágio desenvolvimento de jogos}
 \label{fig:erroestimadopca}
\end{figure}


\begin{figure}
 \centering
 \includegraphics[scale=0.6]{./img/boxplot-eigengaits-parkinsondatabase.png}
 % matrixargseg.png: 296x162 pixel, 100dpi, 7.52x4.11 cm, bb=0 0 213 117
 %\caption{Estágio desenvolvimento de jogos ~\cite{fullerton2008game}}
\caption{Resultado da classificação indivíduos diagnosticados com parkinson versus indivíduos sem o diagnóstico por meio da distância euclidiana no autoespaço}
%  \caption{Estágio desenvolvimento de jogos}
 \label{fig:classificacaopca}
\end{figure}

\subsubsection{Matriz de Confusão e Suas Métricas}
Para avaliar o resultado da classificação, será apresentada a \textbf{matriz de confusão}~\cite{datamining2005}, que permite comparar os valores reais da classe com os valores obtidos no modelo de predição. 

A matriz de confusão para duas classes consiste numa matriz $2$\ x $2$\, contendo os \textit{Verdadeiros Positivos} (\textbf{TP}) e \textit{Verdadeiro Negativo} (\textbf{TN}), que são as classificações corretas. Os \textit{Falsos Negativos} (\textbf{FN}) contém a predição incorreta de um valor que deveria ser positivo e os \textit{Falsos Positivos} (\textbf{FP}) contém os valores positivos quando deveriam ser negativos como pode ser visto na Tabela~\ref{table:descricaomatrizconfusao}.

% Please remember to add \use{multirow} to your document preamble in order to suppor multirow cells
\begin{table}[!htbp]
\caption{Descrição da Matriz de Confusão}
\label{table:descricaomatrizconfusao}
\begin{tabular}{ll|c|c|}
\cline{3-4}
                                                                                                               & \multicolumn{1}{c}{}                         & \multicolumn{2}{|c|}{\textit{\textbf{Classe Preditiva}}} \\ \cline{3-4} 
                                                                                                               &                                              & \textbf{Parkinson}          & \textbf{Não Parkinson}     \\ \hline
\multicolumn{1}{|c}{\multirow{\textit{\textbf{\begin{tabular}[c]{@{}c@{}}Classe\\ Atual\end{tabular}}}}} & \multicolumn{1}{|l|}{\textbf{Parkinson}}     & Verdadeiros Positivos (VP)  & Falsos Negativos (FN)      \\ \cline{2-4} 
\multicolumn{1}{|l}{\textit{\textbf{}}}                                                                        & \multicolumn{1}{|l|}{\textbf{Não Parkinson}} & Falsos Positivo (FP) & Verdadeiros Negativos (VN) \\ \hline
\end{tabular}
\end{table}

A matriz de confusão é uma ferramenta importante para avaliar os resultados de predição, pois facilita o entendimento do que está sendo avaliado e como se comporta o classificador em relação aos erros de classificação obtidos. Esta matriz serve como base para métricas que podem ser aplicadas a classificação e,  consequentemente, exibem a precisão do modelo. A matriz desta pesquisa (Tabela ~\ref{table:resultadomatrizconfusaopca}) foi gerada a partir da repetição de dez vezes da técnica de Validação Cruzada \textit{10-K-Fold}, conforme a Seção \ref{sec:validacao_cruzada_database}.

\begin{table}[!htbp]
\caption{Resultado da Matriz de Confusão}
\label{table:resultadomatrizconfusaopca}
\centering
\begin{tabular}{l|c|c|}
\cline{2-3}
\multicolumn{1}{c}{}                         & \multicolumn{2}{|c|}{\textit{\textbf{Classe Preditiva}}} \\ \cline{2-3} 
                                             & \textbf{Parkinson}      & \textbf{Não-Parkinson}         \\ \hline
\multicolumn{1}{|l|}{\textbf{Parkinson}} & 429       & 71           \\ \hline
\multicolumn{1}{|l|}{\textbf{Não Parkinson}}     & 114           & 386     \\ \hline
\end{tabular}
\end{table}

Para demonstrar a avaliação do modelo de forma quantitativa usou-se um conjunto de métricas derivadas da matriz de confusão ~\cite{datamining2005}.
\begin{description}
	\item [\textit{TpRate}] taxa de acerto obtido: $ TpRate = TP/P $\ ;
	\item [\textit{FpRate}]: taxa de falso alarme obtido: $ FpRate = FP/N $\ ;
	\item [\textit{Precision}]: taxa de acerto de uma instância em determinada classe: $ Precision =  TP/(TP +FP) $\ ;
	\item [\textit{Accuracy}]: taxa de acerto de todo o classificador: $ Accuracy = (TP+TN)/(P+N) $\ ;
	\item [\textit{F-Measure}]: análise de classificador binário que mede a acurácia do teste. Considerando a média harmônica da taxa de \textit{precision} e do \textit{tp rate}: $ F-Measure = 2 * (Precision * TpRate)/(Precision + TpRate) $\ .
\end{description}

\begin{table}[!htbp]
\caption{Métricas da Matriz de Confusão}
\label{table:metricasmatrizconfusaosvm}
\centering
\begin{tabular}{|l|r|}
\hline
\multicolumn{2}{|l|}{\textbf{Métricas}} \\ \hline
\textbf{TpRate}                    & 85,80$\%$\                 \\ \hline
\textbf{FpRate}                    & 22,80$\%$\                \\ \hline
\textbf{Precision}                 & 79,01$\%$\                \\ \hline
\textbf{Accuracy}                  & 81,50$\%$\                \\ \hline
\textbf{F-Measure}                 & 82,26$\%$\                \\ \hline
\end{tabular}
\end{table}

%Comparar Resultados com http://www.medcalc.org/calc/diagnostic_test.php

%Tabela Gerada no http://www.tablesgenerator.com/latex_tables#

%\begin{table}[h]
%\begin{tabular}{l|c|c|}
%\cline{2-3}
%\multicolumn{1}{c}{}                         & \multicolumn{2}{|c|}{\textit{\textbf{Classe Preditiva}}} \\ \cline{2-3} 
                                             %& \textbf{Não Parkinson}      & \textbf{Parkinson}         \\ \hline
%\multicolumn{1}{|l|}{\textbf{Não Parkinson}} & Verdadeiros Positivos       & Falsos Negativos           \\ \hline
%\multicolumn{1}{|l|}{\textbf{Parkinson}}     & Falsos Positivos            & Verdadeiros Negativos      \\ \hline
%\end{tabular}
%\label{table:matrizconfusao}
%\caption{Matriz de Confusão}
%\end{table}


\subsection{Limitações do Método}
O movimento da marcha pode variar de pessoa para pessoa e, atualmente, não existe diagnóstico comprobatório da \ac{dp} ~\cite{visionbased2009,protpar010}. O diagnóstico fica a critério do julgamento clínico do profissional de saúde. Contudo, a presente abordagem permite identificar padrões característicos da marcha de indivíduos considerados saudáveis, ante aos classificados como portadores da \ac{dp} na "\textit{Parkinson´s Disease Database}" ~\cite{physionet}.


\section{H2 - Estudo de Caso 2: Máquina de Vetor de Suporte para Estudo Analítico de Caso Controle Por Intermédio de Sensor de Movimento Usados em Jogos Eletrônicos}\label{sec:resultado_svm}

Partindo da importância de identificar o sintoma da bradicinesia e, consequentemente, avaliar a dificuldade do movimento (Seção ~\ref{section:analise_bradicinesia}), nessa pesquisa buscou-se avaliar esse sintoma com o movimento de adução e abudção dos braços (ver Figura ~\ref{fig:movabducaomet}). A abordagem de aprendizagem de máquina foi utilizada para classificar portadores da doença de parkinson ante indivíduos sem o diagnóstico. Partiu-se do princípio que os indivíduos com~\ac{dp} teriam mais dificuldade ao levantar o Braço e a velocidade angular do mesmo seria reduzida ante os indivíduos que não desenvolveram a doença.

\subsection{Estudo analítico de caso-controle}\label{section:estudo_caso_controle}
Esta etapa da pesquisa foi pautada pelo protocolo de pesquisa submetido à avaliação do Comitê de Ética da UFCG. Somente após a aprovação deste (\textbf{CAAE: 14408213.9.1001.5182}) é que os dados foram coletados. 

Os resultados que se pretendem alcançar com a pesquisa são mecanismos para a identificação e classificação de pessoas saudáveis ante os portadores de doença de parkinson. Durante a pesquisa também analisou-se o sensor de movimento MS-Kinnect ~\cite{kinnect2013} para avaliar a possibilidade de aquisição de dados de saúde baseada na Cinemática Linear do Movimento Humano~\cite{mcginnis2013biomechanics}. A partir dos resultados obtidos, pretende-se avaliar e classificar a normalidade e dificuldade na execução de movimentos como, por exemplo, levantar um braço~\cite{mcginnis2013biomechanics}.

A coleta de dados foi realizada no Hospital Universitário da UFAL, e na Fundação Pestalozzi em Maceió, ambas sob a tutela da Profa. e Neurologista Dra. Cícera Pontes; e na Clínica de Fisioterapia do CESMAC, sob a tutela do Prof. de Fisioterapia Jean Charles Santos. As coletas foram realizadas em local reservado e de forma individual, com a anuência do sujeito pesquisado através da assinatura do Termo de Consentimento.

\subsubsection{Amostra}
A técnica de amostragem utilizada para seleção, foi por conveniência, composta por 15 indivíduos portadores de ~\ac{dp} e 12 sem o diagnostico, como grupo controle.

\subsubsection{Recrutamento dos Sujeitos e Aquisição do Consentimento Livre e Esclarecido}
A forma de recrutamento deste protocolo será circunscrita por intermédio de um profissional de saúde. O profissional conhecia a história clínica do paciente e obteve a permissão do mesmo para que a equipe de pesquisa entrasse em contato. A equipe de pesquisa explicitou os riscos e benefícios da participação da pesquisa buscando a arbitrariedade e espontaneidade da decisão. Depois foi oferecido para assinatura o Termo de Consentimento Livre e Esclarecido.

\subsubsection{Critérios de Inclusão}
Foram inclusos na pesquisa, os indivíduos do grupo diagnosticados com \ac{dp} no estágio 3 segundo a UPDRS ~\cite{updrs87}, sem distinção de gênero ou cor. Os indivíduos, ficaram dentro das facilidades da clínica onde a coleta foi realizada e aceitaram participar do estudo. O grupo de indivíduos que não possuíam diagnóstico da \ac{dp}, informaram que nunca receberam o diagnóstico da doença e que aceitariam participar do estudo como grupo controle.

\subsubsection{Critérios de Exclusão}
Foram excluídos das pesquisas os indivíduos com sintomas motores e que tivessem problemas de equilíbrio ou questionamento de dores ao executar os procedimentos. Foram excluídos também, o indivíduo que por qualquer motivo se negou a participar do estudo.

\subsubsection{Materiais}
Para a presente pesquisa, foram coletados movimentos de abdução e adução dos braços~\cite{mcginnis2013biomechanics}, os quais poderiam ser incorporados a um jogo eletrônico. Foi utilizado um jogo com o arcabouço de software de captura de dados desenvolvido por um aluno de Mestrado da Universidade Federal de Campina Grande ~\cite{antonio2013}, juntamente a um aluno de iniciação científica do Instituto Federal de Alagoas. 

Durante a execução da coleta, houve uma preocupação com a integridade física dos participantes. Então, os movimentos utilizados no jogo foram apenas de adução e abdução dos braços~\cite{mcginnis2013biomechanics}, proporcionado a segurança dos participantes. 

%\subsubsection{Infra-Estrutura}
%A pesquisa será realizada na Clínica em que o paciente está em tratamento, onde são realizados tratamentos fisioterápicos ou consultas. O espaço físico forneceu condições favoráveis e adequadas para aplicação dos jogos. Para a realização da pesquisa foram utilizados:
%
%\begin{itemize}
	%\item Jogo (\textit{Catch the Spheres} rodando em notebook com Sistema Operacional Windows 7.0 e Unity 3d 3.0;
	%\item Projetor (Epson Lcd Powerlite X14 3000l Hdmi) para projetar o jogo na parede e facilitar a visualização;
	%\item Sensor de movimento Ms-Kinnect ~\cite{kinnect2013}.
%\end{itemize}

\subsubsection{Métodos}
Nesta pesquisa foi realizada uma análise de um sensor de movimento utilizado em jogos eletrônicos, e avaliada a possibilidade de aquisição de dados de saúde baseada na Cinemática Angular do Movimento Humano ~\cite{hamill1999bases}.  Através dos resultados obtidos avaliou-se a possibilidade de classificar a normalidade e dificuldade na execução de movimentos como abdução e adução dos braços.

A coleta de dados foi realizada no próprio espaço de tratamento do indivíduo em local reservado e de forma individual. A participação do indivíduo foi consentida por meio da assinatura do Termo de Consentimento. Devido as restrições de tempo (1 minuto e 30 segundos) e da execução de um mesmo movimento por todos os participantes, foram solicitados dos voluntários a execução dos seguintes procedimentos:
\begin{enumerate}
	\item O voluntário se posiciona a uma distância de 2 metros do sensor de movimento, de modo a conseguir capturar toda a extensão superior do braço durante o movimento de abdução; 	
	\item O voluntário inicia o jogo \textit{Catch the Spheres} usando a mão esquerda conforme a interface da aplicação;
	\item O voluntário abduz e aduz 10 vezes o braço esquerdo, e depois o braço direito o mais alto e o mais rápido que consegue, de modo a permitir que fossem capturadas a amplitude de movimento e a velocidade angular do mesmo. 
	\item O voluntário fecha a aplicação e esta realiza o armazenamento dos dados.
\end{enumerate}

\begin{figure}[!htbp]
 \centering
 \includegraphics[scale=0.5]{./img/capturaducaokinnect.png}
 % matrixargseg.png: 296x162 pixel, 100dpi, 7.52x4.11 cm, bb=0 0 213 117
 %\caption{Estágio desenvolvimento de jogos ~\cite{fullerton2008game}}
\caption{Movimentos de Abdução e Adução}
%  \caption{Estágio desenvolvimento de jogos}
 \label{fig:movabducaomet}
\end{figure}

Durante a análise foram comparados os Ângulos Relativos do Tronco e do Levantamento de Braços dos Indivíduos. As grandezas cinemáticas coletadas nesses estudo foram:
\begin{enumerate}
	\item Do movimento de Abdução, a máxima amplitude atingida pelos membros superiores;
	\item Velocidade Angular de Abdução membros esquerdo e direito;
	\item Velocidade Angular de Adução membros esquerdo e direito.
\end{enumerate}

Os dados biomecânicos~\cite{hamill1999bases} coletados tiveram o objetivo de identificar a bradicinesia nos indivíduos diagnosticados com a \ac{dp} ante os indivíduos sem o diagnóstico estabelecido. Pelo quantitativo da pesquisa ter sido de 27 indivíduos, a abordagem de aprendizagem de máquina usando \ac{svm} ~\cite{vapnik95} foi utilizada juntamente com a técnica de validação-cruzada \textit{leave-one-out}, essa técnica será explicada com mais detalhes na Seção \ref{validacao_cruzada_svm}.


\subsubsection{Relação Risco e Benefício da Pesquisa}
Os riscos inerentes podem decorrer da exposição de dados dos sujeitos da pesquisa, o que pode acarretar danos morais e/ou psicológicos. Logo, teve-se um cuidado de preservar a integridade física e psicológica dos sujeitos da pesquisa, garantindo assim, a privacidade e confidencialidade das informações.

Caso houvesse algum constrangimento por parte do sujeito da pesquisa, ao não conseguir realizá-la ou responder alguma pergunta devido ao comprometimento da doença, os pesquisadores prestaram total assistência, orientando-o adequadamente para prosseguir ou encerrar o procedimento.

%Os presentes riscos fazem jus aos benefícios que a pesquisa venha a trazer com a possibilidade de monitoramento dos sintomas da \ac{dp}. A identificação dos sintomas motores e classificação desses dados através do computador podem permitir avanços para um melhor acompanhamento da evolução da doença além de possibilitar que os pacientes venham a ser monitorados de forma não invasiva através de um jogo eletrônico. Os pacientes deverão ter o seu estágio da \ac{dp} previamente diagnosticada por um médico para ser possível comparar os dados do monitoramento com a classificação obtida.




\subsection{Aplicação do Método}
O propósito da classificação é explorar a possibilidade de obter dados de saúde de forma contínua e não invasiva a partir de um sensor de captura de movimento usado em jogos eletrônicos (Ms-Kinnect). Durante a coleta dos dados foi indagado junto aos voluntários sua condição física e possíveis riscos e desconfortos que o voluntário pudesse ter ao realizar o procedimento. 

Durante a pesquisa, partiu-se do princípio, que através da análise do movimento de abdução e adução do braço, seria possível avaliar a biomecânica da amplitude do movimento dos braços e velocidade angular dos mesmos. Então, por intermédio desses dados biomecânicos, seria possível identificar a ocorrência do sintoma de bradicinesia em indivíduos portadores da \ac{dp}.


\subsection{Resultados}
Conforme a abordagem GAHME apresentada no Capítulo~\ref{chapter:abordagem_gahme}, os dados adquiridos foram processados, extraídas as características do movimento angular, filtrados e postos em uma Máquina de Vetor de Suporte, para realizar a classificação entra as duas classes de dados. Para a classificação dos dados foi utilizado um \textit{kernel} Linear (Seção~\ref{sec:svm_linear}) por ter obtido os melhores resultados dentre os \textit{kernels} presentes no Matlab ~\cite{matlab2011}: Polinomial, Radial e de MLP. O resultado do \textit{kernel} linear foi o mais expressivo entre os demais devido a separação linear ter dividido bem as duas classes. Em um trabalho futuro pode-se fazer um estudo de regressão linear nos dados para comprovar essa hipótese.

\subsubsection{Vetor Médio}
Nessa etapa da pesquisa foi calculado o Vetor Médio (Seção~\ref{section:filtro_dados}), para entender melhor a diferença de movimento entre os sujeitos diagnosticados com a \ac{dp} e sujeitos sem o diagnóstico. Como pode ser visto na Figura~\ref{fig:vetor_medio_abducao}, a amplitude de movimento de um indivíduo diagnosticado com a ~\ac{dp} é bem menor do que a de um indivíduo sem o diagnóstico. Entretanto, por ter sido escalonado em 20 \textit{frames}, esse vetor médio perdeu a informação da velocidade do movimento.

\begin{figure}[!htbp]
 \centering
 \includegraphics[scale=0.50]{./img/vetormedioaducao.png}
 \caption{Vetor Médio do Movimento de Abdução e Adução}
 \label{fig:vetor_medio_abducao}
\end{figure}


\subsubsection{Validação Cruzada}\label{validacao_cruzada_svm}
Para uma base de dados pequena contendo 27 indivíduos, o método de Validação Cruzada escolhido deve tentar maximizar o conjunto de treinamento para atingir um melhor resultado de teste. Por esse motivo, foi escolhido o método de validação cruzada \textit{leave-one-out}. 

O método \textit{leave-one-out} é um método de validação cruzada \textit{k-fold} com o mesmo número de \textit{n} indivíduos. Logo, apenas um indivíduo será considerado teste e os demais serão de treinamento. Desta maneira não existe estratificação nos dados, tornando o processo determinístico e repetível com a mesma base de dados, pois não existe o problema de viés na seleção dos dados. A taxa de erro obtida da classificação é a taxa de erro do modelo para aquela base de dados. 

\subsubsection{Matriz de Confusão e Suas Métricas}
A Matriz de Confusão obtida indica que existem três indivíduos classificados como ``Não-Parkinson'' e que possuem a doença. Pode ser visto na Figura~\ref{fig:ciclos_movimento_processados_filtrados}, que alguns ciclos de indivíduos classificados como Parkinsonianos possuem amplitude bastante semelhante aos do indivíduos ``Não-Parkinsonianos''. Logo, estes não apresentam o sintoma de bradicinesia, o que pode indicar que o indivíduo esteja no início da doença, ou bem medicado, ou até mesmo não apresenta este sintoma motor. 
Um resultado, que não era esperado nesta pesquisa é a ocorrência de indivíduos que não possuem a doença de parkinson mas mesmo assim foram classificados com o sintoma. Isso requer um melhor estudo para identificar o que pode ter ocorrido na classificação.

\begin{table}[!htbp]
\caption{Resultado da Matriz de Confusão SVM}
\label{table:resultadomatrizconfusaosvm}
\centering
\begin{tabular}{l|c|c|}
\cline{2-3}
\multicolumn{1}{c}{}                         & \multicolumn{2}{|c|}{\textit{\textbf{Classe Preditiva}}} \\ \cline{2-3} 
                                             & \textbf{Parkinson}      & \textbf{Não-Parkinson}         \\ \hline
\multicolumn{1}{|l|}{\textbf{Parkinson}} & 12       & 3           \\ \hline
\multicolumn{1}{|l|}{\textbf{Não Parkinson}}     & 2           & 10     \\ \hline
\end{tabular}

\end{table}



\begin{table}[!htbp]
\label{table:metricasmatrizconfusao}
\caption{Métricas da Matriz de Confusão}
\centering
\begin{tabular}{|l|r|}
\hline
\multicolumn{2}{|l|}{\textbf{Métricas}} \\ \hline
\textbf{TpRate}                    & 80,00$\%$\                 \\ \hline
\textbf{FpRate}                    & 16,67$\%$\                \\ \hline
\textbf{Precision}                 & 85,71$\%$\                \\ \hline
\textbf{Accuracy}                  & 81,48$\%$\                \\ \hline
\textbf{F-Measure}                 & 82,76$\%$\                \\ \hline
\end{tabular}
\end{table}


\subsection{Limitações do Método}
O método utilizado para diferenciar os movimentos executados de indivíduos diagnosticados com a \ac{dp} ante os indivíduos sem o diagnóstico estabelecido, foi uma técnica estatística de aprendizagem denominada de \ac{svm}. Nesse estudo não se pretende estabelecer um diagnóstico da \ac{dp}, ou até mesmo provar que os movimentos utilizados pelos participantes da pesquisa servem para um diagnóstico. Contudo, este trabalho demonstra que existem diferenças entre essas duas classes, e estas podem ser capturadas por um sensor de movimento usado em jogos eletrônicos, e que essas diferenças podem ser classificadas utilizando uma abordagem de aprendizagem de máquina. 

\section{H3 - \textit{Goal Question Metric} Com Usuários Participantes da Pesquisa}\label{gqm_usuarios}
%Para identificar a possibilidade de integrar o monitoramento da saúde do jogador através de jogos eletrônicos à sua rotina diária, foi utilizada a abordagem \textit{Goal, Question, Metric} (GQM). GQM ~\cite{basili94} é uma abordagem hierárquica que inicia com objetivo principal e o divide em atividades que podem ser mensuradas durante a execução do projeto. É uma abordagem para integrar objetivos a e perspectivas de qualidade de interesse, baseado nas necessidades do projeto~\cite{van1999goal}. Foi preparado o questionário GQM mostrado no Apêndice~\ref{apend:gqm} para avaliar a possibilidade de monitorar dados motores de forma não invasiva e integrada a rotina diária das pessoas.

Com o objetivo de averiguar a possibilidade de integrar o monitoramento da saúde do jogador através de jogos eletrônicos à sua rotina diária, foi utilizada a abordagem \textit{Goal, Question, Metric} (GQM)~\cite{van1999goal}. Essa abordagem é um paradigma de pesquisa utilizado na Engenharia de Software para medição de processos de software e melhoria contínua dos produtos ~\cite{saraiva2006,elicquest05}. A qualidade do produto de software ~\cite{saraiva2006} pode ser compreendida como a adequação a um conjunto de características atingidas em maior ou menor grau para que o produto final venha atender as necessidades do usuário final, identificadas na fase de elicitação de requisitos ~\cite{elicquest05}.

O ~\ac{gqm} é um paradigma de avaliação orientado por metas e tem como componentes elementares: objetivos, questionamentos e a métricas ~\cite{saraiva2006}. Nesse paradigma de pesquisa é definido um objetivo principal, onde são refinados em perguntas que venham extrair as métricas da pesquisa que fornecem informações. De posse das respostas baseada em métricas, estas são comparadas com o objetivo da pesquisa no intuito de identificar se ele foi alcançado. Logo, o paradigma ~\ac{gqm} busca definir métricas partindo de uma perspectiva de ``de cima para baixo''; analisa, interpreta e mensura dados de maneira ``de baixo para cima'' como pode ser graficamente visualizado na Figura~\ref{fig:gqm} ~\cite{van1999goal}. 

\begin{figure}[!htbp]
 \centering
 \includegraphics[scale=0.50]{./img/gqm.png}
 \caption[O Paradigma GQM \copyright]{O Paradigma GQM \copyright~\cite{van1999goal}}
 \label{fig:gqm}
\end{figure}

Segundo Saraiva ~\cite{saraiva2006}, numa análise da aplicação do método de ~\ac{gqm} para o contexto de avaliação de usabilidade de software, os componentes elementares do paradigma ~\ac{gqm} são:

\begin{itemize}
	\item \textbf{Objetivo}: Sua definição envolve o propósito da avaliação, o que deve ser avaliado, a perspectiva e o ambiente proposto.
	\item \textbf{Questão}: A questão anuncia a necessidade de se obter informações em linguagem natural, podendo formular uma ou mais questões para cada categoria. Logo, sua resposta deve estar condicionada ao objetivo proposto.
	\item \textbf{Métrica}: Sua função é especificar os dados que se deseja obter durante as avaliações em termos quantitativos, podendo ter mais de uma métrica para cada questão.	
\end{itemize}

Baseado nos componentes elementares do paradigma, foi elaborado um questionário~\ac{gqm} (Apêndice~\ref{apend:gqm}) com o objetivo principal de avaliar a possibilidade de monitorar dados motores, de forma não invasiva e integrada a rotina diária dos usuários. Para elaboração de métricas para atingir esse objetivo foram formuladas duas questões de pesquisa com o intuito de avaliar:
\begin{enumerate}
	\item se o usuário integraria a abordagem GAHME à sua rotina diária;
	\item se a segurança com a integridade física está de acordo com a faixa etária do usuário.
\end{enumerate}

O questionário consistiu de um conjunto de 10 questões de resposta fechada (quantitativa) ~\cite{elicquest05}, e o entrevistado deve escolher uma resposta dentre as alternativas dadas. Esse método foi escolhido para contribuir por uma maior uniformidade nas respostas e consequentemente facilitar sua análise. Porém, este método impede a expressão das opiniões dos entrevistados~\cite{elicquest05}. 

%\begin{table}
%\begin{longtable}{|p{\textwidth}|}
%\caption{O Questionário GQM}\\
%\label{table:table_gqm}
%\hline
%\endfirsthead
%\multicolumn{1}{c}%
%{\tablename\ \thetable\ -- \textit{Continuação da página anterior}} \\
%\hline
%\endhead
%\hline \multicolumn{1}{r}{\textit{Continua na próxima página}} \\
%\endfoot
%\hline
%\endlastfoot
%\textbf{\textit{Objetivo principal}}: Avaliar a possibilidade de monitorar dados motores de forma não invasiva e integrada a rotina diária das pessoas. \\ \hline
%\textbf{\textit{Questão 1}}: O usuário poderia integrar a abordagem GAHME à sua rotina diária ?\\ \hline
%\textit{Métrica 1.1}: Numa escala de 1 a 5 qual o grau de diversão do jogo? \\ \hline
%\textit{Métrica 1.2}: O jogo traz motivação ao usuário ? (Sim/Não) \\ \hline
%\textit{Métrica 1.3}: Se o usuário tivesse adquirido esse jogo, com que frequência o utilizaria durante a semana? (1 vez/3 vezes/Todos os dias/Nunca usaria) \\ \hline
%\textit{Métrica 1.4}: O usuário considera o jogo simples, sem muitas regras e de fácil entendimento ? Ele pode ser aplicado em diferentes idades? (Sim/ Não) \\ \hline
%\textit{Métrica 1.5}: O usuário tem o costume de jogar esses jogos casuais em casa? (Sim/ Não) \\ \hline
%\textit{Métrica 1.6}: O usuário agregaria um jogo desse estilo em sua rotina diária? (Sim/ Não) \\ \hline
%\textbf{\textit{Questão 2}}: A segurança com a integridade física está de acordo com a faixa etária do usuário ? \\ \hline
%\textit{Métrica 2.1}: Uma criança estaria segura jogando esse jogo, ao efetuar os movimentos dos braços? \\ \hline
%\textit{Métrica 2.2}: Um adulto estaria seguro ao jogar esse jogo, ao efetuar os movimentos dos braços? \\ \hline
%\textit{Métrica 2.3	}: Um idoso estaria seguro ao jogar esse jogo, ao efetuar os movimentos dos braços? \\ \hline
%\textit{Métrica 2.4}: Qual opinião do usuário sobre a faixa etária do jogo? (Livre/Crianças/Adultos/Idosos) \\ \hline
%\end{longtable}
%\end{center}
%\end{table}


\subsection{Aplicação do Método}
%A abordagem de ~\ac{GQM}, tem o intuito de auxiliar na elaboração de planos de avaliação de atributos de qualidade de produtos e processos de software ~\cite{basili94}. Sabemos, que essa abordagem é muito mais empregada na avaliação de processos, porém nessa pesquisa iremos invocar a técnica para a avaliação do produto do ponto de vista do usuário. Essa abordagem, permite identificar métricas apropriadas ao contexto e objetivos da avaliação de modo a facilitar a interpretação e análise dos dados orientados por metas.
%
%Essa etapa da pesquisa teve como objetivo avaliar se os possíveis usuários do sistema iriam aprovar o monitoramento motor usando jogos eletrônicos. Essa avaliação vai ao encontro do usuário final e principal beneficiado pela abordagem apresentada nessa proposta de Tese.

%Essa etapa da pesquisa foi realizada em colaboração com o trabalho de Santos Jr. ~\cite{antonio2013}, que desenvolveu também o mecanismo de captura dos dados Seção~\ref{sec:cliente_game}.  Tendo como objetivo avaliar se possíveis usuários do sistema iriam aprovar o monitoramento motor usando jogos eletrônicos. Essa avaliação vai ao encontro do usuário final e principal beneficiado pela abordagem apresentada nessa proposta de Tese.

Nessa etapa da pesquisa foram entrevistadas 24 pessoas do Laboratório Embedded da Universidade Federal de Campina Grande, do Instituto Federal de Alagoas, e em pacientes da Fundação Pestalozzi e da clínica de Fisioterapia do CESMAC (ambas em Maceió). Os usuários foram selecionados para jogar o \emph{Catch the Spheres} (Seção ~\ref{jogo_catch}), testaram e responderam o questionário para validar a Hipótese \textit{H3} deste trabalho. 

%Buscou-se também validar a viabilidade da Arquitetura de Software do \textit{GAHME}, verificando a possibilidade de capturar dados motores. 
O procedimento para executar as sessões de teste foi:
\begin{enumerate}
	\item O usuário se posicionou a uma distância de 2 metros do sensor de movimento, de modo a adquirir toda a extensão superior do braço; 	
	\item O usuário iniciou o jogo \textit{Catch the Spheres}, conforme a interface da aplicação;
	\item O usuário utilizou o jogo \textit{Catch the Spheres} por volta de 00:01:30s;
	\item O usuário fechou a aplicação.
\end{enumerate}

\subsection{Resultados}
Os resultados do questionário são apresentados na Tabela ~\ref{table:resultados_gqm} contendo as respostas binárias ``Sim/Não'' e nas Figuras~\ref{fig:question1},\ref{fig:question3},\ref{fig:question10} nas perguntas de questões com múltipla escolha.

\textit{Questão 1 - O usuário poderia integrar a abordagem GAHME à sua rotina diária ?}: os 24 usuários deram as seguintes respostas nas Métricas (1.1, 1.2, 1.3,1.4,1.5 e 1.6): 75\% dos usuário atribuíram ao menos nota 4 (de 1 a 5) ao grau de diversão do jogo; 91,67\% sentiram motivados com o jogo; 58\% dos usuários jogariam 3 vezes por semana, 25\% jogariam todos os dias e apenas 17\% jogariam uma vez por semana. 

Então, tem-se um percentual de 83\% de usuários que poderiam integrar o monitoramento motor a sua rotina; 91,67\% consideraram o jogo simples e de fácil entendimento, e isso permite o uso de um maior número de usuários. Uma métrica desfavorável foi que apenas 41,67\% dos usuários possuem o costume de usar jogos casuais em casa. Mas, devido a expectativa de melhora do estado de saúde, 75\% dos usuários responderam que agregariam o jogo a sua rotina diária.

\textit{Questão 2 - A segurança com a integridade física está de acordo com a faixa etária do usuário
?}: nesta questão percebe-se uma grande preocupação dos usuários quanto a risco de quedas. Inicialmente, a pesquisa seria destinada para o movimento de braços e pernas. Devido aos riscos, foi modificada para movimentação somente dos braços, reduzindo a preocupação dos usuários. Mesmo assim, as métricas obtidas demonstraram que o jogo é seguro para crianças e adultos. No caso dos idosos, 75\% dos usuários consideraram o jogo seguro para essa faixa etária, muito embora os mesmos usuários classificaram o jogo com a faixa etária ``livre'', com 88\% de ocorrência.

% Please remember to add \use{multirow} to your document preamble in order to suppor multirow cells
\begin{table}[h]
\caption{Métricas Avaliadas do \textit{GQM}}
\centering
\begin{tabular}{|p{10cm}|p{1.2cm}|p{1.2cm}|}
\hline
\textbf{Métrica} & \textbf{Sim} & \textbf{Não} \\ \hline
1.2: O jogo traz motivação ao usuário? & 91,67\% & 8,33\% \\ \hline
1.4: O usuário considera o jogo simples, sem muitas regras e de fácil entendimento? Ele pode ser aplicado em diferentes idades? & 91,67\% & 8,33\% \\ \hline
1.5: O usuário tem o costume de jogar esses jogos casuais em casa? & 41,67\% & 58,33\% \\ \hline
1.6: O usuário agregaria um jogo desse estilo em sua rotina diária? & 75\% & 25\% \\ \hline
2.1: Uma criança estaria segura jogando esse jogo, ao efetuar os movimentos dos braços? & 100\% & 0\% \\ \hline
2.2: Um adulto estaria seguro ao jogar esse jogo, ao efetuar os movimentos dos braços? & 100\% & 0\% \\ \hline
2.3: Um idoso estaria seguro ao jogar esse jogo, ao efetuar os movimentos dos braços? & 75\% & 25\% \\ \hline
\end{tabular}
\label{table:resultados_gqm}
\end{table}

\begin{figure}[!htb]
     \centering
     \includegraphics[scale=0.7]{./img/chart_1-.png}
     \caption{Resultado da Pergunta 1}
     \label{fig:question1}
\end{figure}


\begin{figure}[!htb]
     \centering
     \includegraphics[scale=0.7]{./img/chart_3-.png}
     \caption{Resultado da Pergunta 3}
     \label{fig:question3}
\end{figure}


\begin{figure}[!htb]
     \centering
     \includegraphics[scale=0.7]{./img/chart_10-.png}
     \caption{Resultado da Pergunta 10}
     \label{fig:question10}
\end{figure}
\FloatBarrier




%\begin{figure}
        %\centering
        %\begin{subfigure}[b]{0.3\textwidth}
                %\centering
                %\includegraphics[width=\textwidth]{./img/chart_1.png}
                %\caption{Pergunta 1 (1: 0\%; 2: 6\%; 3: 28\%; 4: 61\%; 5: 6\%)}
                %\label{fig:question1}
        %\end{subfigure}%
        %~ %add desired spacing between images, e. g. ~, \quad, \qquad etc.
          %%(or a blank line to force the subfigure onto a new line)
        %\begin{subfigure}[b]{0.3\textwidth}
                %\centering
                %\includegraphics[width=\textwidth]{./img/chart_2.png}
                %\caption{Pergunta 2 (Sim: 67\%; Não: 33\%)}
                %\label{fig:question2}
        %\end{subfigure}
        %~
        %\begin{subfigure}[b]{0.3\textwidth}
                %\centering
                %\includegraphics[width=\textwidth]{./img/chart_3.png}
                %\caption{Pergunta 3 (1 vez: 28\%; 3 vezes: 61\%; Todos os dias: 11\%; Nunca: 0\%)}
                %\label{fig:question3}
        %\end{subfigure}
%
        %\begin{subfigure}[b]{0.3\textwidth}
                %\centering
                %\includegraphics[width=\textwidth]{./img/chart_4.png}
                %\caption{Pergunta 4 (Sim: 100\%)}
                %\label{fig:question4}
        %\end{subfigure}
        %~
        %\begin{subfigure}[b]{0.3\textwidth}
                %\centering
                %\includegraphics[width=\textwidth]{./img/chart_5.png}
                %\caption{Pergunta 5 (Sim: 94\%; Não: 6\%)}
                %\label{fig:question5}
        %\end{subfigure}
        %~
        %\begin{subfigure}[b]{0.3\textwidth}
                %\centering
                %\includegraphics[width=\textwidth]{./img/chart_6.png}
                %\caption{Pergunta 6 (Sim: 44\%; Não: 56\%)}
                %\label{fig:question6}
        %\end{subfigure}
%
        %\begin{subfigure}[b]{0.3\textwidth}
                %\centering
                %\includegraphics[width=\textwidth]{./img/chart_7.png}
                %\caption{Pergunta 7 (Sim: 100\%)}
                %\label{fig:question7}
        %\end{subfigure}
        %~
        %\begin{subfigure}[b]{0.3\textwidth}
                %\centering
                %\includegraphics[width=\textwidth]{./img/chart_8.png}
                %\caption{Pergunta 8 (Sim: 100\%)}
                %\label{fig:question8}
        %\end{subfigure}
        %~
        %\begin{subfigure}[b]{0.3\textwidth}
                %\centering
                %\includegraphics[width=\textwidth]{./img/chart_9.png}
                %\caption{Pergunta 9 (Sim: 72\%; Não: 28\%)}
                %\label{fig:question9}
        %\end{subfigure}
%
        %\begin{subfigure}[b]{0.3\textwidth}
                %\centering
                %\includegraphics[width=\textwidth]{./img/chart_10.png}
                %\caption{Pergunta 10 (Livre: 80\%; Crianças: 10\%; Adultos: 10\%; Idosos: 0\%)}
                %\label{fig:question10}
        %\end{subfigure}
        %~
        %\begin{subfigure}[b]{0.3\textwidth}
                %\centering
                %\includegraphics[width=\textwidth]{./img/chart_11.png}
                %\caption{Pergunta 11 (Sim: 17\%; Não: 83\%)}
                %\label{fig:question11}
        %\end{subfigure}
        %\caption{Gráficos das Respostas do questionário}\label{result}
%\end{figure}

\subsection{Conclusão} 
Nessa etapa, buscou-se validar a Hipótese \textit{H3 - É possível desenvolver um jogo que tenha mecanismos de captura de dados motores embutidos e que permita monitorar e quantificar esses dados de maneira não-invasiva.} 

Apesar do questionário ter avaliado a opinião dos jogadores quanto ao jogo apresentado, pode-se generalizar que as opiniões são válidas para outros jogos usando a abordagem \textit{GAHME}. Deve-se levar em consideração, também, que as métricas obtidas nessa pesquisa foram extraídas de um jogo na fase de protótipo. Caso ele fosse aperfeiçoado é possível que sua aceitabilidade seria ainda maior. Por esse motivo, o resultado obtido com a pesquisa~\ac{gqm} foi positivo, e considera-se que é viável desenvolver um jogo com o objetivo de monitorar dados motores, de forma não invasiva, e integrada à rotina diária dos usuários.



