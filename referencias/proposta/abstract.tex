Applying pervasive computing to health's context can improve users' life quality by remotely providing information regarding their health. On the other hand, developing a non-invasive health monitoring pervasive systems is a multidisciplinary challenge. These systems, despite technology advancements, are still visible and stereotyped which complicates its dissemination. So, these systems have not been applied to the users activity daily living, undermining motor symptoms monitoring.

 Given the difficulties of developing such a system, this work proposes using video games to motivate and disregard health monitoring. The American industry stated that, in 2011, video game players averaged 37 years-old and 29\% are over 50 years-old. Since 2005, video games use motion detection sensors to capture users' kinetic actions. This way, this work proposes an approach based in video games to quantify motion signals and monitor health.

 This work's relevance was assessed through qualitative research where a semi-structured interview with health professionals. To validate the approach, we executed a case-control analytic study to detect individuals diagnosed with Parkinson's Disease using motion capturing sensors through video games. We evaluated health data acquisition possibilities based on Human Motion Angular Kinetic characteristics. The data was applied in a Support Vector Machine (SVM) to classify the data. As a result, we had 80\% rate of true positive identification and 16.67\% rate of false positive. This way, we concluded that the proposed approach allows to develop video games that serves to monitor data motion non-invasively.
