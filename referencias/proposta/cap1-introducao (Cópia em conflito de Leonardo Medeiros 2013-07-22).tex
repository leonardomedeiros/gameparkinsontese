\chapter{Introdu\c{c}\~{a}o} \label{sec:intro}
Nas últimas décadas os jogos eletrônicos, tornaram-se bastante presentes no cotidiano das pessoas, desde seu surgimento no final da década de 70 ~\cite{moore2011basics} os jogos eletrônicos acompanharam o crescimento e amadurecimento de seus usuários e evoluíram para plataformas cada vez mais poderosas como PS3 e Xbox 360. A \textit{Entertainment Software Association}, associação formada pelas principais fabricantes americanas de jogos eletrônicos, publicou seu documento anual com estatísticas sobre a indústria, o "\textit{Essential Facts About the Computer and Video Game Industry}" ~\cite{esa:facts2011}. A pesquisa constata que, em 2011, os jogadores de videogame dos Estados Unidos têm, em média, 37 anos e 29$\%$ possuem mais de 50 anos.

O estilo de vida atual das pessoas, comporta  diversos dispositivos eletrônicos em seu cotidiano o que acarreta numa diminuição da atividade física, ocasionando numa diminuição da qualidade de vida dos indivíduos ~\cite{maitland}. Por esse motivo, buscou-se motivar a prática da atividade física por intermédio dos jogos eletrônicos que proporcionasse a execução dos movimentos, queima de calorias e consequentemente uma melhora da saúde ~\cite{Suhonen:2008:SFE:1457199.1457204}.  Os jogos para a prática de exercício físico doravante \textit{Exergames} se tornaram extremamente populares e estudos já comprovam seus benefícios em relação ao aumento da atividade física ~\cite{baran08}. A prática da educação física num ambiente controlado e seguro para a atividade física permite que as pessoas se movimentem ao jogar e o próprio ambiente do jogo busca corrigir posturas e motivar a prática do exercício físico bem como fornecer uma avaliação da atividade física desempenhada ~\cite{Hardy2011,vaghetti2011,Suhonen:2008:SFE:1457199.1457204}. 

%em diferentes momentos do dia dentro de um ambiente de jogo eletrônico o qual pode estar integrado em sua rotina diária. 
 Neste trabalho, busca-se avaliar a possibilidade de monitorar dados de saúde enquanto os usuários estão em um momento de entretenimento. A presente pesquisa parte do pressuposto que sintomas motores possam ser capturados por intermédio dos sensores de movimento ~\cite{visionbased2009,patel_monitoring_2009,bachlin_parkinsons_2009} e que estes podem ser usados dentro de um cenário de jogo eletrônico. Atualmente, os jogos eletrônicos já fazem uso de acelerômetros, giroscópio e detecção de movimento através de vídeo.  Contudo, como visto anteriormente esses dispositivos já são usados para entretenimento e até a melhora do estado de saúde de seus usuários por intermédio de jogos que motivem a prática de exercícios físicos ~\cite{Suhonen:2008:SFE:1457199.1457204,vaghetti2011,bartolome11}. Porém, a presente proposta pretende agregar a capacidade de monitorar sintomas motores do usuário dentro de um ambiente de jogo eletrônico. Todavia, alinhar a jogabilidade e a possibilidade de monitoramento contínuo dos dados de saúde não é uma tarefa trivial. Por esse motivo, propomos um processo de desenvolvimento de jogos que permite realizar o monitoramento de dados motores de um modo não invasivo e integrado a rotina diária dos usuários. A presente pesquisa, parte do pressuposto que utilizar esses mesmos sensores num cenário de jogo eletrônico pode permitir que esses dispositivos sejam utilizados rotineiramente e o monitoramento seja realizado em um momento que o usuário utiliza para o entretenimento e em diferentes horários do dia, isso possibilitará o monitoramento dos sintomas da \ac{dp} em diferentes momentos do dia. 

Como possível cenário de uso para a pesquisa, podemos supor que um usuário o qual é paciente de uma doença crônica como a \ac{dp} faz uso de algum medicamento antiparkinsoniano e possui um jogo de monitoramento de tremores embarcado em um dispositivo móvel como um celular que possui acelerômetro. Por ser um jogo móvel e disponível em um dispositivo que o usuário carrega consigo na sua rotina diária, ele poderá utilizá-lo quando e onde desejar, logo os sintomas de tremor poderão ser detectados em diferentes momentos do dia. Então o médico de posse da informação a cerca da ocorrência dos sintomas motores poderá avaliar melhor a dosagem medicamentosa. Estudos indicam que uma dosagem correta, irá melhorar a qualidade de vida do usuário ao prolongar a efetividade do medicamento utilizado ~\cite{rodrigues2006}.
