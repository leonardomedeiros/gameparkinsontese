%contexto, problema, objetivo, como validou, resultados alcançados. 
O potencial da computação pervasiva pode ser percebido em várias perspectivas e ambiente: hospitais, situações de emergência, na indústria, educação, entre outros. Quando aplicada ao auxílio do tratamento de saúde denomina-se como pervasive healthcare . Dentre os benefícios do uso da pervasive healthcare tem-se: a melhoria do tratamento domiciliar do paciente, a realização do monitoramento remoto , monitoramento contínuo em pacientes que possuem níveis cognitivos e físicos comprometidos. Contudo, a concepção de um sistema de que não seja invasivo ainda é um grande desafio devido aos dispositivos serem pesados, visíveis e estereotipados. Por esse motivo esses dispositivos enfrentam barreiras por parte dos usuários e passam não utilizá-los em sua rotina perdendo a viabilidade do monitoramento do estado da saúde. Uma possível solução a essa barreira, é tentar colocar o monitoramento da saúde num cenário de jogo eletrônico.
É sabido que nas últimas décadas os jogos eletrônicos tornaram-se bastante presentes no cotidiano das pessoas, a  \textit{Entertainment Software Association} publicou em seu documento anual estatísticas sobre a indústria de jogos e constatou que no ano de 2011, os jogadores de videogame dos Estados Unidos possuíam em média 37 anos onde 29$\%$ estavam acima dos 50 anos.
Então, diante da dificuldade de realizar a monitorização aliada à rotina diária dos usuários, enuncia-se o seguinte problema de tese: Como realizar um mecanismo não perceptível de monitoramento motor de dados de saúde e que possa estar integrado à rotina diária dos usuários. Desse modo, tem-se como objetivo principal monitorar dados motores de uma forma não perceptível por intermédio de jogos eletrônicos como forma de motivar e abstrair o monitoramento de dados de saúde e longe do contexto de tratamento de saúde. Por esse motivo, criamos a abordagem de um processo de desenvolvimento de jogos para saúde juntamente com um arcabouço de software adequado a essas práticas que permitam o desenvolvimento de jogos com o objetivo de monitorar dados de saúde. 
Para avaliarmos esse trabalho usamos aspectos qualitativos que permitem identificar a importância deste trabalho junto à comunidade de especialistas da área de saúde.  E para os possíveis usuários finais foram abordados por intermédio de técnicas de questionários (GQM) 24 indivíduos onde buscamos verificar se esses possíveis usuários poderiam integrar o jogo desenvolvido em sua rotina diária onde obtivemos 75$\%$ dos entrevistados responderam que pretendiam integrar esses jogos à sua rotina diária.
O aspecto quantitativo da pesquisa faz uso de da Análise dos Componentes Principais (PCA) aplicada em uma base de dados disponibilizada na internet de um dispositivo wearable, onde seis sensores são postos em cada pé com o intuito de realizar a análise da marcha, onde obtivemos TpRate de 85,80$\%$, FpRate 22,80$\%$ e Precision 79,01$\%$. Posteriormente realizamos um estudo analítico de caso controle de indivíduos diagnósticados com a doença de parkinson e indivíduos sem o diagnóstico utilizando sensores de captura de movimento usado em jogos eletrônicos (Ms-Kinnect). Nessa etapa da pesquisa buscou-se avaliar as possibilidades de aquisição de dados de saúde baseada nas características de Cinemática Linear do Movimento Humano, esses dados foram postos em uma máquina de Aprendizagem Suppport Vector Machine (SVM) para realizar a classificação das duas classes de dados (tendo como resultados TpRate de 80,00$\%$, FpRate 16,67$\%$ e Precision 85,71$\%$. 
Como trabalhos futuros, pretendemos avaliar melhor os dados para obter melhores resultados através de identificação de sinais de tremor utilizando os dados do Ms-Kinnect, além de verificar com mais acurácia através de um maior número de pessoas o jogo produzido com a abordagem apresentada neste trabalho.






%Para atingir os objetivos da pesquisa, foram elencadas hipóteses que foram testadas durante este trabalho:
	%\begin{description}
	%\item[H1] O acompanhamento de sintomas motores integrados à rotina diária do paciente, traz benefícios ao tratamento e qualidade de vida do mesmo, do ponto de vista do profissional da saúde.
	%\item[H2] É possível capturar parâmetros de dados motores por intermédio de sensores de movimento utilizados em jogos eletrônicos, tais parâmetros podem vir a auxiliar o acompanhamento de doenças com comprometimento motor como exemplo a Doença de Parkinson.
	%\item[H3] É possível desenvolver um jogo que tenha mecanismos de captura de dados motores embutidos e que consiga quantificar os sinais motores dos usuários e consequentemente monitorar o seu estado de saúde de maneira imperceptível.
	%\end{description}

%Que permita levar em consideração o uso dos dispositivos, pensar na execução de movimentos ou ações que permitam esse monitoramento. Para que seja possível propor um jogo que consiga obter um monitoramento dos dados de saúde, deve ser realizado um estudo sobre quais os movimentos e ações que o usuário deve executar. Posteriormente, na posse dessas ações, deverá ser testada a execução dessas atividades e sua captura e possível classificação. De posse dos movimentos e da captura dos dados será feito um levantamento de um \textit{game design} que permita executar os movimentos em  um ambiente lúdico e divertido como um jogo para entretenimento.

%Desde 2005 os jogos eletrônicos fazem uso de dispositivos como acelerômetros, giroscópio, dispositivos possibilitando ao usuário estivesse uma maior imersão no universo do jogo através da análise de seus movimentos. Como o uso desses dispositivos já está embutido no contexto do jogo, possivelmente o usuário não iria sentir desconforto caso fossem usados para monitorar os dados enquanto estivesse num momento de entretenimento ao usar um jogo eletrônico para a captura dos sinais motores do indivíduo. 



%Essa pesquisa também fará uma análise de jogos que fazem uso de sensores de movimento e avaliará as possibilidades de aquisição de dados de saúde baseada na Cinemática Angular do Movimento Humano. Através dos resultados obtidos pretendemos avaliar e classificar a normalidade e dificuldade na
%execução de movimentos como levantar de um braço por exemplo.
 %execução de movimentos como levantar um braço, esticar uma perna ou balançar o corpo.


