\chapter{Entrevista Semi-Estruturada com Profissionais de Saúde}\label{chapter:entrevista_semi_estruturada}

A interpretação de dados é o cerne da pesquisa qualitativa. Esse método, tem como função desenvolver a teoria, servindo ao mesmo tempo de base para a decisão sobre quais dados adicionais devem ser coletados \cite{FLI04}. É importante que seja realizada a codificação seletiva, para que seja elaborada a categorização essencial sobre todas as categorias envolvidas \cite{FLI04}. O pesquisador precisa decidir quais os fenômenos salientes e ponderá-los de forma a ter como resultado uma categoria central juntamente com as categorias a ela relacionadas. O procedimento da interpretação dos dados, assim como a integração de material adicional, é encerrado quando atinge a "saturação teórica", quando o avanço da codificação não atinge mais novos conhecimentos \cite{FLI04}.

Para análise dos textos provenientes da pesquisa (transcrição da entrevista com os Neurologistas e Fisioterapeutas especialistas em Neurologia) foi utilizada a codificação seletiva através de criação de categorias a \emph{posteriori}. As categorias foram criadas e organizadas de acordo com o conteúdo de cada texto. As respostas de cada participante foram analisadas e a partir da identificação das categorias, incluídas na árvore de categorias do QDA Miner \cite{qda13}, que pode armazenar a transcrição de cada entrevista. Admitindo-se que uma classificação, para ser adequada não pode ser feita arbitrariamente, a categorização da árvore será criada e reformulada várias vezes, durante o processo de análise de acordo com o método de pesquisa\cite{FLI04}.


\subsubsection{Instrumento de Análise dos Dados da Pesquisa Qualitativa} \label{section:analise_dados} 
A pesquisa qualitativa assistida por computador \textit{software} permite a análise qualitativa das informações obtidas em modo texto
por intermédio do \textit{software} QDA Miner \cite{qda13}. Software dessa natureza auxiliam o pesquisador na organização dos registros da pesquisa e das interpretações dos mesmos. A escolha dessa ferramenta é justificada pela dificuldade de classificar e analisar os dados obtidos. Nessa análise somente foram levadas em consideração as atividades referentes ao acompanhamento dos sintomas motores em pacientes de parkinson e como um possível cenário que permita o monitoramento desses sintomas por intermédio de jogos eletrônicos poderia auxiliar os pacientes e também os profissionais de saúde.

Detalhamos a entrevista semi-estruturada realizada, descrevendo a opinião dos entrevistados e coletando requisitos baseado nas necessidades expostas pelos mesmos. Essa etapa da pesquisa é importante para testar a Hipótese \textbf{H1} usando pesquisa qualitativa para sua avaliação, que realizou uma análise do ponto de vista do profissional da saúde para a comprovação da hipótese:

	\begin{description}
	\item[H1] O acompanhamento de sintomas motores integrados à rotina diária do paciente, traz benefícios ao tratamento e qualidade de vida do mesmo, do ponto de vista do profissional da saúde.
	\end{description}
	
\subsection{Análise da Entrevista Semi-Estruturada}
Nesta seção são apresentados os resultados da Entrevista Semi-Estruturada como exposto na Figura \ref{figure:desenho_pesquisa}}.
A análise qualitativa ~\cite{FLI04} permitirá identificar as práticas dos profissionais de saúde referentes ao acompanhamento dos sintomas motores em pacientes de parkinson e como essas práticas podem ser aperfeiçoadas num cenário em que haja o monitoramento dos sintomas.

A identificação dos requisitos de um sistema representa o início da aquisição das necessidades da solução a ser proposta. Os requisitos definem quais serão os serviços que o sistema deverá prover além de um conjunto de restrições existentes na operação do mesmo \cite{sommerville2011}. Para Nuseibeh \cite{bas00}  a Engenharia de Requisitos é o processo de descobrir o propósito do software, identificando os principais envolvidos do sistema com suas respectivas necessidades e documentando a análise realizada para uma implementação posterior. Contudo, é um processo que deve ser continuamente repetido para que a necessidades dos envolvidos sejam satisfeitas. As técnicas para identificação de requisitos são derivadas principalmente das ciências sociais, que se baseiam em pesquisa qualitativa onde são analisados a teoria do objeto de estudo com a experiência prática dos envolvidos na pesquisa ~\cite{zowghi2005}. Uma das técnicas de identificação de requisitos baseada em pesquisa qualitativa é a entrevista semi-estruturada em que o entrevistador possui um conjunto de perguntas pré-definidas e guia a entrevista de acordo com a opinião do entrevistado ~\cite{FLI04}.


\section{Perfil dos Participantes}
Os participantes da pesquisa foram 4 profissionais de saúde onde 2 são Fisioterapeutas com especializações em Neurologia e 2 são médicos especialistas em Neurologia. A escolha do diferente perfil desses profissionais se fez necessária porque os ambos os profissionais desempenham funções distintas para o tratamento e acompanhamento dos pacientes da ~\ac{dp}. Os neurologistas realizam o diagnóstico e acompanham os sintomas motores juntamente com as informações obtidas do paciente ou cuidador realizando assim o gerenciamento da dosagem medicamentosa da doença. Por outro lado, os fisioterapeutas fazem o acompanhamento dos sintomas motores em sessões de fisioterapia promovendo a aprendizagem motora desses pacientes. Logo esses profissionais possuem visões e preocupações distintas inerentes a sua profissão. 


Para manter a confidencialidade de informação, os entrevistados irão receber uma Legenda que possa identificar unicamente o profissional sem revelar sua identidade como descrito na Tabela~\ref{table:perfil_analise_participantes}.

\begin{table}[h]
\caption{Perfil dos Participantes}
\label{table:perfil_analise_participantes}
\begin{tabular}{|l|l|c|c|}
\hline
\textbf{LEGENDA} & \textbf{PROFISSÃO}             & \multicolumn{1}{|l}{\textbf{IDADE (ANOS)}} & \multicolumn{1}{|l|}{\textbf{EXPERIÊNCIA (ANOS)}} \\ \hline
FIS\_01          & Fisioterapia em Neurologia & 40                                         & 10                                                \\ \hline
FIS\_02          & Fisioterapia em Neurologia     & 39                                         & 10                                                \\ \hline
NEU\_01          & Médico Neurologista            & 42                                         & 15                                                \\ \hline
NEU\_02          & Médico Neurologista            & 67                                         & 30                                                \\ \hline
\end{tabular}

\end{table}

\subsection{Questionário de Pesquisa}
Para a construção do questionário foi realizada uma análise sobre a doença de parkinson e sintomas que pudessem ser monitorados usando sensores, para a construção do questionário foram utilizadas diretrizes médicas da Doença de Parkinson ~\cite{protpar010,national2006parkinson} e da tabela UPDRS ~\cite{updrs87} usada para avaliação do progresso da doença ~\cite{updrs87}. Foram elaboradas 15 perguntas agrupadas em 3 seções Apêndice \ref{apendice:entrevista-semi-estruturada}, com os seguintes temas: sintomas da doença de parkinson, monitoramento da saúde motora e os benefícios advindos do monitoramento. Devido as diferenças existentes na abordagem utilizada por cada profissional o entrevistador pôde selecionar as questões de acordo com as habilidades e responsabilidades do entrevistado.

\section{Análise}
Durante a análise dos textos, alguns foram extraídos para demonstrar os fragmentos encontrados durante esta análise. A nomenclatura utilizada nos fragmentos encontrado nessa análise contém o prefixo \textbf{FRAGMENTO} mais um número sequencial identificando o mesmo. Esse procedimento visou identificar \textbf{requisitos} que orientem a proposta de monitoramento de dados motores por intermédio de jogos eletrônicos, no que se refere a definir parâmetros importantes para o monitoramento motor e a perspectiva do profissional de saúde em relação a abordagem apresentada nesta Proposta de Tese. Para a captura do requisito, ficou estabelecido que ele deve ser importante para os entrevistados onde cada requisito extraído das entrevistas será citado como \textbf{REQ-ENTREVISTAS}, seguido por um número correspondente à sequencia de sua apresentação.

\subsection{Diagnóstico}\label{section:analise_diagnostico}
Inicialmente durante a entrevista com os Médicos Neurologistas que são responsáveis pelo diagnóstico da \ac{dp} como é realizado o diagnóstico da \ac{dp}. Foi identificado tanto na literatura ~\cite{tolosa06,vedolin2003} quanto nas entrevistas a dificuldade de se obter um diagnóstico da \ac{dp}, pois atualmente ainda não existe um diagnóstico estabelecido da doença ~\cite{national2006parkinson,protpar010}. 

Todos os profissionais informaram que o sintoma mais comum é o tremor de repouso, que normalmente é unilateral e seguido de uma bradicinesia como podemos ver em ([FRAGMENTO-01][FRAGMENTO-02]). Ainda no ([FRAGMENTO-01]), existe uma ocorrência que a \textbf{[NEU\_01]} reforça em mais 4 oportunidades durante a entrevista, ressaltando a importância da técnica de \textit{Finger Taps} ~\cite{updrs87} para avaliação da bradicinesia sintomas na \ac{dp}.


\begin{quote}
\textbf{[FRAGMENTO-01][NEU\_01]} -
\emph{O diagnóstico da doença de Parkinson é quando o paciente chega se queixando de tremor. Esse sintoma começa com um tremor unilateral, geralmente pelas mãos, lentamente progressivo e de repouso. Além do tremor esse paciente exibe também uma lentidão que a gente consegue detectar pelo \textit{Finger Taps}. Que é colocando o polegar o primeiro e segundo dedo simultaneamente para ver se há ou não lentidão e comparando sempre com o outro lado você vai ver que há uma diferença. E existe também uma rigidez, quando você colocar o braço e faz uma flexão e extensão e você vê o tônus desse paciente comparado com o outro lado há uma diferença.}
\end{quote}

\begin{quote}
\textbf{[FRAGMENTO-02][NEU\_02]} -
\emph{
O diagnóstico da doença de Parkinson é feito com uma das queixa iniciais do paciente é um tremor de repouso geralmente associado a uma dificuldade na marcha. Os pacientes reclamam de uma perna presa e um tremor de repouso. 
}
\end{quote}

Um ponto que deve ser ressaltado no  ([FRAGMENTO-03]), é o que o entrevistado referiu como ``\textit{boa resposta ao prolopa}''. Essa ocorrência é chamado de diagnóstico diferencial da \ac{dp} ~\cite{protpar010}. Possivelmente com a abordagem utilizada nesta Proposta de Tese, poderíamos avaliar essa ocorrência e fornecer subsídios para que o profissional de saúde venha a decidir se a resposta à medicação foi efetiva e consequentemente realizar o diagnóstico diferencial.

\begin{quote}
\textbf{[FRAGMENTO-03][NEU\_01]} - 
\emph{
Então os sintomas é o tremor em repouso, a lentidão e a rigidez. De um lado só, por exemplo começa no braço direito e depois vai para a perna direita, depois para o braço esquerdo e depois a perna esquerda. Isso lentamente progressivo. E a gente faz a exclusão com outras doenças através de outros exames como tomografia, ressonância e com uma boa resposta ao prolopa.
}
\end{quote}

\subsection{Sintomas}
Nesta seção estão expostos a importância dos sintomas para o acompanhamento da sintomatologia da doença de parkinson do ponto de vista dos profissionais de saúde participantes da pesquisa.

\subsubsection{Tremor}
O sintoma de tremor além de ter sido referenciado durante o diagnóstico da doença ~\ref{section:analise_diagnostico} por todos os entrevistados. Contudo, este sintoma possui particularidades como a dificuldade de controlar o sintoma por intermédio do tratamento medicamentoso ([FRAGMENTO-04]). Outra aspecto que deve ser levado em consideração também nesse sintoma é que ele não é tão incapacitante quanto a bradicinesia ~\ac{dp} ([FRAGMENTO-05]), neste fragmento o [NEU\_01] reforça da importância de controlar os sintomas de lentidão do movimento ante os de tremor ~\cite{do2007parkinson}.

\begin{quote}
\textbf{[FRAGMENTO-04][NEU\_01]} - 
\emph{
Mas a gente tem que ver, porque as vezes o tremor é muito mais difícil de você controlar. Porque tem muito haver com a parte emocional. Quanto mais emocionalmente desequilibrado o paciente tiver, mais tremor ele tem.
}
\end{quote}

\begin{quote}
\textbf{[FRAGMENTO-05][NEU\_01]} - 
\emph{
O controle do tremor é um pouco complicado porque é um dos sintomas mais difíceis de ser controlado com as medicações  que temos hoje.  Então você poderia ver, principalmente nos seus exames seria a lentidão. Porque o paciente quer tremer mas ele não quer ficar lento.
}
\end{quote}


\subsubsection{Bradicinesia}\label{section:analise_bradicinesia}

O pesquisador indagou se o sintoma da bradicinesia era considerado o mais debilitante da ~\ac{dp} e como resposta ele obteve a afirmação que a bradicinesia impacta diretamente na qualidade de vida do paciente, privando-o de realizar as atividades diárias ([FRAGMENTO-06]).

\begin{quote}
\textbf{[FRAGMENTO-06][NEU\_01]} - 
\emph{
É ele atrapalha né, principalmente no levantar no andar, para você se levantar, pentear o cabelo e o tremor é prejudicial, mas mais prejudicial ainda é a lentidão do movimento.
}
\end{quote}


Ao indagar ao [NEU\_01] se o movimento de adução e abdução do braço, seria relevante para a identificação da~\ac{dp}, o profissional informou que a bradicinesia é um sintoma que traz lentidão em todo o corpo e possivelmente iria afetar nesse movimento também devido à redução dos movimentos automáticos ([FRAGMENTO-07]) que corrobora com a afirmação do [FIS\_01] que explica os impactos físicos de um paciente com a ~\ac{dp} ([FRAGMENTO-08]).

\begin{quote}
\textbf{[FRAGMENTO-07][NEU\_01]} - 
\emph{
Na verdade o movimento em si, vai ver o quão lento está. Porque você não tem um déficit motor. O comprometimento na doença de Parkinson está no comprometimento piramidal, o comprometimento extra-piramidal não vai estar alterando a força motora. O que vai estar vai ser exatamente a lentidão. A lentidão e o quão mais específico é que o paciente vai pegar naquele lugar. Por exemplo, é um paciente que está andando você que os movimentos dele automáticos estão reduzidos, principalmente no balançar dos braços. Você vai andando, vai andando, você vê aquele paciente que está com a força, ele está com toda a estrutura piramidal tudo normal. Mas ela anda lento em consequência da lentidão do movimento porque os movimentos automáticos estão reduzidos.
}
\end{quote}


\begin{quote}
\textbf{[FRAGMENTO-08][FIS\_01]} - 
\emph{
Os sintomas mais frequentes a gente tem a bradicinesia que é a lentificação do movimento, a gente tem um padrão postural que começa a ficar bem nítido que o paciente apresentar o Parkinson que é uma perda da movimentação automática da cintura escapular e aí ele começa a apresentar uma diminuição no volume da voz que é uma dipofonia e ele começa a apresentar uma maior rigidez muscular, que eles reclamam bastante e a bradicinesia que tornam os movimentos cada vez mais lentos.
}
\end{quote}



\subsubsection{Marcha}

Encontramos uma menção à dificuldade de andar do paciente de ~\ac{dp}, quando o [NEU\_01] cita no ([FRAGMENTO-07]) (``\textit{Você vai andando, vai andando, você vê aquele paciente que está com a força, ele está com toda a estrutura piramidal tudo normal. Mas ela anda lento em consequência da lentidão do movimento ...}''. O [NEU\_02] corrobora com a mesma opinião ao citar a dificuldade de iniciar a marcha no ([FRAGMENTO-09]).
Por outro lado, o fisioterapeuta tem o papel de realizar o acompanhamento da marcha nas sessões de fisioterapia além de fornecer um aprendizado motor para a melhora da qualidade de vida do paciente de acordo com suas limitações devido a sintomatologia da doença como podemos ver no ([FRAGMENTO-10]).

\begin{quote}
\textbf{[FRAGMENTO-09][NEU\_02]} - 
\emph{
Problema na marcha. Dificuldade de iniciar a marcha, certa dificuldade de um lado comprometido. Mesmo quando o sintoma está unilateral eles sentem dificuldade para iniciar a marcha.
}
\end{quote}

\begin{quote}
\textbf{[FRAGMENTO-10][FIS\_01]} - 
\emph{
Numa marcha, o doente de Parkinson tem a tendência de estar olhando para o chão. Mas a gente sabe que isso não é compatível com uma boa marcha a tendência é cair, para piorar eles têm os passos miúdos e também um passo arrastado, então esse passo favorece a queda. Então, ele perdeu a marcha automática que é aquele que a gente adquire na infância. A gente vai evoluindo vai levando as nossas quedas  corrige.  O doente, ele perde isso. O que a gente faz nas sessões de fisioterapia é tentar aplicar auto-correções para adaptar o paciente à nova realidade para que ele tenha uma aprendizado motor e no futuro um automatismo do movimento.
}
\end{quote}

Devido a essa recorrência de opiniões sobre a importância da análise da marcha ~\ref{section:analise_marcha} para o acompanhamento dos sintomas da doença de Parkinson, esse trabalho que realizou um estudo mais aprofundado desse movimento na Seção~\ref{section:analise_marcha_pca}. Um detalhe importante que tem haver com a integridade física dos sujeitos da pesquisa é a possibilidade de queda como encontrado no [FRAGMENTO-10] \textit{``Mas a gente sabe que isso não é compatível com uma boa marcha a tendência é cair ...''} que vai além do do custo financeiro para a aquisição dos sensores de que capturam a~\ac{fvrs} ~\cite{dyno}) e reforça a importância de bases de dados motoras como a Physionet~\cite{physionet} com o intuito de preservar a integridade física dos pacientes.



\subsection{Monitoramento Motor}
Nesta seção estão expostos a importância do monitoramento dos sinais capturados durante a pesquisa no estudo analítico de caso controle exposto no Método de Pesquisa na Seção ~\ref{section:estudo_caso_controle}. Nesse estudo também pretendemos identificar características dos movimentos que possam ser extraídos desses sinais e que venham fornecer subsídios para diferenciar indivíduos diagnosticados com a ~\ac{dp} ante indivíduos sem o diagnóstico da doença.

\subsubsection{Amplitude do Movimento dos Braços}

Ao indagar ao [FIS\_02] se o movimento de adução e abdução do braço seria relevante para a identificação da ~\ac{dp} o fisioterapeuta informou que mesmo não sendo um teste específico para a identificação da doença, existiam diferenças significativas encontradas em indivíduos diagnosticados com parkinson [FRAGMENTO-11].
\begin{quote}
\textbf{[FRAGMENTO-11][FIS\_02]}-
\emph{
Sim. Existe alterações sim, mas eu nunca vi especificamente esse teste como sendo usado para diagnóstico da doença. Mas que realmente existem mudanças no movimento de adução e abdução de uma pessoa normal ante a um parkinsoniano.
}
\end{quote}

O [FIS\_01], explicou os motivos que levam a perda da mobilidade no movimento de adução e abdução ([FRAGMENTO-12]) e consequentemente, reforça que esse movimento poderia ser monitorado para verificarmos o comprometimento da doença. Em um outro fragmento ([FRAGMENTO-13]) o mesmo fisioterapeuta reforça da importância de monitorar a amplitude do movimento, pois permite visualizar a resposta do paciente ao tratamento oferecido.

\begin{quote}
\textbf{[FRAGMENTO-12][FIS\_01]}-
\emph{
Têm, porque uma das grandes perdas que eles apresentam é na cintura escapular e consequentemente é pegando a parte de ombro. Pois caso ela seja mais fixa, porque geralmente o paciente de Parkinson abduz o ombro. O ombro fica abduzido junto ao tronco e ai ele perde a mobilidade do cotovelo e punho e também o movimento fica comprometido por conta disso.
}
\end{quote}

\begin{quote}
\textbf{[FRAGMENTO-13][FIS\_01]}-
\emph{
Mesmo sabendo que a tendência é uma lentificação (bradicinesia).  As outras doenças também, porque um dos objetivos nossos é o aumento da amplitude. Então é um meio interessante para a gente conseguir visualizar se o tratamento está dando certo ou não.
}
\end{quote}



\subsubsection{Velocidade do Movimento De Adução e Abdução dos Braços}
Um ponto de convergência entre os profissionais foi a importância de monitorar a velocidade angular dos pacientes. Pois os profissionais tentam convergir o tratamento fisioterápico e medicamentoso para a melhora do sintoma da bradicinesia, então para os profissionais a melhora está condicionada a um aumento na velocidade do movimento ([FRAGMENTO-14],[FRAGMENTO-15])

\begin{quote}
\textbf{[FRAGMENTO-14][NEU\_01]} - 
\emph{
É como eu falei para mim seria melhor se capturássemos se ele está mais lento. Se através dessa amplitude você conseguir por intermédio do computador identificar que ele está mais lento de um lado do que do outro. Conseguir visualizar a velocidade de um lado e do outro, então isso é interessante.
}
\end{quote}


\begin{quote}
\textbf{[FRAGMENTO-15][FIS\_01]} - 
\emph{
É e consequentemente a velocidade, porque nesse caso o tratamento é diretamente relacionado a isso quanto mais veloz o parkinsoniano é melhor para a gente melhor prognóstico a gente pode ter lá na frente. Mesmo sabendo que a tendência é uma lentificação.
}
\end{quote}




\subsubsection{Assimetria do Movimento}

A assimetria do movimento acomete os pacientes que estão nos estágios iniciais da doença. Por esse motivo, geralmente ele é  identificado durante o diagnóstico [FRAGMENTO-03]. Porém, alguns pacientes parkinsonianos persistem com a assimetria do movimento em que um dos lados é mais comprometido que o outro. Por esse motivo é que o [NEU\_01] afirmou \textit{``Se através dessa amplitude você conseguir por intermédio do computador identificar que ele está mais lento de um lado do que do outro.}. Porque a tendência natural da evolução da doença de parkinson é que haja redução na assimetria do movimento conforme a opinião do [NEU\_01] no [FRAGMENTO-16] e a Tabela UPDRS em sua escala de avaliação do progresso da doença ~\cite{updrs87} (Seção ~\ref{section:escalas_avaliacao}).

\begin{quote}
\textbf{[FRAGMENTO-16][NEU\_01]}-
\emph{
No início. Geralmente o paciente se queixa de uma diminuição de força de um lado do corpo. Mas na progressão, ele vai sentir dificuldade global. Mas aqueles parkinsonianos iniciais geralmente eles se queixam na diminuição do movimento de um dos lados.
}
\end{quote}

\subsection{Benefícios Advindos do Monitoramento}


\subsubsection{Quantificação dos Sintomas}
As perguntas relacionadas à quantificação dos sintomas como amplitude de movimento, velocidade angular e se seria válido acompanhar esses valores para o monitoramento dos sintomas. Nos trouxe dois grupos de respostas, uma que reconhecia da importância da quantificação dos dados para poder mensurar a melhora ou piora do paciente de forma quantitativa como está no [FRAGMENTO-17]. E outra resposta é que esses valores teriam mais validade científica do que prática, pois na prática os profissionais já monitoram esses sintomas manualmente [FRAGMENTO-18]. Todavia, se esses profissionais tivessem acesso a um sistema que permitisse o monitoramento motor e consequentemente conseguissem perceber os seus benefícios. Esses mesmos profissionais poderiam modificar a prática atual de forma manual e adotar uma nova proposta.

\begin{quote}
\textbf{[FRAGMENTO-17][FIS\_02]}-
\emph{
É preciso ter parâmetros sim. Pois atualmente usamos muito o olho clínico e ai vai de cada profissional. Se tivermos números facilitam bastante porque se tornam fatos e basearmos nossas conclusões em números é bem melhor.
}
\end{quote}

\begin{quote}
\textbf{[FRAGMENTO-18][FIS\_01]}-
\emph{
É interessante em termos de pesquisa. Em termos de clínica a geralmente a gente vai no geral. Por exemplo: Eu faço uma flexão de ombro com bastão e anotei no meu exame que ele ia até mais ou menos 70º e após 15 dias eu vejo que ele está levantando acima de 90º. Então está marcado a minha evolução. Então eu faço a avaliação nesse sentido. Então esse sistema seria bom para pesquisa mesmo.
}
\end{quote}



\subsubsection{Gerenciamento da Dosagem Medicamentosa}
O monitoramento dos dados motores poderia ser bastante utilizado no gerenciamento da dosagem medicamentosa. Os profissionais sentem a necessidade de visualizar a eficácia de seu tratamento junto ao paciente. O [FIS\_01] no [FRAGMENTO-19] cita a importância de avaliar tanto o tratamento medicamentoso quanto se a sua atividade fisioterápica está trazendo benefícios ao paciente. Os Neurologistas citam ([NEU\_01] e [NEU\_01]), citam a importância de reajustar a dosagem medicamentosa e que poderiam visualizar se o medicamento estaria surtindo efeito no paciente. Outra opinião bastante pertinente é que o agravamento da ~\ac{dp} é bastante sutil segundo a [NEU\_01] no [FRAGMENTO-21], logo se houvesse fosse possível mostrar a evolução da doença em períodos mais longos o tratamento poderia ser mais efetivo e consequentemente melhoraria a qualidade de vida dos pacientes.


\begin{quote}
\textbf{[FRAGMENTO-19][FIS\_01]} - 
\emph{
 É interessante porque teremos uma ideia de até que ponto a medicação está sendo efetiva, até quando a patologia está progredindo e também avaliar se o nosso tratamento fisioterápico está dando resultados ao tentar frear a evolução da doença.
}
\end{quote}


\begin{quote}
\textbf{[FRAGMENTO-20][NEU\_02]} - 
\emph{
Sim. Dentro do que você propõe. Com certeza sim. Essa avaliação desses movimentos. Porque a gente consegue visualizar se a medicação está surtindo efeito, se precisa ser reajustada.
}
\end{quote}

\begin{quote}
\textbf{[FRAGMENTO-21][NEU\_01]} - 
\emph{
Se esse mecanismo acontecesse. Você poderia avaliar a dosagem de um paciente por exemplo. Veja avalie durante uma semana, não melhorou. Então a gente poderia fazer um teste com tremor, lentidão e a rigidez, se houvesse esse aspecto.  A gente poderia aumentar a dosagem e visualizaria a eficácia da dosagem com o decorrer do tempo, com o decorrer da evolução. E verificaria se realmente o paciente está melhorando. Porque o paciente da doença de Parkinson ele piora lentamente, as vezes é tão sutil que o próprio paciente não consegue. Então é como eu disse, cada paciente a evolução é diferente num existe. Mas poderia assim, se você conseguisse detectar as amplitudes do tremor por exemplo.
}
\end{quote}






\section{Requisitos Identificados}


%O uso do termo "Engenharia" implica que técnicas sistemáticas e repetidas devem ser aplicadas para certificar que os requisitos de um sistema estejam completos, consistentes e relevantes \cite{sommerville2011}. O emprego correto da Engenharia de Requisitos é um passo fundamental para o desenvolvimento de um bom  produto, onde a satisfação do usuário deve ser o principal objetivo a ser atingido \cite{zowghi2005}.
Os requisitos obtidos durante as fases de análise que compõem o processo de pesquisa descrito neste trabalho indicam a importância de monitorar os dados motores para uma avaliação posterior do profissional de saúde.
Para demonstrar a relevância do requisito identificado confrontamos a teoria com o que é aplicado na prática pelos profissionais de saúde. Por isso ao lado dos requisitos iremos citar referências científicas que corroboram com esses requisitos. 

\begin{description}
	\item[REQ-ENTREVISTAS-01 :] Identificar e quantificar o tremor parkinsonianno ~\cite{tolosa06,keijsers2006,lemoyne2010}.
	\item[REQ-ENTREVISTAS-02 :] Identificar a bradicinesia ~\cite{patel_monitoring_2009}. %Para identificar a bradicinesia pode ser calculada a velocidade angular do movimento de abdução e adução do braço.
	\item[REQ-ENTREVISTAS-03 :] Avaliar bradicinesia usando ~\textit{finger-tapping} ~\cite{finger2012}.
	\item[REQ-ENTREVISTAS-04 :] Considerar e identificar a assimetria do movimento nos estágios iniciais ~\cite{national2006parkinson}.	%Além da velocidade angular podemos calcular a amplitude do movimento, assim identificarems a assimetria do movimento com mais clareza.
	\item[REQ-ENTREVISTAS-05 :] Fornecer mecanismos para possibilitar o Diagnóstico Diferencial \cite{protpar010} da Doença de Parkinson. %caso os sintomas de bradicinesia e marcha sejam identificados. Caso os indivíduos venham a surtir efeito ao medicamento e consequentemente reduzir a gravidade do sintoma, então foi possível realizar um diagnóstico diferencial.
	\item[REQ-ENTREVISTAS-06 :] Analisar a Marcha ~\cite{gaitusingsensorsreview2012}. Medir a marcha e comparar o padrão do movimento com indivíduos com e sem o diagnóstico da ~\ac{dp} para classificar a marcha como saudável ou parkinsoniana.
	\item[REQ-ENTREVISTAS-07 :] Calcular e armazenar a amplitude do movimento de adução e abdução dos braços. Para realizar o monitoramento da saúde motora e poder acompanhar o tratamento.
	\item[REQ-ENTREVISTAS-08 :] Calcular e armazenar a velocidade angular do movimento de adução e abdução dos braços. Para poder avaliar o sintoma da bradicinesia.
	\item[REQ-ENTREVISTAS-09 :] Avaliar estado emocional e avaliar o comprometimento do tremor. 
\end{description}



\subsection{Inviabilidade Técnica}
Algumas requisitos identificados, atualmente devido a tecnologia do sensor de movimento utilizado atualmente é inviável sua implementação como:
\begin{itemize}
  \item O \textbf{REQ-ENTREVISTAS-01}, o tremor de repouso é um dos principais sintomas da doença de parkinson. Sabíamos da sua importância e desenvolvemos até um jogo para \textit{Smartphone} para tentar quantificar o tremor. Porém, ao testar junto aos usuários percebemos que ao usarem o jogo eles paravam de tremer inviabilizando a quantificação do tremor. 
	Contudo, ao realizar a coleta dos dados com pacientes de parkinson identificamos que alguns deles tremiam a mão que não estava em movimento. Ou seja, quando solicitávamos para abduzir o braço direito em alguns pacientes a mão esquerda tremia. Devido o tempo e o objeto de estudo que era o próprio movimento não avaliamos ainda o sinal produzido do braço parado.
	\item O \textbf{[REQ-ENTREVISTAS-03]}, a técnica de \textit{finger-tapping} não pode ser avaliadas utilizando o MS-Kinnect 1.0, pois nessa versão não existe a captura do movimento dos dedos conforme na Figura~\ref{fig:articulacoeskinnect} e consequentemente inviabiliza o seu monitoramento.
	\item O \textbf{REQ-ENTREVISTAS-09}, por envolver estado emocional e parâmetros que não estamos levando em consideração nesse trabalho esse requisito está fora do escopo deste trabalho. Porém com mecanismos de detecção de batimentos cardíacos presente em versões mais atuais do MS-Kinnect poderia averiguar a relação dos batimentos cardíacos com o tremor em trabalhos futuros.
\end{itemize}

\subsection{Matriz Rastreabilidade - Fragmento x Requisitos}
Essa Matriz de Rastreabilidade (Fragmento x Requisitos) vai mapear os \textbf{REQUISITOS} aos \textbf{FRAGMENTOS} que de forma direta ou indireta estejam correlacionados (Tabela \ref{table:matrix_rastreabilidade}). Ao final da matriz, temos um campo de Quantidade de Ocorrências que conta quantas vezes aquele requisito foi citado nos fragmentos.

%\begin{center}
\begin{table}[!htbp]
\caption{Matriz Rastreabilidade: Fragmento x Requisitos}
\label{table:matrix_rastreabilidade}
\begin{tabular}{||p{6.06cm}||ccccccccc|}
\hline
 \multicolumn{1}{|p{6.06cm}|}{\centering \textbf{FRAGMENTOS / REQUISITOS}} &  01 &  02 &  03 &  04 &  05 &  06 &  07 &  08 & 09 \\ 
\hline 
 \multicolumn{1}{|p{6.06cm}|}{\centering 01} &  x &  x &  x &  x &   &   &   &   &  \\ 
 \multicolumn{1}{|p{6.06cm}|}{\centering 02} &  x &   &   &   &   &  x &   &   &  \\ 
 \multicolumn{1}{|p{6.06cm}|}{\centering 03} &  x &  x &   &  x &  x &   &   &   &  \\ 
 \multicolumn{1}{|p{6.06cm}|}{\centering 04} &  x &   &   &   &   &   &   &   & x \\ 
 \multicolumn{1}{|p{6.06cm}|}{\centering 05} &  x &  x &   &   &   &   &   &   &  \\ 
 \multicolumn{1}{|p{6.06cm}|}{\centering 06} &   &  x &   &   &   &  x &   &   &  \\ 
 \multicolumn{1}{|p{6.06cm}|}{\centering 07} &   &  x &   &   &   &  x &  x &  x &  \\ 
 \multicolumn{1}{|p{6.06cm}|}{\centering 08} &   &  x &   &   &   &  x &  x &  x &  \\ 
 \multicolumn{1}{|p{6.06cm}|}{\centering 09} &   &   &   &  x &   &  x &   &   &  \\ 
 \multicolumn{1}{|p{6.06cm}|}{\centering 10} &   &   &   &   &   &  x &   &   &  \\ 
 \multicolumn{1}{|p{6.06cm}|}{\centering 11} &   &   &   &   &   &   &  x &   &  \\ 
 \multicolumn{1}{|p{6.06cm}|}{\centering 12} &   &  x &   &   &   &   &  x &  x &  \\ 
 \multicolumn{1}{|p{6.06cm}|}{\centering 13} &   &   &   &   &   &   &  x &   &  \\ 
 \multicolumn{1}{|p{6.06cm}|}{\centering 14} &   &  x &   &   &   &   &   &   &  \\ 
 \multicolumn{1}{|p{6.06cm}|}{\centering 15} &   &  x &   &  x &   &   &   &  x &  \\ 
 \multicolumn{1}{|p{6.06cm}|}{\centering 16} &   &   &   &  x &   &   &   &   &  \\ 
 \multicolumn{1}{|p{6.06cm}|}{\centering 17} &   &   &   &   &   &   &   &   &  \\ 
 \multicolumn{1}{|p{6.06cm}|}{\centering 18} &  x &   &  x &  x &   &  x &  x &  x & x \\ 
 \multicolumn{1}{|p{6.06cm}|}{\centering 19} &  x &   &   &   &  x &  x &  x &  x &  \\ 
 \multicolumn{1}{|p{6.06cm}|}{\centering 20} &  x &   &   &   &  x &  x &  x &  x &  \\ 
 \multicolumn{1}{|p{6.06cm}|}{\centering 21} &  x &   &   &   &  x &  x &  x &  x &  \\ 
\hline 
 \multicolumn{1}{|p{6.06cm}|}{\centering \textbf{QTD. OCORRÊNCIAS}} &  9 &  9 &  2 &  6 &  4 &  10 &  9 &  8 & 2 \\ 
\hline 
\end{tabular}
\end{table}
%\end{center}

\subsection{Matriz Rastreabilidade - Requisitos x Implementação}
Essa Matriz de Rastreabilidade (Requisitos x Implementaçao) vai mapear os \textbf{REQUISITOS} ao trabalho desenvolvido nessa Proposta de Tese. Isso demonstra o \textit{status} do trabalho e pode direcionar também os trabalhos futuros.

% Please remember to add \use{multirow} to your document preamble in order to suppor multirow cells
% Please remember to add \use{multirow} to your document preamble in order to suppor multirow cells
\begin{table}[!htbp]
\center
\begin{tabular}{|l|c|c|c|}
\hline
\multirow{\textbf{REQUISITO}} & \multicolumn{1}{|l}{\multirow{\textbf{SIM}}} & \multicolumn{1}{|l}{\multirow{\textbf{NÃO}}} & \multicolumn{1}{|l|}{\multirow{\textbf{\begin{tabular}[c]{@{}c@{}}AVERIGUAÇÃO\\ FUTURA\end{tabular}}}} \\
                                    & \multicolumn{1}{|l}{\textbf{}}                     & \multicolumn{1}{|l}{\textbf{}}                     & \multicolumn{1}{|l|}{\textbf{}}                                                                              \\ \hline
REQ-ENTREVISTA-01                   &                                                    &                                                    & X                                                                                                            \\ \hline
REQ-ENTREVISTA-02                   & X                                                  &                                                    &                                                                                                              \\ \hline
REQ-ENTREVISTA-03                   &                                                    & X                                                  &                                                                                                              \\ \hline
REQ-ENTREVISTA-04                   & X                                                  &                                                    &                                                                                                              \\ \hline
REQ-ENTREVISTA-05                   & X                                                  &                                                    &                                                                                                              \\ \hline
REQ-ENTREVISTA-06                   & X                                                  &                                                    &                                                                                                              \\ \hline
REQ-ENTREVISTA-07                   & X                                                  &                                                    &                                                                                                              \\ \hline
REQ-ENTREVISTA-08                   & X                                                  &                                                    &                                                                                                              \\ \hline
REQ-ENTREVISTA-09                   &                                                    & X                                                  &                                                                                                              \\ \hline
\end{tabular}
\end{table}

%\subsection{Requisitos de Jogos Com o Propósito de Monitoramento de Dados Motores}
%Nessa seção serão apresentados requisitos identificados durante a análise qualitativa da Entrevista Semi-Estruturada realizada com os profissionais de saúde conforme a metodologia de pesquisa. Para uma melhor análise dos resultados as considerações identificadas durante a análise foi confrontada com as diretrizes médicas ~\cite{protpar010,Jankovic_2008,ambulatoryparkinson2010} e a revisão da literatura sobre jogos para saúde \ref{section:jogos_saude} que serviram de base científica para esse trabalho.
%
%\subsubsection{Requisitos Essenciais}
%\begin{description}
	%\item[RE-001: – Progresso do jogo]
	%\item[RE-002: 
%\end{description}

%Requisitos Essenciais
%[RE-001] – Progresso do jogo
%Segundo Suhonnen, a característica mais atrativa de um jogo é perceber o progresso do mesmo. É o usuário conseguir visualizar o progresso do jogo [8].
%[RE-002] – Estado de Fluxo ou Escapismo
%O jogador tem que sentir relaxado e com desejo de repetir a atividades em outras oportunidades [8], o jogo tem que permitir que o usuário entre num estágio de fluxo [20] e execute as atividades sem perceber a noção de tempo e espaço. O usuário deverá jogar pelo próprio prazer.
%[RE-003] – Pontuação e Taxas de Acerto
%O jogador deve visualizar suas ações positivas e negativas. O jogo deverá pontuar as atividades do jogador de acordo com seus acertos. O jogador deve visualizar claramente os objetivos e perceber o sucesso ou fracasso alcançado [8], [15].
%[RE-004] – Preocupação física do jogador
%Por promover atividades físicas, ou ações que possam trazer injúria ao jogador, como movimentos de equilíbrio, movimentos repetitivos ou rápidos. O game design do jogo deve ter a preocupação de desenvolver o jogo de acordo com o público-alvo. Isso significa dizer que a faixa etária e limitações físicas e cognitivas em decorrência da idade ou enfermidade devem ser levadas em consideração. Os jogadores devem ter a segurança de usarem o jogo e ter a certeza que o seu uso não acaarretará em nenhum dano físico [1].
%[RE-004] – Monitoramento dos sinais vitais
%O jogo poderá monitorar os sinais vitais através dos sensores usados como interface do jogo. Sensores de movimento, podem ser aplicados para movimentar o personagem. Monitores cardíacos para controlar a intensidades dos exercícios físicos. Eletroencefalograma podem ser usados para acompanhar o nível de concentração do jogador [15]. A partir dos dados capturados, estes podem ser armazenados para uma avaliação a posteriori por um profissional de saúde. Desta maneira teremos um monitoramento dos dados de saúde de maneira não invasiva e presente na rotina do usuário.
%Requisitos Secundários
%[RS-001] – Motivar atividade física
%Os jogos para monitoramento de saúde que fazem uso de sensores de movimento, normalmente mootivam a prática de exercícicio físico [1], [8], [15]. Logo, esse requisito pode ocorrer de forma secundária sem ser o propósito principal do jogo.
%[RS-002] – Promover Reabilitação
%Para alguns tipos de usuário o jogo poderá ser usado para reabilitação. Usuários que tenham passado por acidentes vascular cerebral, cirurgia recente em algum membro. Estudos indicam que os jogos para exercício físico podem ser aplicados para auxiliar a reabilitação do usuário. Então como requisito secundário poderia atender a esse fim.
%[RS-003] – Adequação ao Tratamento
%Através da abordagem do jogo será possível influenciar o jogador a uma maior aderência ao tratamento[13], [14]. Contudo essa abordagem é difícil de ser adequada para um jogo voltado ao entretenimento.

\section{Conclusão}
Como dito no início do capítulo, buscamos nessa entrevista semi-estruturada avaliar a Hipótese \textbf{H1} da presente Proposta de Tese. Com este intuito, verificamos junto aos profissionais de saúde que um sistema de monitoramento de dados motores traria benefícios a qualidade de vida dos pacientes

Segundo o acompanhamento de sintomas motores integrados trazem benefícios no tratamento dos pacientes e como consequência uma melhora na qualidade de vida, do ponto de vista do profissional da saúde. Com base na rastreabilidade dos fragmentos da entrevista e a respectiva identificação dos requisitos. Pudemos concluir que houve muitas ocorrências nos requisitos de Identificação de sintomas como : tremores ([\textbf{REQ-ENTREVISTAS-01}]), bradicinesia [\textbf{REQ-ENTREVISTAS-02}] e análise da marcha [\textbf{REQ-ENTREVISTAS-06}]; para o acompanhamento e monitoramento da doença os profissionais de saúde também citaram da importância de calcular tanto a amplitude dos movimentos ([\textbf{REQ-ENTREVISTAS-07}]) quanto a velocidade angular da abdução e adução dos braços ([\textbf{REQ-ENTREVISTAS-08}]). Baseado nessas considerações, podemos validar qualitativamente a Hipótese \textbf{H1} como propõe o método de pesquisa utilizado nesta Proposta de Tese (Seção \ref{sec:desenho_pesquisa}).
