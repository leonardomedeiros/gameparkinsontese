\documentclass[a4paper,10pt]{article}
\usepackage[utf8]{inputenc}
\usepackage[portuges]{babel}
\usepackage[T1]{fontenc} 
\usepackage{ae}
% \usepackage[latin1]{inputenc}
\usepackage{url}

%opening
\Titulo{\large{Análise Formal de Sistemas Críticos de \\ Automação e Controle}\\ \vspace*{1cm} \normalsize{Plano de Trabalho}}
\Autor{Álvaro Álvares de Carvalho Cesar Sobrinho}
\Local{\normalsize{Maceió-AL}}
\Data{\normalsize{Novembro de 2010}}

\begin{document}

\Capa
\newpage
\renewcommand\contentsname{Sumário}
\tableofcontents
\thispagestyle{empty}
\newpage

\section{Introdução}

\pagenumbering{arabic}

Essa proposta para mestrado está inserida em um contexto da gerencia de informações 
e automação de procedimentos para monitoramento e cuidado a saúde de pacientes.
A necessidade da criação de sistemas para esse tipo de operação 
é uma constante no cenário médico atual. Essa preocupação tem despertado em vários 
pesquisadores a utilização e desenvolvimento de conceitos e tecnologias que automatizem 
esse processo.

Geralmente na realização de consultas de rotina ou tratamento de doenças críticas, os 
pacientes atendidos não realizam um gerenciamento de suas informações clínicas. A não 
preocupação no controle e armazenamento desses dados pode gerar confusão no entendimento 
do diagnóstico, prognóstico ou prescrição de medicamentos fornecidos pelo médico. Ao 
iniciar a gerência desse tipo de procedimentos o paciente  obtém um maior entendimento 
do seu quadro clínico, uma melhoria no controle da ingestão de medicamentos e uma 
documentação de seu histórico de saúde. Portanto, um monitoramento e a realização de 
operações automáticas para o cuidado a saúde do paciente tem a possibilidade de evitar 
complicações clínicas pela falta de cumprimento de obrigações médicas, ou até alertar 
os profissionais que efetuam seu cuidado de um problema que esteja ocorrendo em um 
determinado momento.

A solução proposta neste documento utiliza uma abordagem centrada no usuário e não na 
realização de monitoramento em um ambiente específico, como tecnologias “em casa” \cite{Dearden}. 
O intuito dessa escolha é ampliar a mobilidade e facilidade de manipulação de suas informações clínicas. 
Esse foco é decorrente da utilização de sistemas para cuidado a saúde por diversos tipos de usuários, 
dentre eles idosos, pessoas com diferentes deficiências e escolaridades \cite{Mulder}. 
Uma das propostas da ferramenta é tornar transparentes algumas operações que normalmente teriam que ser executadas 
explicitamente pelo usuário diariamente ou em situações críticas de saúde. Com isso, visa uma 
ação preventiva e também uma agilização do atendimento ao paciente. Um paradigma que está sendo 
bastante aderido pela comunidade de pesquisadores dessa área para auxiliar no desenvolvimento 
de sistemas desse tipo é a computação pervasiva. 

O paradigma da computação pervasiva, idealizado por Mark Weiser \cite{Weiser}, propõe que os 
sistemas computacionais estejam presentes em todos os lugares, através de dispositivos eletrônicos 
de diversos tipos, realizando operações de maneira imperceptível ao usuário. A utilização da 
computação pervasiva em sistemas de cuidado a saúde deu inicio a pesquisas para o desenvolvimento 
de soluções \textit{pervasive healthcare} ou saúde pervasiva.

A presente proposta visa o estudo sobre como utilizar o paradigma da computação pervasiva em sistemas
para o cuidado a saúde, e a partir disso, realizar o desenvolvimento de uma solução 
multiplataforma pervasiva para dispositivos móveis com intuito do apoio a monitoração de informações 
clínicas dos pacientes. Esta proposta está estruturada da seguinte maneira. Na Seção 2 uma problemática. 
Na Seção 3 são descritos os objetivos do trabalho. Na Seção 4 a metodologia da pesquisa. Na Seção 5 um 
cronograma das atividades. 

\section{Problemática}

Sistemas para apoio ao cuidado da saúde são utilizados para monitorar a situação dos seus usuários através 
de “dispositivos inteligentes”, “ambientes inteligentes” e “computação vestida” \cite{Ziefle}. Essas abordagens 
utilizam respectivamente, dispositivos móveis, infraestruturas de hardware com softwares embarcados, e sensores 
através do corpo humano. São desenvolvidas para serem utilizadas, como supracitado, para um publico alvo diversificado, 
contendo informações pessoais críticas sobre o estado de sua saúde. A segurança dos dados trafegados em aplicações 
\textit{pervasive healthcare} é um relevante fator a ser tratado em sua fase de concepção \cite{Moncrieff}. Entretanto, muitas 
aplicações não priorizam a utilização de políticas de segurança, o que as tornaria mais confiáveis e consequentemente 
com melhor aceitabilidade dos usuários.  

Projetar aplicações \textit{pervasive healthcare} que contenham a capacidade de facil adaptação as necessidades 
e desejos dos usuário é também importante para uma melhor aceitação do sistema por seus usuários \cite{Mulder}. O cuidado com valores humanos e a confiabilidade de privacidade 
de informações do sistema são fatores que podem ser cruciais para o seu sucesso ou fracasso, e são 
alguns dos desafios na construção de aplicações \textit{pervasive healthcare}.  

A mobilidade, retirando o foco de um determinado ambiente também é um fator a ser considerado. Ao ampliar a utilização da 
tecnologia através de dispositivos móveis, ao invés de se concentrar na criação de um determinado “ambiente inteligente”, 
as possibilidades de uso da solução pode crescer consideravelmente. Vários trabalhos relacionados propõem o desenvolvimento 
de soluções para dispositivos móveis \cite{Feng,Teles,Munoz,Sharmin}. Porém, utilizam sistemas operacionais 
específicos, podendo diminuir as opções para a utilização da aplicação. Uma forma de retirar limitações da solução é a busca 
do desenvolvimento voltado para aplicações multiplataformas, por não limitarem o uso para uma determinada tecnologia.  

\section{Objetivos}

O objetivo deste trabalho é a realização do desenvolvimento de uma aplicação multiplataforma pervasiva para dispositivos 
móveis com a proposta do monitoramento do cuidado a saúde de pacientes. Mais especificamente os seguintes objetivos:
 
\begin{itemize}

\item Investigar os conceitos da computação pervasiva;
\item Investigar como utilizar os conceitos da computação pervasiva em aplicações para saúde;
\item Investigar os requisitos necessários para o desenvolvimento da aplicação para obter uma aceitação relevante dos usuários;
\item Efetuar uma modelagem e documentação do sistema de acordo com os conceitos e requisitos analisados; 
\item Efetuar o desenvolvimento de uma aplicação multiplataforma para o apoio ao cuidado a saúde que interaja com facilidade com infraestruturas distintas para o compartilhamento de informações com sistemas utilizados por médicos e equipamentos de monitoramento;
\item Efetuar a validação da aplicação desenvolvida através de um estudo de caso.

\end{itemize}

\section{Metodologia}

Para realizar os objetivos pretendidos, as seguintes operações devem ser executadas. 
Os objetivos parciais estão divididos nas seguintes etapas:

\begin{enumerate}
	\item Estudar os conceitos da computação pervasiva e verificar como aplicá-los em sistemas para monitoramento da saúde de pacientes.
	  \begin{description}
	   \item{(a)} Verificar as tecnologias utilizadas para implementação de aplicações pervasivas;
	   \item{(b)} Verificar como utilizar as tecnologias em aplicações para monitoramento da saúde.
	  \end{description}
	\item Averiguar requisitos para a aplicação obter uma aceitação relevante dos usuários.
	  \begin{description}
	   \item{(a)} Verificar as funcionalidades necessárias para suprir as necessidades dos pacientes em geral;
	   \item{(b)} Verificar uma melhor forma para modelar a interface do usuário e acessibilidade do sistema para obter aceitação por diversos tipos de pacientes;
	   \item{(c)} Verificar políticas de segurança para a manipulação de dados privados dos pacientes. 
	  \end{description}
	\item Efetuar a modelagem e documentação do sistema de acordo com os requisitos analisados.
	  \begin{description}
	   \item{(a)} Realizar a modelagem do sistema;
	   \item{(b)} Documentar o sistema.
	  \end{description}
        \item Desenvolver uma aplicação multiplataforma para o apoio ao cuidado da saúde e realizar um estudo de caso.
	  \begin{description}
	   \item{(a)} Realizar o desenvolvimento da aplicação;
	   \item{(b)} Realizar um estudo de caso da aplicação;
	   \item{(c)} Documentar os resultados do estudo de caso. 
	  \end{description}
	\item Escrever artigos científicos.
        \item Escrita e defesa da dissertação de mestrado.
\end{enumerate}

\section{Cronograma}
O cronograma destas atividades é apresentado na Tabela 1.

\pagebreak

\begin{table}
    \begin{center}
	\begin{tabular}{|c|c|c|c|c|}
		\hline Etapa / Ano & 2011 & 2012 & 2013 & 2014 \\ 
		\hline 1 & $\bullet$ 	&  				&  				&  \\ 
		\hline 2 & $\bullet$ 	& $\bullet$			&  				&  \\ 
		\hline 3 & $\bullet$ 	& $\bullet$ 	& $\bullet$ 					&  \\ 
		\hline 4 &  				& $\bullet$ 	& $\bullet$	&  \\ 
		\hline 5 &  				& $\bullet$	&  $\bullet$				&  \\ 
		\hline 6 &  				& $\bullet$	&  $\bullet$				&  \\ 
		\hline 7 &  				&  				& $\bullet$	& $\bullet$ \\ 
		\hline 8 &  				&  				& 					& $\bullet$ \\\hline 
	\end{tabular} 
	\label{crono}
	\caption{Cronograma de atividades}
    \end{center}
\end{table}

\renewcommand{\bibname}{Referências}

\bibliographystyle{alpha.bst} 
\bibliography{referencias}

\end{document}
