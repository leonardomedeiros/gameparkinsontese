O uso da computação pervasiva aplicada ao contexto de saúde pode melhorar a qualidade de vida dos usuários ao fornecer remotamente informações sobre o estado de saúde. Por outro lado, a concepção de um sistema pervasivo não invasivo de monitoramento de dados motores ainda é um grande desafio multidisciplinar. Estes sistemas, apesar do avanço na tecnologia, ainda são visíveis e estereotipados, dificultando assim sua disseminação. Portanto, o uso destes sistemas não tem sido incorporado na rotina dos usuários, inviabilizando o monitoramento dos sintomas motores.
 
Diante da dificuldade de desenvolver um sistema com as características descritas, neste trabalho, propõe-se utilizar jogos eletrônicos para motivar e abstrair o monitoramento de dados de saúde. Estatísticas da indústria americana de jogos constataram que, em 2011, os jogadores de videogame possuíam em média 37 anos e 29$\%$  estão acima dos 50 anos. Desde 2005, os jogos eletrônicos utilizam sensores de detecção de movimento para capturar as ações cinéticas do usuário. Desta forma, na abordagem proposta neste trabalho, o usuário executa movimentos específicos em um jogo eletrônico para quantificar seus sinais motores e monitorar seu estado de saúde.

A relevância deste trabalho foi validada por meio de pesquisa qualitativa na qual foi realizada uma entrevista semi-estruturada junto a profissionais de saúde. Para a validação da abordagem, foi realizado um estudo analítico de caso-controle para detectar indivíduos diagnosticados com a Doença de Parkinson, utilizando sensores de captura de movimento em jogos eletrônicos. Buscou-se avaliar as possibilidades de aquisição de dados de saúde baseada nas características de Cinemática Angular do Movimento Humano. Estes dados foram aplicados em uma Máquina de Vetor de Suporte (SVM) para classificação dos dados. Como resultado, foi obtida uma taxa de identificação de verdadeiros positivos de 80,00\% e falsos positivos de 16,67\%. Desta forma, concluiu-se que a abordagem proposta permite desenvolver jogos eletrônicos que servem como uma forma não invasiva para monitorar dados motores.



