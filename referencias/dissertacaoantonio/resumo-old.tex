Os avan�os alcan�ados na �rea de tecnologia da informa��o e comunica��o trouxeram a possibilidade de realizar em casa o monitoramento da sa�de, o que antes s� podia ser feito em locais especializados, como cl�nicas e hospitais. O monitoramento realizado no conforto da casa dos indiv�duos permite prover servi�os para um maior n�mero de pessoas, otimizando o uso de recursos humanos e financeiros. Por�m, este sistema de monitoramento enfrenta problemas, sendo o principal deles a dificuldade de uso e a falta de motiva��o dos usu�rios. Embora j� exista a possibilidade e viabilidade em usar sensores para monitoramento, essa � uma atividade que ainda n�o est� integrada ao cotidiano das pessoas.

Para resolver o problema de falta 
de motiva��o nos usu�rios, jogos eletr�nicos v�m sendo fortemente utilizados como ferramenta motivadora. Entretanto, os jogos que tentam motivar o jogador a exercitar-se ou a ser monitorado geralmente requerem que o jogador habitue-se com um estilo de jogo que n�o � parte de seu cotidiano. As a��es realizadas dentro do jogo n�o s�o parte de seus jogos preferidos.

Neste trabalho apresenta-se um arcabou�o de software para monitorar dados de sa�de atrav�s de jogos eletr�nicos. Sua forma de captura de dados permite criar jogos para monitoramento ainda mantendo a jogabilidade. A valida��o do arcabou�o se deu com o desenvolvimento de dois jogos, os quais foram testados com 18 pessoas na faixa de idade de 19 a 27 anos do Laborat�rio Embedded, da Universidade Federal de Campina Grande, e do Instituto Federal de Alagoas. Os resultados indicaram que os jogos foram considerados divertidos pela maioria das pessoas, que tamb�m declararam que os incluiriam em sua rotina di�ria. Al�m disso, foi poss�vel coletar os dados de sa�de de maneira a utiliz�-los para a detec��o de doen�as motoras.