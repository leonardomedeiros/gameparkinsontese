%contexto, problema, objetivo, como validou, resultados alcançados.
%-Contexto
A Propaganda Pervasiva destaca-se das demais formas de veiculação de anúncios por permitir a entrega de anúncios sensíveis ao contexto dos consumidores. Dessa forma, é possível entregar aos consumidores os anúncios mais relevantes, proporcionando uma maior efetividade à propaganda. Entretanto, entregar os anúncios mais relevantes, o que tem sido o foco das pesquisas na área de Propaganda Pervasiva, implica na redução do total de anúncios veiculados e essa redução impacta diretamente nos objetivos da campanha publicitária. Ou seja, nessa busca por anúncios mais relevantes para os consumidores os objetivos dos anunciantes têm sido negligenciados.

Assim, é necessário encontrar um equilíbrio entre os interesses dos consumidores e os objetivos dos anunciantes. A hipótese assumida neste trabalho é que agentes de software, representando os consumidores e os anunciantes, sejam capazes de alcançar o equilíbrio almejado por meio de um cenário de negociação. 

Para viabilizar esse cenário de negociação, foram desenvolvidos: um modelo de negociação, no qual anunciantes e consumidores estabelecem rodadas de propostas e contrapropostas, cada um defendendo seus próprios interesses, a fim de obter uma relação de equilíbrio; um modelo de obtenção de informação contextual, que permite que os agentes adquiram, de forma dinâmica e transparente, a informação necessária à negociação; e um modelo de representação da informação contextual, que fornece uma linguagem comum para a comunicação entre os agentes. 

Para validar os modelos desenvolvidos, foi realizada uma série de experimentos com foco na obtenção e utilização de informação contextual como fator predominante para determinar a relevância do anúncio. A análise dos resultados obtidos permite concluir que o modelo de negociação mostra-se eficaz em encontrar o equilíbrio desejado.
