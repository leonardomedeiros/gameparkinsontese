Health monitoring, which could only be done in specialized places, such as clinics and hospitals, can now be performed at home thanks to the advances in Information and Communication Technologies. Monitoring health data in the comfort of one's homes allows the provision of services to a greater number of people, optimizing human and financial resources. However, this system faces problems such as difficulty of use and lack of motivation in users. Although it is feasible to use sensors to monitor people, this is still something not easy to integrate to their daily lives. 

To tackle the lack of motivation among people who need to be monitored, electronic games have been extensively used as a motivational tool. However, games that intend to motivate the player to exercise or to be monitored usually have a different gameplay than what the player would enjoy.

This work presents a software framework to monitor health data through electronic games, allowing it to be acquired while keeping the playability. The framework was used for the development of two games, tested by 18 people with ages ranging from 19 to 27, from the Embedded Laboratory at the Federal University of Campina Grande, and Federal Institute of Alagoas. Results showed that most people found the games to be funny, and also stated that they would include the games in their daily routine. Also, it was possible to collect health data to be used in the detection of motor diseases.